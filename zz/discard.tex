% !TEX root = ../oddp.tex

\section{To be discarded}

\subsection{Comparison with the surjection operad (drafty)}

Consider the functor tensor product $F^*\otimes G$, that is, $N_*(\Delta^\bullet)\otimes_\Delta N_\bullet(\Delta^n)$. It is related to the previous tensor product via Poincaré duality and it is isomorphic to $N_*(\Delta^n)^{\otimes r}$. Similarly, there is a chain complex analogous to $\Omega(r,n)$ that we denote, for the moment as $\Omega^\vee(r,n)$. Instead of using the previous isomorphism, we will use the isomorphism that sends $u_i\in U$ to the $w_i$-th factor of the tensor product.

If $\tau$ is a simplex of dimension $m$, one consider all pairs $(f,A)$ where
\begin{itemize}
	\item $f$ is a surjection, i.e., a non-degenerate word in $EC_r$ that contains every element $0,1,\ldots,r-1$ at least once.
	\item $A$ is a decomposition of a simplex $\tau$ into overlapping intervals.
\end{itemize}
Now,
\begin{itemize}
	\item The pair $(f,A)$ defines a word in $EC_r$ given by replicating each number in $f$ as many times as the length of each interval dictates. For example, if $r=3$ and $f= 0121012$ and $A=(0,1,2)(2)(2,3)(3)(3,4,5)(5,6)(6)$, then the resulting word is $w=0001221000112$.
	\item $A$ gives a simplex in $\Delta^m$ that results from erasing the parentheses: $U=[0,1,2,2,2,3,3,3,4,5,5,6,6]$
\end{itemize}
The triple $(f,A,\tau)$ thus yields an element of $\Omega^\vee(r,n)$, whose image coincides with the image of the operation $(f,A)$ on $\tau$.

If we want to interpret this element in $\Omega(r,n)$, we just need to do the following:
\begin{itemize}
	\item The new $U$ is obtained from the maximally degenerate simplex \[(0,0,0,1,1,1,2,2,2,3,3,3,4,4,4,5,5,5,6,6,6)\] by removing the elements of $U$. In our case, $U=(0,0,1,1,4,4,5,6)$.
	\item The new $w$ is obtained from $w$ by setting $w_i$ equal to any element of $C_r$ different from the ones that where labeling $u_i$ in the old $w$ \emph{provided that the old ones were not repeated. Otherwise, the operation is zero!!. Nonetheless, these cases may be easily identified}. This happens quite often. In our case $w=(12121220)$.
\end{itemize}
Finally, we need to substract $u_i$ from $w_i$ to obtain the word we care about in $\Omega(r,n)$: (12010100)

If $f$ is a surjection, let $w$ be the pieced word that comes from replacing each entry by its complement. Let $Q(f)$ be the collection of all subwords of all representatives of $w$.
\begin{lemma} In this dictionary (without the last substraction), the collection of words associated to a surjection $f$ is $Q(f)$.
\end{lemma}

\subsection{The even prime [TO BE DELETED]} This strategy can be also applied to recover the presentation in [ANIBAL] of the classical formulas of Steenrod for the cup-$i$ products. All the maps remain the same except for $\varphi\colon W_*(r)\otimes N_*(\Delta^n)\to \WW^*(r)\otimes N^r_*(\Delta^n)$, which is no longer well-defined because $(r-1)m$ is not always even. To remedy this handicap, consider the other (suspended) resolution $\VV_*(2)$ of $C_2$:
\[R\langle C_2\rangle \overset{T-1}{\lra} R\langle C_2\rangle \overset{1+T}{\lra} R\langle C_2\rangle \overset{T-1}{\lra} R.\]
There is a flip map $\varphi_k\colon W_*(2)\to \VV^*(2)$ for every odd $k$, and there are suspension maps $\theta\colon \WW^*(r)\to \Sigma \VV^*(r)$ and $\theta\colon \VV^*(r)\to \Sigma \WW^*(r)$ that send a generator $e_q$ to $e_{q-1}$. Consider the complex
\[\mathbf{U}^*(2,n) = \bigoplus_{m\text{ odd}} \VV^m(r)\otimes N^2_m(\Delta^n)
oplus
\bigoplus_{m\text{ even}} \WW^m(r)\otimes N^2_m(\Delta^n)
\]
with differential $\partial(e_q^\vee\otimes \vec{\tau}) = \partial(e_q^\vee)\otimes \vec{\tau} + (-1)^q \theta(e_q)\otimes \partial \vec{\tau}$. Define
\[\varphi\colon \varphi\colon W_*(r)\otimes N_*(\Delta^n)\to
\mathbf{U}^*(2,n)\]
as $\varphi(e_q\otimes \tau) = \varphi_m(e_q)\otimes \vec{\tau}$.

In this case $\Theta(2,n) = \NN^*(\Delta^\bullet)\otimes_{\Delta} N_*^2(\Delta^n)$. Define a map $\Psi\colon U^*(2,n)\to \Theta(2,n)$ as $e_q\otimes \vec{\tau}\mapsto \Psi(e_q)\otimes \vec{\tau}$ where
\begin{align*}
	\WW^*(r)\lra N^*(EC_r) \lra N^*(\Delta^n\times EC_r) \\
	\VV^*(r)\lra N^*(EC_r) \lra N^*(\Delta^n\times EC_r) \\
\end{align*}
{color{red}THIS DOES NOT WORK. I think twisted coefficients might be needed}
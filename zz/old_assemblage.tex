
If $r=5$, then there is a cyclic subdivision map $h\colon \Psd \partial \Delta^4\to \partial \Delta^4$ satisfying both conditions:
\begin{align*}
    h([0],[0]) &= [0] &
	h([01],[01]) &= [0] &
	h([02],[02]) &= [0] \\
	h([012],[012]) &= [0] &
	h([013],[013]) &= [0] &
	h([0123],[0123]) &= [0]
\end{align*}
It yields the following map $f\colon N_*(\partial \Delta^4)\otimes N_*(\partial \Delta^4)\to \Sigma^4N_*(\partial \Delta^4)$ (again, we only indicate the non-zero values of $f$ on a cyclic basis):
\begin{align*}
	f([0]\otimes[1,2,3,4]) &= [0] &
	f([0,1]\otimes [2,3,4]) &= [0] \\
	f([0,2]\otimes [1,3,4]) &= [0] &
	f([0,1,2]\otimes [3,4]) &= [0] \\
	f([0,1,3]\otimes [2,4]) &= [0] &
	f([0,1,2,3]\otimes [4]) &= [0] \\
	f([0,1]\otimes [2,3,4,0]) &= [0,1] &
	f([0,2]\otimes [1,3,4,0]) &= [0,2] \\
	f([0,1,3]\otimes [2,4,0]) &= [0,1] &
	f([0,1,3]\otimes [2,4,1]) &= [0,3] \\
	f([0,1,2,3]\otimes [4,0]) &= [0,1] &
	f([0,1,2,3]\otimes [4,1]) &= [2,0] \\
	f([0,1,2]\otimes [3,4,0]) &= [0,1,2] &
	f([0,1,3]\otimes [2,4,0,1]) &= [0,1,3] \\
	f([0,1,2,3]\otimes [4,0,1]) &= [0,1,2] &
	f([0,1,2,3]\otimes [4,1,2]) &= [0,2,3] \\
	f([0,1,2,3]\otimes [4,0,1,2]) &= [0,1,2,3]
\end{align*}
This map is not unique, the map obtained by redefining $h([013],[013]) = [3]$ also extends.

If $r=7$, then there is a cyclic subdivision map $h\colon \Psd \partial \Delta^6\to \partial \Delta^6$ satisfying both conditions. Below are the vertices that map to the vertex $0$, which cyclically determine the map $h$:
\begin{align*}
			(0) &&
			(0,1) &&
			(0,2) &&
			(0,3) \\
			(0,1,2) &&
			(0,1,3) &&
			(0,1,4) &&
			(0,2,3) &&
			(0,2,4) \\
			(0,1,2,3) &&
			(0,1,2,4) &&
			(0,1,3,4) &&
			(0,2,3,4) &&
			(0,1,3,5) \\
			(0,1,2,3,4) &&
			(0,1,2,3,5) &&
			(0,1,3,4,5) &&
			(0,1,2,3,4,5)
\end{align*}
It yields the following map $f\colon N_*(\partial \Delta^6)\otimes N_*(\partial \Delta^6)\to \Sigma^6N_*(\partial \Delta^6)$ (again, we only indicate the non-zero values of $f$ on a cyclic basis):
\begin{align*}
f( [0] \otimes [1, 2, 3, 4, 5, 6] )&= [0] &
f( [0, 1] \otimes [2, 3, 4, 5, 6] )&= [0] \\
f( [0, 2] \otimes [1, 3, 4, 5, 6] )&= [0] &
f( [0, 3] \otimes [1, 2, 4, 5, 6] )&= [0] \\
f( [0, 1, 2] \otimes [3, 4, 5, 6] )&= [0] &
f( [0, 1, 3] \otimes [2, 4, 5, 6] )&= [0] \\
f( [0, 1, 4] \otimes [2, 3, 5, 6] )&= [0] &
f( [0, 2, 3] \otimes [1, 4, 5, 6] )&= [0] \\
f( [0, 2, 4] \otimes [1, 3, 5, 6] )&= [0] &
f( [0, 1, 2, 3] \otimes [4, 5, 6] )&= [0] \\
f( [0, 1, 2, 4] \otimes [3, 5, 6] )&= [0] &
f( [0, 1, 3, 4] \otimes [2, 5, 6] )&= [0] \\
f( [0, 1, 3, 5] \otimes [2, 4, 6] )&= [0] &
f( [0, 2, 3, 4] \otimes [1, 5, 6] )&= [0] \\
f( [0, 1, 2, 3, 4] \otimes [5, 6] )&= [0] &
f( [0, 1, 2, 3, 5] \otimes [4, 6] )&= [0] \\
f( [0, 1, 3, 4, 5] \otimes [2, 6] )&= [0] &
f( [0, 1, 2, 3, 4, 5] \otimes [6] )&= [0] \\
f( [0, 1] \otimes [0, 2, 3, 4, 5, 6] )&= [0, 1] &
f( [0, 2] \otimes [0, 1, 3, 4, 5, 6] )&= [0, 2] \\
f( [0, 3] \otimes [0, 1, 2, 4, 5, 6] )&= [0, 3] &
f( [0, 1, 2] \otimes [0, 3, 4, 5, 6] )&= [0, 1] \\
f( [0, 1, 3] \otimes [0, 2, 4, 5, 6] )&= [0, 1] &
f( [0, 1, 4] \otimes [0, 2, 3, 5, 6] )&= [0, 1] \\
f( [0, 1, 4] \otimes [1, 2, 3, 5, 6] )&= [0, 4] &
f( [0, 2, 3] \otimes [0, 1, 4, 5, 6] )&= [0, 2] \\
f( [0, 2, 4] \otimes [0, 1, 3, 5, 6] )&= [0, 2] &
f( [0, 2, 4] \otimes [1, 2, 3, 5, 6] )&= [0, 4] \\
f( [0, 1, 2, 3] \otimes [0, 4, 5, 6] )&= [0, 1] &
f( [0, 1, 2, 4] \otimes [0, 3, 5, 6] )&= [0, 1] \\
f( [0, 1, 3, 4] \otimes [0, 2, 5, 6] )&= [0, 1] &
f( [0, 1, 3, 4] \otimes [1, 2, 5, 6] )&= [0, 3] \\
f( [0, 1, 3, 5] \otimes [0, 2, 4, 6] )&= [0, 1] &
f( [0, 1, 3, 5] \otimes [1, 2, 4, 6] )&= [0, 3] \\
f( [0, 1, 3, 5] \otimes [2, 3, 4, 6] )&= [0, 5] &
f( [0, 2, 3, 4] \otimes [0, 1, 5, 6] )&= [0, 2] \\
f( [0, 2, 3, 4] \otimes [1, 2, 5, 6] )&= [0, 3] &
f( [0, 1, 2, 3, 4] \otimes [0, 5, 6] )&= [0, 1] \\
f( [0, 1, 2, 3, 5] \otimes [0, 4, 6] )&= [0, 1] &
f( [0, 1, 2, 3, 5] \otimes [1, 4, 6] )&= [0, 2] \\
f( [0, 1, 2, 3, 5] \otimes [3, 4, 6] )&= [0, 5] &
f( [0, 1, 3, 4, 5] \otimes [0, 2, 6] )&= [0, 1] \\
f( [0, 1, 3, 4, 5] \otimes [1, 2, 6] )&= [0, 3] &
f( [0, 1, 3, 4, 5] \otimes [2, 3, 6] )&= [0, 4] \\
f( [0, 1, 2, 3, 4, 5] \otimes [0, 6] )&= [0, 1] &
f( [0, 1, 2, 3, 4, 5] \otimes [1, 6] )&= [0, 2] \\
f( [0, 1, 2, 3, 4, 5] \otimes [3, 6] )&= [0, 4] &
f( [0, 1, 2] \otimes [0, 1, 3, 4, 5, 6] )&= [0, 1, 2] \\
f( [0, 1, 3] \otimes [0, 1, 2, 4, 5, 6] )&= [0, 1, 3] &
f( [0, 1, 4] \otimes [0, 1, 2, 3, 5, 6] )&= [0, 1, 4] \\
f( [0, 2, 3] \otimes [0, 1, 2, 4, 5, 6] )&= [0, 2, 3] &
f( [0, 2, 4] \otimes [0, 1, 2, 3, 5, 6] )&= [0, 2, 4] \\
f( [0, 1, 2, 3] \otimes [0, 1, 4, 5, 6] )&= [0, 1, 2] &
f( [0, 1, 2, 4] \otimes [0, 1, 3, 5, 6] )&= [0, 1, 2] \\
f( [0, 1, 3, 4] \otimes [0, 1, 2, 5, 6] )&= [0, 1, 3] &
f( [0, 1, 3, 4] \otimes [1, 2, 3, 5, 6] )&= [0, 3, 4] \\
f( [0, 1, 3, 5] \otimes [0, 1, 2, 4, 6] )&= [0, 1, 3] &
f( [0, 1, 3, 5] \otimes [0, 2, 3, 4, 6] )&= [0, 1, 5] \\
f( [0, 1, 3, 5] \otimes [1, 2, 3, 4, 6] )&= [0, 3, 5] &
f( [0, 2, 3, 4] \otimes [0, 1, 2, 5, 6] )&= [0, 2, 3] \\
f( [0, 2, 3, 4] \otimes [1, 2, 3, 5, 6] )&= [0, 3, 4] &
f( [0, 1, 2, 3, 4] \otimes [0, 1, 5, 6] )&= [0, 1, 2] \\
f( [0, 1, 2, 3, 5] \otimes [0, 1, 4, 6] )&= [0, 1, 2] &
f( [0, 1, 2, 3, 5] \otimes [0, 3, 4, 6] )&= [0, 1, 5] \\
f( [0, 1, 2, 3, 5] \otimes [1, 2, 4, 6] )&= [0, 2, 3] &
f( [0, 1, 2, 3, 5] \otimes [1, 3, 4, 6] )&= [0, 2, 5] \\
f( [0, 1, 3, 4, 5] \otimes [0, 1, 2, 6] )&= [0, 1, 3] &
f( [0, 1, 3, 4, 5] \otimes [0, 2, 3, 6] )&= [0, 1, 4] \\
f( [0, 1, 3, 4, 5] \otimes [1, 2, 3, 6] )&= [0, 3, 4] &
f( [0, 1, 3, 4, 5] \otimes [2, 3, 4, 6] )&= [0, 4, 5] \\
f( [0, 1, 2, 3, 4, 5] \otimes [0, 1, 6] )&= [0, 1, 2] &
f( [0, 1, 2, 3, 4, 5] \otimes [0, 3, 6] )&= [0, 1, 4] \\
f( [0, 1, 2, 3, 4, 5] \otimes [1, 2, 6] )&= [0, 2, 3] &
f( [0, 1, 2, 3, 4, 5] \otimes [1, 3, 6] )&= [0, 2, 4] \\
f( [0, 1, 2, 3, 4, 5] \otimes [3, 4, 6] )&= [0, 4, 5]
\end{align*}
\begin{align*}
f( [0, 1, 2, 3] \otimes [0, 1, 2, 4, 5, 6] )&= [0, 1, 2, 3] \\
f( [0, 1, 2, 4] \otimes [0, 1, 2, 3, 5, 6] )&= [0, 1, 2, 4] \\
f( [0, 1, 3, 4] \otimes [0, 1, 2, 3, 5, 6] )&= [0, 1, 3, 4] \\
f( [0, 1, 3, 5] \otimes [0, 1, 2, 3, 4, 6] )&= [0, 1, 3, 5] \\
f( [0, 2, 3, 4] \otimes [0, 1, 2, 3, 5, 6] )&= [0, 2, 3, 4] \\
f( [0, 1, 2, 3, 4] \otimes [0, 1, 2, 5, 6] )&= [0, 1, 2, 3] \\
f( [0, 1, 2, 3, 5] \otimes [0, 1, 2, 4, 6] )&= [0, 1, 2, 3] \\
f( [0, 1, 2, 3, 5] \otimes [0, 1, 3, 4, 6] )&= [0, 1, 2, 5] \\
f( [0, 1, 2, 3, 5] \otimes [1, 2, 3, 4, 6] )&= [0, 2, 3, 5] \\
f( [0, 1, 3, 4, 5] \otimes [0, 1, 2, 3, 6] )&= [0, 1, 3, 4] \\
f( [0, 1, 3, 4, 5] \otimes [0, 2, 3, 4, 6] )&= [0, 1, 4, 5] \\
f( [0, 1, 3, 4, 5] \otimes [1, 2, 3, 4, 6] )&= [0, 3, 4, 5] \\
f( [0, 1, 2, 3, 4, 5] \otimes [0, 1, 2, 6] )&= [0, 1, 2, 3] \\
f( [0, 1, 2, 3, 4, 5] \otimes [0, 1, 3, 6] )&= [0, 1, 2, 4] \\
f( [0, 1, 2, 3, 4, 5] \otimes [0, 3, 4, 6] )&= [0, 1, 4, 5] \\
f( [0, 1, 2, 3, 4, 5] \otimes [1, 2, 3, 6] )&= [0, 2, 3, 4] \\
f( [0, 1, 2, 3, 4, 5] \otimes [1, 3, 4, 6] )&= [0, 2, 4, 5] \\
f( [0, 1, 2, 3, 4] \otimes [0, 1, 2, 3, 5, 6] )&= [0, 1, 2, 3, 4] \\
f( [0, 1, 2, 3, 5] \otimes [0, 1, 2, 3, 4, 6] )&= [0, 1, 2, 3, 5] \\
f( [0, 1, 3, 4, 5] \otimes [0, 1, 2, 3, 4, 6] )&= [0, 1, 3, 4, 5] \\
f( [0, 1, 2, 3, 4, 5] \otimes [0, 1, 2, 3, 6] )&= [0, 1, 2, 3, 4] \\
f( [0, 1, 2, 3, 4, 5] \otimes [0, 1, 3, 4, 6] )&= [0, 1, 2, 4, 5] \\
f( [0, 1, 2, 3, 4, 5] \otimes [1, 2, 3, 4, 6] )&= [0, 2, 3, 4, 5] \\
f( [0, 1, 2, 3, 4, 5] \otimes [0, 1, 2, 3, 4, 6] )&= [0, 1, 2, 3, 4, 5] \\
\end{align*}

\begin{lemma}
    If $f\colon \partial \Delta^{r-1}*\partial\Delta^{r-1}\to \Sigma^{r-1}\partial \Delta^{r-1}$ is a $C_r$-equivariant map, then $r$ is prime.
\end{lemma}
\begin{proof}
    This is due to the fact that the action of $C_r$ of $\partial \Delta^{r-1}$ is always free, while the action of $C_r$ on the join is free only if $r$ is prime: if $r=pq$, then $\rho$ has order $q$ on the simplex $\tau_1 = [0,p,2p,\ldots,(q-1)p]$ and on its complementary simplex $\tau_2 = PD(\tau_1)$. Therefore $f(\rho^q(\tau_1,\tau_2)) = f(\tau_1,\tau_2)\neq \rho^q(f(\tau_1,\tau_2))$.
\end{proof}
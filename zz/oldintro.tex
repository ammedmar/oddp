\subsection{old}



These operations where known to come from generalised $\smile_i$-products
\begin{align}\label{eq:1}
C^{k_1}(X;\bF_r)\otimes\overset{r}{\ldots}\otimes C^{k_r}(X;\bF_r)\lra C^{k_1+\ldots +k_r + i}(X;\bF_r),\qquad i\geq 0
\end{align}
though no explicit formula was pursued until the work of McClure and Smith \cite{MS02} and Berger and Fresse \cite{BF03} based on previous work by Benson \cite{Benson}. These operations were known to be the image of the homology of the symmetric group under a certain equivariant map 
\[E\Sigma_r\lra \hom(C^*(X;\bZ)^{\otimes r},C^*(X;\bZ))\]
that was thoroughly described. All these maps together yield an action of the Barratt--Eccles operad on the cochains of a simplicial complex, or more generally, of a simplicial set. The inclusion $C_r\hookrightarrow \Sigma_r$ of the cyclic group of order $r$ into the symmetric group on $r$ letters is a surjection in homology. Recently \cite{Anibaletal} a homomorphism
\[\cW_*(r)\lra E\Sigma_r\]
from the minimal resolution of $C_r$ has been found. Altogether these maps yield explicit maps as \eqref{eq:1}.

In our study of odd power operations for $\Sigma$-spectra or Khovanov homology, we have found that these cochain operations have three inconveniences for their adaptation to the stable setting:
\federico{Probably it is better to emphasize cochain operations in $\Sigma$-spectra as in \cite{Gill20} more than Khovanov homology, since the interested audience is broader}
\begin{enumerate}
\item {\bf Efficiency:} The definition of the cochain operations is recursive on the degree of the cochain operation. A standard PC cannot perform almost any power operation for the prime $5$ or higher. 
In this article, we introduce new formulas for the cochain operations that solve these three problems at once. In addition, they are described with the language of face maps, which has a more natural taste than the language of overlapping intervals used in \cite{MS02} and \cite{BF03}. We will work with augmented pointed simplicial sets.
\end{enumerate}
The \emph{pair subdivision} of the $n$th simplex is a cubulation $\Psd \Delta^n$ of the standard simplex $\Delta^n$ whose vertices are the barycenters of the faces of $\Delta^n$ (see Definition \ref{TBW}). An \emph{enhanced pair subdivision map} is a map $\Psd \Delta^n\to \Delta^n$ satisfying certain properties (see Definition \ref{TBW}). We write $\Delta^n_+$ for the augmented simplex with a disjoint basepoint and $\NN^*(\Delta^n_+)$ for the normalized cochains on $\Delta^n_+$. %Let $R$ be a commutative ring that is an algebra over $\bZ\left[\frac{1}{\tilde{r}!}\right]$, where $\tilde{r} = \frac{r-1}{2}$.
\begin{theorem} 
	Let $r$ be an odd prime. Each enhanced pair subdivision map $\Psd \Delta^{r-1}\to \Delta^{r-1}$ defines an explicit natural homomorphism on any augmented simplicial set $X$
	\[\Psi\colon W_*(r)\lra \hom(\NN^*(X)^{\otimes r},\NN^*(X)).\]
	The power operations $P^k(x)$ and $\beta P^k(x)$ are computed as $\nu(m)\cdot \Psi(e_{m(r-1)-2i(r-1)})(x\otimes\overset{r}{\ldots}\otimes x)$ and $\nu(m)\cdot \Psi(e_{m(r-1)-2i(r-1)+1})(x\otimes\overset{r}{\ldots}\otimes x)$ respectively, with $R=\bF_r$ and $m$ the dimension of $x$.
\end{theorem}
\federico{Here $\nu(m)$ should be a factor of $\frac{1}{(\tilde{r}!)^m}$, otherwise the operation is not stable}

%Examples of algebras over $\bZ\left[\frac{1}{\tilde{r}!}\right]$ are $\bZ_{(r)}$, the integers localized at $r$, the rationals $\bQ$ or $\bF_r$ the finite field with $r$ elements. The ring $R$ may be taken to be $\bZ$ at the cost of loosing part of the compatibility with suspension. 
This compatibility has the following form:
\begin{theorem}
 Let $X$ be an augmented pointed simplicial set and let $\Sigma X$ be its simplicial suspension. Let $\rho$ be the standard generator of the cyclic group $C_r$. Then 
\[\Psi(\rho(e_{k-r+1)})(\Sigma x_1\otimes\overset{r}{\ldots}\otimes \Sigma x_r) = \tilde{r}!\Sigma\Psi(e_{k})(x_1\otimes \overset{r}{\ldots} \otimes x_r).
\]
\end{theorem}

In Section \ref{sec:BLA} we give three explicit pair subdivision maps for the primes $r=3,5,7$. In Section \ref{sec:BLA} we give a succint description of these formulas. 

The homomorphism $\Psi$ does not seem to factor in general through the Barratt-Eccles operad or the surjection operad. We have empirically tested by computer that our operations for $r=3$ coincide with the operations of \cite{Anibaletal}. For $r=5$ the operations of \cite{Anibaletal} cannot come from our method. 


As we are considering augmented objects, the degree of the operations has a shift of $r-1$ with respect to the classical degree. All the ``old'' operations in classical degree $0$ coincided with the Alexander-Whitney diagonal. Our operations in augmented degree $r-1$ do coincide with the Alexander-Whitney diagonal up to multiplication by a scalar (see \ref{BLA}). The augmented context allows for an extra operation $P^{-1}$ that has augmented degree $0$ and classical degree $r-1$. This operation is always zero except in degree $-1$, in which $P^{-1}(x) = x^r$. This operation is induced by a cochain operation in augmented degree $0$
\[\NN^{k}(X)\otimes \NN^{0}(X)\otimes\overset{r-1}{\ldots}\otimes \NN^{0}(X)\lra \NN^{k}(X)\]
that sends a cochain $(x_1,\ldots,x_r)$ to the cochain $x$ whose value on a $k$-simplex $\tau$ is $x_1(\tau)\cdot \prod_{i\geq 2} x_i(d_{\mathrm{all}}\tau)$, where $d_{\mathrm{all}}\colon \NN_k(X)\to \NN_{-1}(X)$ is the augmentation. 




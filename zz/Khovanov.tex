%intro

\begin{theorem}
    If an $r$-cyclic desymmetrisation for $\cC$ and an $r$-cyclic assymetry are given, then there are explicit $r$-cyclic comultiplications on the chain complex of any simplicial object in the category $\cC$. These operations are natural with respect to maps in the subcategory $\cC'$ while the induced power operations are natural with respect to any map.
\end{theorem}


    The theorem remains true if we work with semi-simplicial objects or an augmented semi-simplicial objects. The theorem allows $\cC$ to be a higher category, in which case the simplicial objects are lax.

%comultiplications

The connected comultiplications that we construct in this work will not be completely natural because we did not impose the diagonal $\Delta$ to be natural. Nonetheless, $\Delta$ is natural at least for inclusions $K\hookrightarrow K\coprod K'$ of a factor of a coproduct (\emph{split maps}). This suffices to show that they will be natural at least for maps $f\colon X\to Y$ between augmented semi-simplicial objects in $\cC$ that are levelwise split. This will suffice to prove that the cohomology operations are natural with respect to any map.

\begin{remark}
    If $\cC$ is a category satisfying the previous hypotheses, then its pointed category $\cC_*$ has coproducts and a tensor product defined as $(A,p)\otimes (B,p) = A\otimes B/(A\otimes p)\amalg (p\otimes B)$. The diagonal $\Delta$ in $\cC$ induced a diagonal in $\cC_*$ and the linearization functor factors through $\cC_*$. 
\end{remark}

\begin{lemma}
    If a connected comultiplication is natural with respect to levelwise split maps, then the operations
    \[
        \power^i\colon H^*(X;\bF_p)\lra H^*(X;\bF_p)
    \]
    are natural with respect to any map.
\end{lemma}
\begin{proof}
    By the previous remark we may assume that the category $\cC$ is pointed with zero object $p$. Let $f\colon X\to Y$ be a map between augmented semi-simplicial objects. Define the \emph{double-suspension mapping cylinder of $f$} as the semi-simplicial object $M(\Sigma^2 f)$ whose $n$-simplices are $X_{n-3}\amalg X_{n-2}\amalg Y_{n-2}$ and where the face maps act as follows on elements $a\in X_{n-3}, \bar{a}\in X_{n-2}$ and $b\in Y_{n-2}$:
    \begin{align*}
        d_i(a) &= \begin{cases}
            \bar{a} & \text{if $i=0$} \\
            f(b) & \text{if $i=1$} \\
            p & \text{if $i=2$}
            d_{i-3}(a) & \text{if $i\geq 3$}
        \end{cases}
        &\\
        d_i(\bar{a}) &= \begin{cases}
            * & \text{if i=0,1} \\
            d_{i-2}(\bar{a}) & \text{if $i\geq 1$} 
        \end{cases}
        & 
        d_i(b) &= \begin{cases}
            * & \text{if i=0,1} \\
            d_{i-2}(b) & \text{if $i\geq 1$} 
        \end{cases}
    \end{align*}
    Then there are inclusions $\Sigma^2 X\overset{\iota}{\to} M(\Sigma^2f) \overset{\iota'}{\leftarrow} \Sigma^2 Y$. The levelwise maps of these inclusions are inclusions of summands, for which the cyclic comultiplication is natural. Since $\abel(M(\Sigma^2 f))$ is the double suspension fo the mapping cylinder of $\abel(f)$, we find that there is an inverse map $g\colon \abel(M(\Sigma^2 f))\to \abel(\Sigma^2(Y))$ to $\abel(\iota')$ such that $\abel(f) = g\circ \abel(\iota)$. Since $\abel(\iota')$ preserves power operations, so does $g$, and so does $g\circ \abel(\iota)$.
\end{proof}




%complexes
A semi-simplicial object $F$ in $\cC$ has dimension $n$ if $F_m$ is the initial object $\emptyset$ for all $m>n$. In that case, define the following semi-simplicial objects of dimension $n-1$: Let $d_j$ be the $j$th face map of $F$.
\begin{itemize}
    \item $(d_iF)_m = F_m$ if $m<n$ and $(d_i F)_n = \emptyset$. The face map $d_j'$ of $(d_iF)$ equals the restriction to $d_iF$ of the face map $d_j$.
    \item $(\delta^iF)_m = F_{m+1}$ if $m<n$. The face map $d_j''$ of $\delta^i F$ equals the face map $d_j$ of $F$ if $j<i$ and the face map $d_{j+1}$ of $F$ if $j\geq i$.
\end{itemize} 
Write $d_i^n$ for the face map $d_i\colon F_n\to F_{n-1}$. 
\begin{definition}
    A \emph{resolution of the diagonal $\Delta$} assigns to each augmented semi-simplicial object $F$ of dimension $n$ in the category $\cC$ a homomorphism $\Theta(F)\colon \abel(F_n)\to \abel(F_{-1})^{\otimes r}$ such that
    \[
        \sum_i (-1)^i\Theta(d_iF)\circ \abel(d_i^n) - \abel(d_0)^{\otimes r}\circ\Theta(\delta^iF) = \begin{cases}
        N\Theta(F) & \text{ if $n$ is odd} \\
        T\Theta(F) & \text{ if $n$ is even},
        \end{cases}
    \]
    and if $F$ has dimension -1, then $\Theta(F) = \abel(\Delta)$.
\end{definition}
\begin{remark}
    Suppose that $F$ is a semi-simplicial object in $\cC$ with $F_n = Y'\amalg Y''$ and $F'$ is the cube that results from $F$ by redefining $F'_n = Y'$ and $F''$ is the cube that results from $F$ by redefining $F''_n = Y''$. If $\Theta$ has been defined on $F'$ and $F''$, then one can extend $\Theta$ to $F$ as $\Theta(F) = \Theta(F')\amalg \Theta(F'')$.
\end{remark}



\begin{proof}
	\begin{align*}
		\delta\hat{\Psi}^{\vee}(e_q)\otimes \vec{\tau}
		&= (\delta\hat{\Psi}^{\vee}(e_q))^{0}\otimes\vec{\tau} + (\delta\hat{\Psi}^{\vee}(e_q))^{1}\otimes\vec{\tau} \\
		&= \hat{\Psi}^{\vee}(\delta e_q)\otimes \vec{\tau} + \sum d_i\Psiom^\vee_{q-r+1}(e_{q-r+1})\otimes\vec{\tau} \\
		%&= \hat{\Psi}^{\vee}(\delta e_{q})\otimes \vec{\tau} + (-1)^q \sum_i (-1)^{\nu_i(-)} D_i(\Psiom^\vee_{q-r+1}(e_{q-r+1}))\otimes \vec{\tau}\\
		&= \hat{\Psi}^{\vee}(\delta e_q)\otimes \vec{\tau} + (-1)^q\sum_i \Psiom^\vee_{q-r+1}(e_{q-r+1})\otimes d_i\vec{\tau} \\
		&= \Psi(\delta(e_q\otimes \vec{\tau}))\qedhere
	\end{align*}
\end{proof}


% !TEX root = ../oddp.tex

\section{A higher cup-product map}

In this section we build, from a certain equivariant map between spheres, a chain homomorphism with coefficients in $\bZ[\frac{1}{\tilde{r}!}]$.
\[\Psi\colon V^*(r)\to \bar{\Omega}^*(r,n)\]
Let $\Omega_*(r,m)$ be generated in degree $q$ by the equivalence classes of pairs $(U,w)$ such that $w$ has at most one full piece, for the relation generated by inner reorderings and standarizations. The differential $\sum_{i} (-1)^i(d_iU,d_iw)$ is well-defined in classes. $\Omega_*(r,m)^{nf}$ is the subcomplex of words without full pieces and $\Omega_*(r,m)^{f}$ is the submodule of words with full pieces.

Let $\Phi\colon \Omega_*(r,m)^f\to \Sigma^{r}\Omega_{*}(r,m-1)^{nf}$ be the (non-chain) homomorphism that sends a pair $(U,w)$ to the result of removing a full piece if it is possible and decrease all entries in $U$ to the right of the piece, and increase all entries in $w$ to the right of the piece.

\begin{remark}
	For the $D_i$ defined in the ``full piece removal'' condition, $\sum_iD_i(U,w) = \Phi^{\vee}(U,w)$ if $\tau$ has dimension $m$.
\end{remark}

\begin{lemma} \label{lemma:omegarm}
	Suppose one is given a family of $C_r$-equivariant maps
	\begin{align*}
		\hat{\Psi}\colon \Omega_*(r,m)^{nf}&\lra W_*(r)
	\end{align*}
	related by the following equation:
	\begin{align}\label{it:1a}
		\hat{\Psi}_{q-1}((\partial (U,w)^{nf}) &= \hat{\Psi}_{q-r}\Phi(U,w) &\text{if $(U,w)$ has a full piece}
	\end{align}
	%\begin{enumerate}
	%\item\label{it:1} $\hat{\Psi}_q\partial = \partial \hat{\Psi}_q + \hat{\Psi}_{q-r}\Phi$ !!!
	%\item\label{it:1a} $\hat{\Psi}_{q-1}((\partial (U,w)^{nf}) = \hat{\Psi}_{q-r}\Phi(U,w)$ if $w$ has a full piece.
	%\item\label{it:2a} $\hat{\Psi}_{q-1}((\partial (U,w)) = \partial \hat{\Psi}_{q}( (U,w))$ if $w$ has no full pieces.
	%\item\label{it:2} $\Phi(S_m(U,w)) = S_{m}(\Phi(U,w))$
	%\item\label{it:3} $\hat{\Psi}S_{m-1}\partial = \hat{\Psi}\partial S_m$ if $w$ has no full pieces
	%\item\label{it:4} $\hat{\Psi}_{r-2} = \theta_{r-1}\hat{\Psi}_{-1}S_{r-2}$
	%\end{enumerate}
	%Then $\hat{\Psi}$ extends recursively to the whole $\Omega_*(r,m)^{nf}$ as
	%\[
	%\hat{\Psi}_q =\begin{cases}
	%\hat{\Psi}_q & \text{if $q\leq r-2$} \\
	%\theta_{r-1}\hat{\Psi}_{q-r+1}S & \text{if $q\geq r-2$}.
	%\end{cases}
	%\]
	%and satisfies \eqref{it:1} and \eqref{it:3}, and
	Then there is a chain homomorphism
	\[\Psi\colon V^*(r)\lra \bar{\Omega}^*(r,n)\]
	defined as $\Psi(e_q\otimes \tau) = \hat{\Psi}^\vee(e_q)\otimes \tau$.
\end{lemma}

\begin{proof}
	\begin{align*}
		\delta\hat{\Psi}^{\vee}(e_q)\otimes \tau
		&= (\delta\hat{\Psi}^{\vee}(e_q))^{nf}\otimes\tau + (\delta\hat{\Psi}^{\vee}(e_q))^f\otimes\tau \\
		&= \hat{\Psi}^{\vee}(\delta e_q)\otimes \tau + \Phi^\vee\check{\Psi}^\vee_{q-r}(e_{q-r})\otimes\tau \\
		&= \hat{\Psi}^{\vee}(\delta e_{q})\otimes \tau + \sum_i D_i(\check{\Psi}^\vee_{q-r}(e_{q-r}))\otimes \tau\\
		&= \hat{\Psi}^{\vee}(\delta e_q)\otimes \tau + \sum_i \check{\Psi}^\vee_{q-r}(e_{q-r})\otimes d_i\tau \\
		&= \Psi(\delta(e_q\otimes \tau))\qedhere
	\end{align*}
\end{proof}

A \emph{pieced word} $w$ is a word in $EC_r$ together with a decomposition of $w$ into pieces of length $\leq r$. Define the following relation on them:
\begin{itemize}
	\item (Inner reorderings) $w\sim (-1)^{|\sigma|}w'$ if all entries of $w$ and $w'$ coincide, except for a single piece where they differ by a permutation $\sigma$.
	%\item (Full piece removal) $w\sim (-1)^rw'$ if $w'$ is obtained from $w$ by removing a piece of length $r$ (a \emph{full piece}).
	\item (Standarization) $w\sim w'$ if $w'$ is obtained from $w$ by removing a piece of length $0$.
\end{itemize}
The complex $\Omega_*(r)$ is generated in degree $q$ by the equivalence classes of words of length $q+1$ and at most one full piece.
%Up to inner reorderings, every class $[w]$ has a maximal representative $\bar{w}$.
%Define $\partial([w]) = [\partial(\bar{w})]$.
The differential $\partial(w) = \sum_{i} (-1)^id_iw$ is well-defined in classes. $\Omega_*(r)^{nf}$ is the subcomplex of words without full pieces and $\Omega_*(r)^{f}$ is the submodule of words with full pieces.

If $(U,w)$ is a generator of $\Omega_*(r,m)$, then $w$ has a canonical piece decomposition: a sequence $(w_i,\ldots,w_{i+k})$ is a piece if and only if $u_{i-1}<u_i =\ldots= u_{i+k}<u_{i+k+1}$. This defines a chain homomorphism
\[\lambda_m\colon \Omega_*(r,m)\to \Omega_*(r)\]
Let $\Phi\colon \Omega_*(r)^f\to \Sigma^{r}\Omega_{*}(r)^{nf}$ be the (non-chain) homomorphism that sends a pair $(U,w)$ to the result of removing a full piece if it is possible and increase all entries in $w$ to the right of the piece.

%If $w'$ is a subword of $w$ and $(U,w)$ is a pair, define the pair $(U|_{w'},w')$ in the obvious way.
%where $w^{*}=w$ if $w$ has no full piece and $0$ otherwise and $w^{**}$ is the opposite.

%A pieced word $w$ and a generator $U\in C^q(\Delta^m)$ are \emph{compatible} if they have the same length and if and only if the sequence $(w_i,\ldots,w_{i+k})$ is a piece. If $U$ and $w$ are compatible, write $(U,w)$ for the generator of $\hat{\Omega}^*(r,m)$ that is obtained by forgetting the pieces of $w$. Otherwise, define $(U,w) = 0$.

\begin{lemma} \label{lemma:omegar}
	Suppose one is given a $C_r$-equivariant map
	\begin{align*}
		\check{\Psi}\colon \Omega_*(r)^{nf}&\lra W_*(r)
	\end{align*}
	such that
	\begin{itemize}
		\item $\hat{\Psi}_{q-1}((\partial (w)^{nf}) = \hat{\Psi}_{q-r}\Phi(w)$ if $w$ has a full piece.
	\end{itemize}
	Then the family of maps $\hat{\Psi}_m = \check{\Psi}\lambda_m$ satisfies the conditions of Lemma \ref{lemma:omegarm}.
\end{lemma}

\begin{proof}
	Immediate.
\end{proof}

\subsection{Maps between spheres}\footnote{Actually we will need the map on unnormalised chains}

We will use the canonical identification $N_*(X*Y) \cong N_*(X)\otimes N_*(Y)$ and write $\sigma$ for the top dimensional class of $N_*(\Delta^{r-1})$. The cyclic group acts on $\Delta_{r-1}$ by permuting its vertices.
Let
\begin{align*}
	\iota_1\colon N_k(\Sigma^{r-1}\partial \Delta^{r-1})&\lra N_{k}(\partial\Delta^{r-1})\otimes N_{r-2}(\partial\Delta^{r-1})\\
	\iota_2\colon N_k(\Sigma^{r-1}\partial \Delta^{r-1})&\lra N_{r-2}(\partial\Delta^{r-1})\otimes N_{k}(\partial\Delta^{r-1})
\end{align*}
be the $C_r$-equivariant chain homomorphisms given by
\begin{align*}
	\iota_1(\tau) &= \tau\otimes \partial \sigma \\
	\iota_2(\tau) &= \partial \sigma\otimes \tau \\
\end{align*}
Let $f\colon N_*(\partial\Delta^{r-1}*\partial\Delta^{r-1})\to N_*(\mathbb{S}^{r-1}*\partial\Delta^{r-1})$ be a $C_r$-equivariant chain homomorphism such that
\renewcommand{\theenumi}{\roman{enumi}}
\begin{enumerate}
	\item\label{cond:1} $f\circ \iota_1 = \Id$
	\item\label{cond:2} $f\circ \iota_2 = \rho$.
	\item\label{cond:3} If $\tau_1\otimes\tau_2$ has degree $r-1$, then $f(\tau_1\otimes \tau_2)$ is a vertex if $\tau_1,\tau_2$ form a permutation and $0$ otherwise.
\end{enumerate}

\begin{remark}
	If we additionally impose the condition that $f$ sends simplices to simplices, then it is determined by its values on $(r-1)$-simplices.
\end{remark}

\begin{remark}
	Condition \eqref{cond:3} can be replaced by the more general condition
	\[
	\partial f(\tau_1\otimes\tau_2) =
	\begin{cases}
	\emptyset & \text{if $\tau_1$ and $\tau_2$ form a full permutation} \\
	0 & \text{otherwise}.
	\end{cases}
	\]
\end{remark}

\begin{remark}
	The former condition $f(\tau_1,\tau_2)\subset \tau_1$ is no longer necessary: It is necessary to define suspension maps for $\Omega_*(r,m)$, but with the current argument that is not needed.
\end{remark}

\begin{remark}
	In the examples constructed, $f(\tau_1,\tau_2)\subset \tau_1\smallsetminus \tau_2$.
\end{remark}

A word $w$ can be canonically broken into overlapping subwords of length at most $r$ called \emph{blocks}: Set the last block to be the last $r$ entries of the word, and recursively define the rest of the blocks by removing the last $r-1$ entries of the last block. Each pair of adjacent blocks share an element, that is called \emph{pivot}. If $w$ is a minimal pieced word, let $w_1$ be the piece that contains the last pivot and let $w_2$ be the complement of that piece in the last block.

\begin{example}
	The pieced word $w=01|24|013|12|4$ has as blocks $01-12401-13124$, the last pivot is $1$, and $w_1 = 013$ and $w_2 = 124$.
\end{example}

\begin{definition}
	Define a homomorphism $S\colon \Omega_*(r)^{nf}\to \Sigma^{r-1}\Omega_*(r)^{nf}$ by sending a pieced word $w$ to the result of replacing $w_1$ and $w_2$ by the piece $f(w_1,w_2)$. %If $w$ is a pieced word with a full piece, define $S(w)$ as the result of inserting back the full piece into $S(\Phi(w))$ (if the full piece was to the right of $w_1$, then it gets inserted at the end of the word).
\end{definition}

\begin{lemma}
	$S$ is a chain homomorphism.
\end{lemma}

\begin{proof}
	Let $\hat{w}$ be the result of removing $w_1$ and $w_2$ from $w$, and let $\ell$ be the dimension of $\hat{w}$. Given two words $w$ and $w'$, let $w*w'$ be its concatenation. Recall that $\partial w = \sum_{i} (-1)^{i}d_iw$, where $d_i$ removed the $i$-th entry of $w$. Then
	\begin{equation}\label{eq:931}
		\partial S(w) = \partial \hat{w}*f(w_1,w_2) + (-1)^{\ell+1}\hat{w}*\partial f(w_1,w_2).
	\end{equation}
	while, if the piece of the pivot contains at least two elements of the penultimate block,
	\[S(\partial w) = \partial \hat{w}*f(w_1,w_2) + (-1)^{\ell+1}\hat{w}*f(\partial(w_1),w_2) + (-1)^{\ell+1}\hat{w}*f(w_1,\partial w_2),\]
	which equals the previous sum. Suppose that the piece of the pivot is contained in the last block, in which case $f(w_1,w_2)$ is of dimension $0$. Then, letting $w_0$ be the piece to the left of $w_1$ and $\check{w}$ be the subword to the left of $w_0$, we have
	\begin{equation}\label{eq:933}
		S(\partial w) = \partial \check{w}*f(w_1,w_2) + \check{w}*f(w_0,\partial(w_1*w_2)).
	\end{equation}
	Now, the last summand of \eqref{eq:931} is zero or $\hat{w}$ depending on whether $f(w_1,w_2)$ vanishes or not. By condition \eqref{cond:3}, this is equivalent to $w_1,w_2$ not being or being a full permutation, which by condition \eqref{cond:1} is equivalent to $f(w_0,\partial(w_1*w_2))$ being $0$ or $w_0$, and therefore equivalent to the last summand of \eqref{eq:933} being zero or $\hat{w}$.
\end{proof}

There is an inclusion $C_*(EC_r)\to \Omega_*(r)^{nf}$ that sends a word to that same word with all pieces of length $1$. There is a map in the opposite direction with $\bZ[\frac{1}{(r-1)!}]$-coefficients that sends a pieced word to the sum of the underlying words of all its representatives, divided by the number of representatives. The composition (here $\Psi$ is the map from the previous section)
\begin{equation} \label{eq:234}
	\Omega_*(r)^{nf}\lra C_*(EC_r)\overset{\Psi^\vee}{\lra} W_*(r)
\end{equation}
is a chain homomorphism with $\bZ[\frac{1}{(r-1)!}]$-coefficients, but in degrees $*\leq r-2$ it lifts to a homomorphism with $\bZ$-coefficients, that we denote by $\check{\Psi}_*\colon \Omega_*(r)^{nf}\to W_*(r)$.

\begin{definition}
	Define a homomorphism $\check{\Psi}\colon \Omega_*(r)^{nf}\to W_*(r)$ recursively as
	\[\check{\Psi}_q(w) = \begin{cases} \check{\Psi}_q(w) & \text{if $*\leq r-2$} \\
		\theta_{1-r}\check{\Psi}_{q-r+1}S(w) & \text{if $*\geq r-2$.}\end{cases}\]
\end{definition}

\begin{lemma}
	$\check{\Psi}$ is a well-defined homomorphism. %$\check{\Psi}\partial = \partial\check{\Psi}$
\end{lemma}

\begin{proof}
	In degrees $*\leq r-2$ it is a homomorphism because it is a composition of the homomorphisms \eqref{eq:234}. The verification in the higher degrees follows from the previous lemma and induction:
	\begin{align*}
		\Psi_{q-1}\partial(w) &= \theta_{r-1}\Psi_{q-r}S\partial(w) = \theta_{r-1}\Psi_{q-r}\partial S(w) = \\
		&=\theta_{r-1}\partial\Psi_{q-r+1}S(w) = \partial\theta_{r-1}\Psi_{q-r+1}S(w) = \partial \Psi_{q}(w).
	\end{align*}
	Finally, one deduces from \eqref{cond:3} that $\theta_{r-1}\check{\Psi}_{-1}S(w) = \check{\Psi}_{r-2}(w)$, hence both definitions coincide at their common case.
\end{proof}

%\begin{lemma} $\hat{\Psi}_{q-1}((\partial w)^{nf}) = \partial\hat{\Psi}_{q-r}\Phi(w)$ if $w$ has a full piece.
%\end{lemma}
%\begin{proof}
%\begin{itemize}
%\item if the full piece is to the left of $w_1$ \fcnote{doublecheck this, if it is exactly to the left an additional argument may be needed}, then $S\partial(w) = \partial S(w)$.
%\item if the full piece is to the right of $w_1$, then let $w_3$ be that full piece, and assume, without loss of generality that is the last piece. Then
%\[\partial S(w) = \partial \hat{w}*f(w_1,w_2)*w_3 + \hat{w}*\partial f(w_1,w_2)*w_3 + \hat{w}*f(w_1,w_2)*\partial w_3\]
%\[S\partial(w) = \partial \hat{w}*f(w_1,w_2)*w_3 + \hat{w}*\partial f(w_1,w_2)*w_3 + \hat{w}*w_1*f(w_2,\partial w_3)\]
%We need to check that the value of $\check{\Psi}$ in the last term coincides. If $q-r+1\geq r-2$, then we have that
%\[\check{\Psi}(\hat{w}*f(w_1,w_2)*\partial w_3) = \check{\Psi}S(\hat{w}*f(w_1,w_2)*\partial w_3) = \check{\Psi}(\hat{w}*f(f(w_1,w_2),\partial w_3)\]
%\[\check{\Psi}(\hat{w}*w_1*f(w_2,\partial w_3)) = \check{\Psi}S(\hat{w}*w_1*f(w_2,\partial w_3)) = \check{\Psi}(\hat{w}*f(w_1*f(w_2,\partial w_3))\]
%and both terms equal $f(w_1,w_2)$.
%\end{itemize}
%
%\end{proof}

\begin{lemma}
	$\hat{\Psi}_{q-1}((\partial w)^{nf}) = \hat{\Psi}_{q-r}\Phi(w)$ if $w$ has a full piece.
\end{lemma}

\begin{proof}
	We will prove it by induction on the position from the right of the left piece. Since $w$ has at least one full piece, $q\geq r-1$.

	If the full piece is the last piece, write $w=\hat{w}*w_1*w_2$ with $w_2$ the full piece. Then by Condition \eqref{cond:1}:
	%\[\partial S(w) = \partial \hat{w}*f(w_1,w_2)*w_3 + \hat{w}*\partial f(w_1,w_2)*w_3 + \hat{w}*f(w_1,w_2)*\partial w_3\]
	\begin{align*}
		\Psi_{q-r}(S(\partial(w))^{nf}) &= \Psi_{q-r}(S(\hat{w}*w_1*\partial w_2)) \\
		&= \Psi_{q-r}(\hat{w}*f(w_1,\partial w_2)) \\
		&= \Psi_{q-r}(\hat{w}*w_1) \\
		&= \Psi_{q-r}\Phi(w).
	\end{align*}
	If the full piece is not the last piece but contains the pivot, write $w=\hat{w}*w_1*w_2$ with $w_2$ the full piece. Then by Condition \eqref{cond:2}:
	\begin{align*}
		\Psi_{q-r}(S(\partial(w))^{nf}) &= \Psi_{q-r}(S(\hat{w}*\partial w_1* w_2)) \\
		&= \Psi_{q-r}(\hat{w}*f(\partial w_1, w_2)) \\
		&= \Psi_{q-r}(\hat{w}*\rho(w_2)) \\
		&= \Psi_{q-r}\Phi(w).
	\end{align*}
	If the full piece $\bar{w}$ is left to the last pivot, write $w=\hat{w}_1*\bar{w}*\hat{w}_2*w_1*w_2$ with $w_1$ the piece that contains the last pivot and $w_2$ the word to the right of $w_1$. Assume by induction that the lemma holds when $\hat{w}*w_1*w_2$ is of smaller length. Then
	\begin{align*}
		\Psi_{q-r}(S(\partial(w))^{nf}) &= \Psi_{q-r}(S(\hat{w}_1*\bar{w}*\hat{w}_2w_1w_2)) \\
		&= \Psi_{q-r}(\hat{w}_1*\bar{w}*\hat{w}_2*f(w_1,w_2)) \\
		&\overset{*}{=} \Psi_{q-r}(\hat{w}_1*\rho(\hat{w}_2*f(w_1,w_2)))\\
		&= \Psi_{q-r}(\hat{w}_1*\rho(\hat{w}_2*w_1*w_2))\\
		&= \Psi_{q-r}(\Phi(w)). \qedhere
	\end{align*}
\end{proof}

\begin{corollary}
	The map $\check{\Psi}$ satisfies the conditions of Lemma \ref{lemma:omegar}, hence together with Lemma \ref{lemma:omegarm} it defines a chain homomorphism $\Psi\colon V^*(r)\to \bar{\Omega}^*(r,n)$.
\end{corollary}

\subsection{Maps $f$}

If $r=3$, then there is a unique map $f\colon N_*(\partial \Delta^2)\otimes N_*(\partial\Delta^2)\to \Sigma^2N_*(\partial\Delta^2)$ satisfying all conditions, namely $f([0]\otimes [1,2]) = [0], f([0,1]\otimes [2])=0$, which extends $C_r$-equivariantly as follows (we only indicate the non-zero terms):
\begin{align*}
	f([0,1]*[2,0]) &= [0,1]
\end{align*}

If $r=5$, then there is no unique map $f\colon N_*(\partial \Delta^4)\otimes N_*(\partial \Delta^4)\to \Sigma^4N_*(\partial \Delta^4)$, but here is one of them:
\begin{align*}
	f([0]\otimes[1,2,3,4]) &= [0] &
	f([0,1]\otimes [2,3,4]) &= [0] &
	f([0,2]\otimes [1,3,4]) &= [0] \\
	f([0,1,2]\otimes [3,4]) &= [0] &
	f([0,1,3]\otimes [2,4]) &= [0] &
	f([0,1,2,3]\otimes [4]) &= [0],
\end{align*}
which extends $C_r$-equivariantly as follows (we only indicate the non-zero terms):
\begin{align*}
	f([0,1]\otimes [2,3,4,0]) &= [0,1] &
	f([0,2]\otimes [1,3,4,0]) &= [0,2] \\
	f([0,1,2]\otimes [3,4,0]) &= [0,1,2] &
	f([0,1,3]\otimes [2,4,0]) &= [0,1] \\
	f([0,1,3]\otimes [2,4,1]) &= [0,3] &
	f([0,1,2,3]\otimes [4,0]) &= [0,1] \\
	f([0,1,2,3]\otimes [4,1]) &= [2,0] &
	f([0,1,3]\otimes [2,4,0,1]) &= [0,1,3] \\
	f([0,1,2,3]\otimes [4,0,1]) &= [0,1,2] &
	f([0,1,2,3]\otimes [4,1,2]) &= [0,2,3] \\
	f([0,1,2,3]\otimes [4,0,1,2]) &= [0,1,2,3] &&
\end{align*}

There is also an explicit map for $r=7$.
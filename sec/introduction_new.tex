% !TEX root = ../oddp.tex

\section{Introduction}\label{s:introduction}

Let $R$ be a commutative ring and $r$ a natural number. A \emph{linearized $r$-cyclic monoidal category with diagonals} is an $r$-cyclic monoidal category $\cC$ with a natural diagonal $\Delta \colon X \to X^{\ot r}$ for each object $X$, and an $r$-cyclic strong monoidal functor $\abel \colon \cC \to \Mod{R}$ to the category of $R$-modules. The category sets and the category of pointed sets are the prototypical examples. A less typical example is the category of signed sets, described in \cref{s:categories}.
If $X$ is an augmented simplicial object in $\cC$, the normalized chain complex of the simplicial $R$-module $\abel\circ X$ is denoted $\chains(X_\bullet;R)$.
Our main result is an effective chain-level construction of operations
\[
	\power^i\colon H^*(X_\bullet;\bF_p)\to H^{*+i}(X_\bullet;\bF_p)
\]
parametrized by the dual $\Wdualaug{r}$ of the augmented minimal resolution of the cyclic group of order $r$
from the following combinatorial structure.

\begin{definition}
	An \emph{$r$-cyclic straightening with duality} is a $\Cyc_r$-equivariant choice of element $x_\tau\in \tau$ for every proper non-empty subset $\tau \subset \{0,1,\dots,r-1\}$ such that the predecessor of $x_\tau$ is not in $\tau$.
\end{definition}

We mention that this structure can exist only if $r$ is prime.

\begin{theorem}\label{thm:main}
	If $r$ is an odd prime, each $r$-cyclic straightening with duality yields an effective construction of chain-level power operations parametrized by $\Wdualaug{r}$ on the cohomology of an augmented simplicial object in $\cC$ .
\end{theorem}


If $r=2$ the theorem still holds, but only if chains are taken with coefficients in $\bF_2$.
In that case there is a unique $r$-cyclic straightening with duality, and therefore our theorem yields a unique family of chain-level operations. 

In Section \cref{Kan} we will explain that our operations fit well with Kan spectra.

\subsection{Other chain-level operations} Chain-level operations for power operations on the cohomology of a (non-augmented) simplicial set were described by \cite{mcclure2003multivariable} and \cite{berger2004combinatorial}. The latter where parametrized by the Barratt-Eccles operad, while the former were parametrized by the surjection operad. In \cite{medina2021may_st} a map from the minimal resolution $W(r)$ of the cyclic group of order $r$ to the Barratt-Eccles operad was constructed, yielding chain-level operations parametrized by the minimal resolution $W(r)$ of the cyclic group of order $r$. 

There is an important difference between these constructions and ours: theirs yield a well-defined product in cohomology, while ours do not in general. On the other hand, our construction satisfies that $\power^0([x]) = [x]$ almost by definition, while that property in their construction is not straightforward\footnote{FC: Does it always hold for an arbitrary symmetric monoidal category $\cC$ with diagonals?}. As a consequence of these different viewpoints, our formulas for $\power^i$ are simpler when $i$ is smaller, while their formulas as simpler when $i$ is large.

In Section \cref{s:diagonals} we will show that Medina's operations parametrized by $W(r)$ and our operations parametrized by $\Wdualaug{r}$ are both special cases of a more general construction that we call \emph{stable $r$-comultiplication}. From this unifying perspective, our operations when $r=2$ and $R=\bF_2$ coincide with the chain-level operations axiomatized in \cite{medina2022axiomatic}. If $r = 3$ there is a unique $r$-cyclic straightening with duality, and we have empirically tested that our operations coincide with those of \cite{medina2021may_st}. When $r\geq 5$, this is no longer true: as we will see in \cref{s:suspension}, our constructions behave differently from Medina's with respect to suspension.




%We will see in Section \ref{s:suspension} that the connected diagonals of our main theorem are compatible with suspension.
%As a consequence, they yield also connected diagonals for the chain complex of a suspension spectra.

%As we have seen, unstable diagonals and connected diagonals on the chain complex of a pointed simplicial set that are compatible with suspension yield stable diagonals in the chain complex of a suspension spectrum. This is no longer true for other models of spectra.
%
%One of these models of spectra are the cubes in the Burnside $2$-category of Lawson, Lipshitz and Sarkar \cite{LLS20} which model Khovanov spectra \cite{LS14}. A connected $2$-cyclic diagonal for these objects was found by the first author in \cite{cantero-moran2020khovanov}, thus giving rise to explicit formulas to compute Steenrod squares in Khovanov homology. In a future work we expect to use the formulas of Theorem \ref{thm:main} to find connected $r$-cyclic diagonals for $r$ an odd prime, thus yielding explicit formulas for the odd power operations in Khovanov homology.
%
%For the reader acquainted with the surjection operad we remark that the operations that are involved in these connected $r$-cyclic diagonals would correspond to summands indexed by non-overlapping intervals.
%
%This new viewpoint has some advantages:
%First, the construction is very explicit since it is manageable to find the coefficient of any given summand. Second, our formulas hold for the chain complexes of augmented simplicial objects too ---these do not admit an unstable diagonal in general.
%Third, computing $\power^i([x])$ for small $i$'s tends to be simpler for a connected diagonal than for an unstable diagonal, while computing them for $i$'s close to $(\bars{x}-1)r$ is simpler for an unstable diagonal than for a connected diagonal.
%
%
%

\subsection{An account of the construction} The construction of these operations is as follows (see \cref{s:preliminaries} for the degree conventions on the chain complexes): 
\martin{I think this has too many steps and that makes it difficult to point out the main ones. Maybe restructuring this with less steps (3 or so) that are more general and writing substeps is better.}
\subsubsection*{Step 1} Let $\Wdualaug{r}$ be the dual of the augmented minimal resolution of the cyclic group of order $r$:
	\[
		R \lra R[\Cyc_r]\langle e_1^{\vee}\rangle \lra R[\Cyc_r]\langle e_2^{\vee}\rangle 
		\lra R[\Cyc_r]\langle e_3^{\vee}\rangle \lra \ldots
	\]
There is a chain endomorphism $\theta^\dd\colon \Wdualaug{r}\to \Wdualaug{r}$ of degree $r-1$ that is an isomorphism in degrees $\geq r$.
Let $\asimplex^n$ be the augmented simplex of dimension $n$, and let $\rchains(\asimplex^{\bullet})$ be the cosimplicial chain complex whose $n$-simplices of degree $q$ form the free $R$-module generated by the $m$-simplices of $\asimplex^n$ if $q=rm$ (and the trivial $R$-module otherwise).
The usual formula $\partial(\tau) = \sum_{j}(-1)^j \d_j\tau$ endows it with a differential of degree $-r$. We then can construct the tensor product of these two chain complexes, with a differential twisted by the chain map $\theta$: \martin{I think this paragraph is really confusing, specially when defining $N^r$. Note that $q$ is immediately used for something different.} %Idea: first define the N^r notation, and the differential. Then \theta and the "twisted" differential.
	\[
		\partial(e_q^\dd\otimes \tau) = \partial(e_q^\dd)\otimes \tau + (-1)^q \sum_j (-1)^j\theta^\dd(e_q^\dd)\otimes \d_j(\tau)
	\]
	obtaining a cosimplicial chain complex with a differential of degree $-1$ and face maps of degree $-r$ that we denote by
	\[
			\Wdualaug{r}\otimes_{\theta} \rchains(\asimplex^{\bullet})
	\]
	We will show that any cosimplicial equivariant chain map
		\[
	\Wdualaug{r}\otimes_{\theta} \rchains(\asimplex^{\bullet}) \lra \chains(\asimplex^{\bullet})^{\otimes r}
	\]
	yields chain-level operations for power operations. This perspective is related to May's \cite{may1970general}, as will be explained in \cref{s:2bdiagonals}. Such map will be called ``connected $r$-comultiplication''.
	
\subsubsection*{Step 2} We then use that there is a Poincaré duality isomorphism between the cosimplicial chain complex $\chains(\asimplex^{\bullet})$ and its linear dual $\cochains(\asimplex^{\bullet})$, which therefore is also a cosimplicial chain complex (concentrated in non-positive degrees). Similarly, there is an isomorphism between $\rchains(\asimplex^\bullet)$ and its linear dual $\chains^{r,\vee}(\asimplex^\bullet)$.
\subsubsection*{Step 3} We then observe that there is an isomorphism between $\cochains(\asimplex^{\bullet})^{\otimes r}$ and $\cochains(\asimplex^{r-1})^{\otimes (\bullet+1)}$, thus we need to build a cosimplicial map 
	\[
		\Wdualaug{r}\otimes_{\theta} \chains^{r,\vee}(\asimplex^{\bullet})\lra \cochains(\asimplex^{r-1})^{\otimes (\bullet+1)}.
	\]
	Taking linear duals, we need to build a simplicial map
	\begin{equation}\label{eq:330}
		\chains(\asimplex^{r-1})^{\otimes (\bullet+1)} \lra \Waug{r}\otimes_{\theta} \chains^r(\asimplex^\bullet).
	\end{equation}
\subsubsection*{Step 4} Now, for each $n$, we can isolate the top simplices of $\chains(\asimplex^{r-1})^{\otimes n+1}$, obtaining an additive isomorphism
	\begin{equation}\label{eq:331}	
		\chains(\asimplex^{r-1})^{\otimes (n+1)} \cong \bigoplus_{p+k=n+1} \chains(\partial \asimplex^{r-1})^{\otimes p}\otimes \rchains(\asimplex^n)_{rk}.
	\end{equation}
		There is a map $\theta_{j}\colon \chains(\partial \asimplex^{r-1})^{\otimes p}\to \chains(\partial \asimplex^{r-1})^{\otimes p+1}$ of degree $r-1$ that inserts the oriented generator of the top homology group of $\chains(\partial \asimplex^{r-1})$ in a certain position. Under the isomorphism \eqref{eq:331}, the differential on the left hand-side takes the form
		\[
			\partial(a\otimes \tau) = \partial(a)\otimes \tau + (-1)^{|a|} \sum_{j}(-1)^j\theta_{j}(a)\otimes \d_j (\tau).
		\]
\subsubsection*{Step 5} On the other hand, the chain complex $\chains(\partial\simplex^{r-1})$ can be glued to itself to obtain a $(r-1)$-periodic augmented resolution $\Per{r}$ of the cyclic group of order $r$. %The periodic isomorphism endows $\Lambda(r)_+$ with an augmented  simplicial structure, and 
	We will show that every $r$-cyclic straightening with duality gives rise to a map
	\begin{equation}\label{eq:499}
		\chains(\partial \asimplex^{r-1})\otimes \chains(\partial \asimplex^{r-1}) \lra \Per{r}
	\end{equation}
that is later used to obtain a chain map
	\[
		\phi\colon \chains(\partial \asimplex^{r-1})^{\otimes n} \lra \Per{r}.
	\]
Although we do not use it here, we remark the following: The complex $\Per{r}$ with the degree conventions of the article corresponds to the negative part of the Tate resolution of the cyclic group on $r$ elements associated to the periodic resolution. From this perspective, the map \eqref{eq:499} yields a chain level lift of the cup product on Tate cohomology of the cyclic group, in non-positive degrees.
\subsubsection*{Step 6} We then construct a map $\chains(\partial \asimplex^{r-1})\to \Waug{r}$. Using the $(r-1)$-periodicity of $\Per{r}$ and $\Waug{r}$, we then obtain a chain map
	\[
		\psi\colon \Per{r}\to \Waug{r}
	\]
\subsubsection*{Step 7} Finally, using the isomorphism \eqref{eq:331}, we define a map as in \eqref{eq:330} by the formula $(\phi\circ \psi)\otimes \id$. We also check that it intertwines the maps $\theta_j$ and $\theta$.

\subsection{Proof of the theorem} The theorem is obtained from \cref{def:connected_diagonal}, \cref{prop:cosimplicial}, \cref{prop:milnor_to_minimal}, \cref{prop:milnor_to_per} (where one uses \cref{prop:construction_xi} and \cref{{prop:straightenings}}) and \cref{prop:per_to_minimal}. 

\subsection{Outline}

In \cref{s:preliminaries} we set some sign and degree conventions for chain complexes and the conventions for augemented simplicial objects.
In \cref{s:2bdiagonals} we develop Step 1, introducing the notions of unstable $r$-comultiplication, stable $r$-comultiplication and connected $r$-comultiplication. 
In \cref{s:3complexes} we carry on Steps 2, 3 and 4.
In \cref{s:milnor_to_per} we continue with Step 5, and in \cref{s:per_to_minimal} we finish with Step 6.

In \cref{s:suspension} we show that these operations behave well with respect to suspensions.
In \cref{s:9Kanspectra} we briefly explain how our method yields natural connected diagonals in the chain complex of a Kan spectrum.
Finally, in \cref{s:formulas} we give an algorithmic presentation of the formulas.
These formulas are meant to be understandable after reading Sections \ref{s:preliminaries} and \ref{s:2bdiagonals}.
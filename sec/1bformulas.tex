% !TEX root = ../oddp.tex

\section{Explicit formulae}\label{s:formulas} 

Let $r$ be an odd prime and let $\tau\mapsto g(\tau)$ be the $r$-cyclic asymmetry with duality. Let $\chains(X)$ denote the right suspension of the normalised chain complex $\cadenas(X)$ of $X$. If $x$ is a cochain of dimension $m$ in the cochain complex of an augmented simplicial set $X$, then $\power^i([x]) = [y]$ with
\[	
	y(\tau) = \frac{1}{(\tilde{r}!)^{m}}x^{\otimes r}(\mu(e_{ri}\hotimes \tau)).
\]
The element $\mu(e_{q}\hotimes \sigma)\in \chains(X)^{\otimes r}$ is computed as follows: Let $\cP(r,q)$ be the collection of pairs $(U,A)$ such that 
\renewcommand{\theenumi}{\roman{enumi}}%
\begin{enumerate}
	\item $U = (u_0,\ldots,u_{q-1})$ is a non-decreasing sequence with $u_j\in \{0,1,\ldots,n\}$,
	\item $A = (a_0,\ldots,a_{q-1})$ is a sequence with $a_i\in \{0,1,\ldots,r-1\}$, and
	\item\label{condfor:3} $a_i<a_j$ if $u_i=u_j$.
\end{enumerate} 
Each pair $(U,A)\in \cP(r,q)$ and each chain $\tau\in \chains(X)$ of dimension $n$ (and degree $n+1$) defines the following generator of $\chains(X)^{\otimes r}$:
\[
	d_{U_A^{r-1}}(\tau)\otimes d_{U_A^{r-2}}(\tau)\otimes \ldots\otimes d_{U_A^{0}}(\tau)\in \chains(X)^{\otimes r}.
\] 
where $U_A^k = \{u_i\mid u_i + a_i \equiv k\mod r\}$ for each $j\in 0,1,\ldots,r-1$. We will now define recursively the coefficient $\nu_{U,A}$ of this generator. Let $A'$ be the sequence $\{a_i+u_i\mod r\mid i=0,\ldots,q-1\}$ and let $q_k$ be the amount of elements of $A'$ that are equal to $k$, so $q=\sum q_k$, and define
\renewcommand{\theenumi}{\arabic{enumi}}%
\begin{enumerate}
    \item $\signo_0 = nq$,
	\item $ \signo_1 = \sum_k\binom{q_k}{2}$,
	\item $ \signo_2$ is the sum of all entries of $U$
	\item $ \signo_3$ is $(n+1)$ times the sum of all the entries of $A'$,
	\item $ \signo_4$ is the parity of the permutation that orders $A'$.
\end{enumerate}
Define $\signo =\signo_0+\signo_1+\signo_2+\signo_3+\signo_4$. If there is a subsequence $u_j = u_{j+1} = \ldots = u_{j+r-1}$ of length at least $r$, then $\nu_{U,A}=0$, otherwise:
\begin{itemize}
	\item If $q<r$, let $A' = (a'_0,\ldots,a'_{q-1})$ be the result of ordering $A$ with $\signot$ the parity of that permutation. 
	\begin{itemize}
		\item[$\bullet$] If $a_{2j}'$ is even and different from $r-1$ and $a_{2j+1}'$ is odd, then define $\nu_{U,A} = (-1)^{\signot}\varphi(q)!(\varphi(0)!)^n$, where $\varphi(q) = \lfloor \frac{r-1-q}{2}\rfloor$. 
		\item[$\bullet$] Otherwise, $\nu_{U,A}=0$.
	\end{itemize}
\item  If $q\geq r$, let $\ell$ be the biggest number smaller than $r$ such that $u_{q-\ell-1}<u_{q-\ell}$.  Let $k$ be such that $u_{q-k-1}<u_{q-k} = u_{q-\ell-1}$. Define
\begin{align*}
w &= (a_{q-\ell},a_{q-\ell+1},\ldots,a_{q-1})
\\
z &= (a_{q-k},\ldots,a_{q-\ell-1})
\end{align*}
\begin{itemize}
	\item[$\bullet$] If $\{0,1,\ldots,r-1\}\not\subset w\cup z$, then $\nu_{U,A} = 0$.
	\item[$\bullet$] Otherwise, define 
\begin{itemize}
	\item $x = (x_0,\ldots,x_{r-\ell-1})$, the ordered complement of $w$ in the set $\{0, \ldots ,r-1\}$,
	\item	$y = (y_0,\ldots,y_{k-r-1})$, the ordered complement of $x$ in $z$,
	\item $\signot_0$ the parity of the permutation that orders $w\cup x$
	\item $\signot_1$ the parity of the permutation that reorders $x\cup y$ to $z$.
	\item $\signot_2=(|x|-1)|y|$.
\end{itemize}
and for each permutation $\pi$ of the entries of $y$, let $\signot_3$ be its parity and define the sequence 
\[
\quad\quad \omega_\pi = (g(x),g(x\cup y_{\pi(0)}), g(x \cup y_{\pi(1)}\cup y_{\pi(1)}), \ldots, g(x\cup y)).
\]
\begin{itemize}
	\item[$\bullet$] If $\omega_\pi$ has repetitions, then set $\nu_{U,A,\pi} = 0$.
	\item[$\bullet$] Otherwise, let $(\omega'_0,\ldots,\omega'_{k-r})$ be the result of ordering the sequence and let $\signot_4$ be the parity of that reordering. Define a new pair $(U_{\pi},A_{\pi})$ as
	\begin{itemize}
		\item[] $\quad U_{\pi} = (u_0,\ldots,u_{q-r})$
		\item[] $\quad A_{\pi} = (a_0,\ldots,a_{q-k-1},\omega'_0,\ldots,\omega'_{k-r})$.
	\end{itemize}
Define $\signot_\pi = \signot_0+\signot_1+\signot_2+\signot_3+\signot_4$.
\end{itemize}
Finally, define
\[
	\nu_{U,A} = \sum_{\pi}(-1)^{\signot_\pi}\nu_{U_\pi,A_\pi}.
\]
\end{itemize}
\end{itemize}
The coefficient of $(U,A)$ as $(-1)^{\signo}\nu_{U,A}$.

\begin{example}\label{example:formulas1_3}
	If $r=3$, $X= \asimplex^7$, $U = (0,0,1,3,4,6), A = (0,1,2,0,2,1)$ and $\tau = [0,1,2,3,4,5,6,7]$ , we have:
	\begin{align*}
		\signo_0 &\equiv 0
		&
		\signo_1 &\equiv 1
		&
		\signo_2 &\equiv 0
		&
		\signo_3 &\equiv 0
		&
		\signo_4 &\equiv 1
		&
		\signo &\equiv 0
	\end{align*}
	To compute $\nu_{U,A}$, in the first iteration we obtain that $x=(0)$, $y=\emptyset$, thus there is only one permutation $\pi$ to be considered, for which
	\begin{align*}
		\signot_0 &\equiv 1
		&
		\signot_1 &\equiv 0
		&
		\signot_2 &\equiv 0
		&
		\signot_3 &\equiv 0
		&
		\signot_4 &\equiv 0
		&
		\signot_{\pi} &\equiv 1
	\end{align*}
	The new pair is $U_{\pi} = (0,0,1,3), A_\pi = (0,1,2,0)$ and $\nu_{U,A} = -\nu_{U_\pi,A_\pi}$. Renaming $U'= U_\pi$ and $A'=A_\pi$, in the second iteration we obtain that $x=(1)$, $y=(0)$, thus there is only one permutation $\pi$ to be considered, for which
	\begin{align*}
		\signot_0 &\equiv 0
		&
		\signot_1 &\equiv 1
		&
		\signot_2 &\equiv 0
		&
		\signot_3 &\equiv 1
		&
		\signot_4 &\equiv 0
		&
		\signot_{\pi} &\equiv 0
	\end{align*}
	The new pair is $U'_{\pi} = (0,0), A'_\pi = (0,1)$, and $\nu_{U',A'} = \nu_{U'_\pi,A'_\pi}$. Renaming $U'' = U'_\pi$ and $A'' = A'_\pi$, we compute that $\nu_{U'',A''} = 1$. Therefore the coefficient of the summand given by $(U,A)$ is
	\[
		(-1)^{\signo} \nu_{U,A} = -\nu_{U',A'} = -\nu_{U'',A''} = -1
	\]
\end{example}

\begin{example}
	If $r=5$, $X= \asimplex^2$, $U = (0,1,1,1,2,2,2), A = (1,2,3,0,1,3,4)$ and $\tau = [0,1,2]$ , we have:
	\begin{align*}
	\signo_0 = 1	
  \signo_0 &\equiv 0
		&
		\signo_1 %&\equiv 0
		&
		\signo_2 \equiv 1
		&
		\signo_3 &\equiv 1
		&
		\signo_4 &\equiv 1
		&
		\signo &\equiv 1
	\end{align*}
	To compute $\nu_{U,A}$, in the first iteration we obtain that $x=(0,2)$, $y=(3)$, thus there is only one permutation $\pi$ to be considered, for which
	\begin{align*}
		\signot_0 &\equiv 1
		&
		\signot_1 &\equiv 0
		&
		\signot_2 &\equiv 1
		&
		\signot_3 &\equiv 0
		&
		\signot_4 &\equiv 0
		&
		\signot_{\pi} &\equiv 0
	\end{align*}
	The new pair is $U_{\pi} = (0,1,1), A_\pi = (1,0,2)$ and $\nu_{U,A} = \nu_{U_\pi,A_\pi}$. Renaming $U' = U_\pi$ and $A' = A_\pi$, we have $\nu_{U',A'} = -2^2$. Therefore the coefficient of the summand given by $(U,A)$ is
	\[
		(-1)^{\signo} \nu_{U,A} = -\nu_{U',A'} = 2^2.
	\]
\end{example}

\subsection{The case \texorpdfstring{$r=3$}{r = 3}} If $r=3$, there is a simpler description of the operations. First, we replace Condition \eqref{condfor:3} for the following:
\begin{itemize}
	\item[(iii')] $a_i\neq a_{i+1}$ for all $i=0,\ldots,q-1$.
\end{itemize}
Define the \emph{regular blocks} of $A$ as the subsequences $(a_{q-2k-1},a_{q-2k},a_{q-2k+1})$ of length $3$. Define the \emph{exceptional block} of $A$ as $a_0$ if $q$ is odd and $(a_0,a_1)$ if $q$ is even. A block is \emph{ascending} (resp. descending) if it has no repeated entries and it is cyclically ordered (resp. not cyclically ordered). The coefficient $\nu_{U,A}$ is non-zero if and only if
\begin{enumerate}
	\item no block has repeated entries,
	\item if a block $(a_{q-2k-1},a_{q-2k},a_{q-2k+1})$ is descending, then $u_{q-2k-1}<a_{q-2k}<a_{q-2k+1}$,
	\item if a block $(a_{q-2k-1},a_{q-2k},a_{q-2k+1})$ is ascending, then $u_{q-2k-1}\neq a_{q-2k}$ or $a_{q-2k}\neq a_{q-2k+1}$,
	\item The exceptional block is either of the following: $(0),(0,1),(1,0)$.
\end{enumerate} 
If that is the case, define $\signo$ as in the previous section and let $\signot$ be the number of descending blocks. Then the coefficient of $(U,A)$ is $\nu_{U,A} = (-1)^{\signo+\signot}$.

\begin{example}
	If $r=3$, $X= \asimplex^7$, $U = (0,0,1,3,4,6), A = (0,1,2,0,2,1)$ and $\tau = [0,1,2,3,4,5,6,7]$ as in Example \ref{example:formulas1_3}, we have the exceptional block $(0,1)$ and the regular blocks $(1,2,0)$ and $(0,2,1)$. Only the last one is descending, thus $\signot = 1$ and $\nu_{U,A} = (-1)^{\signo+\signot} = -1$.
\end{example}

\subsection{Explanation of the formulas for any prime number \texorpdfstring{$r$}{r}} Here we justify the formulas of this section using the main theorem of the paper. A complete read of the paper is necessary to follow this section. Let $\tau\in \chains(X)$ be a chain of dimension $n$ (hence degree $n+1$). Let $(U,A)$ be a generator of $\chains(\asimplex^n\times \EC_r)$, and let $(U',A') = \beta(U,A)$, let $q_k$ be the number of elements of $A'$ that equal $k$ and let $q = \sum_{k}q_k$ the degree of $(U,A)$.

The sign of the map $\alpha$ is the parity of 
\[
	\sum_{k}\lambda((U_A^{k})^c,U_A^k) + (n+1)\sum_k q_k + (n+1)\sum_kk(r-1-q_k),
\]
where the first term is the sign of $\Lambda$, the second is the sign of the functor tensor product and the third is the sign of the reordering of the tensor factors. Notice that we have to reorder $(U^{r-1}_A\otimes \ldots U^0_A)\otimes \tau\otimes\ldots\otimes \tau$, hence the sign is $\sum_k(r-1-q_k)$ instead of $\sum_k q_k$. By Remark \ref{remark:alex}, we can write these signs as:
\begin{align}
\label{signo:lambda}	\lambda((U_A^k)^c,U^k_A) &\equiv \sum_i u_i + \sum_k(n+1-q_k)q_k + \binom{q_k}{2}
\\
\label{signo:prod} (n+1)q
\\
 \label{signo:reord} \sum_kk(r-1-q_k)&\equiv (n+1)\sum_{i} a'_i
\end{align}
The sign of the map $\beta$ is the sign that orders $A'  = \{a_i+u_i\mid i=0,\ldots,q-1\}$. The map $\gamma$ does not have sign and the map $\lambda$ has the sign $(n+1)q$. After the natural cancellations, we obtain $\signo$.

For the map $f$, we have first the sign of turning the right suspension into left suspension $(-1)^{(|w|+|z|)r}$, which can be written as $(-1)^{|y|+1}$. Then, the sign of $\twist$ is the parity of $(r-|x|)(|x|+|y|) \equiv \signot_2$, the sign of $\Lambda$ is $\signot_0$ and the sign of $s_*^{\Psd}$ is the parity $\signot_0$ of the permutation that arranges $x\cup y$ to $z$ plus $|y|+1$. Altogether yield $\signot_0+\signot_1+\signot_2$. For the map $\Phi$ we follow Remark \ref{remark:phidual}.
\subsection{Explanation of the formulas for \texorpdfstring{$r=3$}{r = 3}} A pieced word $A = (A_0\barra\ldots\barra A_j)$ admits many underlying words $(a_0,\ldots,a_{q-1})$, because the entries of each piece may be permuted. The formulas for $r=3$ follow from the following observation: each pieced word $A$ has a preferred underlying word $A = (a_0,\ldots,a_{q-1})$ obtained by ordering each piece of size two so that the second element is the cyclic succesor of the first one. Moreover, with this choice we have that the preferred underlying word of $S(A_0\barra\ldots\barra A_{j})$ is always the result of removing the last two elements from $A$, with positive sign unless the last block is descending (this implies that the last two pieces must be singletons). 

To check this last statement it suffices to check that
\begin{align*}
	f([0]\otimes [1,2]) &= [0]
	&
	f([0,1]\otimes [2,0]) &= [0,1]
	\\
		f([0,1]\otimes [2]) &= [0]
	&
	f([0,1]\otimes [2,1]) &= 0
\end{align*}
since all the other cases are permutations of these three. 







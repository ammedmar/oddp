% !TEX root = ../oddp.tex
\section{Cyclic comultiplications and power operations} \label{s:2bcomultiplications}

Let $\rho$ denote the standard generator of the cyclic group $\Cyc_r$ on $r$ elements, which we identify with $\{0,1,\ldots,r-1\}$ via $i\mapsto \rho^i$. The symmetric group on $\{0,1,\ldots,r-1\}$ is denoted $\Sym_r$, and the regular representation includes $\Cyc_r$ into $\Sym_r$. The letter $\tilde{r}$ will denote the quantity $\frac{r-1}{2}$. The definitions of Section \ref{s:unstable} are taken from \cite{may1970general} and \cite{medina2021may_st}. Chain complexes in this section are assumed to be non-negatively graded.

\subsection{Unstable comultiplications}\label{s:unstable}

The \emph{minimal $\Cyc_r$-resolution} of the commutative ring $R$ with trivial action is the chain complex $\W(r) = \bigoplus_{q\geq 0} \W[q](r)$ defined as follows:
\begin{align*}
	\W[q](r) &= R[\Cyc_r]\langle e_q\rangle  &
	\partial(e_q) &= \begin{cases}
		N(e_{q-1}) & \text{if $q$ is even} \\
		T(e_{q-1}) & \text{if $q$ is odd,}
	\end{cases}
\end{align*}
where $N = \sum_j \rho^j$ and $T = \rho - \Id$. 

\begin{definition} An \emph{unstable ($r$-cyclic) comultiplication} on a chain complex $(\ucadenas,\partial)$ is a $\Cyc_r$-equivariant chain map
\[\mu\colon \W(r)\otimes \ucadenas\lra \ucadenas^{\otimes r}.\]
\end{definition}

\begin{definition}
	An \emph{($r$-cyclic) May structure} on a chain complex $(\ucadenas,\partial)$ is an unstable comultiplication together with
	\begin{itemize}
		\item A homotopy associative comultiplication such that $\mu(e_0\otimes x)$ is homotopic to the $r$-fold comultiplication,
		\item a factorization 
	\[
		\mu\colon \W(r)\otimes \ucadenas\lra V_*(r)\otimes \W(r)\lra  \ucadenas^{\otimes r}
	\]
where $V_*(r)$ is a free resolution of the symmetric group on $r$ letters, the first map is of the form $f\otimes \id$ for some $f$ equivariant with respect to the regular representation of $\Cyc_r$, and the second map is $\Sym_r$-equivariant.
	\end{itemize}
\end{definition}


\begin{proposition}\label{prop:unstable}
	If $r$ is prime, an unstable comultiplication $\mu$ on a chain complex $\ucadenas$ induces operations indexed by the integers
	\begin{align*}
		\power^{i}\colon H_{-n}(\ucocadenas;\mathbb{F}_r)&\lra H_{-n-i}(\ucocadenas;\mathbb{F}_r)
	\end{align*}
	that send a cohomology class $[x]$ in degree $-n$ to the cohomology class $[y]$ with
	\[
	y(a) = \frac{1}{(\tilde{r}!)^n}(x\otimes \overset{r}{\ldots}\otimes x)(\mu(e_{(r-1)n-i}\otimes a))
	\]
	These operations satisfy the following equation
	\begin{itemize}
		\item $\power^i(x) = 0$ if $i>(r-1)n$ and $x$ has degree $-n$.
	\end{itemize}
\end{proposition}
\begin{remark} If $\mu$ is part of a May structure, then $\power^i$ vanishes unless $i = 2(r-1)k$ or $i = 2(r-1)k+1$.
\end{remark}

\begin{proof} See \cite{may1970general}.    %The operations are linear because $\mu$ is linear and Fermat's little theorem in $\mathbb{F}_r$ guarantees that $(\lambda\cdot y)^\otimes{r} = \lambda\cdot (y)^{\otimes r}$ for a scalar $\lambda$. That the operations are well-defined follows through the usual argument: if $x$ is a representative of a cohomology class, its coboundary vanishes, and so does the coboundary of $y\otimes\ldots\otimes y$, hence it maps cocycles to cocycles. If $x = \delta y$ is a coboundary, then FILL.
	For the last claim, note that if $i>(r-1)n$, then $(r-1)n-i$ is negative, and therefore $\mu(e_{(r-1)n-i},a)$ vanishes.
\end{proof}
\begin{example}
		The operations in \cite{steenrod1947products} yield a natural $2$-cyclic unstable comultiplication on the normalised chains of simplicial sets that is part of a May structure \cite{medina2021may_st} (see also \cite{gonzalez-diaz1999steenrod, medina2021fast_sq}). This extends to a natural $2$-cyclic connected stable comultiplication. A May structure for arbitrary $r$ is constructed in \cite{medina2021may_st} (see also \cite{brumfiel2023explicit}).
\end{example}



\subsection{Stable comultiplications}\label{s:stable}

Let $\Wst(r)$ be the unbounded chain complex
\[
\ldots\lra \Cyc_r\langle e_2\rangle \overset{N}{\lra}
\Cyc_r\langle e_1\rangle \overset{T}{\lra}
\Cyc_r\langle e_0\rangle \overset{N}{\lra}
\Cyc_r\langle e_{-1}\rangle \overset{T}{\lra}
\Cyc_r\langle e_{-2}\rangle \overset{N}{\lra}
\ldots
\]
If either $r$ is odd or $r=2$ and $R=\bF_2$, we understand it as the limit of
\[
\ldots\lra \sus{2(1-r)}\W(r)\overset{\theta_{1-r}}{\lra} \sus{1-r} \W(r) \overset{\theta_{1-r}}{\lra} \W(r).
\]
In general, we understand it as the limit of
\[
\ldots\lra \sus{4(1-r)}\W(r)\overset{\theta_{2(1-r)}}{\lra} \sus{2(1-r)} \W(r) \overset{\theta_{2(1-r)}}{\lra} \W(r).
\]
where $\theta_{k}\colon \W(r)\to \sus{-k} \W(r)$ is the \emph{suspension homomorphism} that sends $e_{q}$ to $e_{q+k}$, which only exists if $k$ is even. %If $r$ is odd or $r=2$ and $\R=\bF_2$, one can instead use the maps $\theta_{1-r}$ to form the limit.
 This complex is isomorphic to the desuspension of the complex that computes the Tate homology of $\Cyc_r$, which is part of a homotopy fiber sequence
\[
	\Wst(r)\lra \W(r)\overset{N}{\lra} \Wd(r)
\]
where the second map is the norm map that sends $e_q$ to zero unless $q=0$, in which case it is sent to $N(e_0^\dd)$. Notice that $\Wst(r)$ is not equal to the standard model of the homotopy fiber: in the homotopy fibre we have, for negative $i$, that $\partial(e_i) = -N(e_{i-1})$ or $T(e_{i-1})$ depending on whether $i$ is even or odd. 



\begin{definition} A \emph{stable ($r$-cyclic) comultiplication} on a chain complex $\ucadenas$ is a $\Cyc_r$-equivariant chain map
\[
\mu\colon \Wst(r)\otimes \ucadenas\lra \ucadenas^{\otimes r}
\]
\end{definition}

Following \cite{Gill2020} we define $\ESst_r$ as the inverse limit of
\[
	\ldots \lra \sus{2(1-r)} \ES_r\overset{\theta_{r-1}}{\lra }\sus{1-r} \ES_r\overset{\theta_{r-1}}{\lra} \ES_r
\]
where $\theta_{r-1}$ is the operadic suspension of \cite{berger2004combinatorial} that sends a sequence of permutations $(\sigma_0,\ldots,\sigma_q)$ to zero if $(\sigma_0(0),\ldots,\sigma_{r-1}(0))$ has repeated elements, and to $\pm (\sigma_{r-1},\sigma_r,\ldots,\sigma_q)$ otherwise, where $\pm$ is the sign of the permutation that orders the sequence $(\sigma_0(0),\ldots,\sigma_{r-1}(0))$.

\begin{definition} A \emph{stable ($r$-cyclic) May structure} on a chain complex $\ucadenas$ is a stable $r$-cyclic comultiplication $\mu$ that factors, up to homotopy, as
\[
\mu\colon \Wst(r)\otimes \ucadenas\lra \ESst_r\otimes \ucadenas\lra  \ucadenas^{\otimes r}
\]
where the first map is of the form $f\otimes \id$ with $f$ equivariant with respect to the inclusion $\Cyc_r\subset \Sym_r$ and the second map is $\Sym_r$-equivariant.
\end{definition}

Every unstable comultiplication yields a stable comultiplication by precomposing $\mu$ with the projection $\Wst(r)\to \W(r)$. A stable comultiplication comes from an unstable comultiplication if it vanishes on $\Wst[i](r)\otimes \ucadenas$ for $i<0$.

\begin{proposition}
	If $r$ is prime, a stable $r$-cyclic comultiplication $\mu$ on a non-negatively graded chain complex $\ucadenas$ induces operations indexed by the integers
	\begin{align*}
		\power^{i}\colon H_{-n}(\ucocadenas;\mathbb{F}_r)&\lra H_{-n-i}(\ucocadenas;\mathbb{F}_r)
	\end{align*}
	that send a cohomology class $[x]$ in degree $-n$ to the cohomology class $[y]$ with
	\[
	y(a) = \frac{1}{(\tilde{r}!)^n}(x\otimes \overset{r}{\ldots}\otimes x)(\mu(e_{(r-1)n-i}\otimes a))
	\]
\end{proposition}
\alert{
\begin{proof} If $\ucadenas$ has a stable comultiplication, then $\sus{} \ucadenas$ has a stable comultiplication as well, defined as $\mu(e_q\otimes \sus{} a) = \frac{1}{\tilde{r}!}\sus{\otimes r}\mu(e_{q+r-1}\otimes a)$. Therefore, after enough suspensions, we may assume that $\ucadenas$ vanishes in negative degrees and that $q$ is positive. Then a truncation reduces this to the situation of an unstable comultiplication.
\end{proof}
}
\begin{example}
	As we will see in Section \ref{s:suspension}, if $\mu$ is the unstable comultiplication for normalized chains in simplicial sets of the previous example, then $\mu(e_q\otimes \sus{} x) = \frac{1}{\tilde{r}!}\sus{\otimes r} \mu(e_{q+r-1}\otimes x)$, hence it does define a stable comultiplication on the normalised chains of simplicial spectra (these are defined in \cite{Gill2020}) that is also part of a stable May structure.
\end{example}


\begin{figure}

\[
	\xymatrix@R=.5cm@C=.5cm{
		\Wst[4] \ot  C_{-2}\ar[d] &
		\Wst[4] \ot  C_{-1}\ar[d]\ar[l] &
		\Wst[4] \ot  C_{0}\ar[d]\ar[l] &
		\Wst[4] \ot  C_{1}\ar[d]\ar[l] &
		\Wst[4] \ot  C_{2}\ar[d]\ar[l]
		\\
		\Wst[3] \ot  C_{-2}\ar[d] &
		\Wst[3] \ot  C_{-1}\ar[d]\ar[l] &
		\Wst[3] \ot  C_{0}\ar[d]\ar[l] &
		\Wst[3] \ot  C_{1}\ar[d]\ar[l] &
		\Wst[3] \ot  C_{2}\ar[d]\ar[l]
		\\
		\Wst[2] \ot  C_{-2}\ar[d] &
		\Wst[2] \ot  C_{-1}\ar[d]\ar[l] &
		\Wst[2] \ot  C_{0}\ar[d]\ar[l] &
		\Wst[2] \ot  C_{1}\ar[d]\ar[l] &
		\Wst[2] \ot  C_{2}\ar[d]\ar[l]
		\\
		\Wst[1] \ot  C_{-2}\ar[d] &
		\Wst[1] \ot  C_{-1}\ar[d]\ar[l] &
		\Wst[1] \ot  C_{0}\ar[d]\ar[l] &
		\Wst[1] \ot  C_{1}\ar[d]\ar[l] &
		\Wst[1] \ot  C_{2}\ar[d]\ar[l]
		\\
		\Wst[0] \ot  C_{-2}\ar[d] &
		\Wst[0] \ot  C_{-1}\ar[d]\ar[l] &
		\Wst[0] \ot  C_{0}\ar[d]\ar[l] &
		\Wst[0] \ot  C_{1}\ar[d]\ar[l] &
		\Wst[0] \ot  C_{2}\ar[d]\ar[l]
		\\
		\green{\Wst[-1] \ot  C_{-2}}\ar[d] &
		\green{\Wst[-1] \ot  C_{-1}}\ar[d]\ar[l] &
		\green{\Wst[-1] \ot  C_{0}}\ar[d]\ar[l] &
		\green{\Wst[-1] \ot  C_{1}}\ar[d]\ar[l] &
		\green{\Wst[-1] \ot  C_{2}}\ar[d]\ar[l]
		\\
		\green{\Wst[-2] \ot  C_{-2}} &
		\green{\Wst[-2] \ot  C_{-1}}\ar[l] &
		\green{\Wst[-2] \ot  C_{0}}\ar[l] &
		\green{\Wst[-2] \ot  C_{1}}\ar[l] &
		\green{\Wst[-2] \ot  C_{2}}\ar[l]
}
\]

\caption{A depiction of $\Wst(3)\otimes \ucadenas$. A stable comultiplication comes from an unstable comultiplication if it vanishes on the part colored in green. }
\end{figure}



\subsection{Connected comultiplications}\label{s:connected} Let $\mu$ be a stable comultiplication. A way of guaranteeing the desirable property that $\power^i(x)$ vanishes for $i<0$ is to ask for $\mu$ to vanish on $\Wst[i]\otimes \ucadenas$ for $i>(r-1)n$.
We will now introduce a convenient chain complex $\rWd(r)\hotimes \rucadenas$ such that giving a chain map $\mu\colon \rWd(r)\hotimes \rucadenas\to \ucadenas^{\otimes r}$ is equivalent to give a stable comultiplication that vanishes on $\Wst[i](r)\otimes \ucadenas$ for $i>(r-1)n$.
\begin{figure}

\[
	\xymatrix@R=.5cm@C=.5cm{
		\green{\Wst[4]\otimes C_{-2}}\ar[d] &
		\green{\Wst[4]\otimes C_{-1}}\ar[d]\ar[l] &
		\green{\Wst[4]\otimes C_{0}}\ar[d]\ar[l] &
		\green{\Wst[4]\otimes C_{1}}\ar[d]\ar[l] &
		\Wst[4]\otimes C_{2}\ar[d]\ar[l]
		\\
		\green{\Wst[3]\otimes C_{-2}}\ar[d] &
		\green{\Wst[3]\otimes C_{-1}}\ar[d]\ar[l] &
		\green{\Wst[3]\otimes C_{0}}\ar[d]\ar[l] &
		\green{\Wst[3]\otimes C_{1}}\ar[d]\ar[l] &
		\Wst[3]\otimes C_{2}\ar[d]\ar[l]
		\\
		\green{\Wst[2]\otimes C_{-2}}\ar[d] &
		\green{\Wst[2]\otimes C_{-1}}\ar[d]\ar[l] &
		\green{\Wst[2]\otimes C_{0}}\ar[d]\ar[l] &
		\Wst[2]\otimes C_{1}\ar[d]\ar[l] &
		\Wst[2]\otimes C_{2}\ar[d]\ar[l]
		\\
		\green{\Wst[1]\otimes C_{-2}}\ar[d] &
		\green{\Wst[1]\otimes C_{-1}}\ar[d]\ar[l] &
		\green{\Wst[1]\otimes C_{0}}\ar[d]\ar[l] &
		\Wst[1]\otimes C_{1}\ar[d]\ar[l] &
		\Wst[1]\otimes C_{2}\ar[d]\ar[l]
		\\
		\green{\Wst[0]\otimes C_{-2}}\ar[d] &
		\green{\Wst[0]\otimes C_{-1}}\ar[d]\ar[l] &
		\Wst[0]\otimes C_{0}\ar[d]\ar[l] &
		\Wst[0]\otimes C_{1}\ar[d]\ar[l] &
		\Wst[0]\otimes C_{2}\ar[d]\ar[l]
		\\
		\green{\Wst[-1]\otimes C_{-2}}\ar[d] &
		\green{\Wst[-1]\otimes C_{-1}}\ar[d]\ar[l] &
		\Wst[-1]\otimes C_{0}\ar[d]\ar[l] &
		\Wst[-1]\otimes C_{1}\ar[d]\ar[l] &
		\Wst[-1]\otimes C_{2}\ar[d]\ar[l]
		\\
		\green{\Wst[-2]\otimes C_{-2}} &
		\Wst[-2]\otimes C_{-1}\ar[l] &
		\Wst[-2]\otimes C_{0}\ar[l] &
		\Wst[-2]\otimes C_{1}\ar[l] &
		\Wst[-2]\otimes C_{2}\ar[l]
}
\]
\caption{If a stable comultiplication for $r=3$ vanishes on the part colored in green, then $\power^i$ vanishes for $i<0$.}
\label{figure:2}


\[
	\xymatrix@R=.4cm@C=.4cm{
		\phantom{W_4\hotimes C^{3}_{-3}} &
		\phantom{W_4\hotimes C^{3}_{-3}} &
		\phantom{W_4\hotimes C^{3}_{0}} &
		\phantom{W_4\hotimes C^{3}_{3}} &
		\bF_3\hotimes C^{3}_{6}\ar[d]
		\\
		\phantom{W_3\hotimes C^{3}_{-3}} &
		\phantom{W_3\hotimes C^{3}_{-3}} &
		\phantom{W_3\hotimes C^{3}_{0}} &
		\phantom{W_3\hotimes C^{3}_{3}} &
		\rWd[1]\hotimes C^{3}_{6}\ar[d]
		\\
		\phantom{W_2\hotimes C^{3}_{-3}} &
		\phantom{W_2\hotimes C^{3}_{-3}} &
		\phantom{W_2\hotimes C^{3}_{0}} &
		\bF_3\hotimes C^{3}_{3}\ar[d] &
		\rWd[-2]\hotimes C^{3}_{6}\ar[d]\ar[l]
		\\
		\phantom{W_1\hotimes C^{3}_{-3}} &
		\phantom{W_1\hotimes C^{3}_{-3}}&
		\phantom{W_1\hotimes C^{3}_{0}} &
		\rWd[1]\hotimes C^{3}_{3}\ar[d] &
		\rWd[-3]\hotimes C^{3}_{6}\ar[d]\ar[l]
		\\
		\phantom{W_0\hotimes C^{3}_{-3}} &
		\phantom{W_0\hotimes C^{3}_{-3}}&
		\bF_3\hotimes C^{3}_{0}\ar[d] &
		\rWd[-2]\hotimes C^{3}_{3}\ar[d]\ar[l] &
		\rWd[-4]\hotimes C^{3}_{6}\ar[d]\ar[l]
		\\
		\phantom{\rWd[-1]\hotimes C^{3}_{-3}} &
		\phantom{\rWd[-1]\hotimes C^{3}_{-3}} &
		\rWd[-1]\hotimes C^{3}_{0}\ar[d] &
		\rWd[-3]\hotimes C^{3}_{3}\ar[d]\ar[l] &
		\rWd[-5]\hotimes C^{3}_{6}\ar[d]\ar[l]
		\\
		\phantom{\rWd[-2]\hotimes C^{3}_{-3}} &
		\bF_3\hotimes C^{3}_{-3} &
		\rWd[-2]\hotimes C^{3}_{0}\ar[l] &
		\rWd[-4]\hotimes C^{3}_{3}\ar[l] &
		\rWd[-6]\hotimes C^{3}_{6}\ar[l]
}
\]

\caption{The complex $\rWd(r)\hotimes \rucadenas$ for $r=3$ and $R=\bF_3$. The map $\varphi$ goes from Figure \ref{figure:2} to here, is an epimorphism and its kernel is the smallest chain subcomplex that contains the green part.}
\label{figure:3}

\end{figure}
Let $\rucadenas$ be the chain complex $\bigoplus_m \sus{(r-1)m}\ucadenas[m]$ with the same differential as $\ucadenas$, which now has degree $-r$. An element $x\in \ucadenas$ viewed in $\rucadenas$ will be denoted $\vec{x}$. We write $\rW(r)$ for the right suspension of the augmented chain complex $\W(r)\to R$, i.e., the chain complex
\[
\ldots
\overset{T}{\lra} 
R[\Cyc_r]\langle e_{3}^\dd\rangle 
\overset{N}{\lra} 
R[\Cyc_r]\langle e_{2}^\dd\rangle 
\overset{T}{\lra} 
R[\Cyc_r]\langle e_{1}^\dd\rangle 
\overset{N}{\lra} 
R
\]
(note the reindexing, so that $e_i$ has degree $i$) and $\rWd(r)$ for its dual 
\[
R\overset{-N}{\lra} R[\Cyc_r]\langle e_{-1}^\dd\rangle \overset{T}{\lra} R[\Cyc_r]\langle e_{-2}^\dd\rangle \overset{-N}{\lra} R[\Cyc_r]\langle e_{-3}^\dd\rangle \overset{T}{\lra} \ldots
\]
Observe that $\rho$ acts ``backwards'' in the dual complex, in the sense that $\rho (e_{-i}^\dd) = (\rho^{-1} e_i)^{\dd}$. The generator of the copy of $R$ in the left will be denoted as $1$ or as $e_0^\dd$.

Again, there is a suspension map $\theta_k\colon \rWd(r)\to \susp{-k}\rWd(r)$ defined for all even $k$ (or all $k$ in case $r=2$ and $R=\bF_2$) as
\begin{align*}
	\theta_k(e_{-i}^\dd) &= e_{k-i}^\dd 
	&
	\theta_k(1) &= \begin{cases} N(e_k^\dd) &\text{if $k<0$} \\ 1 &\text{if $k=0$} \\ 0 &\text{if $k>0$.} \end{cases}
\end{align*}
If either $r$ is odd or $r=2$ and $R=\mathbb{F}_2$, define the chain complex $\rWd(r)\hotimes \rucadenas$ as the graded $R[\Cyc_r]$-module $\rWd(r)\otimes \rucadenas$ with the differential
\[
\partial(e^\dd_{-q}\otimes \vec{x}) = \partial (e^\dd_{-q})\otimes \vec{x} + (-1)^q \theta_{r-1}(e^\dd_{-q})\otimes \partial (\vec{x}).
\]
If $r$ is even and $R$ is arbitrary, define the cochain complex $\rWhatd(r)$ as
\[
R[\Cyc_r]\langle e_{0}^\dd\rangle / N \overset{-T}{\lra} R[\Cyc_r]\langle e_{-1}^\dd\rangle \overset{N}{\lra} R[\Cyc_r]\langle e_{-2}^\dd\rangle \overset{-T}{\lra}\ldots
\]
This chain complex admits chain maps 
\begin{align*}
    \hat{\theta}_{r-1}\colon \rWd(r)&\lra \susp{1-r}\rWhat(r)
    &
    \hat{\theta}_{r-1}\colon \rWhat(r)&\lra \susp{1-r}\rW(r).
\end{align*} 
Define $\ucadenas[\mathrm{even}] = \bigoplus_{m}\ucadenas[2m]$ and $\ucadenas[\mathrm{odd}] = \bigoplus_{m}\ucadenas[2m+1]$ and write $\rWd(r)\hotimes \rucadenas$ for the graded $R[\Cyc_r]$-module $\left(\rWd(r)\otimes \rucadenas[\mathrm{odd}]\right)\oplus \left(\rWhatd(r)\otimes \rucadenas[\mathrm{even}]\right)$ with differential
\[
\partial(e^\vee_{-q}\otimes \vec{x}) = \partial (e^\vee_{-q})\otimes \vec{x} + (-1)^q \theta_{1-r}(e^\vee_{-q})\otimes \partial (\vec{x}).
\]
Observe that if $r=2$ and $R=\bF_2$, the complexes $\rWd(r)$ and $\rWhatd(r)$ coincide, and so do both definitions of $\rWd(r)\hotimes \rucadenas$.
\begin{definition}
	A \emph{connected ($r$-cyclic) comultiplication} on a \alert{bounded below} chain complex $\ucadenas$ is a $\Cyc_r$-equivariant chain map
	\[
	\mu\colon \rWd(r)\hotimes \rucadenas\lra \ucadenas^{\otimes r}.
	\]
\end{definition}

If either $k$ is even or $r=2$ and $R=\bF_2$, let $\varphi_k\colon \Wst(r)\to \susp{-k}\rWd(r)$ be the chain map that sends $e_q$ to $(-1)^{\binom{k-q}{2}}e_{k-q}^{\vee}$. One can obtain this chain map as follows: We already saw $\Wst(r)$ as the homotopy fiber of the norm map $N\colon \W(r)\to \Wd(r)$, but it is also the homotopy fiber of the map $\rWd(r)\to \sus{}\W(r)$ that is the identity on $R$ and zero elsewhere. The map $\varphi_k$ is the composition
\[
	\Wst(r)\overset{\theta_k}{\lra} \sus{-k} \Wst(r) \overset{\cong}{\lra} \sus{-k}\hofib(\rWd(r)\to \rW(r)) \lra \susp{-k}\rWd(r).
\]
This composition is an epimorphism with kernel the smallest chain subcomplex that contains $\Wst[>k]$. In other words, it is isomorphic to the truncation $\tau_k \Wst(r)$. If $k$ is odd, let $\varphi_k\colon \Wst(r)\to \sus{k}\rWhatd(r)$ be the chain map that sends $e_q$ to $e_{k-q}^{\vee}$. 

Define the chain map $\varphi\colon \Wst(r)\to \rWd(r)\hotimes \rucadenas$ as 
\[
	\varphi(e_q\otimes x) = \varphi_{-n(r-1)}(e_{q})\otimes \vec{x}
\]
if $x$ has degree $n$. This is an epimorphism with kernel the smallest chain subcomplex that contains $\bigoplus_n\bigoplus_{i>(r-1)n} \Wst[i]\otimes \ucadenas[n]$. As a consequence,



\begin{proposition}
	If $r$ is prime, a connected $r$-cyclic comultiplication induces operations indexed by the integers
	\begin{align*}
		\power^{i}\colon H_{-n}(\ucocadenas;\mathbb{F}_r)&\lra H_{-n-i}(\ucocadenas;\mathbb{F}_r)
	\end{align*}
	that send a cohomology class $[x]$ in degree $n$ to the cohomology class $[y]$ with
	\[
	y(a) = \frac{1}{(\tilde{r}!)^n}(y\otimes \overset{r}{\ldots}\otimes y)(\mu(e_{ri}^\vee\otimes a))
	\]
	These operations satisfy the following equation
	\begin{itemize}
		\item $\power^i(x) = 0$ for $i<0$.
	\end{itemize}
\end{proposition}

\begin{example}
As far as we know, the concept of connected comultiplication has not appeared before in the literature, nonetheless, this viewpoint is implicit in the construction of the unstable comultiplication for $r=2$ on the cochain operations on the normalised cochains of a simplicial set of \cite{medina2021fast_sq}. It is also implicit in the construction of the stable comultiplication for $r=2$ of the cochain operations on the cochains of augmented semi-simplicial objects in the Burnside $2$-category \cite{cantero-moran2020khovanov}.
\end{example}

\begin{question}
	Give a good definition of ``connected May structure''. Give a homotopy equivalence between the Tate complex of $\ES_r$ and the unbounded complex $\ESst_r$.
\end{question}


% !TEX root = ../oddp.tex

\subsection{From \texorpdfstring{$\Omega_*(r)$}{Omega(r)} to the minimal resolution}\label{ss:mapf}

The \emph{pivotal piece} of a pieced word $A_0\barra \dots\barra A_k$ is the leftmost piece $A_j$ such that $A_{j+1}*\dots*A_k$ has degree smaller than $r$ (i.e., $A_{j+1}\cup\dots\cup A_k$ has less than $r$ entries).

\begin{example}
	If $r=5$, the pivotal piece of the pieced word $A=01|24|013|12|4$ is $A_1 = 013$, while $A_2 = 12|4$ and $\hat{A} = 01|24$. If $r=3$, the pivotal piece of the pieced word $01|20|1|02|1|20|12$ is $20$, while $A_2 = 12$ and $\hat{A} = 01|20|1|02|1$.
\end{example}

The join homomorphism $\chains(\asimplex^{r-1}\join \asimplex^{r-1}) \to \chains(\asimplex^{r-1})$ of Definition \ref{d:join_product} induces an endomorphism
\[
\kappa \colon \Omega_*(r)^{\nf} \lra \Omega_*(r)^{\nf}
\]
that sends a pieced word $A_0\barra \dots\barra A_k$ with pivotal piece $A_j$ to the pieced word $(-1)^{\lambda(A_{j+1},\dots,A_k)}A_0\barra \dots\barra A_j\barra A_{j+1}\join\dots\join A_k$, where the sign is that of the permutation that orders the entries of $A_{j+1},\dots,A_k$.

The cyclic group $\Cyc_r$ acts on $\partial \asimplex^{r-1}$ by permuting its vertices forwards. Let $\sigma = [0,1,\dots,r-1]$ be the top simplex of $\asimplex^{r-1}$ and let
\begin{align*}
	\iota_1 \colon \sus{r-1}\chains(\partial \simplex^{r-1})& \lra \chains(\partial\simplex^{r-1}) \ot \chains(\partial\simplex^{r-1})\\
	\iota_2 \colon \sus{r-1}\chains(\partial \simplex^{r-1})& \lra \chains(\partial\simplex^{r-1}) \ot \chains(\partial\simplex^{r-1})
\end{align*}
be the $\Cyc_r$-equivariant chain maps of degree $r-1$ given by
\begin{align*}
	\iota_1(\tau) &= \tau \ot \partial \sigma &
	\iota_2(\tau) &= \partial \sigma \ot \tau
\end{align*}

\begin{construction}\label{cons:1} Let $f \colon \chains(\partial\simplex^{r-1}) \ot \chains(\partial\simplex^{r-1}) \to \sus{r-1}\chains(\partial\asimplex^{r-1})$ be a $\Cyc_r$-equivariant chain map such that
\renewcommand{\theenumi}{\roman{enumi}}
\begin{enumerate}
	\item\label{cond:1} $f\circ \iota_1 = \Id$
	\item\label{cond:2} $f\circ \iota_2 = \rho$.
	\item\label{cond:3} if $\tau_1 \ot \tau_2$ has degree $r$, then $ \partial f(\tau_1 \ot \tau_2) =
	\begin{cases}
		(-1)^{\lambda(\tau_1,\tau_2)}\emptyset & \text{if $\tau_1^c=\tau_2$} \\
		0 & \text{otherwise},
	\end{cases}$
	where $\lambda(\tau_1,\tau_2)$ is the sign of the permutation that orders $\tau_1\cup \tau_2$.
\end{enumerate}
\end{construction}

\begin{remark}\label{remark:3prime}
	Condition \eqref{cond:3} is implied by the following condition:
	\begin{itemize}
		\item[(iii')] If $\tau_1 \ot \tau_2$ has degree $r$, then $f(\tau_1 \ot \tau_2)$ is $(-1)^{\lambda(\tau_1,\tau_2)} v$ for some vertex of $\partial \asimplex^{r-1}$ if $\tau_1\cup\tau_2$ is non-degenerate and $0$ otherwise.
	\end{itemize}
\end{remark}

\begin{proposition} Let $f$ be a chain map as in Construction \ref{cons:1}, satisfying Conditions \eqref{cond:1} and \eqref{cond:3}. Then, there is a chain map
	\[
	S \colon \Omega_*(r)^{\nf} \lra \sus{r-1}\Omega_*(r)^{\nf}
	\]
	that sends a pieced word $A = A_0\barra \dots\barra A_k$ with pivotal piece $A_j$ to the pieced word $ A_0\barra \dots\barra f(A_j \ot \kappa(A_{j+1}\barra \dots\barra A_k))$.
\end{proposition}

\begin{proof}
	Replacing the word $A$ by $\kappa(A)$, we may assume that $A_{j+1} = A_k$. Let $\bar{A} = A_0\barra \dots\barra A_{j-1}$ and let $\ell$ be the degree of $\bar{A}$ (i.e., the number of entries of $A$). Then 
	\begin{equation}\label{eq:931}
		\partial S(A) = \partial \bar{A}\mid f(A_{j} \ot A_{j+1}) + (-1)^{\ell}\bar{A}\barra \partial f(A_{j} \ot A_{j+1}).
	\end{equation}
	We distinguish two cases: if $A_j\cup A_{j+1}$ has length at least $r+1$,
	\[S(\partial A) = \partial \bar{A}\barra f(A_j \ot A_{j+1}) + (-1)^{\ell}\bar{A}\barra f(\partial (A_j \ot A_{j+1})),\]
	which equals the previous sum. If $A_j\cup A_{j+1}$ has length $r$, let $\check{A} = A_0\mid \dots\mid A_{j-2}$. Then
	\begin{equation}\label{eq:933}
		S(\partial A) = \partial \bar{A}\barra f(A_j \ot A_{j+1}) + (-1)^\ell \check{A}\barra f(A_{j-1} \ot \partial(A_j*A_{j+1})).
	\end{equation}
	Let $\pm$ be the sign of the permutation that orders $A_j*A_{j+1}$. By condition \eqref{cond:3}, the last summand of \eqref{eq:931} is zero or $\pm(-1)^\ell\bar{A}$ depending on whether $A_j*A_{j+1}$ is degenerate or not, which by condition \eqref{cond:1} is equivalent to $f(A_{j-1},\partial(A_j*A_{j+1}))$ being $0$ or $\pm(-1)^{\ell}A_{j-1}$, and therefore equivalent to the last summand of \eqref{eq:933} being zero or $\pm(-1)^{\ell}\bar{A}$.
\end{proof}


\begin{definition}\label{def:psiom}
	Let $f$ be a chain map as in Construction \ref{cons:1}, satisfying Conditions \eqref{cond:1} and \eqref{cond:3}. Define a homomorphism $\Psi \colon \Omega_*(r)^{\nf} \ot \to \rW(r) \ot \bZ[\frac{1}{\tilde{r}!}]$ recursively as
	\[\Psiom_q(A) = \begin{cases} \Phi_q(\kappa(A)) & \text{if $q\leq r-1$} \\
		\frac{1}{\tilde{r}!}\theta_{r-1}\circ \Psiom_{q-r+1}\circ S(A) & \text{if $q \geq r$,}\end{cases}\]
\end{definition}

\begin{example}\label{example:psi3}
Let $r=3$. We will use the computations of Examples \ref{example:Phi}, \ref{example:f3_1} and \ref{example:f3_2} with the $f$ constructed in \cref{s:assembly}. To lighten the notation, we will write each block $A_j = [a_0,\dots,a_{m-1}]$ as $a_0a_1\dots a_{m-1}$, so $[0,1]$ is written $01$.
	\begin{align*}
		\Psi_6(01\barra 2\barra 0\barra 2\barra 1)
		&= \theta_2\circ \Psi_4\circ S(01\barra 2\barra 0\barra 2\barra 1)
		\\
		&= \theta_2\circ \Psi_4(01\barra 2\barra f([0] \ot [2,1]))
		\\
		&= -\theta_2\circ \Psi_4\circ S(01\barra 2\barra 0)
		\\
		&= -\theta_4\circ \Psi_2(f([0,1] \ot [2,0]))
		\\
		&= -\theta_4\circ \Psi_2(01)
		\\
		&= -\theta_4\circ \Phi_2(01)
		\\
		&= -\theta_4(e_2 + \rho^3e_2)
		\\
		&= -e_6 - \rho^3e_6
	\end{align*}
\end{example}

\begin{example}\label{example:psi5}
Let $r=5$. We will use the computations of Examples \ref{example:Phi} and \ref{example:f5_1} with the $f$ constructed in \cref{s:assembly}. We use the same lightened notation as in the previous example
	\begin{align*}
		\Psi_7(1\barra 230\barra 413)
		&= 2^{-1}\theta_4\circ \Psi_3\circ S(1\barra 230\barra 413)
		\\
		&= 2^{-1}\theta_4\circ \Psi_3(1\barra f([2,3,0] \ot [4,1,3]))
		\\
		&= 2^{-1}\theta_4\circ \Psi_3(1\barra 02)
		\\
		&= 2^{-1}\theta_4\circ \Phi_3([1,0,2])
		\\
		&= -2^{-1}\theta_4(e_3)
		\\
		&= -2^{-1}e_7
	\end{align*}
\end{example}

\begin{proposition}\label{prop:chain}
	$\Psiom$ is a chain map. 
\end{proposition}

\begin{proof}
	$\Psiom_*$ is already a chain map up to degree $*\leq r-1$. To check that it commutes with the differential in degree $r$, we distinguish two cases:

	First, if $\Psiom(A) = 0$, then $A$ has repeated entries. If it has one repeated entry, then the image by $\Psiom$ of its boundary has two cancelling terms, while if it has more repeated entries, it is zero on the nose.

	Second, if $\Psiom(A)\neq 0$, then $A$ has no repetitions. We may assume, as before, that $A = A_1|A_2$ has only two pieces, and write $A_1*A_2$ for their join and $\diff_iA_1*A_2$ for the result of removing the $i$th entry. Then:
	\begin{align*}
		\Psi_{r-1}(\partial A)
		&= \sum_i (-1)^i \Phi(\diff_i(A_1*A_2))
		\\
		&= (-1)^{\lambda(A_1,A_2)}\sum_i (-1)^i(-1)^{i}{\textstyle\frac{1}{\tilde{r}!}}\rho^{i+1} e_{r-1}
		\\
		&= (-1)^{\lambda(A_1,A_2)} {\textstyle\frac{1}{\tilde{r}!}} N(e_{r-1}).
	\end{align*}
	On the other hand,
	\begin{align*}
		\partial \circ \Psi_r(A) &= {\textstyle\frac{1}{\tilde{r}!}}\partial \circ \theta_{r-1}\circ \Psi_1\circ S(A)
		\\
		&= {\textstyle\frac{1}{\tilde{r}!}}\theta_{r-1}\circ \Psi_0\circ \partial\circ S(A)
		\\
		&= {\textstyle\frac{1}{\tilde{r}!}}\theta_{r-1}\circ \Psi_0\circ \partial\circ f(A_1 \ot A_2)
		\\
		&= (-1)^{\lambda(A_1,A_2)}{\textstyle\frac{1}{\tilde{r}!}} \theta_{r-1}(e_0)
		\\
		&= (-1)^{\lambda(A_1,A_2)}{\textstyle\frac{1}{\tilde{r}!}} N(e_{r-1}).
	\end{align*}
	Finally, to check that $\Psi$ commutes with the differential in degrees $q>r$, one proceeds by induction.
\end{proof}


\begin{proposition}\label{prop:condition}
	Let $f$ be a chain map as in Construction \ref{cons:1}, satisfying Conditions \eqref{cond:1}, \eqref{cond:2} and \eqref{cond:3}. If $\Psi$ is defined as in Definition \ref{def:psiom}, and $A\in \Omega_*(r)$ has a full piece,
	\[
	\tilde{r}!\Psiom_{q-1}(\partial^{\nf} (A)) = (-1)^q\theta_{r-1}\circ \Psiom_{q-r}\circ\DDD(A).
	\]
\end{proposition}

\begin{proof}
	We will prove it by induction on the position of the full piece from the right. Since $A$ has one full piece, $q \geq r$. We assume without loss of generality that the full piece is $[0,1,\dots,r-1]$ (i.e., it is ordered), and we distinguish three cases:

	If the full piece is the last piece, write $A=\bar{A}\barra A_1\barra A_2$ with $A_2 = \sigma$ the full piece and $A_1$ the penultimate piece, and let $\ell$ be the length of $\bar{A}*A_1$. Then by Condition \eqref{cond:1}:
	\begin{align*}
		\tilde{r}!\Psiom_{q-1}(\partial^{\nf} (A)) &=
		\theta_{r-1} \circ \Psiom_{q-r}\circ S (\partial^{\nf}(A)) \\
		&= (-1)^{\ell} \theta_{r-1}\circ\Psiom_{q-r}\circ S(\bar{A}\barra A_1\barra \partial A_2) \\
		&= (-1)^{\ell} \theta_{r-1}\circ\Psiom_{q-r}(\bar{A}\barra f(A_1 \ot \partial \sigma)) \\
		&= (-1)^{\ell} \theta_{r-1}\circ\Psiom_{q-r}(\bar{A}\barra A_1) \\
		&= (-1)^q\theta_{r-1}\circ \Psiom_{q-r}\circ\DDD(A).
	\end{align*}
	If the full piece is the penultimate piece, write $A=\bar{A}\barra A_1\barra A_2$ with $A_1$ the full piece and $\ell$ for the length of $\bar{A}$. Then by Condition \eqref{cond:2}:
	\begin{align*}
		\tilde{r}!\Psiom_{q-1}(\partial^{\nf} A) &=
		\theta_{r-1}\circ \Psiom_{q-r}\circ S(\partial^{\nf}(A)) \\
		&= (-1)^{\ell} \theta_{r-1}\circ \Psiom_{q-r}\circ S(\bar{A}\barra \partial A_1\barra A_2) \\
		&= (-1)^{\ell} \theta_{r-1}\circ \Psiom_{q-r}(\bar{A}\barra f(\partial A_1 \ot A_2)) \\
		&= (-1)^{\ell} \theta_{r-1}\circ \Psiom_{q-r}(\bar{A}\barra \rho(A_2)) \\
		&= (-1)^q\theta_{r-1}\circ \Psiom_{q-r}\circ \DDD(A).
	\end{align*}
	Otherwise, $q \geq 2r$, and we reason by induction as follows:
	\begin{align*}
		(\tilde{r}!)^2\Psiom_{q-1}(\partial^{\nf} A)
		&= (-1)^q\tilde{r}!\theta_{r-1}\circ \Psiom_{q-r}\circ S(\partial^{\nf} A) \\
		&= (-1)^q\tilde{r}!\theta_{r-1}\circ \Psiom_{q-r}(\partial^{\nf} SA) \\
		&= (-1)^{q+q-r}\theta_{r-1}\circ \theta_{r-1}\circ \Psiom_{q-2r+1}\circ \DDD (SA)) \\
		&= (-1)^{q+q-r}\theta_{r-1}\circ \theta_{r-1}\circ \Psiom_{q-2r+1}\circ S\circ \DDD (A) \\
		&= (-1)^{q}\tilde{r}!\theta_{r-1}\circ \Psiom_{q-r}\circ \DDD (A)\qedhere
	\end{align*}
\end{proof}
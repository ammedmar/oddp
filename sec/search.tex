% !TEX root = ../oddp.tex

\section{Some notes in the search for $f$}

Let $Z_\bullet(r)$ be the semi-simplicial set whose $k$-simplices are partitions $(q_0,\ldots,q_k)$ of $r$ with face maps given by
\begin{align*}
	\partial_i(q_0,\ldots,q_k) &= \begin{cases}
		(q_0,\ldots,q_{i-1}+q_{i},q_{i+1},\ldots,q_k) & \text{if $i>0$} \\
		(q_1,\ldots,q_{k}+q_0) & \text{if $i=0$}
	\end{cases}
\end{align*}
We write $\Lambda(r)$ for the quotient of $N_*(Z(r))$ by the equivalence relation generated by $(q_0,q_1,\ldots,q_k)\sim (q_k,q_0,\ldots,q_{k-1})$.

\begin{example}
	$r=7$, $(q_0,q_1,q_2) = (2,2,3)$ and its boundary is $(2,5)-(4,3)+(2,5)$.
\end{example}

Write $\Theta(r)$ for the quotient of $N_*(\partial \Delta^{r-1})$ by the equivalence relation generated by $(a_0,a_1,\ldots,a_k)\sim (\rho(a_0),\rho(a_1),\ldots,\rho(a_k))$. If $(a_0,\ldots,a_k)$ is a $k$-simplex, we will use the cyclic notation $a_{k+1} = a_0$.

\begin{definition}
	Let $a = (a_0,\ldots,a_k)\in \partial \Delta^{r-1}$ be a simplex. The \emph{pattern} $q(a)$ of $a$ is the sequence of cyclic differences $(a_0-a_k+r,a_1-a_0,a_2-a_1,\ldots,a_k-a_{k-1})$.
\end{definition}

\begin{lemma}
	The map $q\colon \Theta(r)\to \Lambda(r)$ is an isomorphism of chain complexes.
\end{lemma}

\begin{proof}
	Immediate.
\end{proof}

\subsection{Reflections on the map $f$}

As we said before, the map $f$ is determined by its value on $(r-1)$-simplices. Therefore, we face the problem of choosing wisely a $C_r$-equivariant function $f\colon (\partial\Delta^{r-1}*\partial\Delta^{r-1})_{r-1}\to \{0,1,\ldots,r-1\}$ satisfying condition \eqref{cond:3}, and then check that
\begin{enumerate}
	\item $f$ extends to the whole join.
	\item $f$ satisfies condition \eqref{cond:1}
	\item $f$ satisfies condition \eqref{cond:2}.
\end{enumerate}
The set of $C_r$-isomorphism classes of $(r-1)$-simplices of $\partial\Delta^{r-1}*\partial\Delta^{r-1}$ is in bijection with the generators of $\Theta(r)$ in two ways: The first isomorphism sends a pair $(a,b)$ to the class of $a$. The second isomorphism sends $(a,b)$ to $b$. I believe that the extra structure (the differential) in $\Lambda(r)$ (or $\Theta(r)$) will be helpful.

I will start working with $\Theta(r)$, though many of the ideas come from thinking about $\Lambda(r)$. The restriction of the unknown $f$ to $\Theta(r)$ along the two isomorphisms above will be denoted $\ell,\bar{\ell}\colon \Theta(r)\to \{0,1,\ldots,r-1\}$. If $a$ is a generator of $\Theta(r)$, its complement in $\{0,1,\ldots,r-1\}$ wil be denoted $b$.

\subsection{First assumption}

I will make the following assumption:
\begin{assumption}
	$\ell(a)\in a$ and $\ell(a)-1\notin a$.
\end{assumption}

\begin{remark}
	This assumption on $\ell$ translates to $\bar{\ell}$ as follows: Since $\ell(a)-1\in b$, $\rho^{-1}(\bar{\ell}(b))\in b$
\end{remark}

\begin{definition}
	Given a generator $\sigma$ of $\Theta(r)$, a \emph{spike} of $\sigma$ is any $a_i$ such that $\ell(\partial_i a) \neq \ell(a)$. Observe that $\ell(a)$ is always a spike under the assumption above. The set of all spikes of $a$ is called $\mathrm{spk}(a)$. The assignment $a\mapsto \partial_i a$ is \emph{standar} if $\ell(\partial_i a) = \ell(a)$, otherwise we say that it is \emph{exceptional}.
\end{definition}

\begin{lemma}
	Under the above assumption, $f(a,b) \neq 0$ if and only if $a\cap b$ is a spike, in which case its value is $(\ell(a),\ell(a\smallsetminus b))$.
\end{lemma}

\begin{proof}
	All the face maps yield non-full pairs (pairs where $a\cup b \neq \{0,1,\ldots,r-1\}$) except for the two face map that remove the element $a\cap b$ from $a$ and from $b$. Removing it from $b$ yields the pair $(a,b\smallsetminus b)$ whose image under $f$ is $\ell(a)$. Removing it from $a$ yields $\ell(\partial_i a)$, which cancels with the other image unless $a_i$ is a spike.
\end{proof}

\subsection{Second assumption}

We introduce from now on the following stronger assumption.

\begin{assumption}
	If $a\mapsto \partial_ia$ is exceptional (i.e., $a_i$ is a spike of $a$), then $\ell(\partial_i a) = a_{i+1}$.
\end{assumption}

\begin{remark}
	This assumption translates to $\bar{\ell}$ as follows: if $b\mapsto \partial_i b$ is exceptional, then $\ell(\partial_ib)$ is a number in $b$ such that all the numbers between it and $\ell(b)-1$ belong to $b$.
\end{remark}

\begin{lemma}
	If $\sigma$ is a simplex in $\Theta(r)$ and $i\leq j$, consider the square
	\[\xymatrix @=2pc{
		&\sigma \ar[dl]_{\partial_i}\ar[dr]^{\partial_{j+1}} &\\
		\partial_i\sigma\ar[dr]_{\partial_j} && \partial_{j+1}\sigma\ar[dl]^{\partial_i}\\
		&\partial_j\partial_{i}\sigma &
	}\]
	and write $\circ$ if an arrow is standard and $\times$ if it is exceptional. Then the only possible configurations are
	\[\xymatrix @=1pc{
		&\ar[dl]_{\times} \ar[dr]^{\times}&& \\
		\ar[dr]_{\times} && \ar[dl]^{\circ} \\
		&&
	}
	\xymatrix @=1pc{
		&\ar[dl]_{\times} \ar[dr]^{\times}&& \\
		\ar[dr]_{\circ} && \ar[dl]^{\times} \\
		&&
	}
	\xymatrix @=1pc{
		&\ar[dl]_{\times} \ar[dr]^{\circ}&& \\
		\ar[dr]_{\circ} && \ar[dl]^{\times} \\
		&&
	}
	\xymatrix @=1pc{
		&\ar[dl]_{\circ} \ar[dr]^{\times}&& \\
		\ar[dr]_{\times} && \ar[dl]^{\circ} \\
		&&
	}
	\xymatrix @=1pc{
		&\ar[dl]_{\circ} \ar[dr]^{\circ}&& \\
		\ar[dr]_{\circ} && \ar[dl]^{\circ} \\
		&&
	}
	\]

	%\begin{assumption} If $\ell(a) = a_i$, then $\ell(\partial_i a) = a_{i+1}$
	%Suppose that for each $k\geq 0$ there is a function $\ell\colon Z_k(r)\to \{0,\ldots,k\}$ such that either
	%\[
	%\ell(\partial_i\sigma) \overset{*}{=}
	%\begin{cases}
	%\ell(\sigma) & \text{if $\ell(\sigma)<i$} \\
	%\ell(\sigma)-1 & \text{if $\ell(\sigma)\geq i$}
	%\end{cases}
	%\quad \text{or}\quad
	%\ell(\partial_i\sigma)\overset{**}{=}i-1.
	%\]
	%Additionally, suppose that $\ell$ is $C_{k+1}$-equivariant.
	%\end{assumption}
	%\begin{example}
	%$r=5$, define
	%\begin{align*}
	%\ell(1,4) &= 1 &
	%\ell(2,3) &= 1 &
	%\ell(1,1,3) &= 2 &
	%\ell(1,2,2) &= 2 &
	%\ell(1,1,1,2) &= 3
	%\end{align*}
	%and extend equivariantly.
	%\end{example}
	\begin{example}
		$r=5$, define
		\begin{align*}
			\ell(0,1) &= 0 &
			\ell(0,2) &= 0 &
			\ell(0,1,2) &= 0 &
			\ell(0,1,3) &= 0 &
			\ell(0,1,2,3) &= 0
		\end{align*}
		and extend equivariantly. For $r=7$ define $\ell$ to take value $0$ on the following simplices, and extend equivariantly:
		\begin{align*}
			(0) &&
			(0,1) &&
			(0,2) &&
			(0,3) \\
			(0,1,2) &&
			(0,1,3) &&
			(0,1,4) &&
			(0,2,3) &&
			(0,2,4) \\
			(0,1,2,3) &&
			(0,1,2,4) &&
			(0,1,3,4) &&
			(0,2,3,4) &&
			(0,1,3,5) \\
			(0,1,2,3,4) &&
			(0,1,2,3,5) &&
			(0,1,3,4,5) &&
			(0,1,2,3,4,5)
		\end{align*}
	\end{example}

	%\begin{itemize}
	%\item if $i$ and $j+1$ are non-leading spikes of $\sigma$, then all arrows are exceptional but one of the lower ones.
	%\item if $i$ and $j+1$ are not non-leading spikes of $\sigma$, then all arrows are standard.
	%\item if $i$ is a non-leading spike and $j+1$ is not, then....
	%\end{itemize}
\end{lemma}

\begin{lemma}
	There is an inclusion $\mathrm{spk}(\partial_i\sigma)\subset \mathrm{spk}(\sigma)$.
\end{lemma}

\begin{proof}
	Follows from the previous lemma
\end{proof}

\begin{lemma}
	If $T\subset \mathrm{spk}(\sigma)$ is a collection of spikes of $\sigma$, then there is exactly one spike $i$ such that $T\subset \mathrm{spk}(\partial_i\sigma)$. We call it the \emph{principal spike} of $(\sigma,T)$.
\end{lemma}

\begin{proof}
	Not so easy, should follow from the previous lemma.
\end{proof}

\subsection{A map $f$}

Here we construct a map $f$ that satisfies Condition \eqref{cond:1} but not necessarily condition \eqref{cond:2}. The maps we already have for $r=5,7$ can be constructed using this method.

\begin{definition}
	%The \emph{spikes} of $a$ (denoted $\mathrm{spk}(a)$) are the spikes of $q(a)$.
	A \emph{snippet} of $a$ is a sequence $v = (a_j,a_{j+1},\ldots,a_{j+m})$ of consecutive elements of $a$ such that $a_j$ is a spike, there are no other spikes in the sequence and $j+m\neq \ell(a)-1$. For that snippet, its \emph{slided snippet} $v^+$ is the sequence $(a_{j+1},a_{j+2},\ldots,a_{j+m+1})$.
\end{definition}

Define a map $f\colon \partial N_*(\Delta^{r-1}*\Delta^{r-1})\to \Sigma^{r-1}\partial N_*(\Delta^{r-1})$ as follows:
\[
f(a,b) =
\begin{cases}
	\{\ell(a)\}\cup v_0^+\cup v_1^+\cup\ldots \cup v_d^+ & \begin{array}{l} \text{if $\{0,1,\ldots,r\}\subset a\cup b$ and} \\ \text{$a\cap b$ is a union of snippets $v_0,v_1,\ldots,v_d$}\end{array} \\
	0 & \text{otherwise.}
\end{cases}
\]

\begin{lemma}
	If $\ell$ satisfies both assumptions, then $f$ is a map of chain complexes.
\end{lemma}

\begin{proof}
	Recall that the differential of $(a,b)$ is the sum of all possible removals of elements of $a$ and $b$. Observe first that removing an element from $a$ or $b$ that is not in $a\cap b$ yields a simplex $(a',b')$ that does not satisfy Condition \eqref{cond:711}, and therefore $f(a',b') = 0$.

	We will prove the lemma by induction, the base case being in dimension $r-1$, when $a\cap b = \emptyset$ and there are no snippets.

	Suppose first that we are in the first case, i.e.,
	\begin{align}\label{cond:711}
		&\{0,1,\ldots,r\}\subset a\cup b \\ \label{cond:712}
		&a\cap b\text{ is a union of snippets }v_0,v_1,\ldots,v_d.
	\end{align}

	Now, let $T(a,b)$ be the set of those spikes of $a$ that are in some snippet but not in any slided snippet.
	\begin{enumerate}
		\item Removing a highest element from a snippet $v$ in $b$ yields a simplex $(a,b')$ that satisfies both conditions and $f(a,b')$ results from removing the highest element of $v^+$ from $f(a,b)$.
		\item Removing any non-spike element from a snippet $v$ in $a$ yields a simplex $(a',b)$ that satisfies both conditions and $f(a',b)$ results from removing that element from $f(a,b)$.
		\item Removing the principal spike of $(a,T(a,b))$ from $a$ yields a simplex $(a',b)$ that satisfies both conditions and $f(a',b)$ results from removing $a_0$ from $f(a,b)$
	\end{enumerate}
	In the remaining cases, $f(a',b')=0$:
	\begin{enumerate}
		\item Removing a non-highest element $\hat{a}$ from a snippet $v$ in $b$ yields a simplex $(a,b')$ such that $a\cap b$ is not a union of snippets (the part of $v$ to the right of $\hat{a}$ is not a snippet and is not part of any other snippet).
		\item Removing a non-principal spike of $(a,T(a,b))$ from $a$ yields a simplex $(a',b)$ where some spike of $(a,\mathrm{spk}(a)\cap b)$ is not a spike of $a'$.
		\item Removing a spike of a snippet that does not belong to $T(a,b)$ yields a simplex whose highest element is contained in $a\cap b$ (hence cannot be part of any snippet).
	\end{enumerate}
	Suppose now that \eqref{cond:711} is not satisfied. Then removing any element of $a\cap b$ from $a$ or $b$ yields a simplex that does not satisfy \eqref{cond:711}.

	Finally, suppose that \eqref{cond:711} is satisfied, but not \eqref{cond:712}. Then removing any element of $a\cap b$ from $a$ or $b$ yields a simplex that does not satisfy \eqref{cond:711} unless $a\cap b$ is the result of removing from a union of snippets $v_0,\ldots,v_d$ the penultimate element $\hat{a}$ of some snippet. If $a' = a\smallsetminus \{\hat{a}\}$ and $b' = b\smallsetminus \{a\}$, then $(a',b)$ and $(b',a)$ are the only summands in the differential of $(a,b)$ that take non-zero value under $f$, and both take value $\{a_0\}\cup v_0^+\cup \ldots \cup V_d^+$.
\end{proof}

\begin{lemma}
	$f(\iota_1(a)) = a$.
\end{lemma}

\begin{proof}
	The elements in $\iota_1(a)$ consist on all pairs $(a,b)$ with $|b|=r-1$, hence $a\cap b$ consists on all elements of $a$ but one. Since the term $\ell(a)-1$ can never belong to $a\cap b$, there is only one such pair where $f$ is non-zero: $b = \{0,1,\ldots,r-1\}\smallsetminus \{\ell(a)-1\}$, and the value of $f$ on it is $a$.
\end{proof}

\begin{remark}
	There is no reason why $f(\iota_2(a))$ should equal $\rho(a)$. Our choices of $\ell$ for $r=5,7$ do satisfy all the assumptions, and do indeed satisfy Condition \eqref{eq:2}. Nonetheless, these are the only cases in which this construction works:
\end{remark}

\begin{lemma}
	The above construction cannot yield Condition \eqref{cond:2} if $r$ is a prime number bigger than 7 and smaller than 100.
\end{lemma}

\begin{proof}
	Let $a$ be the generator of $\Theta(r)_{r-2}$. Consider the set $Q=(i_0,\ldots,i_q)$ whose elements are
	\begin{itemize}
		\item the number of elements between consecutive spikes, where the right spike is not $\ell(\sigma)$.
		\item the number of elements minus one between consecutive spikes, where the right spike is $\ell(\sigma)$.
	\end{itemize}
	Then the number of possible combinations of snippets is $\prod_j{(i_j+2)}$.

	On the other hand, the number of simplices in $N_*(\partial\Delta^{r-1})$ of dimension at least $0$ up to cyclic ordering is $\frac{2^r-2}{r}$. If $r=5$, then the number of simplices is $6$, and in the example we have of $f$ for $r=5$, $Q=(2,3)$. If $r=7$, the number of simplices is $18$ and in the example we have of $f$ we have $Q=(2,3,3)$. For $r=11$, the quatity is $186 = 2*3*31$, hence there should be a snippet of length $29$, and this is impossible. The same issue happens for all higher primes up to 100.
\end{proof}

\subsection{Other methods}

Suppose that we have constructed already a map $f$. Then the map $f'$ defined as $f'(a,b) = f(b)$ is a chain map, and it satisfies condition \eqref{cond:2} if $f$ satisfies condition \eqref{cond:1}.

$\ell\colon \Theta(r)\to \{0,1,\ldots,r-1\}$ is a map that extends to the whole join. Then defining $\ell'(a) = \rho^{-1}(\bar{\ell}(a))$ we obtain a new map that extends. If $\ell$ satisfies condition \eqref{cond:1}, then $\ell'$ satisfies condition \eqref{cond:2}.

$\Theta(r)$ has an automorphism $\varphi(x)= -x$ that reverses the cyclic order: $(0,1,3)\mapsto (0,4,2)$. Precomposing with that automorphism yields new $\ell$'s.

In the examples above: If $r=5$, then $\ell = \rho^{-1}(\bar{\ell}(\varphi))$. If $r=7$, then $\ell$ and $\bar{\ell})$ are not related in this way. In fact, $\rho^{-1}(\bar{\ell})$ yields another solution that does not satisfy the assumptions made.

Hence one should try to relax the assumptions.

\begin{remark}
	One could be tempted to ask at the same time for the second assumption both on $\ell$ and $\bar{\ell}\circ \varphi$. I have found that there is no such $\ell$ even for $r=7$.
\end{remark}
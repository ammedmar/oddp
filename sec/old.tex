% !TEX root = ../oddp.tex

\section{Odd prime operations}

In the first section we construct a map from the normalized chain complex of the Milnor resolution to the minimal resolution of $EC_r$.
In the second section we construct a $C_r$-equivariant map $W_*(r)\otimes N_*(\Delta^n)\to N_*(\Delta^n)^{\otimes r}$.

All chain complexes are augmented. Chain complexes are taken with generic ring coefficients unless specified. The linear dual of a chain complex $C_*$ is denoted $C^*$. The boundary of a chain complex is denoted $\partial$, the coboundary of a cochain complex is denoted $\delta$ and the face maps of a simplicial set are denoted $d_i$. The linear dual of a map $f$ or a vector $v$ is denoted by $f^\vee$ and $v^\vee$, though we will drop the superscript when the context allows. Chain complexes and cochain complexes are positively graded, though we understand a positively graded cochain complex as a negatively graded chain complex. All the (co)chain complexes in this work come with a prescribed basis, and since they are of finite type, this basis can be used to generate the dual complex.

We will construct four maps from the dual of the minimal resolution of $C_r$. Sometimes it will be convenient to treat their linear duals: It is easier to construct a map $W^*(r)\to N^*(EC_r)$ because $W^*(r)$ is free on one object in each dimension, but it is easier to construct maps $N_*(EC_r)\to N_*(EC_r)$ because the differential in chains is easier than the differential in cochains.

\subsection*{To be done}
\begin{itemize}
	\item The first section we suspend in the left, in the second section we suspend in the right. What is the best convention?
	\item Words are called $A$ in the first section and $w$ in the second section.
	\item I've added a section ``other considerations'' with possible comments to be made around the main theorem.
\end{itemize}

\subsection*{Recent changes}
\begin{itemize}
	\item One has to take $C^*(EC_r)$ instead of $N^*(EC_r)$ because that is what is needed in $N^*(\Delta^n\times EC_r)$.
	\item The case $r=3$ was badly written, I've corrected it.
	\item The whole proof has been rewritten.
\end{itemize}

\section{The minimal resolution of $C_r$ and the Milnor construction}
Let $r$ be an odd number, let $C_r$ be the cyclic group of order $r$ and let $\rho$ be a generator.

\subsection{Cyclic resolutions}
Recall the augmented minimal resolution $W_*(r)$ of the cyclic group $C_r$ by $R$-modules:
\[R[C_r]\lra\ldots \lra R[C_r]\overset{T}{\lra} R[C_r]\overset{N}{\lra}R[C_r]\overset{T}{\lra} R[C_r] \overset{N}{\lra} R.\]
The generator $\rho$ in degree $k\geq -1$ will be denoted by $e_k$ (here $e_{-1}=1$), and
\begin{align*}
	N(e_k) &= \sum_{i=0}^{r-1} \rho^ie_{k-1} &
	T(e_k) &= (1-\rho)e_{k-1}
\end{align*}

Recall the Milnor construction applied to the cyclic group $C_r$, which yields a contractible augmented simplicial set with a free $C_r$-action:
\begin{align*}
	(EC_r)_k &= C_r^{k+1}\\
	d_i(a_0\ldots a_k) &= (a_0\ldots\hat{a}_i\ldots a_k) \\
	s_i(a_0\ldots a_k) &= (a_0\ldots a_i,a_i\ldots a_k)
\end{align*}
The $i$th coface $d^i\colon C^*(EC_r)\to C^{*+1}(EC_r)$ is given by inserting a number $b\in \{0,\ldots,r-1\}$ in $i$-th position in all possible ways (some of them yield degenerate cochains). Write $d^{i,b}(a_0\ldots a_n) = (a_0\ldots a_{i-1},b,a_i,\ldots, a_n)$, so that
\[\delta(a_0\ldots a_n) = \sum_{0\leq i\leq n}\sum_{0\leq b\leq r-1}(-1)^i \delta^{i,b}(a_0\ldots a_n).\]
\begin{warning} Observe that $\rho$ acts on the dual complex as $\rho^{-1}$.
\end{warning}
There are \emph{suspension maps}
\begin{align*}
	\theta_{r-1}\colon W_*(r) & \lra  \Sigma^{r-1}W_*(r) \\
	e_k & \longmapsto e_{k-r+1} \\
	%   S\colon \Lambda(r)^*&\lra \Sigma^{r-1} \Lambda(r)^* \\
	%   (a_0,\ldots,a_k) & \longmapsto \sum_{(b_{0},\ldots,b_{r-1})\in \Perm_r} (-1)^{\sigma(a_k\ldots a_{k+r-1})} %(a_0,\ldots,a_{k},a_{k+1},\ldots,a_{k+r-1})
	S\colon  C_*(EC_r)&\lra \Sigma^{r-1}C_*(EC_r) \\
	(a_0,\ldots,a_k) & \longmapsto \begin{cases}
		{\sgn(a_{0},\ldots,a_{r-1})} (a_{r-1},\ldots,a_{k}) & \text{ if $k\geq r-1$} \\
		\emptyset & \text{if $k = r-2$ and $a_i\neq a_j$ for all $i,j$} \\
		0 & \text{otherwise.}
	\end{cases}
\end{align*}
\begin{remark}
	For any even integer $k$, the suspension $\theta_{k}\colon W_*(r)\to \Sigma^{k}W_*(r)$ is a homomorphism of chain complexes.
\end{remark}
%The following lemma is very similar to its version for the surjection operad (whose complex in arity $r$ is a quotient of $\Lambda(r)_*$).
\begin{lemma}
	The suspension $S\colon C_*(EC_r)\to \Sigma^{r-1}C_*(EC_r)$ is a homomorphism of chain complexes.
\end{lemma}
\begin{proof}
	Consider first a simplex $(a_0,\ldots,a_k)$ of dimension $k\geq r$. If $A=(a_{0},\ldots,a_{r-1})$ has no repeated entries (i.e., it is a permutation), we choose the unique $j<r-1$ such that $a_j = a_r$. Let $A' = (a_{0},\ldots,\hat{a}_j,\ldots,a_r)$, and observe that the permutations $A$ and $A'$ differ by $r-j-1$ transpositions, hence $\sgn(A') = (-1)^{r-j+1}\sgn(A) = (-1)^j\sgn(A)$.
	\begin{align*}
		S\left(\sum_{i=0}^k (-1)^id_i (a_0,\ldots,a_k)\right)
		&= \sum_{i=0}^{r-1} (-1)^iS(d_i (a_0,\ldots,a_k)) + \sum_{i=r}^{k} (-1)^iS(d_i (a_0,\ldots,a_k))
	\end{align*}
	All the terms in the first summation on the RHS are zero but the one indexed by $j$, because $(a_0,\ldots,\hat{a}_i,\ldots,a_r)$ is not a permutation in those cases. Additionally, $Sd_i = d_{i-r+1}S$ in each term in the second summation on the RHS, hence we have:
	\begin{align*}
		S\left(\sum_{i=0}^k (-1)^id_i (a_0,\ldots,a_k)\right)
		&= (-1)^jS(d_j (a_0,\ldots,a_k)) + \sum_{i=1}^{k-r+1} (-1)^i d_i S(a_0,\ldots,a_k)\\
		&= (-1)^{j}\sgn(A')(a_r,\ldots,a_k) + \sum_{i=1}^{k-r+1} (-1)^i d_i S(a_0,\ldots,a_k)\\
		&= (-1)^{0}\sgn(A)(a_r,\ldots,a_k) + \sum_{i=1}^{k-r+1} (-1)^i d_i S(a_0,\ldots,a_k)\\
		&= \sum_{i=0}^{k-r+1} (-1)^i d_i S(a_0,\ldots,a_k)
	\end{align*}
If $A$ has repeated entries, then $S(a_0,\ldots,a_k) = 0$. We consider two cases: if the repeated entries of $A$ are different from $a_k$ or there are more than two repeated entries, then all summands in $d(a_0,\ldots,a_k)$ have repeated entries in its first $r$ positions, therefore $S(dA)=0$. If there are exactly two repeated entries $a_j=a_m = a_r$ with $j,m<r$, let $A' = (a_0,\ldots,\hat{a}_j,\ldots,a_{r})$, $A'' = (a_0,\ldots,\hat{a}_m,\ldots,a_{r})$, and observe that these two words differ by $|m-j-1|$ transpositions, hence $\sgn(A') = (-1)^{m-j-1}\sgn(A'')$.
\begin{align*}
	S\left(\sum_{i=0}^k (-1)^id_i (a_0,\ldots,a_k)\right)
	&= \sum_{i=0}^{r-1} (-1)^iS(d_i (a_0,\ldots,a_k)) + \sum_{i=r}^{k} (-1)^iS(d_i (a_0,\ldots,a_k)) \\
	&= (-1)^jS(d_j (a_0,\ldots,a_k)) + (-1)^mS(d_m (a_0,\ldots,a_k)) \\
	&= \left((-1)^{j}\sgn(A') + (-1)^m\sgn(A'')\right)(a_r,\ldots,a_k)\\
	&= 0
\end{align*}
The case $k=r-1$ is left to the reader.
\end{proof}

\subsection{A comparison homomorphism} Here we introduce a homomorphism between the previous two cochain complexes.

\begin{definition} For each $0\leq k<r-1$, let $L_k\subset C_k(EC_r)$ be the set of those increasing words $(a_0\ldots a_k)$ such that $a_{2i}$ is even and different from $r-1$, and $a_{2i+1}$ is odd. Define
\begin{equation}\label{eq:111}
	\psi_k = \sum_{(a_0,\ldots,a_k)\in L_k}\sum_{\sigma\in \Sigma_{k+1}} \sgn(\sigma)(a_{\sigma(0)},\ldots,a_{\sigma(k)})\in C^k(EC_r)
\end{equation}
For each $k = \ell (r-1) + s\geq r-2$ with $s<r-1$, define
\begin{equation}\label{eq:112}
	\psi_{k} = (\dual{S})^\ell(\psi_s)\in C^k(EC_r).
\end{equation}
\end{definition}
\begin{notation} If $A= (a_{\sigma(0)},\ldots,a_{\sigma(k)})$, we will sometimes denote the sign of the permutation $\sigma$ as $\sgn(A)$. If $a$ is a non-negative number, write $\tilde{a} = \left\lfloor \frac{a}{2}\right\rfloor$.
\end{notation}
Define the $C_r$-equivariant map $\Psi\colon W^*(r;\bZ[\frac{1}{\tilde{r}!}])\lra C^*(EC_r; \bZ[\frac{1}{\tilde{r}!}])$ as
\[e_k\mapsto \frac{(\tilde{r}-\tilde{s}-1)!}{(\tilde{r}!)^{\ell+1}}\psi_k\]



\begin{theorem}
$\Psi$ is a homomorphism of cochain complexes.
\end{theorem}
%If $r>3$ and $R\neq \bZ_r$, then the $C_r$-equivariant map $\Psi\colon W^{\leq m(r-1)}(r)\lra N^*(EC_r)_{\leq m(r-1)}$ given by $e_k\mapsto (\tilde{r}!)^{m-\ell-1}(\tilde{r}-\tilde{s}-1)!\psi_k$ is a homomorphism of cochain complexes.
\begin{remark} If $r$ is prime, the coefficient $\tilde{r}!\mod r$ is, by Wilson's theorem, one of the two values of $\sqrt{(-1)^{\tilde{r}+1}}$, though determining which of them is not immediate\footnote{\url{https://mathoverflow.net/questions/121678/the-value-pm-1-for-the-square-root-of-wilsons-theorem-p-1-2-mod-p}}.
\end{remark}
\begin{remark} If $r=3$, then $\bZ[\frac{1}{\tilde{r}!}] = \bZ$ and if $r$ is prime, then $\bZ/r$ is a $\bZ[\frac{1}{\tilde{r}!}]$-algebra, hence these coefficients are also allowed.
\end{remark}

\begin{proof}
We need to prove that
\begin{align*}
	N\psi_{k} & = (\tilde{r}-\tilde{s})\delta(\psi_{k-1}) & \text{ if $k$ is even} \\
	T\psi_{k} &= d(\psi_{k-1}) & \text{ if $k$ is odd.}
\end{align*}
For the case $k\leq r-2$ we need to check that
\begin{align*}
	\sum_{A\in L_{k-1}}\sum_{\sigma\in \Sigma_{k-1}}\sum_{0\leq b\leq r-1}(-1)^i\sgn(A)\delta^{i,b}A &= \bar{\psi}_k. & \text{ if $k$ is even,}\\
	\sum_{A\in L_{k-1}}\sum_{\sigma\in \Sigma_{k-1}}\sum_{0\leq b\leq r-1}(-1)^i\sgn(A)\delta^{i,b}A &= T\psi_k & \text{if $k$ is odd.}
\end{align*}
This will be proved in Lemmas \ref{lemma:pair} and \ref{lemma:cancel}. In higher degrees we have, by induction  ($\lambda = \tilde{r}-\tilde{s})$,
\begin{align*}
	N\psi_{k} &= N\dual{S}\psi_{k-r+1} = \dual{S} N\psi_{k-r+1} = \lambda\dual{S} d\psi_{k-r} = \lambda d\dual{S} \psi_{k-r} = \lambda d\psi_{k-1}\\
	T\psi_{k} &= T\dual{S}\psi_{k-r+1} = \dual{S} T\psi_{k-r+1} = \dual{S} d\psi_{k-r} = d\dual{S} \psi_{k-r} = d\psi_{k-1}.
\end{align*}
This also sets the case when because both definitions of $\psi_{r-2}$ agree.
\end{proof}



%If $k<r$, the \emph{shape} of such a simplex $(a_0,\ldots,a_k)$ is the cyclic set of cyclic differences $\{a_i - \bar{a}_{i-1}\}_{i=0}^k$. By this we mean that two shapes are equivalent if they differ by a cyclic permutation.
%\begin{lemma} If $k<r$ is even, the cyclic orbit of each summand of $\psi_k$ has $(\tilde{r}-\tilde{k}-1)$ elements in $\psi_k$.
%\end{lemma}
%\begin{proof}
%Since $(a_0,\ldots,a_k)$ has $\tilde{k}+1$ elements, this is the number of cyclic permutations that contain $r-1$. Since there are $\tilde{r}$ cyclic permutations in total, the result follows.
%\end{proof}



A number $b$ is \emph{friend} of a word $A$ of $\psi_k$ under any of the following conditions:
\begin{itemize}
\item $b>A$ and $b$ has the same parity as $a_{k}$,
\item $b<A$ and $b$ is odd,
\item $b=r-1$.
\end{itemize}
A number is \emph{enemy} of a word $A$ of $\psi_k$ if it is not its friend. Write $b\vdash A$ if $b$ and $A$ are friends and $b\not\vdash A$ if they are enemies.

Say that a word $A = (a_{\sigma(0)},\ldots,a_{\sigma(k)})\in N_k(EC_r)$ is \emph{even} if it b (\emph{odd}) if $a_{2i}$ is even (odd) and $a_{2i+1}$ is odd (even). A word is \emph{good even} if it is even and $r-1\notin A$. A word $A$ is \emph{good odd} if $\rho^{-1}(A)$ is even good, i.e., if $r-2\notin A$ and either $A$ is odd, or $r-1\in A$ and $A\smallsetminus \{r-1\}$ is even. Observe that the summands of $\psi_k$ are precisely on the good even words, and that if $k$ is odd, then all even words are good even. Let $\bar{\psi}_k$ be the sum of all even and odd words in $\Lambda(r)_k$ with their permutation signs.
%For the next lemma, observe that if $A$ is even, then $\rho^b(A)$ is either even or odd. Observe also that, if $k$ is even, removing the minimum of an odd sequence, or removing the maximum of an even sequence yields a summand of $\psi_{k-1}$.
\begin{lemma}\label{lemma:simp} If $k\leq r-2$ is even, then $N\psi_k = (\tilde{r}-\tilde{k})\bar{\psi}_k$. If $k$ is odd, then $T\psi_k$ is the sum of all good odd words minus the (good) even words with their permutation signs.
\end{lemma}
\begin{proof}
Suppose first that $k$ is even. We will first establish that every summand in $N\psi_k$ is a summand in $\bar{\psi}_k$. Then, we will prove that every summand in $\bar{\psi}_k$ appears exactly $(\tilde{r}-\tilde{k})$ times in $N\psi_k$, with the same signs.

If $A$ is even and $0\notin A$ ($0\in A$), then $\rho(A)$ is odd (even), while if $A$ is odd, then $\rho(A)$ is even,. This has two consequences: first, since the summands in $\psi_k^0$ are good even summands, every summand in $N\psi_k^0$ is either even or odd; second, if $A$ is an even (odd) word, then $\rho^{b+1}(A)$ is a summand in $\psi_k$ if and only if $b\notin A$ (this condition assures that $r-1\notin \rho^{b+1}(A)$) and has opposite parity to $|\{a\in A\mid a<b\}|$ (this assures that $\rho^{b+1}(A)$ is even). There are $r-k-1$ many summands satisfying the first condition, and half of them satisfy the second condition, hence there are exactly $\tilde{r}-\tilde{k}$ such $b$'s, therefore the first statement follows. Regarding the sign, observe that if $A$ is even, then the permutation that reorders $\rho(A)$ is either the same as the permutation $\sigma$ that reorders $A$, or it is the result of precomposing $\sigma$ with a cyclic permutation of $\Sigma_k$, hence its sign is that of $\sigma$. If $A$ is odd, then the permutation that reorders $\rho(A)$ is $\sigma$.

The second statement follows by inspection.
%
%The second statement follows because there is a summand in $\psi_k^0$ for each orbit of $C_r$ in $\psi_k$.
%
%On the other hand, if $A$ is an even (odd) word such that either $0\notin A$ or $r-1\notin A$, then $\rho^{\mi}A\in \psi_0$. If both $0\in A$ and $r-1\in A$, then choose two consecutive values $b,b+1\notin A$. Then either $\rho^{b+1}(A)$ or $\rho^{b+1}(A)$ is a summand in $\psi_k$, and therefore a summand in $\psi_k^0$, by the previous lemma.
\end{proof}

\newcommand{\mi}{{\sigma^{-1}(0)}}
\newcommand{\ma}{{\sigma^{-1}(k)}}

\begin{lemma}\label{lemma:pair} If $k\leq r-2$ is even,
\begin{align*}
	\sum_{A\in L_{k-1}}\sum_{\sigma\in \Sigma_{k-1}}\sum_{b\vdash A}(-1)^i\sgn(A)\delta^{i,b}A &= \bar{\psi}_k.
\end{align*}
If $k$ is odd,
\begin{align*}
	\sum_{A\in L_{k-1}}\sum_{\sigma\in \Sigma_{k-1}}\sum_{b\vdash A}(-1)^i\sgn(A)\delta^{i,b}A &= T\psi_k.
\end{align*}
\end{lemma}
\begin{proof}
Since each summation has no repeated summands, it will be enough to prove that every summand in the left appears in the right with the same sign, and viceversa.

%Observe first that the summands in $N\psi_k^0$ and $T\psi_k$ are the precisely the even and the odd words. Moreover, when $k$ is odd, the summands in $\psi_k$ are the even words, while the summands in $\rho\psi_k$ are the odd words.

Let $k$ be even. Since $b$ is friend of $A$, and $A$ is good even, the word $\delta^{i,b}A$ is either even or odd (depending on whether $b>A$ or $b<A$).  Hence the summands in the left appear as summands in the right. Reciprocally, if $A$ is an even (odd) word, then removing its maximal (minimal) element $b=a_{\ma}$ ($b=a_{\mi}$) yields a good even word in $\psi_{k-1}$, hence $A = \delta^{\ma,b}A\smallsetminus\{b\}$ ($A = \delta^{\mi,b}A\smallsetminus\{b\}$).

Let $k$ be odd. Since $b$ is friend of $A$ and $A$ is good even, the word $\delta^{i,b}A$ is again even or odd (depending on whether $b>A$ or $b<A$). If $b<A$, since $A$ is even, we have that $\max \delta^{i,b}(A) = \max A \leq r-3$, hence $\delta^{i,b}A$ is good. If $b>A$, then either $b$ is odd, in which case $\delta^{i,b}$ is good even, or $b=r-1$, in which case removing $r-1$ from $\delta{i,b}A$ yields the good even word $A$, hence $A$ is good odd. Reciprocally, if $A$ is a good even (odd) word, then removing its maximal (minimal) element $b=a_{\ma}$ ($b=a_{\mi}$) yields a good even word in $\psi_{k-1}$, hence $A = \delta^{\ma,b}A\smallsetminus\{b\}$ ($A = \delta^{\mi,b}A\smallsetminus\{b\}$).

%If $b>A$, as $b$ is odd, it is different from $r-1$, hence $\delta^{i,b}A$ is good. If $b<A$, then
%%
%%if $A = (a_{\sigma(0)},\ldots,a_{\sigma(k)})$ is a summand in the right (i.e., an even or odd word), then
%%\[
%%\rho^{b}(A) = \begin{cases}
	%%\delta^{\sigma(s),a_{\sigma(s)}}\left(A\smallsetminus \{a_{\sigma(s)}\}\right) & \text{if $A$ is odd and $a_{\sigma(s)} = \min A$}, \\
	%%\delta^{\sigma(s),a_{\sigma(s)}}\left(A\smallsetminus \{a_{\sigma(s)}\}\right) & \text{if $A$ is even and $a_{\sigma(s)} = \max A$},
	%%\end{cases}
	%%\]
	%%with $A\smallsetminus \{a_{\sigma(s)}\}\in \psi_{k-1}$ because it is even and does not contain $r-1$.

	%Moreover, if $k$ is odd, then $\delta^{i,b}A$ is always good.

	Regarding the signs, observe that the sign of the summand in the left indexed by $A = (a_{\sigma(0)},\ldots,a_{\sigma(k-1)})$, $b$ and $i$ is $(-1)^i\sgn(A)$ while the sign of the summand $\delta^{i,b}A$ in the right is computed as follows:
	\[
	\sgn(\delta^{i,b}A) = \begin{cases}
		(-1)^i\sgn(\delta^{0,b}A) = (-1)^i\sgn(A) &\text{ if $b<A$,}\\
		(-1)^{k-i}\sgn(\delta^{k,b}A) = (-1)^k(-1)^i\sgn(A) &\text{ if $b>A$,}
	\end{cases}
	\]
	where the first equality follows because $\delta^{i,b}A$ and $\delta^{0,b}A$ (\resp $\delta^{k,b}A$) differ by $i$ (\resp $k-i$) transpositions, and the second equality follows because the permutation that reorders $\delta^{0,b}A$ (\resp $\delta^{k,b}A$) fixes the first (last) entry, and because $k$ is even.

	Hence, if $k$ is even, then all summands in the left appear with the sign $\sgn(\delta^{i,b}A)$, as in the right, while if $k$ is odd, the even summands appear with opposite sign and the odd summands with the same sign.
\end{proof}
\begin{lemma}\label{lemma:cancel} Let $k\leq r-2$.
	\begin{align*}
		\sum_{A\in L_{k-1}}\sum_{\sigma\in \Sigma_{k-1}}\sum_{b\not{\vdash} A}(-1)^i\sgn(A)\delta^{i,b}A &= 0.
	\end{align*}
\end{lemma}
\begin{proof}  Let $A$ be a summand of $\psi_k$ and let $0\leq b\leq r-1$ be enemy of $A$. Then, if $b >\min(A)$ and $b<\max(A)$, then both $\max\{a\in A\mid a<b\}$ and $\min\{a\in A\mid a>b\}$ are well defined and have different parity, hence one of them ($c$) has the same parity as $b$. If $b<\min(A)$ or $b>\max(A)$, we let $c = \min(A)$ or $c = \max(A)$ respectively. Let $j$ be the position of $c$ in $A$  and let $A' = (\delta^{i,b} A)\smallsetminus \{c\}$. Then $c$ is enemy of $A'$ and $\delta^{j,c}A = \delta^{i,b}A'$. Let us show that they appear with opposite signs: The permutation that reorders $A'$ differs from the permutation that reorders $A$ by $|j-i-1|$ transpositions, therefore
	\[(-1)^{i}\sgn(A) = (-1)^{i}(-1)^{j-i-1}\sgn(A') = -(-1)^j\sgn(A').\]
	Finally, observe that $A'' = A$, hence the summands cancel in pairs.
\end{proof}





\section{Some cochain complexes}
There is a Poincaré duality isomorphism
\[PD\colon N_*(\Delta^n)\lra N^{n-1-*}(\Delta^n)\]
that sends a simplex to its complement in the maximal simplex.


\subsection{Models of $N_*(\Delta^n)^{\otimes r}$}
Consider the functor tensor product $F\otimes_{\Delta}G$ of the following two functors:
\begin{align*}
	F\colon \Delta&\lra \Ch(R)&  G\colon \Delta^{\op}&\lra \Ch(R)\\
	F([m])&=\Sigma^{(r-1)m-r}(N^*(\Delta^m)^{\otimes r})&  G[m] &= N_m(\Delta^n).
\end{align*}
\begin{lemma} The map $\alpha\colon F\otimes_{\Delta} G\lra N_*(\Delta^n)^{\otimes r}$ given by
	\[\alpha((U_1,\ldots,U_r)\otimes \tau) = (d_{U_1}^\tau(\tau),\ldots, d_{U_r}^\tau(\tau))\]
	is a  $C_r$-isomorphism of chain complexes.
\end{lemma}
Consider now the chain complex $\bar{\Omega}_*(r,n)$ that in degree $k$ is generated by equivalence classes of triples $(U,w,\tau)$ where \fcnote{This is actually a quotient of a functor tensor product. Maybe it is worthy defining it like that.}
\begin{itemize}
	\item $U$ is a generator of $C_{(r-1)m-1-k}(\Delta^m)$,
	\item $w$ is a generator of $C_{(r-1)m-1-k}(EC_r)$ (a word),
	\item $\tau$ is a generator of $N_m(\Delta^n)$
\end{itemize}
such that
\begin{itemize}
	\item $(U,w)$ is a non-degenerate simplex in $EC_r\times \Delta^m$
\end{itemize}
subject to the following relations:
\begin{itemize}
	\item (Inner reorderings) $(U,w,\tau)\sim (-1)^{|\sigma|}(U,w',\tau)$ if there is an interval $u_{i-1}<u_i =\ldots u_{i+k}<u_{i+k+1}$ such that the sequences $(w_i,\ldots,w_{i+k})$ and $(w'_i,\ldots,w'_{i+k})$ differ by a permutation $\sigma$ and $w$ and $w'$ agree outside that interval. Notice that if all elements of $U$ are different, then $w=w'$.
	\item (Full piece removal) $(U,w,d_i(\tau))\sim (-1)^r(D_i(U,w),\tau)$, where $D_i(U,w) = (U',w')$ and
	\begin{align*}
		U'_j &= \begin{cases} U_j &\text{ if $j<i$} \\ i & \text{ if $i\leq j < i+r$} \\ U_{j-r} + 1 & \text{ if $j\geq i+r$.}\end{cases} &
		w'_j &= \begin{cases} w_j &\text{ if $j<i$} \\ w_{j-r} - 1 & \text{ if $j\geq i+r$.}\end{cases}
	\end{align*}
	and $(w'_{i},w'_{i+1},\ldots,w'_{i+r-1}) = (0,1,\ldots,r-1)$ (a \emph{full piece}).
\end{itemize}
This complex has a differential defined as follows: The full piece removal defines an order relation on each equivalence class. Then $\partial([(U,w,\tau)]) = [\partial(\bar{U},\bar{w},\tau)]$ where $(\bar{U},\bar{w},\tau)$ is a maximal representative of $[(U,w,\tau)]$ (in order to compute it it is enough to consider representatives that have a single full piece).

\begin{lemma} The map $\beta'\colon \bar{\Omega}^*(r,n)\lra \bar{\Omega}^*(r,n)$ given by sending $(U,w,\tau)$ to $(U,w',\tau)$, where $w_i' = u_i+w_1\mod r$ is a $C_r$-equivariant isomorphism of chain complexes.
\end{lemma}

Given a generator $[(U,w,\tau)]$, define for each $0\leq i\leq r-1$, $U_w^i \subset U$ as the subset of those entries $u_j$ such that $w_j=i$. Define the word $w_U$ as $(\omega_U)_j = \omega_j$ and $\sigma(U,w)$ as the sign of the permutation that arranges $\omega_U$ in ascending order. \fcnote{This map may be further factored?}
\begin{lemma} The map $\beta\colon \bar{\Omega}^*(r,n)\lra F\otimes_\Delta G$ given by
	\[\beta([U,w,\tau]) = (-1)^{\sigma(U,w)}(U_w^0\otimes\ldots\otimes U_w^{r-1})\otimes \tau\]
	is a $C_r$-equivariant isomorphism of chain complexes.
\end{lemma}








\subsection{Models of $W_*(r)\otimes N_*(\Delta^n)$}
Recall that $W_*(r)$ is the augmented minimal resolution of $C_r$.
\begin{definition} If $k\geq -1$ is odd, define the \emph{flip map} $\varphi_k\colon W_*(r)\to W^{k-*}(r)$ as $e_q\mapsto e^{\vee}_{k-q}$. If $k$ is an even integer, define the \emph{suspension map} $\theta_k\colon W_*(r)\to W_{*-k}(r)$ as $\theta_k(e_q) = e_{q-k}$.
\end{definition}
Let $r$ be odd. Define the graded complex
\[V^{*}(r) = \bigoplus_{m}W^{(r-1)m-1-*}(r)\otimes N_m(\Delta^n)\]
with the differential $d(e^\vee_q\otimes \tau) = d(e^\vee_q)\otimes \tau + (-1)^q \theta_{r-1}(e^\vee_q)\otimes d(\tau)$%\fcnote{the sign works even better with $(-1)^n$. Maybe the other order is better?}. Define a $C_r$-equivariant homomorphism of chain complexes
\[\varphi\colon W_*(r)\otimes N_*(\Delta^n)\lra V^{*}(r)\]
as $e_{q}\otimes \tau\mapsto \varphi(e_q)\otimes \tau$.

\section{A higher cup-product map}
In this section we build, from a certain equivariant map between spheres, a chain homomorphism with coefficients in $\bZ[\frac{1}{\tilde{r}!}]$.
\[\Psi\colon V^*(r)\to \bar{\Omega}^*(r,n)\]
Let $\Omega_*(r,m)$ be generated in degree $q$ by the equivalence classes of pairs $(U,w)$ such that $w$ has at most one full piece, for the relation generated by inner reorderings and standarizations. The differential $\sum_{i} (-1)^i(d_iU,d_iw)$ is well-defined in classes. $\Omega_*(r,m)^{nf}$ is the subcomplex of words without full pieces and $\Omega_*(r,m)^{f}$ is the submodule of words with full pieces.

Let $\Phi\colon \Omega_*(r,m)^f\to \Sigma^{r}\Omega_{*}(r,m-1)^{nf}$ be the (non-chain) homomorphism that sends a pair $(U,w)$ to the result of removing a full piece if it is possible and decrease all entries in $U$ to the right of the piece, and increase all entries in $w$ to the right of the piece.
\begin{remark} For the $D_i$ defined in the ``full piece removal'' condition, $\sum_iD_i(U,w) = \Phi^{\vee}(U,w)$ if $\tau$ has dimension $m$.
\end{remark}


\begin{lemma}\label{lemma:omegarm} Suppose one is given a family of $C_r$-equivariant maps
	\begin{align*}
		\hat{\Psi}\colon \Omega_*(r,m)^{nf}&\lra W_*(r)
	\end{align*}
	related by the following equation:
	\begin{align}\label{it:1a}
		\hat{\Psi}_{q-1}((\partial (U,w)^{nf}) &= \hat{\Psi}_{q-r}\Phi(U,w) &\text{if $(U,w)$ has a full piece}
	\end{align}
	%\begin{enumerate}
	%\item\label{it:1} $\hat{\Psi}_q\partial = \partial \hat{\Psi}_q + \hat{\Psi}_{q-r}\Phi$ !!!
	%\item\label{it:1a} $\hat{\Psi}_{q-1}((\partial (U,w)^{nf}) = \hat{\Psi}_{q-r}\Phi(U,w)$ if $w$ has a full piece.
	%\item\label{it:2a} $\hat{\Psi}_{q-1}((\partial (U,w)) = \partial \hat{\Psi}_{q}( (U,w))$ if $w$ has no full pieces.
	%\item\label{it:2} $\Phi(S_m(U,w)) = S_{m}(\Phi(U,w))$
	%\item\label{it:3} $\hat{\Psi}S_{m-1}\partial = \hat{\Psi}\partial S_m$ if $w$ has no full pieces
	%\item\label{it:4} $\hat{\Psi}_{r-2} = \theta_{r-1}\hat{\Psi}_{-1}S_{r-2}$
	%\end{enumerate}
	%Then $\hat{\Psi}$ extends recursively to the whole $\Omega_*(r,m)^{nf}$ as
	%\[
	%\hat{\Psi}_q =\begin{cases}
		%\hat{\Psi}_q & \text{if $q\leq r-2$} \\
		%\theta_{r-1}\hat{\Psi}_{q-r+1}S & \text{if $q\geq r-2$}.
		%\end{cases}
		%\]
		%and satisfies \eqref{it:1} and \eqref{it:3}, and
		Then there is a chain homomorphism
		\[\Psi\colon V^*(r)\lra \bar{\Omega}^*(r,n)\]
		defined as $\Psi(e_q\otimes \tau) = \hat{\Psi}^\vee(e_q)\otimes \tau$.
	\end{lemma}
	\begin{proof}
		\begin{align*}\delta\hat{\Psi}^{\vee}(e_q)\otimes \tau
			&= (\delta\hat{\Psi}^{\vee}(e_q))^{nf}\otimes\tau + (\delta\hat{\Psi}^{\vee}(e_q))^f\otimes\tau \\
			&= \hat{\Psi}^{\vee}(\delta e_q)\otimes \tau + \Phi^\vee\check{\Psi}^\vee_{q-r}(e_{q-r})\otimes\tau \\
			&= \hat{\Psi}^{\vee}(\delta e_{q})\otimes \tau + \sum_i D_i(\check{\Psi}^\vee_{q-r}(e_{q-r}))\otimes \tau\\
			&= \hat{\Psi}^{\vee}(\delta e_q)\otimes \tau + \sum_i \check{\Psi}^\vee_{q-r}(e_{q-r})\otimes d_i\tau \\
			&= \Psi(\delta(e_q\otimes \tau))\qedhere
		\end{align*}
	\end{proof}

	A \emph{pieced word} $w$ is a word in $EC_r$ together with a decomposition of $w$ into pieces of length $\leq r$. Define the following relation on them:
	\begin{itemize}
		\item (Inner reorderings) $w\sim (-1)^{|\sigma|}w'$ if all entries of $w$ and $w'$ coincide, except for a single piece where they differ by a permutation $\sigma$.
		%\item (Full piece removal) $w\sim (-1)^rw'$ if $w'$ is obtained from $w$ by removing a piece of length $r$ (a \emph{full piece}).
		\item (Standarization) $w\sim w'$ if $w'$ is obtained from $w$ by removing a piece of length $0$.
	\end{itemize}
	The complex $\Omega_*(r)$ is generated in degree $q$ by the equivalence classes of words of length $q+1$ and at most one full piece.
	%Up to inner reorderings, every class $[w]$ has a maximal representative $\bar{w}$.
	%Define $\partial([w]) = [\partial(\bar{w})]$.
	The differential $\partial(w) = \sum_{i} (-1)^id_iw$ is well-defined in classes. $\Omega_*(r)^{nf}$ is the subcomplex of words without full pieces and $\Omega_*(r)^{f}$ is the submodule of words with full pieces.



	If $(U,w)$ is a generator of $\Omega_*(r,m)$, then $w$ has a canonical piece decomposition: a sequence $(w_i,\ldots,w_{i+k})$ is a piece if and only if $u_{i-1}<u_i =\ldots= u_{i+k}<u_{i+k+1}$. This defines a chain homomorphism
	\[\lambda_m\colon \Omega_*(r,m)\to \Omega_*(r)\]
	Let $\Phi\colon \Omega_*(r)^f\to \Sigma^{r}\Omega_{*}(r)^{nf}$ be the (non-chain) homomorphism that sends a pair $(U,w)$ to the result of removing a full piece if it is possible and increase all entries in $w$ to the right of the piece.

	%If $w'$ is a subword of $w$ and $(U,w)$ is a pair, define the pair $(U|_{w'},w')$ in the obvious way.
	%where $w^{*}=w$ if $w$ has no full piece and $0$ otherwise and $w^{**}$ is the opposite.

	%A pieced word $w$ and a generator $U\in C^q(\Delta^m)$ are \emph{compatible} if they have the same length and  if and only if the sequence $(w_i,\ldots,w_{i+k})$ is a piece. If $U$ and $w$ are compatible, write $(U,w)$ for the generator of $\hat{\Omega}^*(r,m)$ that is obtained by forgetting the pieces of $w$. Otherwise, define $(U,w) = 0$.

	\begin{lemma}\label{lemma:omegar} Suppose one is given a $C_r$-equivariant map
		\begin{align*}
			\check{\Psi}\colon \Omega_*(r)^{nf}&\lra W_*(r)
		\end{align*}
		such that
		\begin{itemize}
			\item $\hat{\Psi}_{q-1}((\partial (w)^{nf}) = \hat{\Psi}_{q-r}\Phi(w)$ if $w$ has a full piece.
		\end{itemize}
		Then the family of maps $\hat{\Psi}_m = \check{\Psi}\lambda_m$ satisfies the conditions of Lemma \ref{lemma:omegarm}.
	\end{lemma}
	\begin{proof}
		Immediate.
	\end{proof}






	\subsection{Maps between spheres}\footnote{Actually we will need the map on unnormalised chains} We will use the canonical identification $N_*(X*Y) \cong N_*(X)\otimes N_*(Y)$ and write $\sigma$ for the top dimensional class of $N_*(\Delta^{r-1})$. The cyclic group acts on $\Delta_{r-1}$ by permuting its vertices. Let
	\begin{align*}
		\iota_1\colon N_k(\Sigma^{r-1}\partial \Delta^{r-1})&\lra N_{k}(\partial\Delta^{r-1})\otimes N_{r-2}(\partial\Delta^{r-1})\\
		\iota_2\colon N_k(\Sigma^{r-1}\partial \Delta^{r-1})&\lra N_{r-2}(\partial\Delta^{r-1})\otimes N_{k}(\partial\Delta^{r-1})
	\end{align*}
	be the $C_r$-equivariant chain homomorphisms given by
	\begin{align*}
		\iota_1(\tau) &= \tau\otimes \partial \sigma \\
		\iota_2(\tau) &= \partial \sigma\otimes \tau \\
	\end{align*}

	\renewcommand{\theenumi}{\roman{enumi}}
	Let $f\colon N_*(\partial\Delta^{r-1}*\partial\Delta^{r-1})\to N_*(\mathbb{S}^{r-1}*\partial\Delta^{r-1})$ be a $C_r$-equivariant chain homomorphism such that
	\begin{enumerate}
		\item\label{cond:1} $f\circ \iota_1 = \Id$
		\item\label{cond:2} $f\circ \iota_2 = \rho$.
		\item\label{cond:3} If $\tau_1\otimes\tau_2$ has degree $r-1$, then $f(\tau_1\otimes \tau_2)$ is a vertex if $\tau_1,\tau_2$ form a permutation and $0$ otherwise.
	\end{enumerate}
	\begin{remark} If we additionally impose the condition that $f$ sends simplices to simplices, then it is determined by its values on $(r-1)$-simplices.
	\end{remark}
	\begin{remark} Condition \eqref{cond:3} can be replaced by the more general condition
		\[\partial f(\tau_1\otimes\tau_2) = \begin{cases}
			\emptyset & \text{if $\tau_1$ and $\tau_2$ form a full permutation} \\
			0 & \text{otherwise}.\end{cases}\]
	\end{remark}
	\begin{remark} The former condition $f(\tau_1,\tau_2)\subset \tau_1$ is no longer necessary: It is necessary to define suspension maps for $\Omega_*(r,m)$, but with the current argument that is not needed.
	\end{remark}
	\begin{remark} In the examples constructed, $f(\tau_1,\tau_2)\subset \tau_1\smallsetminus \tau_2$.
	\end{remark}

	A word $w$ can be canonically broken into overlapping subwords of length at most $r$ called \emph{blocks}: Set the last block to be the last $r$ entries of the word, and recursively define the rest of the blocks by removing the last $r-1$ entries of the last block. Each pair of adjacent blocks share an element, that is called \emph{pivot}. If $w$ is a minimal pieced word, let $w_1$ be the piece that contains the last pivot and let $w_2$ be the complement of that piece in the last block.
	\begin{example} The pieced word $w=01|24|013|12|4$ has as blocks $01-12401-13124$, the last pivot is $1$, and $w_1 = 013$ and $w_2 = 124$.
	\end{example}
	\begin{definition} Define a homomorphism $S\colon \Omega_*(r)^{nf}\to \Sigma^{r-1}\Omega_*(r)^{nf}$ by sending a pieced word $w$ to the result of replacing $w_1$ and $w_2$ by the piece $f(w_1,w_2)$. %If $w$ is a pieced word with a full piece, define $S(w)$ as the result of inserting back the full piece into $S(\Phi(w))$ (if the full piece was to the right of $w_1$, then it gets inserted at the end of the word).
	\end{definition}
	\begin{lemma} $S$ is a chain homomorphism.
	\end{lemma}
	\begin{proof}
		Let $\hat{w}$ be the result of removing $w_1$ and $w_2$ from $w$, and let $\ell$ be the dimension of $\hat{w}$. Given two words $w$ and $w'$, let $w*w'$ be its concatenation. Recall that $\partial w = \sum_{i} (-1)^{i}d_iw$, where $d_i$ removed the $i$-th entry of $w$. Then
		\begin{equation}\label{eq:931}
			\partial S(w) = \partial \hat{w}*f(w_1,w_2) + (-1)^{\ell+1}\hat{w}*\partial f(w_1,w_2).
		\end{equation}
		while, if the piece of the pivot contains at least two elements of the penultimate block,
		\[S(\partial w) = \partial \hat{w}*f(w_1,w_2) + (-1)^{\ell+1}\hat{w}*f(\partial(w_1),w_2) + (-1)^{\ell+1}\hat{w}*f(w_1,\partial w_2),\]
		which equals the previous sum. Suppose that the piece of the pivot is contained in the last block, in which case $f(w_1,w_2)$ is of dimension $0$. Then, letting $w_0$ be the piece to the left of $w_1$ and $\check{w}$ be the subword to the left of $w_0$, we have
		\begin{equation}\label{eq:933}
			S(\partial w) = \partial \check{w}*f(w_1,w_2) + \check{w}*f(w_0,\partial(w_1*w_2)).
		\end{equation}
		Now, the last summand of \eqref{eq:931} is zero or $\hat{w}$ depending on whether $f(w_1,w_2)$ vanishes or not. By condition \eqref{cond:3}, this is equivalent to $w_1,w_2$ not being or being a full permutation, which by condition \eqref{cond:1} is equivalent to $f(w_0,\partial(w_1*w_2))$ being $0$ or $w_0$, and therefore equivalent to the last summand of \eqref{eq:933} being zero or $\hat{w}$.
	\end{proof}

	There is an inclusion $C_*(EC_r)\to \Omega_*(r)^{nf}$ that sends a word to that same word with all pieces of length $1$. There is a map in the opposite direction with $\bZ[\frac{1}{(r-1)!}]$-coefficients that sends a pieced word to the sum of the underlying words of all its representatives, divided by the number of representatives. The composition (here $\Psi$ is the map from the previous section)
	\begin{equation}\label{eq:234}
		\Omega_*(r)^{nf}\lra C_*(EC_r)\overset{\Psi^\vee}{\lra} W_*(r)
	\end{equation}
	is a chain homomorphism with $\bZ[\frac{1}{(r-1)!}]$-coefficients, but in degrees $*\leq r-2$ it lifts to a homomorphism with $\bZ$-coefficients, that we denote by $\check{\Psi}_*\colon \Omega_*(r)^{nf}\to W_*(r)$.

	\begin{definition} Define a homomorphism $\check{\Psi}\colon \Omega_*(r)^{nf}\to W_*(r)$ recursively as
		\[\check{\Psi}_q(w) = \begin{cases} \check{\Psi}_q(w) & \text{if $*\leq r-2$} \\
			\theta_{1-r}\check{\Psi}_{q-r+1}S(w) & \text{if $*\geq r-2$.}\end{cases}\]
	\end{definition}



	\begin{lemma} $\check{\Psi}$ is a well-defined homomorphism. %$\check{\Psi}\partial = \partial\check{\Psi}$
	\end{lemma}
	\begin{proof}
		In degrees $*\leq r-2$ it is a homomorphism because it is a composition of the homomorphisms \eqref{eq:234}. The verification in the higher degrees follows from the previous lemma and induction:
		\begin{align*}
			\Psi_{q-1}\partial(w) &= \theta_{r-1}\Psi_{q-r}S\partial(w) = \theta_{r-1}\Psi_{q-r}\partial S(w) = \\
			&=\theta_{r-1}\partial\Psi_{q-r+1}S(w) = \partial\theta_{r-1}\Psi_{q-r+1}S(w) = \partial \Psi_{q}(w).
		\end{align*}
		Finally, one deduces from \eqref{cond:3} that $\theta_{r-1}\check{\Psi}_{-1}S(w) = \check{\Psi}_{r-2}(w)$, hence both definitions coincide at their common case.
	\end{proof}


	%\begin{lemma} $\hat{\Psi}_{q-1}((\partial w)^{nf}) = \partial\hat{\Psi}_{q-r}\Phi(w)$ if $w$ has a full piece.
	%\end{lemma}
	%\begin{proof}
	%\begin{itemize}
	%\item if the full piece is to the left of $w_1$ \fcnote{doublecheck this, if it is exactly to the left an additional argument may be needed}, then $S\partial(w) = \partial S(w)$.
	%\item if the full piece is to the right of $w_1$, then let $w_3$ be that full piece, and assume, without loss of generality that is the last piece. Then
	%\[\partial S(w) = \partial \hat{w}*f(w_1,w_2)*w_3 + \hat{w}*\partial f(w_1,w_2)*w_3 + \hat{w}*f(w_1,w_2)*\partial w_3\]
	%\[S\partial(w) = \partial \hat{w}*f(w_1,w_2)*w_3 + \hat{w}*\partial f(w_1,w_2)*w_3 + \hat{w}*w_1*f(w_2,\partial w_3)\]
	%We need to check that the value of $\check{\Psi}$ in the last term coincides. If $q-r+1\geq r-2$, then we have that
	%\[\check{\Psi}(\hat{w}*f(w_1,w_2)*\partial w_3) = \check{\Psi}S(\hat{w}*f(w_1,w_2)*\partial w_3) = \check{\Psi}(\hat{w}*f(f(w_1,w_2),\partial w_3)\]
	%\[\check{\Psi}(\hat{w}*w_1*f(w_2,\partial w_3)) = \check{\Psi}S(\hat{w}*w_1*f(w_2,\partial w_3)) = \check{\Psi}(\hat{w}*f(w_1*f(w_2,\partial w_3))\]
	%and both terms equal $f(w_1,w_2)$.
	%\end{itemize}
	%
	%\end{proof}

	\begin{lemma} $\hat{\Psi}_{q-1}((\partial w)^{nf}) = \hat{\Psi}_{q-r}\Phi(w)$ if $w$ has a full piece.
	\end{lemma}
	\begin{proof} We will prove it by induction on the position from the right of the left piece. Since $w$ has at least one full piece, $q\geq r-1$.

		If the full piece is the last piece, write $w=\hat{w}*w_1*w_2$ with $w_2$ the full piece. Then by Condition \eqref{cond:1}:
		%\[\partial S(w) = \partial \hat{w}*f(w_1,w_2)*w_3 + \hat{w}*\partial f(w_1,w_2)*w_3 + \hat{w}*f(w_1,w_2)*\partial w_3\]
		\begin{align*}
			\Psi_{q-r}(S(\partial(w))^{nf}) &= \Psi_{q-r}(S(\hat{w}*w_1*\partial w_2)) \\
			&= \Psi_{q-r}(\hat{w}*f(w_1,\partial w_2)) \\
			&= \Psi_{q-r}(\hat{w}*w_1) \\
			&= \Psi_{q-r}\Phi(w).
		\end{align*}
		If the full piece is not the last piece but contains the pivot, write $w=\hat{w}*w_1*w_2$ with $w_2$ the full piece. Then by Condition \eqref{cond:2}:
		\begin{align*}
			\Psi_{q-r}(S(\partial(w))^{nf}) &= \Psi_{q-r}(S(\hat{w}*\partial w_1* w_2)) \\
			&= \Psi_{q-r}(\hat{w}*f(\partial w_1, w_2)) \\
			&= \Psi_{q-r}(\hat{w}*\rho(w_2)) \\
			&= \Psi_{q-r}\Phi(w).
		\end{align*}
		If the full piece $\bar{w}$ is left to the last pivot, write $w=\hat{w}_1*\bar{w}*\hat{w}_2*w_1*w_2$ with $w_1$ the piece that contains the last pivot and $w_2$ the word to the right of $w_1$. Assume by induction that the lemma holds when $\hat{w}*w_1*w_2$ is of smaller length. Then
		\begin{align*}
			\Psi_{q-r}(S(\partial(w))^{nf}) &= \Psi_{q-r}(S(\hat{w}_1*\bar{w}*\hat{w}_2w_1w_2)) \\
			&= \Psi_{q-r}(\hat{w}_1*\bar{w}*\hat{w}_2*f(w_1,w_2)) \\
			&\overset{*}{=} \Psi_{q-r}(\hat{w}_1*\rho(\hat{w}_2*f(w_1,w_2)))\\
			&= \Psi_{q-r}(\hat{w}_1*\rho(\hat{w}_2*w_1*w_2))\\
			&=  \Psi_{q-r}(\Phi(w)). \qedhere
		\end{align*}
	\end{proof}
	\begin{corollary} The map $\check{\Psi}$ satisfies the conditions of Lemma \ref{lemma:omegar}, hence together with Lemma \ref{lemma:omegarm} it defines a chain homomorphism $\Psi\colon V^*(r)\to \bar{\Omega}^*(r,n)$.
	\end{corollary}






	\subsection{Maps $f$}
	If $r=3$, then there is a unique map $f\colon N_*(\partial \Delta^2)\otimes N_*(\partial\Delta^2)\to \Sigma^2N_*(\partial\Delta^2)$ satisfying all conditions, namely $f([0]\otimes [1,2]) = [0], f([0,1]\otimes [2])=0$, which extends $C_r$-equivariantly as follows (we only indicate the non-zero terms):
	\begin{align*}
		f([0,1]*[2,0]) &= [0,1]
	\end{align*}

	If $r=5$, then there is no unique map $f\colon N_*(\partial \Delta^4)\otimes N_*(\partial \Delta^4)\to \Sigma^4N_*(\partial \Delta^4)$, but here is one of them:
	\begin{align*}
		f([0]\otimes[1,2,3,4]) &= [0] &
		f([0,1]\otimes [2,3,4]) &= [0] &
		f([0,2]\otimes [1,3,4]) &= [0] \\
		f([0,1,2]\otimes [3,4]) &= [0] &
		f([0,1,3]\otimes [2,4]) &= [0] &
		f([0,1,2,3]\otimes [4]) &= [0],
	\end{align*}
	which extends $C_r$-equivariantly as follows (we only indicate the non-zero terms):
	\begin{align*}
		f([0,1]\otimes [2,3,4,0]) &= [0,1] &
		f([0,2]\otimes [1,3,4,0]) &= [0,2] \\
		f([0,1,2]\otimes [3,4,0]) &= [0,1,2] &
		f([0,1,3]\otimes [2,4,0]) &= [0,1] \\
		f([0,1,3]\otimes [2,4,1]) &= [0,3] &
		f([0,1,2,3]\otimes [4,0]) &= [0,1] \\
		f([0,1,2,3]\otimes [4,1]) &= [2,0] &
		f([0,1,3]\otimes [2,4,0,1]) &= [0,1,3] \\
		f([0,1,2,3]\otimes [4,0,1]) &= [0,1,2] &
		f([0,1,2,3]\otimes [4,1,2]) &= [0,2,3] \\
		f([0,1,2,3]\otimes [4,0,1,2]) &= [0,1,2,3] &&
	\end{align*}

	There is also an explicit map for $r=7$.
	\section{Other considerations}
	\begin{enumerate}
		\item These operations are compatible with suspension.
		\item Cup product is the Alexander cup product (possibly divided by a scalar).
		\item Succint description of the higher power operations.
		\item A measurement of the computational complexity of these operations.
		\item Comparison with previous formulas.
		\item The case $r=2$. It is not trivial: $C^*(EC_2)\neq W^*(2)$.
	\end{enumerate}

	\section{Comparison with the surjection operad (drafty)}
	Consider the functor tensor product $F^*\otimes G$, that is, $N_*(\Delta^\bullet)\otimes_\Delta N_\bullet(\Delta^n)$. It is related to the previous tensor product via Poincaré duality and it is isomorphic to $N_*(\Delta^n)^{\otimes r}$. Similarly, there is a chain complex analogous to $\Omega(r,n)$ that we denote, for the moment as $\Omega^\vee(r,n)$. Instead of using the previous isomorphism, we will use the isomorphism that sends $u_i\in U$ to the $w_i$-th factor of the tensor product.

	If $\tau$ is a simplex of dimension $m$, one consider all pairs $(f,A)$ where
	\begin{itemize}
		\item $f$ is a surjection, i.e., a non-degenerate word in $EC_r$ that contains every element $0,1,\ldots,r-1$ at least once.
		\item $A$ is a decomposition of a simplex $\tau$ into overlapping intervals.
	\end{itemize}
	Now,
	\begin{itemize}
		\item The pair $(f,A)$ defines a word in $EC_r$ given by replicating each number in $f$ as many times as the length of each interval dictates. For example, if $r=3$ and $f= 0121012$ and $A=(0,1,2)(2)(2,3)(3)(3,4,5)(5,6)(6)$, then the resulting word is $w=0001221000112$.
		\item $A$ gives a simplex in $\Delta^m$ that results from erasing the parentheses: $U=[0,1,2,2,2,3,3,3,4,5,5,6,6]$
	\end{itemize}
	The triple $(f,A,\tau)$ thus yields an element of $\Omega^\vee(r,n)$, whose image coincides with the image of the operation $(f,A)$ on $\tau$.

	If we want to interpret this element in $\Omega(r,n)$, we just need to do the following:
	\begin{itemize}
		\item The new $U$ is obtained from the maximally degenerate simplex \[(0,0,0,1,1,1,2,2,2,3,3,3,4,4,4,5,5,5,6,6,6)\] by removing the elements of $U$. In our case, $U=(0,0,1,1,4,4,5,6)$.
		\item The new $w$ is obtained from $w$ by setting $w_i$ equal to any element of $C_r$ different from the ones that where labeling $u_i$ in the old $w$ \emph{provided that the old ones were not repeated. Otherwise, the operation is zero!!. Nonetheless, these cases may be easily identified}. This happens quite often. In our case $w=(12121220)$.
	\end{itemize}
	Finally, we need to substract $u_i$ from $w_i$ to obtain the word we care about in $\Omega(r,n)$: (12010100)

	If $f$ is a surjection, let $w$ be the pieced word that comes from replacing each entry by its complement. Let $Q(f)$ be the collection of all subwords of all representatives of $w$.
	\begin{lemma} In this dictionary (without the last substraction), the collection of words associated to a surjection $f$ is $Q(f)$.
	\end{lemma}

	\section{Some notes in the search for $f$}

	Let $Z_\bullet(r)$ be the semi-simplicial set whose $k$-simplices are partitions $(q_0,\ldots,q_k)$ of $r$ with face maps given by
	\begin{align*}
		\partial_i(q_0,\ldots,q_k) &= \begin{cases}
			(q_0,\ldots,q_{i-1}+q_{i},q_{i+1},\ldots,q_k) & \text{if $i>0$} \\
			(q_1,\ldots,q_{k}+q_0) & \text{if $i=0$}
		\end{cases}
	\end{align*}
	We write $\Lambda(r)$ for the quotient of $N_*(Z(r))$ by the equivalence relation generated by $(q_0,q_1,\ldots,q_k)\sim (q_k,q_0,\ldots,q_{k-1})$.
	\begin{example} $r=7$, $(q_0,q_1,q_2) = (2,2,3)$ and its boundary is $(2,5)-(4,3)+(2,5)$.
	\end{example}
	Write $\Theta(r)$ for the quotient of $N_*(\partial \Delta^{r-1})$ by the equivalence relation generated by $(a_0,a_1,\ldots,a_k)\sim (\rho(a_0),\rho(a_1),\ldots,\rho(a_k))$. If $(a_0,\ldots,a_k)$ is a $k$-simplex, we will use the cyclic notation $a_{k+1} = a_0$.
	\begin{definition} Let $a = (a_0,\ldots,a_k)\in \partial \Delta^{r-1}$ be a simplex. The \emph{pattern} $q(a)$ of $a$ is the sequence of cyclic differences $(a_0-a_k+r,a_1-a_0,a_2-a_1,\ldots,a_k-a_{k-1})$.
	\end{definition}
	\begin{lemma} The map $q\colon \Theta(r)\to \Lambda(r)$ is an isomorphism of chain complexes.
	\end{lemma}
	\begin{proof} Immediate.
	\end{proof}
	\subsection{Reflections on the map $f$} As we said before, the map $f$ is determined by its value on $(r-1)$-simplices. Therefore, we face the problem of choosing wisely a $C_r$-equivariant function $f\colon (\partial\Delta^{r-1}*\partial\Delta^{r-1})_{r-1}\to \{0,1,\ldots,r-1\}$ satisfying condition \eqref{cond:3}, and then check that
	\begin{enumerate}
		\item $f$ extends to the whole join.
		\item $f$ satisfies condition \eqref{cond:1}
		\item $f$ satisfies condition \eqref{cond:2}.
	\end{enumerate}
	The set of $C_r$-isomorphism classes of $(r-1)$-simplices of $\partial\Delta^{r-1}*\partial\Delta^{r-1}$ is in bijection with the generators of $\Theta(r)$ in two ways: The first isomorphism sends a pair $(a,b)$ to the class of $a$. The second isomorphism sends $(a,b)$ to $b$. I believe that the extra structure (the differential) in $\Lambda(r)$ (or $\Theta(r)$) will be helpful.

	I will start working with $\Theta(r)$, though many of the ideas come from thinking about $\Lambda(r)$. The restriction of the unknown $f$ to $\Theta(r)$ along the two isomorphisms above will be denoted $\ell,\bar{\ell}\colon \Theta(r)\to \{0,1,\ldots,r-1\}$. If $a$ is a generator of $\Theta(r)$, its complement in $\{0,1,\ldots,r-1\}$ wil be denoted $b$.

	\subsection{First assumption} I will make the following assumption:
	\begin{assumption} $\ell(a)\in a$ and $\ell(a)-1\notin a$.
	\end{assumption}
	\begin{remark} This assumption on $\ell$ translates to $\bar{\ell}$ as follows: Since $\ell(a)-1\in b$, $\rho^{-1}(\bar{\ell}(b))\in b$
	\end{remark}
	\begin{definition} Given a generator $\sigma$ of $\Theta(r)$, a \emph{spike} of $\sigma$ is any $a_i$ such that $\ell(\partial_i a) \neq \ell(a)$. Observe that $\ell(a)$ is always a spike under the assumption above. The set of all spikes of $a$ is called $\mathrm{spk}(a)$. The assignment $a\mapsto \partial_i a$ is \emph{standar} if $\ell(\partial_i a) = \ell(a)$, otherwise we say that it is \emph{exceptional}.
	\end{definition}
	\begin{lemma} Under the above assumption, $f(a,b) \neq 0$ if and only if $a\cap b$ is a spike, in which case its value is $(\ell(a),\ell(a\smallsetminus b))$.
	\end{lemma}
	\begin{proof} All the face maps yield non-full pairs (pairs where $a\cup b \neq \{0,1,\ldots,r-1\}$) except for the two face map that remove the element $a\cap b$ from $a$ and from $b$. Removing it from $b$ yields the pair $(a,b\smallsetminus b)$ whose image under $f$ is $\ell(a)$. Removing it from $a$ yields $\ell(\partial_i a)$, which cancels with the other image unless $a_i$ is a spike.
	\end{proof}
	\subsection{Second assumption}We introduce from now on the following stronger assumption.
	\begin{assumption} if $a\mapsto \partial_ia$ is exceptional (i.e., $a_i$ is a spike of $a$), then $\ell(\partial_i a) = a_{i+1}$.
	\end{assumption}
	\begin{remark} This assumption translates to $\bar{\ell}$ as follows: if $b\mapsto \partial_i b$ is exceptional, then $\ell(\partial_ib)$ is a number in $b$ such that all the numbers between it and $\ell(b)-1$ belong to $b$.
	\end{remark}


	\begin{lemma} If $\sigma$ is a simplex in $\Theta(r)$ and $i\leq j$, consider the square
		\[\xymatrix @=2pc{
			&\sigma \ar[dl]_{\partial_i}\ar[dr]^{\partial_{j+1}} &\\
			\partial_i\sigma\ar[dr]_{\partial_j} && \partial_{j+1}\sigma\ar[dl]^{\partial_i}\\
			&\partial_j\partial_{i}\sigma &
		}\]
		and write $\circ$ if an arrow is standard and $\times$ if it is exceptional. Then the only possible configurations are
		\[\xymatrix @=1pc{
			&\ar[dl]_{\times} \ar[dr]^{\times}&& \\
			\ar[dr]_{\times} && \ar[dl]^{\circ} \\
			&&
		}
		\xymatrix @=1pc{
			&\ar[dl]_{\times} \ar[dr]^{\times}&& \\
			\ar[dr]_{\circ} && \ar[dl]^{\times} \\
			&&
		}
		\xymatrix @=1pc{
			&\ar[dl]_{\times} \ar[dr]^{\circ}&& \\
			\ar[dr]_{\circ} && \ar[dl]^{\times} \\
			&&
		}
		\xymatrix @=1pc{
			&\ar[dl]_{\circ} \ar[dr]^{\times}&& \\
			\ar[dr]_{\times} && \ar[dl]^{\circ} \\
			&&
		}
		\xymatrix @=1pc{
			&\ar[dl]_{\circ} \ar[dr]^{\circ}&& \\
			\ar[dr]_{\circ} && \ar[dl]^{\circ} \\
			&&
		}
		\]





		%\begin{assumption} If $\ell(a) = a_i$, then $\ell(\partial_i a) = a_{i+1}$
		%Suppose that for each $k\geq 0$ there is a function $\ell\colon Z_k(r)\to \{0,\ldots,k\}$ such that either
		%\[
		%\ell(\partial_i\sigma) \overset{*}{=}
		%\begin{cases}
		%\ell(\sigma) & \text{if $\ell(\sigma)<i$} \\
		%\ell(\sigma)-1 & \text{if $\ell(\sigma)\geq i$}
		%\end{cases}
		%\quad \text{or}\quad
		%\ell(\partial_i\sigma)\overset{**}{=}i-1.
		%\]
		%Additionally, suppose that $\ell$ is $C_{k+1}$-equivariant.
		%\end{assumption}
		%\begin{example}
		%$r=5$, define
		%\begin{align*}
		%\ell(1,4) &= 1 &
		%\ell(2,3) &= 1 &
		%\ell(1,1,3) &= 2 &
		%\ell(1,2,2) &= 2 &
		%\ell(1,1,1,2) &= 3
		%\end{align*}
		%and extend equivariantly.
		%\end{example}
		\begin{example}
			$r=5$, define
			\begin{align*}
				\ell(0,1) &= 0 &
				\ell(0,2) &= 0 &
				\ell(0,1,2) &= 0 &
				\ell(0,1,3) &= 0 &
				\ell(0,1,2,3) &= 0
			\end{align*}
			and extend equivariantly. For $r=7$ define $\ell$ to take value $0$ on the following simplices, and extend equivariantly:
			\begin{align*}
				(0) &&
				(0,1) &&
				(0,2) &&
				(0,3) \\
				(0,1,2) &&
				(0,1,3) &&
				(0,1,4) &&
				(0,2,3) &&
				(0,2,4) \\
				(0,1,2,3) &&
				(0,1,2,4) &&
				(0,1,3,4) &&
				(0,2,3,4) &&
				(0,1,3,5) \\
				(0,1,2,3,4) &&
				(0,1,2,3,5) &&
				(0,1,3,4,5) &&
				(0,1,2,3,4,5)
			\end{align*}
		\end{example}

		%\begin{itemize}
		%\item if $i$ and $j+1$ are non-leading spikes of $\sigma$, then all arrows are exceptional but one of the lower ones.
		%\item if $i$ and $j+1$ are not non-leading spikes of $\sigma$, then all arrows are standard.
		%\item if $i$ is a non-leading spike and $j+1$ is not, then....
		%\end{itemize}
	\end{lemma}
	\begin{lemma} There is an inclusion $\mathrm{spk}(\partial_i\sigma)\subset \mathrm{spk}(\sigma)$.
	\end{lemma}
	\begin{proof} Follows from the previous lemma
	\end{proof}
	\begin{lemma} If $T\subset \mathrm{spk}(\sigma)$ is a collection of spikes of $\sigma$, then there is exactly one spike $i$ such that $T\subset \mathrm{spk}(\partial_i\sigma)$. We call it the \emph{principal spike} of $(\sigma,T)$.
	\end{lemma}
	\begin{proof} Not so easy, should follow from the previous lemma.

	\end{proof}
	\subsection{A map $f$} Here we construct a map $f$ that satisfies Condition \eqref{cond:1} but not necessarily condition \eqref{cond:2}. The maps we already have for $r=5,7$ can be constructed using this method.
	\begin{definition}  %The \emph{spikes} of $a$ (denoted $\mathrm{spk}(a)$) are the spikes of $q(a)$.
		A \emph{snippet} of $a$ is a sequence $v = (a_j,a_{j+1},\ldots,a_{j+m})$ of consecutive elements of $a$ such that $a_j$ is a spike, there are no other spikes in the sequence and $j+m\neq \ell(a)-1$. For that snippet, its \emph{slided snippet} $v^+$ is the sequence $(a_{j+1},a_{j+2},\ldots,a_{j+m+1})$.
	\end{definition}
	Define a map $f\colon \partial N_*(\Delta^{r-1}*\Delta^{r-1})\to \Sigma^{r-1}\partial N_*(\Delta^{r-1})$ as follows:
	\[
	f(a,b) = \begin{cases}
		\{\ell(a)\}\cup v_0^+\cup v_1^+\cup\ldots \cup v_d^+ & \begin{array}{l} \text{if $\{0,1,\ldots,r\}\subset a\cup b$ and} \\ \text{$a\cap b$ is a union of snippets $v_0,v_1,\ldots,v_d$}\end{array} \\
		0 & \text{otherwise.}
	\end{cases}
	\]
	\begin{lemma} If $\ell$ satisfies both assumptions, then $f$ is a map of chain complexes.
	\end{lemma}
	\begin{proof} Recall that the differential of $(a,b)$ is the sum of all possible removals of elements of $a$ and $b$. Observe first that removing an element from $a$ or $b$ that is not in $a\cap b$ yields a simplex $(a',b')$ that does not satisfy Condition \eqref{cond:711}, and therefore $f(a',b') = 0$.

		We will prove the lemma by induction, the base case being in dimension $r-1$, when $a\cap b = \emptyset$ and there are no snippets.

		Suppose first that we are in the first case, i.e.,
		\begin{align}\label{cond:711}
			&\{0,1,\ldots,r\}\subset a\cup b \\ \label{cond:712}
			&a\cap b\text{ is a union of snippets }v_0,v_1,\ldots,v_d.
		\end{align}


		Now, let $T(a,b)$ be the set of those spikes of $a$ that are in some snippet but not in any slided snippet.
		\begin{enumerate}
			\item Removing a highest element from a snippet $v$ in $b$ yields a simplex $(a,b')$ that satisfies both conditions and $f(a,b')$ results from removing the highest element of $v^+$ from $f(a,b)$.
			\item Removing any non-spike element from a snippet $v$ in $a$ yields a simplex $(a',b)$ that satisfies both conditions and $f(a',b)$ results from removing that element from $f(a,b)$.
			\item Removing the principal spike of $(a,T(a,b))$ from $a$ yields a simplex $(a',b)$ that satisfies both conditions and $f(a',b)$ results from removing $a_0$ from $f(a,b)$
		\end{enumerate}
		In the remaining cases, $f(a',b')=0$:
		\begin{enumerate}
			\item Removing a non-highest element $\hat{a}$ from a snippet $v$ in $b$ yields a simplex $(a,b')$ such that $a\cap b$ is not a union of snippets (the part of $v$ to the right of $\hat{a}$ is not a snippet and is not part of any other snippet).
			\item Removing a non-principal spike of $(a,T(a,b))$ from $a$ yields a simplex $(a',b)$ where some spike of $(a,\mathrm{spk}(a)\cap b)$ is not a spike of $a'$.
			\item Removing a spike of a snippet that does not belong to $T(a,b)$ yields a simplex whose highest element is contained in $a\cap b$ (hence cannot be part of any snippet).
		\end{enumerate}
		Suppose now that \eqref{cond:711} is not satisfied. Then removing any element of $a\cap b$ from $a$ or $b$ yields a simplex that does not satisfy \eqref{cond:711}.

		Finally, suppose that \eqref{cond:711} is satisfied, but not \eqref{cond:712}. Then removing any element of $a\cap b$ from $a$ or $b$ yields a simplex that does not satisfy \eqref{cond:711} unless $a\cap b$ is the result of removing from a union of snippets $v_0,\ldots,v_d$ the penultimate element $\hat{a}$ of some snippet. If $a' = a\smallsetminus \{\hat{a}\}$ and $b' = b\smallsetminus \{a\}$, then $(a',b)$ and $(b',a)$ are the only summands in the differential of $(a,b)$ that take non-zero value under $f$, and both take value $\{a_0\}\cup v_0^+\cup \ldots \cup V_d^+$.
	\end{proof}
	\begin{lemma} $f(\iota_1(a)) = a$.
	\end{lemma}
	\begin{proof} The elements in $\iota_1(a)$ consist on all pairs $(a,b)$ with $|b|=r-1$, hence $a\cap b$ consists on all elements of $a$ but one. Since the term $\ell(a)-1$ can never belong to $a\cap b$, there is only one such pair where $f$ is non-zero: $b = \{0,1,\ldots,r-1\}\smallsetminus \{\ell(a)-1\}$, and the value of $f$ on it is $a$.
	\end{proof}
	\begin{remark} There is no reason why $f(\iota_2(a))$ should equal $\rho(a)$. Our choices of $\ell$ for $r=5,7$ do satisfy all the assumptions, and do indeed satisfy Condition \eqref{eq:2}. Nonetheless, these are the only cases in which this construction works:
	\end{remark}
	\begin{lemma} The above construction cannot yield Condition \eqref{cond:2} if $r$ is a prime number bigger than 7 and smaller than 100.
	\end{lemma}
	\begin{proof}
		Let $a$ be the generator of $\Theta(r)_{r-2}$. Consider the set $Q=(i_0,\ldots,i_q)$ whose elements are
		\begin{itemize}
			\item the number of elements between consecutive spikes, where the right spike is not $\ell(\sigma)$.
			\item the number of elements minus one between consecutive spikes, where the right spike is $\ell(\sigma)$.
		\end{itemize}
		Then the number of possible combinations of snippets is $\prod_j{(i_j+2)}$.

		On the other hand, the number of simplices in $N_*(\partial\Delta^{r-1})$ of dimension at least $0$ up to cyclic ordering is $\frac{2^r-2}{r}$. If $r=5$, then the number of simplices is $6$, and in the example we have of $f$ for $r=5$, $Q=(2,3)$. If $r=7$, the number of simplices is $18$ and in the example we have of $f$ we have $Q=(2,3,3)$. For $r=11$, the quatity is $186 = 2*3*31$, hence there should be a snippet of length $29$, and this is impossible. The same issue happens for all higher primes up to 100.
	\end{proof}
	\subsection{Other methods} Suppose that we have constructed already a map $f$. Then the map $f'$ defined as $f'(a,b) = f(b)$ is a chain map, and it satisfies condition \eqref{cond:2} if $f$ satisfies condition \eqref{cond:1}.

	$\ell\colon \Theta(r)\to \{0,1,\ldots,r-1\}$ is a map that extends to the whole join. Then defining $\ell'(a) = \rho^{-1}(\bar{\ell}(a))$ we obtain a new map that extends. If $\ell$ satisfies condition \eqref{cond:1}, then $\ell'$ satisfies condition \eqref{cond:2}.

	$\Theta(r)$ has an automorphism $\varphi(x)= -x$ that reverses the cyclic order: $(0,1,3)\mapsto (0,4,2)$. Precomposing with that automorphism yields new $\ell$'s.

	In the examples above: If $r=5$, then $\ell = \rho^{-1}(\bar{\ell}(\varphi))$. If $r=7$, then $\ell$ and $\bar{\ell})$ are not related in this way. In fact, $\rho^{-1}(\bar{\ell})$ yields another solution that does not satisfy the assumptions made.

	Hence one should try to relax the assumptions.
	\begin{remark} One could be tempted to ask at the same time for the second assumption both on $\ell$ and $\bar{\ell}\circ \varphi$. I have found that there is no such $\ell$ even for $r=7$.
	\end{remark}

% !TEX root = ../oddp.tex

\section{Odd prime operations}

In the first section we construct a map from the normalized chain complex of the Milnor resolution to the minimal resolution of $EC_r$.
In the second section we construct a $C_r$-equivariant map $W_*(r)\otimes N_*(\Delta^n)\to N_*(\Delta^n)^{\otimes r}$.

All chain complexes are augmented. Chain complexes are taken with generic ring coefficients unless specified. The linear dual of a chain complex $C_*$ is denoted $C^*$. The boundary of a chain complex is denoted $\partial$, the coboundary of a cochain complex is denoted $\delta$ and the face maps of a simplicial set are denoted $d_i$. The linear dual of a map $f$ or a vector $v$ is denoted by $f^\vee$ and $v^\vee$, though we will drop the superscript when the context allows. Chain complexes and cochain complexes are positively graded, though we understand a positively graded cochain complex as a negatively graded chain complex. All the (co)chain complexes in this work come with a prescribed basis, and since they are of finite type, this basis can be used to generate the dual complex.

We will construct four maps from the dual of the minimal resolution of $C_r$. Sometimes it will be convenient to treat their linear duals: It is easier to construct a map $W^*(r)\to N^*(EC_r)$ because $W^*(r)$ is free on one object in each dimension, but it is easier to construct maps $N_*(EC_r)\to N_*(EC_r)$ because the differential in chains is easier than the differential in cochains.

\subsection*{To be done}
\begin{itemize}
	\item Words are called $A$ in the first section and $w$ in the second section. Additional conflicting notation is that $a_i$ inthe 
	\item I've added a section ``other considerations'' with possible comments to be made around the main theorem.
	\item IMPORTANT: Find the sign that relates the join and the pair barycentric subdivision.
\end{itemize}



% !TEX root = ../oddp.tex

\section{Alexander duality on the simplex and tensor products}\label{s:3complexes}

In this section we present a duality on the normalised chains of the augmented simplex, and use it to build manageable models of the $r$-fold tensor product of the normalised chain complex of an augmented semi-simplicial object in $\cC$. In Proposition \ref{prop:omegarm} and Lemma \ref{lemma:omegar} we reduce the construction of a connected comultiplication to the construction of a map between two resolutions of the cyclic group $\Cyc_r$ satisfying an additional hypothesis.

\subsection{Alexander duality on the augmented standard simplex}

\anibal{What is the structure on $\sSet_+$?}

\anibal{What is the bar construction of this algebra?} \federico{See the complex $\Omega_*(r)$ in Section 4}

\begin{definition}\label{d:poincare_duality_algebra}
	Let $A$ be a connected commutative algebra that is finite dimensional for each degree.
	We say $A$ is a \textit{Poincar\'e duality algebra} of \textit{formal dimension} $d$ if:
	\begin{enumerate}
		\item\label{i:pd1} $A_i = 0$ for $i > d$,
		\item\label{i:pd2} $\dim A_d = 1$,
		\item\label{i:pd3} $A_i \ot A_{d-i} \to A_d$ is non-degenerate.
	\end{enumerate}
\end{definition}

\begin{definition}\label{d:join_product}
	For any $[n] \in \ob\asimplex$ the \textit{join product} $\ast \colon \chains(\asimplex^{n})^{\ot 2} \to \chains(\asimplex^{n})$ is the linear map defined by sending a basis element $[v_0, \dots, v_{p-1}] \ot [v_{p},\dots,v_{m-1}]$ to $(-1)^{\sign \pi}[v_{\pi(0)}, \dots, v_{\pi(m-1)}]$	if $v_i \neq v_j$ for all $i \neq j$, where $\pi$ is the permutation ordering the vertices, and to $0$ otherwise.
\end{definition}

\begin{theorem}
	The join product is a chain map naturally defining on each $\chains(\asimplex^{\bn})$ the structure of a Poincar\'e duality algebra with unit the empty simplex $[-1] \to [n]$ and formal dimension $n+1$.
\end{theorem}
We will refer to this duality as \emph{Alexander duality}.

\begin{proof}
	The complex $\chains(\asimplex^{\bn})$ is connected and satisfies \cref{i:pd2,i:pd3} in \cref{d:poincare_duality_algebra} since $\chains(\asimplex^{\bn})_0 \cong \Z\{[-1] \to [n]\}$, $\chains(\asimplex^{n})_{n+1} \cong \Z\{[n] \to [n]\}$, and $\chains(\asimplex^{n})_{n+k} \cong 0$ for $k>1$.

	That the join product is a natural chain map can be easily verified and a complete proof is presented in \cite[p.19]{medina2020prop1}.

	Thinking about the join product in terms of the union of sets with a permutation sign leads to a direct verification of its commutativity (in the graded sense) and unitality with respect to the empty simplex.

	To verify \cref{i:pd3} consider a basis element $x = [v_1,\dots,v_i]$.
	Let $x^c$ be the ordered complement of $\set{v_1,\dots,v_i}$ in $\{0,\dots,n\}$ and notice that $x \ast x^c = \pm [0,\dots,n]$ as required.
\end{proof}

\begin{definition}
    Let $\Lambda\colon \chains_*(\asimplex^\bn)\to \sus{n+1}\chains^*(\asimplex^\bn)$ be the isomorphism induced by the pairing above. In detail, $\Lambda(\tau) = (-1)^{\lambda(\tau)}\tau^c$, where $\lambda(\tau)$ is the sign of the permutation that orders the vertices of $\tau*\tau^c$.
\end{definition}

\begin{remark}
    The sign of the permutation that orders the vertices of $\tau*\tau^c$, for $\tau = [v_1,\ldots,v_m]$ and $\tau^c  = [u_1,\ldots,u_k]$ has the parity of either of the following numbers:
    \begin{align*}
        \sum_{i=1}^m (v_i-i) &= \sum_{i=1}^m v_i - \binom{m+1}{2} \\
        \sum_{i=1}^m (n-u_i-(m-i)) &\equiv \sum_{i=1}^m u_i +m(n+1)+\binom{m+1}{2}.
    \end{align*}
\end{remark}
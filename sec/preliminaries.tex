% !TEX root = ../oddp.tex

\section{Preliminaries} \label{s:preliminaries}


\subsection{Augmented pointed simplicial sets} For each non-negative integer $n$, let $\bar{n} = \{1,\ldots,n\}$ denote the $n$th ordinal.
\begin{definition}
    The \emph{augmented simplex category} $\Delta_+$ is the category of finite ordinals and order-preserving maps between them.
\end{definition}
Every map in this category factors as an injection followed by a surjection. A \emph{face map} is an injection $f\colon \bar{n}\to \bar{m}$, and it is completely determined by the set $U = \bar{m}\smallsetminus f(\bar{n})$, and we will denote $\delta_U:= f$. If $U$ has a single element $i$, we will write $\delta_i :=\delta_U$. A \emph{degeneracy map} is a surjection $f\colon \bar{n}\to \bar{m}$. We will write $\sigma_i\colon \bar{n}\to \bar{n-1}$ for the surjection that satisfies $\sigma_i(i) = \sigma_i(i+1)$. 

The category $\Delta_+$ has a strict monoidal product called the \emph{join}, that sends a pair of ordinals $\bar{n},\bar{m}$ to the ordinal $\overline{n+m}$ and a pair of functions $f\colon \bar{n}\to \bar{s}$, $g\colon \bar{m}\to \bar{r}$ to the function $f*g\colon \overline{n+m}\to \overline{s+r}$ defined as $f(x) = x$ if $x\leq n$ and $f(x) = s+f(x-n)$ if $x>n$. The unit is the ordinal $\bar{0}$.

Let $\Setp$ be the symmetric monoidal category of pointed sets with the smash product. The category $\Set$ of sets with the cartesian product is a symmetric monoidal subcategory of $\Setp$.  
\begin{definition}
    An \emph{augmented pointed simplicial set} is a functor $X\colon \Delta_+^\mathrm{op}\to \Setp$.
\end{definition}
The pointed set $X(\bar{n})$ is denoted $X_n$. The image of the face maps $\delta_U$ is denoted $d_U$ and the image of the degeneracies $\Sigma_i$ is denoted $s_i$.

The join product in the augmented simplicial category induces a monoidal product on the category of augmented simplicial sets: If $F$ and $G$ are augmented pointed simplicial sets, consider the functor $F\times G\colon \Delta_+^\op\times \Delta_+^\op\to \Setp$ that sends a pair $([n],[m])$ to the pointed set $F[n]\wedge G[m]$. Then define $F*G\colon \Delta_+^\op\to \Setp$ as the left Kan extension of this functor along the opposite of the join product $\Delta_+^\op\times \Delta_+^\op\to \Delta_+^\op$. 

Let $R$ be a commutative ring and let $\Mod{R}$ and $\Ch{R}$ be the categories of $R$-modules and non-negatively graded chain complexes of $R$-modules respectively. If $A$ is a pointed set, write $R\langle A\rangle$ for the free $R$-module generated by $A$ modulo the basepoint. If $X$ is an augmented pointed simplicial set, define the \emph{complex of unnormalised chains $C_*(X;R)$ with coefficients in $R$} as
\begin{align*}
\uchains_n(X;R) &= R\langle X_n\rangle 
&
\partial(x) &= \sum_{j=1}^n d_j(x)
\end{align*}
The \emph{complex of normalised chains $\chains_*(X;R)$ with coefficients in $R$} is defined as 
\begin{align*}
\chains_n(X;R) &= R\langle X_n'\rangle 
&
\partial(x) &= \sum_{j=1}^n d_j(x)
\end{align*}
where $X_n'$ is obtained from $X_n$ by identifying the image of the degeneracies with the basepoint. Their dual cochain complexes are denoted $\uchains^*(X;R)$ and $\chains^*(X;R)$ and the dual of the differential $\partial$ is denoted $\delta$.
\subsection{Simplicial sets and simplicial pointed sets} Write $[n] = \{0,1,\ldots,n\}$ for the $n$th non-empty finite ordinal. The \emph{simplex category} is the category of non-empty finite ordinals and order-preserving maps between them. A \emph{pointed simplicial set} is a functor $\Delta^\op\to \Setp$. Sending the non-empty ordinal $[n]$ to the ordinal $\overline{n+1}$, we have an inclusion $\Delta\hookrightarrow \Delta_+$. By taking left Kan extension along this inclusion we may regard every pointed simplicial set as an augmented pointed simplicial set. Explicitely, if $Y$ is the image of a pointed simplicial set $X$, then $Y_n = X_{n-1}$ if $n>0$ and $Y_0$ contains only the basepoint. 

The category $\Set$ of sets includes into the category of pointed sets, and therefore every simplicial set may be regarded as a pointed simplicial set. The normalised and unnormalised chains on a simplicial set are the same as the normalised and unnormalised chains on its associated augmented pointed simplicial set with a shift of degree. We will consistently use the degree conventions for augmented pointed simplicial sets. To translate these conventions to the case of simplicial sets, observe the following:
\begin{itemize}
    \item A homomorphism $\chains_*(X)^{\otimes r}\to \chains_*(X)$ of degree $k$ with the classical convention, becomes a homomorphism of degree $k-r+1$ with the convention for augmented pointed simplicial sets.
\end{itemize}


$\simplex_+$, objects: $\bar{0}, \bar{1}, \dots$
\begin{itemize}
    \item Join of augmented pointed simplicial sets.
    \item chains and normalised chains.
    \item symmetric comultiplications and power operations. Minimal resolution
\end{itemize}
% !TEX root = ../oddp.tex

\section{Poincar\'e duality algebra}\label{s:statement}

\begin{definition}\label{d:poincare_duality_algebra}
	Let $A$ be a connected commutative algebra that is finite dimensional for each degree.
	We say $A$ is a \textit{Poincar\'e duality algebra} of \textit{formal dimension} $d$ if:
	\begin{enumerate}
		\item\label{i:pd1} $A_i = 0$ for $i > d$,
		\item\label{i:pd2} $\dim A_d = 1$,
		\item\label{i:pd3} $A_i \ot A_{d-i} \to A_d$ is non-degenerate.
	\end{enumerate}
\end{definition}

\begin{definition}\label{d:join_product}
	For any $\bn \in \ob\asimplex$ the \textit{join product} $\ast \colon \chains(\asimplex^{\bn})^{\ot 2} \to \chains(\asimplex^{\bn})$ is the linear map defined by sending a basis element $[v_1, \dots, v_p] \ot [v_{p+1},\dots,v_{m}]$ to $(-1)^{\sign \pi}[v_{\pi(1)}, \dots, v_{\pi(m)}]$	if $v_i \neq v_j$ for all $i \neq j$, where $\pi$ is the permutation ordering the vertices, and to $0$ otherwise.
\end{definition}

\begin{theorem}
	The join product is a chain map naturally defining on each $\chains(\asimplex^{\bn})$ the structure of a Poincar\'e duality algebra with unit the empty simplex $\bar{0} \to \bar{n}$ and formal dimension $n+1$.
\end{theorem}

\begin{proof}
	The complex $\chains(\asimplex^{\bn})$ is connected and satisfies \cref{i:pd2,i:pd3} in \cref{d:poincare_duality_algebra} since $\chains(\asimplex^{\bn})_0 \cong \Z\{\bar{0} \to \bn\}$, $\chains(\asimplex^{\bn})_{n+1} \cong \Z\{\bn \to \bn\}$, and $\chains(\asimplex^{\bn})_{n+k} \cong 0$ for $k>1$.

	That the join product is a natural chain map can be easily verified and a complete proof is presented in \cite[p.19]{medina2020prop1}.

	Thinking about the join product in terms of the union of sets with a permutation sign leads to a direct verification of its commutativity (in the graded sense) and unitality with respect to the empty simplex.

	To verify \cref{i:pd3} consider a basis element $x = [v_1,\dots,v_i]$.
	Let $\check{x}$ be the ordered complement of $\set{v_1,\dots,v_i}$ in $\{0,\dots,n\}$ and notice that $x \ast \check{x} = \pm [0,\dots,n]$ as required.
\end{proof}

\begin{corollary}
	For every $\bn \in \asimplex$ the chain complexes $\chains(\asimplex^{\bn})$ and $\sus{n+1}\cochains(\asimplex^{\bn})$ are naturally isomorphic.
\end{corollary}

The inverse of the canonical isomorphism $\chains(\asimplex^{\bn}) \to \sus{n+1}\cochains(\asimplex^{\bn})$ is given by
\[
[v_0, \dots, v_m] \mapsto \pm \, d_{v_0} \dotsm \, d_{v_m} [0, \dots, m],
\]
where the sign is that of the shuffle permutation.

\anibal{What is the structure on $\sSet_+$?}

\anibal{What is the bar construction of this algebra?}

\federico{If $X$ is an augmented simplicial set, there is a join coproduct $\chains(X)\to \chains(X)^{\otimes r}$ that sends a simplex to all its join decompositions. This is like the Alexander-Whitney diagonal, but the decomposition is into \emph{non}-overlapping intervals. With the augmented degree convention the Alexander-Whitney diagonal has degree $r-1$, and the join coproduct has degree $0$. In our case (the operations built in this paper), the operation that we get in degree $0$ sends $\tau$ to $\tau\otimes d_{\mathrm{all}}\tau\otimes\ldots\otimes d_{\mathrm{all}}\tau$ (here $d_{\mathrm{all}}(\tau)$ is the maximal generalised face map applied to $\tau$). In particular, for $\Delta_+^n$, it sends $\tau$ to $\tau\otimes \emptyset\otimes \ldots\otimes \emptyset$
}

\section{Symmetric comultiplications} Let $\mu\colon W_*(r)\otimes N_*(\Delta^n)\lra N_*(\Delta^n)^{\otimes r}$ be the composition 
\[
\mu =  \alpha\circ\beta\circ\gamma\circ\Psi^{\vee}\circ\varphi.
\]
Let $\mu_k\colon N_*(\Delta^n)\to N_*(\Delta^n)^{\otimes r}$ be defined as $\mu_k(\tau) = \mu(e_k\otimes \tau)$.
\begin{theorem} Given a pair barycentric subdivision map $h$, the family $\{\mu_k\}$ is a symmetric comultiplication in $N_*(\Delta^n;\bZ)$.
\end{theorem}
\federico{Here $\bZ$ coefficients are allowed because of the factor $(\tilde{r}!)^m$ in the definition of Lemma \eqref{lemma:omegar} that I did not notice until the very end}
\begin{example}
    From Examples \ref{ex:101},\ref{ex:102},\ref{ex:103},\ref{ex:104},\ref{ex:105},\ref{ex:106} and \ref{ex:107} we deduce that, for $r=3$, the coefficient of $[0,2,3]\otimes [3,4,5]\otimes [5,7,9]$ in $\mu_0([0,2,3,4,5,7,9])$ is $-1$.
\end{example}
Let us give a self-contained description of these comultiplication maps: Recall that the last pivot of a word $A = (A_0,\ldots,A_q)\in (EC_r)_q$ is the entry $A_{q-r+1}$. Recall that if $U\in C^q(\Delta^n)$ and $A\in C^q(EC_r)$ are generators, a piece in $A$ is a sequence $A_i,A_{i+1},\ldots,A_{i+l}$ such that $u_{i-1}<u_i = u_{i+1} = \ldots = u_{i+k} < u_{i+k+1}$. Write $A_1$ for the piece that contains the last pivot and $A_2$ for the part of $A$ to the right of $A_1$ and $\hat{A}$ for the part of $A$ to the left of $A_1$. Recall that a pieced word is a word with a decomposition into subwords of length $<r$.

\begin{definition} A pieced word $A$ of length $<r$ is \emph{admissible} if there is a permutation $\sigma$ such that $(A_{\sigma(0)},A_{\sigma(1)},\ldots,A_{\sigma(q)})$ is an increasing sequence and $A_{i}$ has the same parity as $i$. Its sign is the sign of the permutation $\sigma$.

For longer pieced words the definition is recursive: a word of length $\geq r$ is \emph{admissible} if $h(w_1,w_1\smallsetminus w_2)$ is non-zero and $\hat{A}*h(w_1,w_1\smallsetminus w_2)$ is admissible. The sign of $A$ is the product of the signs SIGNS OF THE MAP FROM THE JOIN TO THE BARYCENTRIC SUBDIVISION.
\end{definition}
\begin{definition}
    A pair $(U,A)\in N^*(\Delta^n\times EC_r)$ is \emph{compatible} if the pieced word $A$ is admissible.
\end{definition}
\begin{definition}
    The \emph{sign} of a pair $(U,A)\in N^*(\Delta^n\times EC_r)$ is the product of the following signs:
    \begin{itemize}
    \item the sign of $A$.
    \item The sign of the permutation that orders $A$ [I have the impression that it almost cancels with the sign of A].
    \item $\mu(U_0,\ldots,U_{r-1},\tau)$
    \item $\lambda(U)$ (the sum of the entries of $U$)
    \item the sign of $\mathrm{rev}$.
\end{itemize}
\end{definition}
\begin{example}
    ($r=3$) Recall that a word of length $2\ell+s$ in $EC_3$ has a canonical block decomposition into $\ell+1$ overlapping blocks, the first having length $s$ and the rest having length $3$. A block $(a_1,a_2,a_3)$ is \emph{ascending} if $a_2=a_1+1$ and $a_3=a_2+1$. A pieced word $A\in EC_3$ is \emph{admissible} if no block as repeated elements and every block containing a piece of length $2$ is ascending. The sign of $A$ is the number of non-ascending blocks in $A$.
\end{example}
The comultiplication map $\mu_k$ is given by
\[\mu_k(\tau) = \sum_{\begin{array}{c}\text{\footnotesize $(U,A)$ compatible }\\ \text{ \footnotesize of length $(r-1)m-k$}\end{array}} (-1)^{s(U,A)}(d_{U^{r-1}_{A}}(\tau)\otimes d_{U^{r-2}_{A}}(\tau)\otimes \ldots \otimes d_{U^{0}_{A}}(\tau) )\]

We have empirically found that our formulas coincide with the formulas in [ANIBAL] in arity $3$, and that they are different from the formulas in [ANIBAL] in arity $5$.

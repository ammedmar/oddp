% !TEX root = ../oddp.tex

\section{Preliminaries}\label{s:preliminaries}
%\federico{Convention: $N_*(X)$ the usual chain complex. $\chains_*(X)$ the shifted up chain complex. $\NN_*(X)$ with the special cosimplicial structure. Same for $W(r)$ and $\rW(r)$ and for the linearization $\abel(X)$ and $\rA(X)$. }

Let $r$ be a positive integer, and let $R$ be a commutative ring such that $\bF_r$ is a $R$-algebra. Let $\Mod{R}$ be the category of $R$-modules and let $\Ch{R}$ be the category of chain complexes over $R$.

\subsection{Chain complexes} There are two suspension functors in the category of chain complexes, that we denote by $\sus{}$ (right suspension) and $\susp{}$ (left suspension), defined on a chain complex $(C_*,\partial_*)$ as
\begin{align*}
	\sus{} C_i &= C_{i-1} & \sus{} \partial_i = \partial_{i-1} \\
	\susp{} C_i &= C_{i-1} & \susp{} \partial_i = -\partial_{i-1}. \\
\end{align*}
Consequently, there are suspension isomorphisms $(C_*,\partial)\to (\sus{} C_*,\sus{}\partial)$ and $(C_*,\partial)\to (\susp{} C_*,\susp{}\partial)$ given by $x\mapsto x$ and $x\mapsto (-1)^{|x|} x$.

\begin{convention}
	The chain complexes with their natural grading will be denoted with italic letters $C_*$. Often it will be convenient to work with their suspension $\sus{}C_*$, that will be denoted with an upright symbol $\rC_*$. This has the following consequence:
	\begin{itemize}
		\item A homomorphism $C_*\to C_*^{\otimes r}$ of degree $k$ with the classical convention, becomes, after suspension, a homomorphism $\rC_*\to \rC_*^{\otimes r}$ of degree $k+r-1$.
	\end{itemize}
\end{convention}

Most of the complexes in this work will be finitely generated free complexes with a prescribed basis.
The linear dual of a complex $C_*$ is denoted $C^*$ and the linear dual of a chain map $f$ will be denoted $f^\vee$. Cochain complexes will be considered as negatively graded chain complexes. This allows to define chain maps and tensor products between chain complexes and cochain complexes.

We will denote the dual basis with the same letter as the original basis unless we want to emphasize the duality, in which case we will denote the dual of $u$ by $u^\vee$. There are two conventions for the differential in the dual complex $(C^*,\delta)$:
\begin{align*}
	(\delta f)(\tau) &= (-1)^{|f|+1} f(\partial \tau) \\
	%(\delta^\dagger f)(\tau) &= (-1)^{|\tau|} f(\partial \tau) \\
	(\delta^\dagger f)(\tau) &= f(\partial \tau)
\end{align*}
The first differential makes the evaluation maps $C^*\otimes C_*\to R$ and $C_*\otimes C^*\to R$ chain maps, and will be used from now on. The second one is not compatible with the evaluation maps. There is a chain isomorphism
\begin{align*}
	(C^*,\delta) &\lra (C^*,\delta^\dagger) & f&\mapsto (-1)^{\binom{|f|+1}{2}} f
\end{align*}

%The letter $r$ will denote a positive odd number, whereas the letter $p$ will always stand for a prime number.
\subsection{Categories}

Let $\cC$ be a category with finite coproducts endowed with the following structure:
\begin{itemize}
	\item A \emph{$r$-cyclic distributive monoidal product}, i.e., an associative monoidal product $\otimes$ that distributes over the coproduct and a natural endomorphism of the $r$-fold tensor product
	\[
	\rho\colon Y_1\otimes  \overset{r}{\ldots}\otimes Y_r\lra Y_r\otimes Y_1\otimes Y_2\otimes \ldots \otimes Y_{r-1}
	\]
	such that $\rho^r$ is the identity.
	\item A \emph{natural cyclic diagonal}, i.e., a natural transformation from the identity to the $r$-fold tensor product
	\[
	\Delta\colon Y\lra Y\otimes \overset{r}{\ldots}\otimes Y
	\]
	such that $\rho\circ \Delta = \Delta$.
	\item a \emph{cyclic linearization functor}, i.e., a coproduct preserving monoidal functor
	\[
	\abel\colon \cC\lra \Mod{R}
	\]
	such that the following diagram commutes
	\[
	\xymatrix{
		L(Y_1\otimes \overset{r}{\ldots} \otimes Y_r)  \ar[r]\ar[d]^{\rho} & \abel(Y_1)\otimes \overset{r}{\ldots} \otimes \abel(Y_r)\ar[d]^{\rho} \\
		L(Y_{r}\otimes \overset{r}{\ldots} \otimes Y_{r-1})  \ar[r] & \abel(Y_r)\otimes \overset{r}{\ldots} \otimes \abel(Y_{r-1}) }
	\]
\end{itemize}
The main example of this work will be the symmetric monoidal category $\Setp$ of pointed sets with the smash product or its subcategory $\Set$ of sets with the cartesian product. The functor $\abel$ takes a set to the free module on it, and a pointed set to the quotient of its free module by the basepoint.

\subsection{Augmented simplicial objects}

For each integer $n\geq 1$, let $[n] = \{0,\ldots,n\}$ denote the $n$th ordinal, with $[-1]$ denoting the empty ordinal.

\begin{definition}
	The \emph{augmented simplex category} $\asimplex$ is the category of finite ordinals and order-preserving maps between them. The \emph{augmented semi-simplex category} $\asimplexinj$ is the category of finite ordinals and injective order-preserving maps.
\end{definition}

An injective map $f\colon [n]\to [m]$ is called a \emph{face map}, while a surjective map is called a \emph{degeneracy}. A face map $f$ is determined by the subset $U = [m]\smallsetminus f([n])$, and is denoted $\delta_U$. If $U$ has a single element $i$, we will write $\delta_i$ instead of $\delta_{\{i\}}$. There are $n+1$ surjections $\{\sigma_i\colon [n+1]\to [n]\}_{i=0}^n$ characterised by $\sigma_i(i) = \sigma_i(i+1)$. Every map in this category factors as an injection followed by a surjection.

The category $\simplex_+$ has a strict monoidal product called the \emph{join}, that sends a pair of ordinals $[n],[m]$ to the ordinal $[n+m+1]$ and a pair of functions $f\colon [n]\to [s]$, $g\colon [m]\to [r]$ to the function $f*g\colon [n+m+1]\to [s+r+1]$ defined as $f(x) = x$ if $x\leq  n$ and $f(x) = s+1+f(x-n)$ if $x\geq n$. The unit is the empty ordinal $[-1]$.

\begin{definition}
	An \emph{augmented simplicial object in $\cC$} is a functor $X\colon \asimplex^\mathrm{op}\to \cC$. An \emph{augmented semi-simplicial object in $\cC$} is a functor $X\colon \asimplexinj^\mathrm{op}\to \cC$.
\end{definition}

The object $X([n])$ is denoted $X_n$. The morphisms $X(\delta_U)$ and $X(\sigma_i)$ are denoted $d_U$ and $s_i$ respectively.

The inclusion $\simplex\subset \asimplex$ of the non-empty ordinals induces a restriction functor $\cC^{\asimplex^{\op}}\to \cC^{\simplex^{\op}}$ to the category of simplicial objects on $\cC$ that forgets $X_{-1}$. It has a left adjoint that sets $X_{-1}$ to be the colimit of $\cC$ and a right adjoint that sets $X_{-1}$ to be the terminal object of $\cC$.

\footnote{\federico{this may be useful, but currently is not used}The join product in the augmented simplicial category induces a monoidal product on the category of augmented simplicial sets: If $F$ and $G$ are augmented pointed simplicial sets, consider the functor $F\times G\colon \Delta_+^\op\times \Delta_+^\op\to \Setp$ that sends a pair $([n],[m])$ to the pointed set $F[n]\wedge G[m]$. Then define $F*G\colon \Delta_+^\op\to \Setp$ as the left Kan extension of this functor along the opposite of the join product $\Delta_+^\op\times \Delta_+^\op\to \Delta_+^\op$.}

If $X$ is an augmented pointed (semi-)simplicial object, postcomposing with the linearization functor yields a simplicial $R$-module $\abel(X)$. Define the \emph{complex of unnormalised chains $C_*(X;R)$ with coefficients in $R$} as
\begin{align*}
	C_n(X;R) &= \abel(X)_n
	&
	\partial(x) &= \sum_{j=0}^{n} (-1)^jd_j(x)
\end{align*}
If $X$ is an augmented simplicial object, the \emph{complex of normalised chains $\chains_*(X;R)$ with coefficients in $R$} is defined as
\begin{align*}
	N_n(X;R) &= L'(X_n)
	&
	\partial(x) &= \sum_{j=0}^{n} (-1)^jd_j(x)
\end{align*}
where $L'(X_n)$ is obtained from $L(X_n)$ by quotienting the image of the degeneracies.


We will mostly work with their shifted versions
\begin{align*}
	\rA(X) &= \sus{}A(X)  & \uchains_*(X;R) &= \sus{} C_*(X;R) & \chains_*(X;R) &= \sus{} N_*(X;r)
\end{align*}
Their dual cochain complexes are denoted $\uchains^*(X;R)$ and $\chains^*(X;R)$ and the dual of the differential $\partial$ is denoted $\delta$. We will omit the coefficient ring $R$ unless necessary.

%\federico{better: Use the adjoint to the left Kan extension to produce semi-simplicial objects out of simplicial objects. Define a single complex out of semi-simplicial objects. The chains-functor is monoidal with respect to the join and the tensor product.}

The \emph{standard augmented $n$-simplex} is the augmented simplicial set $\simplex^n_+$ given by $[m]\to \mathrm{Mor}([m],[n])$. The normalised chain complex $N_*(\asimplex^n)$ is generated by the non-degenerate simplices, which are indexed by the inclusions $\partial_U\colon [m]\to [n]$. The simplex indexed by $\partial_U$ is denoted $[v_0,\ldots,v_{m}]$, where $\{v_i\}\subset \{0,\ldots,n\}$ is the image of $\partial_U$.

If $\cC$ has a zero object $*$ (for example, $\Setp$), define the left and the right suspensions of two semi-simplicial objects in $\cC$ as
\begin{align*}
	\susp{} X_n &= X_n & \susp{}d_i &= \begin{cases} d_{i-1} & \text{if $1\leq i<n$} \\
		\ast & \text{if $i=0$}
	\end{cases}
	\\
	\sus{} X_n &= X_n & \sus{}d_i &= \begin{cases} d_i & \text{if $0\leq i<n-1$} \\
		\ast & \text{if $i=n-1$}
	\end{cases}
\end{align*}
Taking normalised chains converts each of these suspensions into the left and right suspensions of chain complexes.

The product $X\otimes Y$ of two simplicial objects $X,Y$ on $\cC$ is given by $(X\otimes Y)_n = X_n\otimes Y_n$ and $d_i(\tau,\tau') = (d_i\tau,d_i\tau')$ and $s_i(\tau,\tau') = (s_i\tau,s_i\tau')$.

\subsection{The bar/Milgram resolution of $\Cyc_r$}

Recall the bar/Milgram construction applied to the cyclic group $\Cyc_r\cong \{0,1,\ldots,r-1\}$, which yields a contractible augmented simplicial set $\EC_r$ with a free $\Cyc_r$-action:
\begin{align*}
	(\EC_r)_k &= \Cyc_r^{k+1}\\
	d_i(a_0\ldots a_k) &= (a_0\ldots\hat{a}_i\ldots a_k) \\
	s_i(a_0\ldots a_k) &= (a_0\ldots a_i,a_i\ldots a_k)
\end{align*}

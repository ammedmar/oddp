% !TEX root = ../oddp.tex

\section{Preliminaries}\label{s:preliminaries}

Let $r$ be a positive integer, and let $R$ be a commutative ring such that $\bF_r$ is a $R$-algebra. Let $\Mod{R}$ be the category of $R$-modules.




\subsection{Chain complexes and sign conventions}

A \emph{chain complex} is a graded $R$-module $C$ with a graded endomorphism $\partial \colon C\to C$ such that $\partial\circ \partial =0$. This graded homomorphism will usually have degree $-1$, but not necessarily (we will stress it when it is not the case).  

If $(C,\partial)$ and $(D,\partial)$ are chain complexes, then a \emph{left chain map of degree $j$} is a graded map $f\colon C\to D$ of degree $|f|=j$ such that $f\circ \partial = (-1)^{|\partial|\cdot |f|}\partial \circ f$. A \emph{right chain map of degree $j$} is a graded map $f\colon C\to D$ of degree $j$ such that $f\circ \partial = \partial \circ f$. Chain maps will usually have degree $0$, but not necessarily (we will stress it when it is not the case). Let $\Ch{R}$ be the category of chain complexes over $R$ and chain maps between them.


\subsubsection{Tensor products}	If $A$ and $B$ are chain complexes whose differentials have the same degree, their tensor product $A \ot B$ is the graded tensor product of $A$ and $B$ with differential $\partial(a \ot b) = \partial a \ot b + (-1)^{|a|}a \ot \partial b$.

If $(C,\partial)$ is a chain complex, its left suspension $\susp{}C$ is the chain complex with differential $\susp{}\partial = -\partial$. Its right suspension $\sus{}C$ is the chain complex with differential $\sus{}\partial = \partial$. If $R[k]$ denotes the chain complex that consists of a single copy of $R$ in degree $k$, there are isomorphisms 
\begin{align}\label{eq:401}
	\sus{k} \ucadenas &\cong \ucadenas \ot R[k] & \susp{k} \ucadenas \cong R[k] \ot \ucadenas.
\end{align}
that send $a$ to $a \ot 1$ and to $1 \ot a$ respectively. Left and right suspensions are functors that act on a chain map $f$ as $(\sus{}f)(a\ot 1) = f(a)\ot 1$ and $(\susp{} f)(1\ot a) = 1\ot f(a)$.

Each notion of suspension yields a notion of chain map $A \to B$ of degree $k$: A right (left) chain map is related to a chain map $A \to \sus{-k} B$ ($A \to \susp{-k} B$) or degree $0$.% but keep in mind that dualizing interchanges both notions. One can switch between them using the isomorphism
%\[
%\sus{k}\ucadenas \lra \susp{k}\ucadenas
%\]
%that sends $x$ to $(-1)^{k|x|}x$. %It is therefore advisable to write a map $f \colon A \to B$ of degree $k$ with the first notion as $f \colon \sus{k}A \to B$ or $f \colon A \to \sus{-k}B$ and with the second notion as $f \colon \susp{k}A \to B$ or $f \colon A \to \susp{-k}B$.

\subsubsection{Linear dual} The linear dual of a complex $\ucadenas$ is denoted $\ucocadenas$. The differential in the dual complex $(\ucocadenas,\partial^\dd)$ is
\begin{align*}
	(\partial^\dd f)(\tau) &= (-1)^{|f|+1} f(\partial \tau). 
\end{align*}Cochain complexes will be considered as negatively graded chain complexes. This allows to define chain maps and tensor products between chain complexes and cochain complexes.

Most of the chain complexes in this work will be finitely generated free complexes with a prescribed basis. We will denote the dual of a basis element $u$ by $u^\dd$. 


The linear dual of a chain map $f$ will be denoted $f^{\dd}$.  
There is an isomorphism $A^\dd \ot B^\dd \to (B \ot A)^{\dd}$ that sends $f \ot g$ to $h$ where $h(b \ot a) = g(b)\cdot f(a)$. As a consequence of this and \eqref{eq:401}, there is an isomorphism
\begin{align*}
	(\sus{}\ucadenas )^\dd &\cong \susp{-1}\ucocadenas
\end{align*}
that sends a cochain $f$ to the cochain $\susp{}f$ defined as $(\susp{}f)(a) = f(a\ot 1)$.
\subsubsection{Tensor products of chain maps with the right suspension}

Given chain maps $f \colon A \to \sus{-|f|}C$ and $g \colon B \to \sus{-|g|}D$, the chain map
\[
f \ot g \colon A \ot B \to \sus{-|f|-|g|}C \ot D
\]
is defined as $(f \ot g)(a \ot b) = (-1)^{|f||b|} f(a) \ot g(b)$. Given chain maps
\begin{align*}
	A & \overset{f_1}{\lra} \sus{-|f_1|} C \overset{f_2}{\lra} \sus{-|f_1|-|f_2|} E \\
	B & \overset{g_1}{\lra} \sus{-|g_1|} D \overset{g_2}{\lra} \sus{-|g_1|-|g_2|} F \\
\end{align*}
then the chain maps
\begin{align*}
	(f_2\circ f_1) \ot (g_2\circ g_1) \colon &A \ot B \lra \sus{-|f_1|-|f_2|-|g_1|-|g_2|}E \ot F
	\\
	(f_2 \ot g_2)\circ (f_1 \ot g_1) \colon &A \ot B \lra \sus{-|f_1|-|f_2|-|g_1|-|g_2|}E \ot F
\end{align*}
are related by
\begin{align*}
	(f_2\circ f_1) \ot (g_2\circ g_1) &= (-1)^{|f_2||g_1|}(f_2 \ot g_2)\circ (f_1 \ot g_1)
\end{align*}

\subsubsection{Tensor products of chain maps with the left suspension}

Given chain maps $f \colon A \to \susp{-|f|}C$ and $g \colon B \to \susp{-|g|}D$, the chain map
\[
f \ot g \colon A \ot B \to \susp{-|f|-|g|}C \ot D
\]
is defined as $(f \ot g)(a \ot b) = (-1)^{|g||a|} f(a) \ot g(b)$. Given chain maps
\begin{align*}
	A & \overset{f_1}{\lra} \susp{-|f_1|} C \overset{f_2}{\lra} \susp{-|f_1|-|f_2|} E \\
	B & \overset{g_1}{\lra} \susp{-|g_1|} D \overset{g_2}{\lra} \susp{-|g_1|-|g_2|} F \\
\end{align*}
then the chain maps
\begin{align*}
	(f_2\circ f_1) \ot (g_2\circ g_1) \colon &A \ot B \lra \susp{-|f_1|-|f_2|-|g_1|-|g_2|}E \ot F
	\\
	(f_2 \ot g_2)\circ (f_1 \ot g_1) \colon &A \ot B \lra \susp{-|f_1|-|f_2|-|g_1|-|g_2|}E \ot F
\end{align*}
are related by
\begin{align*}
	(f_2\circ f_1) \ot (g_2\circ g_1) &= (-1)^{|f_1||g_2|}(f_2 \ot g_2)\circ (f_1 \ot g_1)
\end{align*}

%\begin{notation}
	%We will often use an upright symbol (for example, $\uchains$) to denote a variation of a previously built chain complex $\ucadenas$. For example, in \cref{s:unstable} we will introduce a chain complex $\W(r)$ while in \cref{s:connected} we will introduce a variation $\rW(r)$: the right suspension of the augmented version of $\W(r)$. In \cref{s:simplices} we introduce the normalized chain complex $\cadenas(X)$ while in \cref{s:alexander} we introduce a variation $\chains(X)$: the right suspension of $\cadenas(X)$.
%\end{notation}

\subsection{Categories}\label{s:categories}

Let $\Cyc_r$ be the cyclic group $\langle\rho\mid \rho^r\rangle$.
\begin{definition}
	A \emph{linearized category with $r$-fold diagonals} is a monoidal category $\cC$ endowed with the following structure:
	\begin{itemize}
		\item a natural transformation between $r$-fold tensor products
		\[
		\rho \colon Y_1 \ot \overset{r}{\dots} \ot Y_r \lra Y_r \ot Y_1 \ot Y_2 \ot \dots \ot Y_{r-1}
		\]
		such that $\rho^r$ is the identity.
		\item A \emph{natural cyclic diagonal}, i.e., a natural transformation from the identity to the $r$-fold tensor product
		\[
		\Delta \colon Y \lra Y \ot \overset{r}{\dots} \ot Y
		\]
		such that $\rho\circ \Delta = \Delta$.
		\item a \emph{cyclic linearization functor}, i.e., a strong monoidal functor
		\[
		\abel \colon (\cC, \ot ) \lra (\Mod{R}, \ot )
		\]
		such that the following diagram commutes
		\[
		\xymatrix{
			L(Y_1 \ot \overset{r}{\dots} \ot Y_r) \ar[r]\ar[d]^{\rho} & \abel(Y_1) \ot \overset{r}{\dots} \ot 	\abel(Y_r)\ar[d]^{\rho} \\
			L(Y_{r} \ot \overset{r}{\dots} \ot Y_{r-1}) \ar[r] & \abel(Y_r) \ot \overset{r}{\dots} \ot \abel(Y_{r-1}) }
		\]
	\end{itemize}
\end{definition}
The main example of this work will be the symmetric monoidal category $\Setp$ of pointed sets with the smash product or its subcategory $\Set$ of sets with the Cartesian product. The functor $\abel$ takes a set to the free module on it, and a pointed set to the quotient of its free module by the base point. The following is an example of a category that has only odd diagonals.

%The category of pointed sets and signed maps $\Setp^\pm$ has as objects pointed sets, and a morphism from $A$ to $B$ is a pair $(f,f_0)$ where $f\colon A\to B$ is a function of pointed sets and $f_0\colon A\to \{0,1\}$ is a function. The composition of $(f,f_0)$ and $(g,g_0)$ is the pair $(h,h_0)$ where $h = g\circ f$ and $h_0(a) = f_0(a) + g_0(a)$. 

\begin{example} The category of pointed sets and signed maps $\Setp^\pm$ has as objects pointed sets, and a morphism from $X$ to $Y$ is a map $f\colon X\times \{+,-\}\to Y\times \{+,-\}$ that is equivariant with respect to the involution that interchanges $+$ and $-$. Maps compose as expected.

The tensor product is the smash product of pointed sets $X\wedge Y = X\times Y / (x_0\times Y\cup X\times y_0)$. If $f\colon X\to A$ and $g\colon Y\to B$ are maps, their tensor product is the composition
\[
	X\times Y \times \{+,-\} \to X\times \{+,-\}\times Y\times \{+,-\} \overset{f\times g}{\to}
	A\times \{+,-\}\times B\times \{+,-\} \to A\times B \times \{+,-\}
\] 
where the first map sends a triple $(x,y,\pm)$ to the quadruple $(x,+,y,\pm)$ and the last map sends a quadruple $(a,+,b,\pm)$ to $(a,b,\pm)$, and extends equivariantly.


If $r$ is odd, then every object $X$ has a natural diagonal $\Delta^r\colon X\to X\wedge \ldots \wedge X$ given by $\Delta^r(x,+) = (x,\ldots,x,+)$. The delicate point is naturality: if $f\colon X\to Y$ is a signed map, then $\Delta^r\circ f(x) = f^{\otimes r}\circ \Delta^r(x)$ if and only ir either $f(x)$ has positive sign, or $f(x)$ has negative sign and $r$ is odd. Thus, diagonal maps are well-defined only for odd $r$.

Finally, there is a linearization functor $\abel\colon \Setp^\pm\to \Mod{R}$ that sends a set $X$ to the quotient $R\langle X\rangle/\langle x_0\rangle$ and a map $f$ to the map $g$ given on generators by $g(x) = f_2(x)\cdot f_1(x)$, where $f_1(x)\in Y$ is the first component of $f$ and $f_2(x)\in \{+,-\}$ is the second component.
\end{example}
\subsection{Augmented simplicial objects}\label{s:simplices}

For each integer $n \geq 0$, denote the $n$th ordinal by $[n] = \{0,\dots,n-1\}$ , with $[0]$ denoting the empty ordinal. 

\begin{warning}
	This is not the standard notation, but is very convenient for the join product.
\end{warning}

\begin{definition}
	The \emph{augmented simplex category} $\asimplex$ is the category of finite ordinals and order-preserving maps between them. 
\end{definition}

An injective map $f \colon [n] \to [m]$ is called a \emph{face map}, while a surjective map is called a \emph{degeneracy}. A face map $f$ is determined by the subset $U = [m]\smallsetminus f([n])$, and is denoted $d^U$. If $U$ has a single element $i$, we will write $d^i$ instead of $d^{\{i\}}$. There are $n$ surjections $s^i \colon [n+1] \to [n]$, $i = 0\dots n-1$ characterized by $s^i(i) = s^i(i+1)$. Every map in this category factors as an injection followed by a surjection.

The category $\asimplex$ has a strict monoidal product called the \emph{join}, that sends a pair of ordinals $[n],[m]$ to the ordinal $[n+m]$ and a pair of functions $f \colon [n] \to [q]$, $g \colon [m] \to [p]$ to the function $f*g \colon [n+m] \to [q+p]$ defined as $f(x) = x$ if $x< n$ and $f(x) = q+f(x-n)$ if $x \geq n$. The unit is the empty ordinal $[0]$.

\begin{definition}
	An \emph{augmented simplicial object in $\cC$} is a functor $X \colon \asimplex^\mathrm{op} \to \cC$.
	An \emph{augmented cosimplicial object in $\cC$} is a functor $X \colon \asimplex \to \cC$.
\end{definition}

If $X$ is an augmented simplicial object, $X([n])$ is denoted $X_n$ and the morphisms $X(d^U)$ and $X(s^i)$ are denoted $\d_U$ and $s_i$ respectively. If $X$ is an augmented cosimplicial object, $X([n])$ is denoted $X_n$ and the morphisms $X(d^U)$ and $X(s^i)$ are denoted $d^U$ and $s^i$ respectively.

The inclusion $\simplex \subset \asimplex$ of the non-empty ordinals induces a restriction functor $\cC^{\asimplex^{\op}} \to \cC^{\simplex^{\op}}$ to the category of simplicial objects on $\cC$ that forgets $X_{0}$. It has a left adjoint that sets $X_{0}$ to be the colimit of $X$ and a right adjoint that sets $X_{0}$ to be the terminal object of $\cC$.

If $X$ is an augmented simplicial object in one of the categories of Section \ref{s:categories}, postcomposing with the linearization functor $\abel \colon \cC \to \Mod{R}$ yields an augmented simplicial $R$-module $\abel(X)$.

Define the 
%\emph{complex of unnormalized chains $\ucadenas(X;R)$ with coefficients in $R$} as
%\begin{align*}
	%\ucadenas[n](X;R) &= \abel(X)_n
	%&
	%\partial(x) &= \sum_{j=0}^{n} (-1)^jd_j(x)
%\end{align*}
%
%If $X$ is an augmented simplicial object, the 
\emph{complex of normalized chains $\chains(X;R)$ with coefficients in $R$} is defined as
\begin{align*}
	\chains[n](X;R) &= L'(X)_n
	&
	\partial(x) &= \sum_{j=0}^{n} (-1)^j \d_j(x)
\end{align*}
where $L'(X)_n$ is obtained from $L(X)_n$ by quotienting the image of the degeneracies.

%\begin{notation}\label{notation:chains}
	%We will denote by $\chains(X)$ the right suspension of $\cadenas(X)$.
%\end{notation}

\begin{notation}
	A few times we will be working with normalized chain complexes of simplicial objects that they themselves are part of a cosimplicial or simplicial object. We will denote the ``internal'' face maps that are used to build the differential as $\diff_i$ and the external face maps that yield the cosimplicial or simplicial structure as $\d_i$.\footnote{ver donde se usa eso}
\end{notation}

The \emph{standard augmented $n$-simplex} is the augmented simplicial set $\simplex^n_+$ given by $[m] \to \mathrm{Mor}_{\asimplex}([m],[n+1])$. The normalized chain complex $\chains(\asimplex^n)$ is generated by the non-degenerate simplices, which are indexed by the inclusions $\d_U \colon [m] \to [n+1]$. The simplex indexed by $\d_U$ is denoted $[v_0,\dots,v_{m-1}]$, where $\{v_0,\dots,v_{m-1}\} \subset \{0,\dots,n\}$ is the ordered image of $[m]$ by $\d_U$. If $\perm$ is a permutation of $0,\dots,m-1$, then $[v_{\perm(0)},\dots,v_{\perm(m-1)}]$ will denote the generator $(-1)^{\sign{\perm}}[v_0,\dots,v_{m-1}]$.

\begin{warning}
	It would have been more coherent with our notation to name this augmented simplicial set the $(n+1)$-simplex, but this might be too confusing for the reader. 
\end{warning}

%\subsubsection{Other tensor products} If $(A^\bullet,\partial_A)$ is a cosimplicial chain complex, and $(B_\bullet,\partial_B)$ is a simplicial chain complex, then define the following chain complex:
%
%
%
%are two chain complexes with differentials of different degree, their tensor product is not well-defined. Suppose that the differential in $B$ decomposes as $\partial_B = \partial_1+\ldots+\partial_k$ and that there are chain endomorphisms $\eta_1,\ldots,\eta_k\colon A\to A$ of degree $|\partial_A|-|\partial_B|$. Then one can define a differential of degree $|\partial_A|$ on the graded module $A\otimes B$ as
%\[
	%\partial(a\otimes b) = \partial_A(a)\otimes b + (-1)^{|a||\partial_B|} \sum_j \eta_j(a)\otimes \partial_j(b).
%\]
%We will denote this chain complex by $A\otimes_\eta B$.
%
%
%


% !TEX root = ../oddp.tex

\section{Preliminaries}\label{s:preliminaries}

Let $r$ be a positive integer, and let $R$ be a commutative ring such that $\bF_r$ is a $R$-algebra.
Let $\Mod{R}$ be the category of $R$-modules and let $\Ch{R}$ be the category of chain complexes over $R$.

\subsection{Chain complexes and sign conventions}

Most of the chain complexes in this work will be finitely generated free complexes with a prescribed basis. The linear dual of a complex $\ucadenas$ is denoted $\ucocadenas$ and the linear dual of a chain map $f$ will be denoted $f^\dd$. Cochain complexes will be considered as negatively graded chain complexes. This allows to define chain maps and tensor products between chain complexes and cochain complexes.

We will denote the dual basis with the same letter as the original basis unless we want to emphasize the duality, in which case we will denote the dual of $u$ by $u^\dd$. The differential in the dual complex $(\ucocadenas,\partial^\dd)$ is
\begin{align*}
	(\partial^\dd f)(\tau) &= (-1)^{|f|+1} f(\partial \tau). %\\
	%(\delta^\dagger f)(\tau) &= (-1)^{|\tau|} f(\partial \tau) \\
	%(\delta^\dagger f)(\tau) &= f(\partial \tau)
\end{align*}

\begin{convention}
	If $A$ and $B$ are chain complexes, their tensor product $A \ot B$ has differential $\partial(a \ot b) = \partial a \ot b + (-1)^{|a|}a \ot \partial b$.
\end{convention}
There is an isomorphism $A^\dd \ot B^\dd \to (B \ot A)^{\dd}$ that sends $f \ot g$ to $h$ where $h(b \ot a) = g(b)\cdot f(a)$. There are two suspension functors that we denote by $\sus{}$ (right suspension) and $\susp{}$ (left suspension), defined on a chain complex $(\ucadenas,\partial)$ as
\begin{align*}
	\sus{} \ucadenas[i] &= \ucadenas[i-1] & \susp{} \ucadenas[i] &= \ucadenas[i-1] \\
	\sus{} \partial_i &= \partial_{i-1} & \susp{} \partial_i &= -\partial_{i-1},
\end{align*}
and on a chain map $f$ as $(\sus{}f)(a) = f(a)$ and $(\susp{} f)(a) = f(a)$. If $R[k]$ denotes the chain complex that consists of a single copy of $R$ in degree $k$, then there are isomorphisms
\begin{align*}
	\sus{k} \ucadenas &\cong \ucadenas \ot R[k] & \susp{k} \ucadenas \cong R[k] \ot \ucadenas.
\end{align*}
that send $a$ to $a \ot 1$ and to $1 \ot a$ respectively. As a consequence, there is an isomorphism
\begin{align*}
	(\sus{}\ucadenas )^\dd &\cong \susp{-1}\ucocadenas
\end{align*}
that sends a cochain $f$ to the cochain $\susp{}f$ defined as $(\susp{}f)(a) = f(\sus{}a)$.
Each notion of suspension yields a notion of chain map $A \to B$ of degree $k$: It may be a graded homomorphism $A \to B$ of degree $k$ that commutes with the differential (i.e., a chain map $A \to \sus{-k} B$), or a graded homomorphism $A \to B$ of degree $k$ that commutes with the differential if $k$ is even and anticommutes with the differential if $k$ is odd (i.e., a chain map $A \to \susp{-k}B$). We will mostly work with the second notion, but keep in mind that dualizing interchanges both notions. One can switch between them using the isomorphism
\[
\sus{k}\ucadenas \lra \susp{k}\ucadenas
\]
that sends $x$ to $(-1)^{k|x|}x$. It is therefore advisable to write a map $f \colon A \to B$ of degree $k$ with the first notion as $f \colon \sus{k}A \to B$ or $f \colon A \to \sus{-k}B$ and with the second notion as $f \colon \susp{k}A \to B$ or $f \colon A \to \susp{-k}B$.

\subsubsection{Tensor products of chain maps with the right suspension}

Given chain maps $f \colon A \to \sus{-|f|}C$ and $g \colon B \to \sus{-|g|}D$, the chain map
\[
f \ot g \colon A \ot B \to \sus{-|f|-|g|}C \ot D
\]
is defined as $(f \ot g)(a \ot b) = (-1)^{|f||b|} f(a) \ot g(b)$. Given chain maps
\begin{align*}
	A & \overset{f_1}{\lra} \sus{-|f_1|} C \overset{f_2}{\lra} \sus{-|f_1|-|f_2|} E \\
	B & \overset{g_1}{\lra} \sus{-|g_1|} D \overset{g_2}{\lra} \sus{-|g_1|-|g_2|} F \\
\end{align*}
then the chain maps
\begin{align*}
	(f_2\circ f_1) \ot (g_2\circ g_1) \colon &A \ot B \lra \sus{-|f_1|-|f_2|-|g_1|-|g_2|}E \ot F
	\\
	(f_2 \ot g_2)\circ (f_1 \ot g_1) \colon &A \ot B \lra \sus{-|f_1|-|f_2|-|g_1|-|g_2|}E \ot F
\end{align*}
are related by
\begin{align*}
	(f_2\circ f_1) \ot (g_2\circ g_1) &= (-1)^{|f_2||g_1|}(f_2 \ot g_2)\circ (f_1 \ot g_1)
\end{align*}

\subsubsection{Tensor products of chain maps with the left suspension}

Given chain maps $f \colon A \to \susp{-|f|}C$ and $g \colon B \to \susp{-|g|}D$, the chain map
\[
f \ot g \colon A \ot B \to \susp{-|f|-|g|}C \ot D
\]
is defined as $(f \ot g)(a \ot b) = (-1)^{|g||a|} f(a) \ot g(b)$. Given chain maps
\begin{align*}
	A & \overset{f_1}{\lra} \susp{-|f_1|} C \overset{f_2}{\lra} \susp{-|f_1|-|f_2|} E \\
	B & \overset{g_1}{\lra} \susp{-|g_1|} D \overset{g_2}{\lra} \susp{-|g_1|-|g_2|} F \\
\end{align*}
then the chain maps
\begin{align*}
	(f_2\circ f_1) \ot (g_2\circ g_1) \colon &A \ot B \lra \susp{-|f_1|-|f_2|-|g_1|-|g_2|}E \ot F
	\\
	(f_2 \ot g_2)\circ (f_1 \ot g_1) \colon &A \ot B \lra \susp{-|f_1|-|f_2|-|g_1|-|g_2|}E \ot F
\end{align*}
are related by
\begin{align*}
	(f_2\circ f_1) \ot (g_2\circ g_1) &= (-1)^{|f_1||g_2|}(f_2 \ot g_2)\circ (f_1 \ot g_1)
\end{align*}

\begin{notation}
	We will often use an upright symbol (for example, $\uchains$) to denote a variation of a previously built chain complex $\ucadenas$. For example, in \cref{s:unstable} we will introduce a chain complex $\W(r)$ while in \cref{s:connected} we will introduce a variation $\rW(r)$: the right suspension of the augmented version of $\W(r)$. In \cref{s:simplices} we introduce the normalized chain complex $\cadenas(X)$ while in \cref{s:alexander} we introduce a variation $\chains(X)$: the right suspension of $\cadenas(X)$.
\end{notation}

%\begin{notation}
%The chain complexes with their natural grading will be denoted with italic letters $\ucadenas$. Often it will be convenient to work with their suspension $\sus{}\ucadenas$, that will be denoted with an upright symbol $\uchains$. The notation $\ucochains$ stands for the dual of $\uchains$.
%\end{notation}

\subsection{Categories}\label{s:categories}

Let $\Cyc_r$ be the cyclic group $\langle\rho\mid \rho^r\rangle$.
\begin{definition}
	A linearized category with $r$-fold diagonals is a monoidal category $\cC$ %with finite coproducts
	endowed with the following structure:
	\begin{itemize}
		%\item A \emph{$r$-cyclic distributive monoidal product}, i.e., an associative monoidal product $ \ot $ that distributes over the coproduct $\amalg$ and
		\item a natural transformation between $r$-fold tensor products
		\[
		\rho \colon Y_1 \ot \overset{r}{\dots} \ot Y_r \lra Y_r \ot Y_1 \ot Y_2 \ot \dots \ot Y_{r-1}
		\]
		such that $\rho^r$ is the identity.
		\item A \emph{natural cyclic diagonal}, i.e., a natural transformation from the identity to the $r$-fold tensor product
		\[
		\Delta \colon Y \lra Y \ot \overset{r}{\dots} \ot Y
		\]
		such that $\rho\circ \Delta = \Delta$.
		\item a \emph{cyclic linearization functor}, i.e., a %coproduct preserving
		strong monoidal functor
		\[
		%\abel \colon (\cC,\amalg, \ot ) \lra (\Mod{R},\oplus, \ot )
		\abel \colon (\cC, \ot ) \lra (\Mod{R}, \ot )
		\]
		such that the following diagram commutes
		\[
		\xymatrix{
			L(Y_1 \ot \overset{r}{\dots} \ot Y_r) \ar[r]\ar[d]^{\rho} & \abel(Y_1) \ot \overset{r}{\dots} \ot 	\abel(Y_r)\ar[d]^{\rho} \\
			L(Y_{r} \ot \overset{r}{\dots} \ot Y_{r-1}) \ar[r] & \abel(Y_r) \ot \overset{r}{\dots} \ot \abel(Y_{r-1}) }
		\]
	\end{itemize}
\end{definition}
The main example of this work will be the symmetric monoidal category $\Setp$ of pointed sets with the smash product or its subcategory $\Set$ of sets with the Cartesian product. The functor $\abel$ takes a set to the free module on it, and a pointed set to the quotient of its free module by the base point.

\add{A coalgebra in a monoidal category $(\cD,\otimes)$ is an object $Y$ of $\cD$ together with a diagonal map $\Delta\colon Y\to Y\otimes Y$. The coalgebra is coassociative if $\Delta\circ (\Delta\otimes \id) = (\Delta\circ (\id \otimes \Delta))$.

	If the monoidal category is symmetric, then a coalgebra $(Y,\Delta)$ is cocommutative if $\Delta\circ \twist = \Delta$. If the monoidal category has a unit $I$, one defines a counital coalgebra as a coalgebra $(Y,\Delta)$ together with a morphism $u\colon Y\to I$ such that $\Delta\circ (u\otimes \id) = \Delta\circ (\id\otimes u) = \id$.

	\begin{definition}
		A \emph{linearized category of coalgebras} is a category of coalgebras in a monoidal category with  a strong monoidal functor $L$ to the category of $R$-modules. If the functor is symmetric monoidal or unital, then one can consider linearized categories of cocommutative or counital coalgebras.
	\end{definition}

	A cocommutative coassociative coalgebra $Y$ has a well-defined $r$-fold diagonal $\Delta^r\colon Y\lra Y^{\otimes r}$, which is invariant under the action of $\Cyc_r$.
}

\subsection{Augmented simplicial objects}\label{s:simplices}

For each integer $n \geq -1$, let $[n] = \{0,\dots,n\}$ denote the $n$th ordinal, with $[-1]$ denoting the empty ordinal.

\begin{definition}
	The \emph{augmented simplex category} $\asimplex$ is the category of finite ordinals and order-preserving maps between them. %The \emph{augmented semi-simplex category} $\asimplexinj$ is the category of finite ordinals and injective order-preserving maps.
\end{definition}

An injective map $f \colon [n] \to [m]$ is called a \emph{face map}, while a surjective map is called a \emph{degeneracy}. A face map $f$ is determined by the subset $U = [m]\smallsetminus f([n])$, and is denoted $d^U$. If $U$ has a single element $i$, we will write $d^i$ instead of $d^{\{i\}}$. There are $n+1$ surjections $s^i \colon [n+1] \to [n]$, $i = 0\dots n$ characterized by $s^i(i) = s^i(i+1)$. Every map in this category factors as an injection followed by a surjection.

The category $\asimplex$ has a strict monoidal product called the \emph{join}, that sends a pair of ordinals $[n],[m]$ to the ordinal $[n+m+1]$ and a pair of functions $f \colon [n] \to [q]$, $g \colon [m] \to [p]$ to the function $f*g \colon [n+m+1] \to [q+p+1]$ defined as $f(x) = x$ if $x\leq n$ and $f(x) = q+1+f(x-n)$ if $x \geq n$. The unit is the empty ordinal $[-1]$.

\begin{definition}
	An \emph{augmented simplicial object in $\cC$} is a functor $X \colon \asimplex^\mathrm{op} \to \cC$.
	%An \emph{augmented semi-simplicial object in $\cC$} is a functor $X \colon \asimplexinj^\mathrm{op} \to \cC$.
	An \emph{augmented cosimplicial object in $\cC$} is a functor $X \colon \asimplex \to \cC$.
	%An \emph{augmented cosemi-simplicial object in $\cC$} is a functor $X \colon \asimplexinj \to \cC$.
\end{definition}

If $X$ is an augmented simplicial object, $X([n])$ is denoted $X_n$ and the morphisms $X(d^U)$ and $X(s^i)$ are denoted $\d_U$ and $s_i$ respectively. If $X$ is an augmented cosimplicial object, $X([n])$ is denoted $X_n$ and the morphisms $X(d^U)$ and $X(s^i)$ are denoted $d^U$ and $s^i$ respectively.

The inclusion $\simplex \subset \asimplex$ of the non-empty ordinals induces a restriction functor $\cC^{\asimplex^{\op}} \to \cC^{\simplex^{\op}}$ to the category of simplicial objects on $\cC$ that forgets $X_{-1}$. It has a left adjoint that sets $X_{-1}$ to be the colimit of $X$ and a right adjoint that sets $X_{-1}$ to be the terminal object of $\cC$.

If $X$ is an augmented simplicial object in one of the categories of Section \ref{s:categories}, postcomposing with the linearization functor $\abel \colon \cC \to \Mod{R}$ yields a simplicial $R$-module $\abel(X)$.
%If we further quotient by the degeneracies, we obtain a semi-simplicial $R$-module $\abel(X)$

Define the \emph{complex of unnormalized chains $\ucadenas(X;R)$ with coefficients in $R$} as
\begin{align*}
	\ucadenas[n](X;R) &= \abel(X)_n
	&
	\partial(x) &= \sum_{j=0}^{n} (-1)^jd_j(x)
\end{align*}

If $X$ is an augmented simplicial object, the \emph{complex of normalized chains $\cadenas(X;R)$ with coefficients in $R$} is defined as
\begin{align*}
	\cadenas[n](X;R) &= L'(X)_n
	&
	\partial(x) &= \sum_{j=0}^{n} (-1)^jd_j(x)
\end{align*}
where $L'(X)_n$ is obtained from $L(X)_n$ by quotienting the image of the degeneracies.

\begin{notation}\label{notation:chains}
	We will denote by $\chains(X)$ the right suspension of $\cadenas(X)$.
\end{notation}

\begin{notation}
	A few times we will be working with normalized chain complexes of simplicial objects that they themselves are part of a cosimplicial or simplicial object. We will denote the ``internal'' face maps that are used to build the differential as $\diff_i$ and the external face maps that yield the cosimplicial or simplicial structure as $\d_i$.
\end{notation}

The \emph{standard augmented $n$-simplex} is the augmented simplicial set $\simplex^n_+$ given by $[m] \to \mathrm{Mor}_{\asimplex}([m],[n])$. The normalized chain complex $\cadenas(\asimplex^n)$ is generated by the non-degenerate simplices, which are indexed by the inclusions $\d_U \colon [m] \to [n]$. The simplex indexed by $\d_U$ is denoted $[v_0,\dots,v_{m}]$, where $\{v_0,\dots,v_{m}\} \subset \{0,\dots,n\}$ is the ordered image of $[m]$ by $\d_U$. As usual, if $\perm$ is a permutation of $0,\dots,m$, then $[v_{\perm(0),\dots,v_{\perm(m)}}]$ will denote the generator $(-1)^{\sign{\perm}}[v_0,\dots,v_m]$.

%If $\cC$ has a zero object $*$ (for example, $\Setp$), define the left and the right suspensions of two semi-simplicial objects in $\cC$ as {\color{red}KAN SUSPENSIONS}
%\begin{align*}
%\susp{} X_n &= X_n & \susp{}\d_i &= \begin{cases} \d_{i-1} & \text{if $1\leq i<n$} \\
	%\ast & \text{if $i=0$}
	%\end{cases}
	%\\
	%\sus{} X_n &= X_n & \sus{}\d_i &= \begin{cases} \d_i & \text{if $0\leq i<n-1$} \\
		%\ast & \text{if $i=n-1$}
		%\end{cases}
		%\end{align*}
		%Taking normalized chains converts each of these suspensions into the left and right suspensions of chain complexes.
		%
		%The product $X \ot Y$ of two simplicial objects $X,Y$ on $\cC$ is given by $(X \ot Y)_n = X_n \ot Y_n$ and $\d_i(\tau,\tau') = (\d_i\tau,\d_i\tau')$ and $s_i(\tau,\tau') = (s_i\tau,s_i\tau')$.

\subsection{The Milgram resolution of a group \texorpdfstring{$G$}{G}}

Recall the that the Milgram construction applied to a group $G$, yields a contractible augmented simplicial set $\EG$ with a free $G$-action:
\begin{align*}
	(\EG)_k &= G^{k+1}\\
	\d_i(g_0\dots g_k) &= (g_0\dots\hat{g}_i\dots g_k) \\
	\s_i(g_0\dots g_k) &= (g_0\dots g_i,g_i\dots g_k)
\end{align*}

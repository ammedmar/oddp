% !TEX root = ../oddp.tex

\subsection{The bar resolution}\label{ss:milnor}
There is an inclusion $\chains(\EC_r) \to \Omega_*(r)^{\nf}$ that sends a tuple $(a_0,a_1,\dots,a_k)$ to the pieced word $(a_0|a_1|\dots|a_k)$. The composition
\[
\chains(\EC_r) \lra \Omega_*(r)^{\nf} \lra \rW(r)
\]
is an equivariant chain map. This composition has a description analogous to the one used to define $\Psiom$, using the following suspension map:
\[
S(a_0,\dots,a_k) = \begin{cases}
	(-1)^{\sign{\perm}} (a_{0},\dots,a_{k-r+1}) & \text{ if $k \geq r-1$} \\
	\emptyset & \text{if $k = r-2$ and $a_i\neq a_j$ for all $i,j$} \\
	0 & \text{otherwise.}
\end{cases}
\]
where $\pi$ is the permutation that orders $(a_{k-r+1},\dots,a_{k})$. The image of $e_{-q}^{\vee}$ is the sum of all words $A = (a_0,\dots,a_{q-1})\in \cochains[-q] \EC_r$ such that
\begin{itemize}
\item none of the subwords of the form $(a_{q-1-\ell(r-1)},a_{q-1-(\ell-1)(r-1))}$ of length $r$ contains repeated elements,
\item the only subword of the form $(a_0,\dots,a_{q-1-\ell(r-1)})$ of length smaller than $r$ becomes, after having been ordered, an increasing sequence that alternates even and odd entries different from $r-1$, starting with an even entry.
\end{itemize}
each word has the sign needed to order each of the subwords mentioned above, and the coefficient $\frac{\varphi(m)!}{(\tilde{r}!)^j}$ where $m$ is the length of the last of the subwords mentioned, and $j$ is the total number of subwords and $\varphi(m) = \lfloor\frac{r-m-1}{2}\rfloor$.
\begin{example} If $r=3$,
\begin{align*}
	\Psi^\vee(e_{-1}^\vee) &= (0)
	\\
	\Psi^\vee(e_{-2}^{\vee}) &= (0,1)-(1,0)
	\\
	\Psi^\vee(e_{-3}^{\vee}) &= (0,1,2) - (0,2,1)
\\
	\Psi^\vee(e_{-4}^{\vee}) &= (0,1,2,0) - (0,1,0,2) + (1,0,2,1) - (1,0,1,2)
\end{align*}
if $r= 5$,
\begin{align*}
	\Psi^\vee(e^\vee_{-1}) &= \frac{1}{2}\left((0) + (2)\right)
	\\
	\Psi^\vee(e^\vee_{-2}) &= \frac{1}{2}\left((0,1) - (1,0) + (0,3) - (3,0) + (2,3) + (3,2)\right)
	\\
	\Psi^\vee(e^\vee_{-3}) &= \frac{1}{2}\left((0,1,2) -(0,2,1) + (1,2,0) - (1,0,2) + (2,0,1) - (2,1,0)\right)
\end{align*}
$\Psi^\vee(e^\vee_{-4})$ is the signed sum of all permutations of $(0,1,2,3)$, so it has $24$ summands, again with coefficient $\frac{1}{2}$.
\end{example}

This map is different from the one obtained in \cite[Prop.~6.16]{brumfiel2023explicit}, and in fact that map cannot be used to construct $\Psiom$ following the recipe of Definition \ref{def:psiom}.


\subsection{The even prime}\label{ss:even} If $r= 2$ and $R=\bF_2$, then $\Omega_*(r)^{\nf}$ is isomorphic to the unnormalized chain complex of $\EC_2$, while $\rW(r)$ is isomorphic to the normalized chain complex of $\EC_2$. The quotient map $\uchains(\EC_2) \to \chains(\EC_2)$ induces therefore a map
\[
\Omega_*(r)^{\nf} \to \rW(r)
\]
that satisfies the conditions of Lemma \ref{lemma:omegar}.
\begin{remark}
	The combinatorics in pages 13 and 14 of \cite{medina2021fast_sq} translate here to the fact that the map from the unnormalized chains to the normalized chains is well-defined, i.e., to the fact that the degenerate simplices form a subcomplex of the unnormalized chain complex.
\end{remark}
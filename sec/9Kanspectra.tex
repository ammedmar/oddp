% !TEX root = ../oddp.tex

\section{Kan spectra}\label{s:9Kanspectra}

The category of Kan spectra is one of the many categories that model spectra. Introduced by Kan in \cite{Kan1963} and later developed by several articles \cite{burghelea_kanspectraI1967, burghelea_kanspectraII_1968, burghelea_kanspectraIII1969,Brown1973}, it fell in disgrace due to the lack of a good smash product. It has recently received some attention \cite{Stephan2015, CKP2023}.

\begin{definition}
    The \emph{stable simplex category} $\kansimplex$ is the colimit of the functors
    \[
        \simplex\lra \simplex\lra \simplex\lra \ldots
    \]
    that send an ordinal $[n]$ to the ordinal $[n+1]$ and extend a map $f\colon [n]\to [m]$ to a map $f'\colon [n+1]\to [m+1]$ by setting $f'(n+1) = m+1$. It has one object $[n]$ for each integer $n\in \bZ$, and there are morphisms $\delta^i\colon [n-1]\to [n]$ and $\sigma^i\colon $, for all $i\geq 0$, satisfying the simplicial identities.
\end{definition}
\begin{definition}
    A \emph{Kan spectrum} is a presheaf $X\colon \kansimplex^{\op}\to \Setp$ on pointed sets such that for each $x\in X([n])$ there is an $m$ such that $d_i(x) = *$ for all $i>m$. A map of Kan spectra is a natural transformation.
\end{definition}

This condition allows to define chain complexes $C_*(X)$ and $N_*(X)$ as for simplicial sets: In degree $n$ they are generated by the pointed set $X([n])$ modulo the basepoint, and the differential is the alternate sum of the face maps.

Write as before $\uchains_*(X)$ and $\chains_*(X)$ for their shifted versions. There is a covariant functor $\kansimplex\to \Ch{R}$ that sends any object $[n]$ to the chain complex of the augemented infinite simplex $\chains_*(\asimplex^\infty)$, and the action of faces and degeneracies are the stabilisation of the faces and degeneracies of the finite simplices. There is another covariant functor $\kansimplex\to \Ch{R}$ that sends any object $[n]$ to the cochains $\chains^*(\asimplex^\infty)$, and the action of the faces and degeneracies is the limit of their finite counterparts (note that this second complex is not free). Then, we have, as in Lemma \ref{lemma:1}:

\begin{lemma}
    There is a natural isomorphism
\[
    \chains_*(X)\cong \NN^*(\asimplex^{\infty})\otimes_{\kansimplex} \AA(X)
\]
\end{lemma}

Then, the rest of \cref{s:3complexes} remains true: One obtains a stable cosimplicial cochain complex $\NN^*(\asimplex^\infty)$ and a stable cosimplicial cochain complex $\Omega^*(r,\infty)$, that is the dual of a chain complex $\Omega_*(r,\infty)$ that admits a map $\lambda$ to $\Omega_*(r)$. With these changes, Proposition \ref{prop:omegarm} yields a connected comultiplication on the chain complex of a Kan spectrum.

Note that the previous lemma does not make sense without dualising: the interesting simplex in $N_*(\asimplex^\infty)$ is far away, at infinite dimension.



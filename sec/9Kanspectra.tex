% !TEX root = ../oddp.tex

\section{Kan spectra}\label{s:9Kanspectra}

The category of Kan spectra is one of the many categories that model spectra. Introduced by Kan in \cite{Kan1963} and later developed by several articles \cite{burghelea_kanspectraI1967, burghelea_kanspectraII_1968, burghelea_kanspectraIII1969,Brown1973}. It has recently received some attention \cite{Stephan2015, CKP2023}.

\begin{definition}
	The \emph{stable simplex category} $\kansimplex$ is the colimit of the functors
	\[
	\simplex \lra \simplex \lra \simplex \lra \dots
	\]
	that send an ordinal $[n]$ to the ordinal $[n+1]$ and extend a map $f \colon [n] \to [m]$ to a map $f' \colon [n+1] \to [m+1]$ by setting $f'(n+1) = m+1$. It has one object $[n]$ for each integer $n\in \bZ$, and there are morphisms $\d^i \colon [n-1] \to [n]$ and $\s^i \colon [n] \to [n-1]$, for all $i \geq 0$, satisfying the simplicial identities.
\end{definition}

The morphisms from $[n]$ to $[m]$ are in bijection with the endomorphisms $f$ of the first infinite ordinal $\omega$ such that
\begin{itemize}
	\item $|f^{-1}(k)|<\infty$ for all $k\in \omega$,
	\item $|f^{-1}(k)|\neq 1$ for finitely many $k\in \omega$,
	\item $\sum_k (1-f^{-1}(k)) = m-n$.
\end{itemize}

\begin{definition}
	A \emph{Kan spectrum} is a presheaf $X \colon \kansimplex^{\op} \to \Setp$ on pointed sets such that for each $x\in X_n$ there is an $m$ such that $d_i(x) = *$ for all $i>m$. A map of Kan spectra is a natural transformation.
\end{definition}

This condition allows to define chain complexes $\ucadenas(X)$ and $\cadenas(X)$ as for simplicial sets: In degree $n$ they are generated by the pointed set $X_n$ modulo the base point, and the differential is the alternate sum of the face maps.

The standard augmented infinite simplex $\asimplex$ is the Kan spectrum with one $n$-simplex for each order-preserving map $[n] \to [\omega]$.

Write as before $\uchains(X)$ and $\chains(X)$ for their shifted versions. There is a covariant functor $\kansimplex \to \Ch{R}$ that sends any object $[n]$ to the chain complex of the augmented infinite simplex $\chains(\asimplex^\infty)$, and the action of faces and degeneracies are the stabilization of the faces and degeneracies of the finite simplices. There is another covariant functor $\kansimplex \to \Ch{R}$ that sends any object $[n]$ to the cochains $\cochains(\asimplex^\infty)$, and the action of the faces and degeneracies is the limit of their finite counterparts (note that this second complex is not free). Then, we have, as in Lemma \ref{lemma:2}:

\begin{lemma}
	There is a natural isomorphism
	\[
	\chains(\asimplex^{\infty}) \ot_{\kansimplex} \rA(X)_\bullet \cong \cadenas(X)
	\]
\end{lemma}

Then, the rest of \cref{s:3complexes} remains true: One obtains a stable cosimplicial cochain complex $\chains(\asimplex^\infty)$ and a stable cosimplicial cochain complexes $\Omd(r,\infty)$ and $\Omhatd(r,\infty)$, and the former is the dual to a chain complex $\Om(r,\infty)$ that admits maps $\eta^{n}$ to $\Om(r)$. With these changes, Proposition \ref{prop:omegarm} yields a connected diagonal on the chain complex of a Kan spectrum.

Note that the previous lemma does not make sense without dualizing: the top simplex in $N_*(\asimplex^\infty)$ is far away, at infinite dimension.
% !TEX root = ../oddp.tex


\subsection{Simplicial functor tensor products}
Recall that if $A$ is an augmented cosimplicial chain complex and $B$ is an augmented simplicial chain complex, the functor tensor product $A \ot_{\asimplex} B$ is the quotient of the tensor product $A \ot B$ by the relation generated by
\begin{align*}
	\d^i a \ot b &\sim (-1)^{|\d^i||a|}a \ot \d_i b
	&
	\s^ia \ot b &\sim (-1)^{|\s^i||a|}a \ot \s_ib.
\end{align*}
The chain complexes $\chains(\asimplex^\bullet)$ form an augmented cosimplicial chain complex with
\begin{align*}
	\d^i \colon \chains(\asimplex^n)& \lra \chains(\asimplex^{n+1})
	&
	\s^i \colon \chains(\asimplex^{n})& \lra \chains(\asimplex^{n-1})
\end{align*}
given on an ordered representative $[v_0,\dots,v_{k-1}]\in \chains(\asimplex^n)$ by
\begin{align*}
\d^i([v_0,\dots,v_{k-1}]) &= [v_0,\dots,v_{j-1},v_j+1,\dots,v_{k-1}] \quad \text{if $v_{j-1}<i\leq v_j$}
\\
\s^i([v_0,\dots,v_{k-1}]) &= [v_0,\dots,v_{j-1},v_j-1,\dots,v_{k-1}-1] \quad \text{if $v_{j-1} \leq i < v_j$}.
\end{align*}
If $X$ is an augmented simplicial object in $\cC$, the chain complex $\chains(X)$ can alternatively be described as the functor tensor product over $\asimplex$ of the augmented cosimplicial chain complex $\chains(\asimplex^{\bullet})$ and the augmented simplicial $R$-module $\abel(X)_\bullet$:
\begin{equation}\label{eq:1}
	\chains(\asimplex^\bullet) \ot_{\asimplex} \abel(X)_\bullet \cong \chains(X).
\end{equation}
This sends a simplex $[k] \to [n]$ and a simplex $[n] \to X$ to the composition $[k] \to X$.

Via Alexander duality, we may endow $\cochains(\asimplex^n)$ with the following augmented \emph{cosimplicial} structure


\begin{align*}
	\d^i \colon \cochains(\asimplex^n) & \lra \susp{}\cochains(\asimplex^{n+1})
	&
	\s^i \colon \cochains(\asimplex^{n+1}) & \lra \susp{}\cochains(\asimplex^n)
\end{align*}
given on a generator $U=(u_0,\dots,u_{m-1})$ by
\begin{align*}
	\d^i(U) &=
	(-1)^{i+j+m+n+1}(u_0,\dots,u_{j-1},i,u_{j}+1,\dots,u_{m-1}+1)
	\quad \text{if $u_{j-1}<i \leq u_{j}$}
	\\
	\s^i(U) &= \begin{cases}
	(-1)^{i+j+m+n+1}(u_0,\dots,u_{j-1},u_{j+1}-1\dots,u_{m-1}-1) 	& \text{if $u_{j} = i$,}
	\\
	0 & \text{otherwise.}
	\end{cases}
\end{align*}
 Let $\rA(X)$ be the augmented simplicial graded $R$-module with $\rA(X)_n = \susp{n+1}\abel(X)_{n}$, and face maps and degeneracies are induced by those of $\abel(X)$ and have degree $-1$ and $+1$ respectively.

\begin{lemma}\label{lemma:2} The map
	\[
	\cochains(\asimplex^\bullet) \ot_{\asimplex} \rA(X)_\bullet \lra \chains(\asimplex^\bullet) \ot_{\asimplex} \abel(X)_\bullet
	\]
	that sends $U \ot \tau$ with to $\Lambda^{-1}(U) \ot \tau$ is an isomorphism of chain complexes.
\end{lemma}
More specifically, the map sends $U \ot \tau$ to $(-1)^{\lambda(U^c,U)+(n+1)m} U^c \ot \tau$.
\begin{example}\label{example:first3} If $U \ot \tau\in \cochains(\asimplex^\bullet) \ot_{\asimplex} \rA(X)_\bullet$ is given, one can compute its image on $\chains(X)$ using the relations
\[
	\d^i U \ot \tau = (-1)^{m} U \ot \d_i\tau
\]
 (with $U$ of degree $m$) in the functor tensor product and that $\emptyset \ot \tau \mapsto \tau$. For example, if $r=3$ and $X=\asimplex^7$ and $\tau = [0,1,2,3,4,5,6,7]$ is the top simplex,
\begin{align*}
		(0,6) \ot \tau &= (-1)^{6+1+2+8}(-1)^2(0) \ot [0,1,2,3,4,5,7]
		\\
		&= -(-1)^{0+0+1+8}(-1)^1\emptyset \ot [1,2,3,4,5,7]\mapsto -[1,2,3,4,5,7].
\end{align*}
Alternatively, using the map of Lemma \ref{lemma:2},
\[
	(0,1,3,4) \ot \tau = +(-1)^{8\cdot 4}[2,5,6,7].
\]
	\end{example}
	\begin{example}\label{example:first3'} If $r=3$ and $X=\asimplex^{19}$ and $\tau = [0,2,3,4,5,6,9]$,
\begin{align*}
		((0,6) \ot \tau) &= -(-1)^{20\cdot 2} [2,3,4,5,6]
		\\
		((0,1,3,4) \ot \tau) &= +(-1)^{20\cdot 4}[3,6,9].
	\end{align*}
	\end{example}
	\begin{example}\label{example:first5} If $r=5$ and $X=\asimplex^2$ and $\tau = [0,1,2]$ is the top simplex,
	\[
	\begin{split}
		(1) \ot \tau &= - (-1)^{3\cdot 1} [0,2]
		\\
		(1,2) \ot \tau &= +(-1)^{3\cdot 2} [0]
		\\
		(0,1,2) \ot \tau &= +(-1)^{3\cdot 3}\emptyset
		\\
		(2) \ot \tau &= +(-1)^{3\cdot 1} [0,1].
		\end{split}
	\]
	\end{example}

The $r$-fold tensor product $\cochains(\asimplex^\bullet)^{\ot r}$ becomes an augmented cosimplicial chain complex with face maps of degree $-r$, while the $r$-fold tensor product $\rA(X)_\bullet^{\ot r}$ becomes a simplicial chain complex with face maps of degree $-r$. Specifically, the faces and degeneracies of the former are as follows:
\begin{align*}
	\d^i \colon \cochains(\asimplex^n)^{\ot r}& \lra \susp{r}\cochains(\asimplex^{n+1})^{\ot r}
	&
	\s^i \colon \cochains(\asimplex^{n+1})^{\ot r}& \lra \susp{-r}\cochains(\asimplex^{n})^{\ot r}
\end{align*}
given by
\begin{align*}
	\d^i(U_0 \ot \dots \ot U_{r-1}) &= (-1)^{\sum_k (r-1-k)m_k}\d^i(U_0) \ot \dots \d^i(U_{r-1})
	\\
	\s^i(U_0 \ot \dots \ot U_{r-1}) &= (-1)^{\sum_k (r-1-k)m_k}\d^i(U_0) \ot \dots \d^i(U_{r-1})
\end{align*}
where $k = 0\dots r-q$ and $m_k$ is the degree of $U_k$ (the sign comes from the fact that the maps $\d^i$ in each tensor factor have degree $1$, hence one has to pay a sign for moving it accross the other tensor factors). Similarly, the $r$-fold tensor product $\rA(X)^{\ot r}_{\bullet}$ becomes an augmented simplicial graded $R$-module with face maps of degree $-r$ and degeneracy maps of degree $r$
\begin{align*}
	\d_i \colon \rA(X)_n^{\ot r} & \lra \rA(X)_{n-1}^{\ot r}
	&
	\s_i \colon \rA(X)_n^{\ot r} & \lra \rA(X)_{n+1}^{\ot r}
\end{align*}
given by
\begin{align*}
	\d_i(\tau_0 \ot \dots \ot \tau_{r-1}) &= (-1)^{\sum_k (r-1-k)(n+1)} \d_i(\tau_0) \ot \dots \ot \d_i(\tau_{r-1})
\\
	\s_i(\tau_0 \ot \dots \ot \tau_{r-1}) &= (-1)^{\sum_k (r-1-k)(n+1)} \s_i(\tau_0) \ot \dots \ot \s_i(\tau_{r-1})
	\end{align*}
\begin{lemma}\label{lemma:3}
	The chain map
	\[\cochains(\asimplex^\bullet)^{\ot r} \ot_{\asimplex} \rA(X)_\bullet^{\ot r} \lra \left(\cochains(\asimplex^\bullet) \ot_{\asimplex} \rA(X)_\bullet\right)^{\ot r}\]
	given by
	\begin{align*}
		(U_0 \ot \dots \ot U_{r-1}) \ot &(\tau_0 \ot \dots \ot \tau_{r-1})
	\\
 &\mapsto (-1)^{(n+1)\sum_{k} km_k}(U_0 \ot \tau_0) \ot \dots \ot (U_{r-1} \ot \tau_{r-1})
	\end{align*}
where $k=0\dots r-1$, the element $\tau$ has degree $n+1$ and $U_k$ has degree $m_k$, is a $\Cyc_r$-equivariant monomorphism of chain complexes.
\end{lemma}
\begin{remark}
	Had we worked with augmented semi-simplicial objects, this map would be only a monomorphism.
\end{remark}
\begin{example}\label{example:reord} If $r=3$ and $X=\asimplex^7$ and $\tau = [0,1,2,3,4,5,6,7]$ is the top simplex,
\[
		\alpha((\emptyset \ot (0,6) \ot (0,1,3,4)) \ot \tau^{\ot 3}) = (-1)^{8\cdot 10}(\emptyset \ot \tau) \ot ((0,6) \ot \tau) \ot ((0,1,3,4) \ot \tau)
	\]
 If $r=3$ and $X=\asimplex^{19}$ and $\tau = [0,2,3,4,5,6,9]$,
\[
		\alpha((\emptyset \ot (0,6) \ot (0,1,3,4)) \ot \tau^{\ot 3}) = (-1)^{7\cdot 10}(\emptyset \ot \tau) \ot ((0,6) \ot \tau) \ot ((0,1,3,4) \ot \tau)
	\]
 If $r=5$ and $X=\asimplex^2$ and $\tau = [0,1,2]$ is the top simplex,
	\[
	\begin{split}
		\alpha(((1) \ot (1,2) \ot \emptyset \ot (0,1,2) \ot (2)) \ot \tau^{\ot 5}) &=
		\\
		=(-1)^{3\cdot 15}((1) \ot \tau) \ot ((1,2) \ot \tau)& \ot (\emptyset \ot \tau) \ot ((0,1,2) \ot \tau) \ot ((2) \ot \tau)
		\end{split}
	\]
	\end{example}

\begin{definition} Define the chain map
	\[\alpha \colon \cochains(\asimplex^\bullet)^{\ot r} \ot_{\asimplex} \rA(X)_\bullet^{\ot r} \lra \chains(X)^{\ot r}\]
as the composition of the chain isomorphism of Lemma \ref{lemma:3} and the $r$-fold tensor product of the composition of the chain isomorphisms of \eqref{eq:1} and Lemma \ref{lemma:2}.
\end{definition}


From Examples \ref{example:first3}, \ref{example:first3'}, \ref{example:first5} and \ref{example:reord}, we can compute the following:
\begin{example}\label{example:alpha} If $r=3$ and $X=\asimplex^7$ and $\tau = [0,1,2,3,4,5,6,7]$ is the top simplex,
\[
		\alpha((\emptyset \ot (0,6) \ot (0,1,3,4)) \ot \tau^{\ot 3}) = -\tau \ot [1,2,3,4,5] \ot [2,5,6,7]
	\]
	where the sign coming from the permutation of the factors of the tensor product is $(-1)^{10\cdot 7}$ and the sign coming from the $\d_i$'s is negative. If $r=3$ and $X=\asimplex^{19}$ and $\tau = [0,2,3,4,5,6,9]$,
\[
		\alpha((\emptyset \ot (0,6) \ot (0,1,3,4)) \ot \tau^{\ot 3}) = -\tau \ot [2,3,4,5,6] \ot [3,6,9]
	\]
	where the sign coming from the permutation of the factors of the tensor product is $(-1)^{10\cdot 6}$ and the sign coming from the $\d_i$'s is negative. If $r=5$ and $X=\asimplex^2$ and $\tau = [0,1,2]$ is the top simplex,
	\[
		\alpha(((1) \ot (1,2) \ot \emptyset \ot (0,1,2) \ot (2)) \ot \tau^{\ot 5}) = -[0,2] \ot [0] \ot [0,1,2] \ot \emptyset \ot [0,1]
	\]
		where the sign coming from the permutation of the factors of the tensor product is $(-1)^{3\cdot 15}$ and the sign coming from the $\d_i$'s is positive.
	\end{example}
\subsection{The chain complex \texorpdfstring{$\Om(r,n)$}{Omega(r,n)}} If $X$ is an augmented simplicial object, we denote by $\uchains(X)$ the right suspension $\sus{}\ucadenas(X)$. Let $\Om(r,n)$ be the quotient of $\chains(\asimplex^{n}\times \EC_r)$ by the equivalence relation generated as follows: If $U = (u_0,\dots,u_{m-1})\in \uchains(\asimplex^n)$ is an unnormalized generator with $u_i\leq u_{i+1}$ and $A = (a_0,\dots,a_{m-1})\in \uchains(\EC_r)$ is another unnormalized generator, then
\begin{itemize}
	\item $(U,A)\sim (-1)^{\sign{\perm}}(U,A')$ if there is an interval $u_{i-1}<u_i =\dots =u_{i+k}<u_{i+k+1}$ such that the sequences $(a_i,\dots,a_{i+k})$ and $(a'_i,\dots,a'_{i+k})$ differ by a permutation $\perm$ and $A$ and $A'$ agree outside that interval.
\end{itemize}

Given a pair $(U,A)$ as above, define $U_A^i = \{u_j\in U\mid a_j=i\}$. There is an isomorphism
\[
	\beta \colon \Om(r,n) \lra \chains(\asimplex^n)^{\ot r}
\]
that sends a pair $(U,A)$ to $(-1)^{\sign{\perm(A)}}\ U_A^0 \ot \dots \ot U_A^{r-1}$, where $\perm(A)$ is the permutation that arranges $A$ in ascending order without permuting entries with the same value.
\begin{example}\label{example:beta} If $r=3$ and $n \geq 6$:
\[
		\beta((0,0,1,3,4,6),(0,1,0,0,0,1)) = -(0,1,3,4) \ot (0,6) \ot \emptyset
	\]
If $r=5$ and $n \geq 2$:
	\[
		\beta((0,1,1,1,2,2,2),(1,3,4,1,1,3,0)) = -(2) \ot (0,1,2) \ot \emptyset \ot (1,2) \ot (1)
	\]
	\end{example}



Given a pair $(U,A)$ as above, define $A' = (a_0+u_0,\dots,a_m+u_m)$ where the sum is taken modulo $r$. There is also an automorphism
\[
	\gamma \colon \Om(r,n) \lra \Om(r,n)
\]
that sends a pair $(U,A)$ to the pair $(U,A')$.
	\begin{example}\label{example:gamma} If $r=3$ and $n \geq 6$:
	\[
		\gamma((0,0,1,3,4,6),(0,1,2,0,2,1)) = ((0,0,1,3,4,6),(0,1,0,0,0,1))
	\]
If $r=5$ and $n \geq 2$:
	\[
		\gamma((0,1,1,1,2,2,2),(1,2,3,0,4,1,3)) = ((0,1,1,1,2,2,2),(1,3,4,1,1,3,0))
		\]
\end{example}

The compositions
\begin{align*}
	\Omd(r,n)&\overset{\beta^\dd}{\lra} \left(\chains(\asimplex^n)^{\ot r}\right)^\dd \lra \cochains(\asimplex^n)^{\ot r}
	\\
	\Omd(r,n)\overset{\gamma^\dd}{\lra}	\Omd(r,n)&\overset{\beta^\dd}{\lra} \left(\chains(\asimplex^n)^{\ot r}\right)^\dd \lra \cochains(\asimplex^n)^{\ot r}
\end{align*}
	yield two augmented cosimplicial chain complexes with $[n]\mapsto \Omd(r,n)$. The first is denoted $\Omhatd(r,\bullet)$ and the second is denoted simply by $\Omd(r,\bullet)$.
\begin{example}\label{example:betadual3} If $r=3$ and $n=7$:
\[
		\beta^\dd((0,0,1,3,4,6),(0,1,0,0,0,1)) = -\emptyset \ot (0,6) \ot (0,1,3,4)
	\]
	\end{example}
	\begin{example}\label{example:betadual5} If $r=5$ and $n=2$:
	\[
		\beta^\dd((0,1,1,1,2,2,2),(1,3,4,1,1,3,0)) = -(1) \ot (1,2) \ot \emptyset \ot (0,1,2) \ot (2)
	\]
	\end{example}

	The face maps of these cosimplicial chain complexes have degree $-1$ and act as follows: Let $k$ be such that $u_{k-1}<i\leq u_k$, and let $U = (u_0,\dots,u_{m-1})$ and $A = (a_0,\dots,a_{m-1})$, then,
	\[
		\d^i \colon \Omhatd(r,n) \to \susp{}\Omhatd(r,n+1)
	\]
		is given by
	\[\d^i(U, A) = (-1)^{r(i+j+m+n+1)}((u'_0,\dots,u'_{m+r-1}),(a'_0,\dots,a'_{m+r-1}))\] with
\begin{align}\label{eq:Omegahat}
	u'_j &=
	\begin{cases}
		u_j &\text{ if $j<k$} \\
		i & \text{ if $k\leq j < k+r$} \\
		u_{j-r} + 1 & \text{ if $j \geq k+r$.}
	\end{cases}
	&
	a'_j &=
	\begin{cases}
		a_j &\text{ if $j<k$} \\
		j-i & \text{if $k\leq j<k+r$} \\
		a_{j-r} & \text{ if $j \geq k+r$.}
	\end{cases}
\end{align}
whereas
\[
	\d^i \colon \Omd(r,n) \to \susp{}\Omd(r,n+1)
	\]
	is given by
\begin{align*}
	\d^i(U, A) &= (-1)^{r(i + j + m + n + 1)}((u'_0,\dots,u'_{m+r-1}),(a'_0,\dots,a'_{m+r-1}))
\end{align*}
with
\begin{align}\label{eq:Theta}
	u'_j &=
	\begin{cases}
		u_j &\text{ if $j<k$} \\
		i & \text{ if $k\leq j < k+r$} \\
		u_{j-r} + 1 & \text{ if $j \geq k+r$.}
	\end{cases}
	&
	a'_j &=
	\begin{cases}
		a_j &\text{ if $j<k$} \\
		r-i+j-k & \text{if $k\leq j<k+r$} \\
		a_{j-r}-1 & \text{ if $j \geq k+r$.}
	\end{cases}
\end{align}
The cyclic group $\Cyc_r$ acts on $\Om(r,n)$ and $\Omhat(r,n)$ as follows: $\rho(U,A) = (U,\rho A)$, with $\rho(a_0,\dots,a_{m-1}) = (\rho a_0,\dots,\rho a_{m-1})$. On $\Omd(r,n)$ and $\Omhatd(r,n)$ we have the dual action (recall that $\rho(a_0^\vee) = (\rho^{-1}a_0)^\vee$).

\begin{remark}
	Despite the complex $\Om(r,n)$ is defined as a quotient of $\chains(\asimplex^n\times \EC_r)$, the augmented cosimplicial structure on $\Omd(r,n)$ is not induced by any augmented cosimplicial structure on $\cochains(\asimplex^n\times \EC_r)$ for $n \geq -1$.
\end{remark}
As a consequence, there are $\Cyc_r$-equivariant natural isomorphisms
\[\Omd(r,\bullet)\overset{\gamma^\dd}{\lra} \Omhatd(r,\bullet)\overset{\beta^\dd}{\lra} \cochains(\asimplex^\bullet)^{\ot r}.\]
Taking functor tensor products, we obtain $\Cyc_r$-equivariant chain isomorphisms
\[\Omd(r,\bullet) \ot_{\asimplex} \rA(X)_\bullet^{\ot r}\overset{\gamma}{\lra} \Omhatd(r,\bullet) \ot_{\asimplex} \rA(X)_\bullet^{\ot r}\overset{\beta}{\lra} \cochains(\asimplex^\bullet)^{\ot r} \ot_{\asimplex} \rA(X)_\bullet^{\ot r}.\]




\subsection{A connected diagonal}\label{s:mainresult} A \emph{full piece} in a pair $(U,A)\in \Om(r,n)$ is a nondegenerate subpair $(u_i,\dots,u_{i+r-1}),(a_i,\dots,a_{i+r-1})$ of $(U,A)$ such that $u_i = u_{i+1} = \dots = u_{i+r-1}$ (and therefore $a_i,\dots,a_{i+r-1}$ are all distinct).

Let $\Om(r,n)^{\nf} \subset \Om(r,n)$ be the subcomplex generated by those pairs $(U,A)$ without full pieces. If $(U,A)$ has a single full piece, then $\partial(U,A) = \partial^{\nf}(U,A) + \partial^{\f}(U,A)$, where the summands in $\partial^{\nf}(U,A)$ contain no full pieces and the summands in $\partial^{\f}(U,A)$ contain one full piece. Recall that $\Delta$ is the natural diagonal in $\cC$ of \ref{s:categories}, and induces a map $L(\Delta) \colon \rA(X)_n \to \rA(X)_n^{\ot r}$ of degree $(r-1)n$ that we will denote also by $\Delta$. Recall the definition of $\rcadenas(X)$ from \cref{s:connected}, and observe that $\Delta$ also induces a homomorphism $\Delta \colon \rcadenas[rn](X) \to \rA(X)_n^{\ot r}$ of degree $0$.

\renewcommand{\Psiom}{\Psi}

\begin{proposition}\label{prop:omegarm}
	Let $r$ be odd or $r=2$ with $R=\mathbb{F}_2$. Suppose given a family of $\Cyc_r$-equivariant homomorphisms
	\begin{align*}
		\Psiom_*^n \colon \Om(r,n)^{\nf}& \lra \rW(r) & n \geq 0
	\end{align*}
	related by the following equation: if $(U,A)\in \Om[q](r,n)$ has a single full piece,
	\begin{equation}
		 \label{it:1a}
		\Psiom^n_{q-1}(\partial^{\nf} (U,A)) = (-1)^q\theta_{1-r}\circ \Psiom^{n-1}_{q-r}\circ \left(\sum_i (-1)^i \d_i(U,A)\right),
	\end{equation}
	where $\d_i$ is the dual of $\d^i$. Then there is a chain map
	\[
	\Psi \colon \rWd(r) \hotimes \rchains(X) \lra \Omd(r,\bullet) \ot_{\asimplex} \rA(X)_\bullet^{\ot r}
	\]
	defined as $\Psi(e_{-q} \ot \vec{\tau}) = \left(\Psiom^n_q\right)^\vee(e_{-q}) \ot \Delta(\tau)$ if $\tau\in \rA(X)_{n}$.
\end{proposition}
\def\diaglin{\Delta}


\begin{proof}
The suspension homomorphism $\theta_{2k}$ always rises the degree by $2k$, so the dual of $\theta_{2k}$ is $\theta_{-2k}$. We will write $\delta = \partial^\vee$, where $\partial$ is the differential of $\rW(r)$ or $\Om(r,n)$. We have
\begin{align*}
	\delta \Psi(e_{-q}^\dd \ot \vec{\tau})
		&= \delta \left(\Psiom^n\right)^{\vee}(e_{-q}^\dd) \ot \diaglin(\tau)\\
		&= \left(\delta\left(\Psiom^n\right)^{\vee}(e_{-q}^\dd)\right)^{\nf} \ot \diaglin(\tau) + \left(\delta\left(\Psiom^n\right)^{\vee}(e_{-q}^\dd)\right)^{\f} \ot \diaglin(\tau)
\end{align*}
while
\begin{align*}
	\Psi(\delta(e_{-q}^\dd \ot \vec{\tau}))
		&=\Psi(\delta(e_{-q}^\dd) \ot \vec{\tau}) + (-1)^{q} \sum_i (-1)^{i} \Psi(\theta_{r-1}(e_{-q}^\dd) \ot \d_i \vec{\tau})
	\end{align*}
The first summand of the second expression is then $\left(\Psiom^n\right)^\vee(\delta e_{-q}) \ot \diaglin(\tau)$, which equals the first summand of the first expression because $\Psiom^{n}$ is a chain map. For the second summands, we first use the dual of the condition of the Proposition: If $(U,A)$ has a full piece and degree $q+1$,
\begin{align*}
	\left(\delta\left(\Psiom^n\right)^\vee (e_{-q}^\dd)\right)^{\f}(U,A)
	&= \left\langle (-1)^{q+1}\Psiom^n\circ \partial^{\nf}(U,A),e_{q}\right\rangle
	\\
	&= \left\langle (-1)^{q+1}(-1)^{q+1}\theta_{1-r}\circ \Psiom^{n-1}\circ \sum_i (-1)^i \d_i(U,A),e_q\right\rangle
	\\
	&= \left(\sum_i (-1)^i\d^i\circ \left(\Psiom^{n-1}\right)^{\vee}\circ \theta_{r-1}(e_{-q}^\dd)\right)(U,A).
\end{align*}
Hence,
\begin{align*}
	\left(\delta\left(\Psiom^n\right)^{\vee}(e_{-q}^\dd)\right)^{\f} \ot \diaglin(\tau)
		&= \sum_i (-1)^i \d^i\circ \left(\Psiom^{n-1}\right)^\vee\circ \theta_{r-1} (e_{-q}^\dd) \ot \diaglin(\tau)
		\\
		&=(-1)^{r-1-q}\sum_i (-1)^i\left(\Psiom^{n-1}\right)^\vee \circ \theta_{r-1} (e_{-q}^\dd) \ot \d_i\diaglin(\tau)
		\\
		&= (-1)^{r-1-q}\sum_i (-1)^i\left(\Psiom^{n-1}\right)^\vee \circ \theta_{r-1} (e_{-q}^\dd) \ot \diaglin(\d_i\tau)
		\\
	&= (-1)^q\sum_i (-1)^{i} \Psi(\theta_{r-1}(e_{-q}^\dd) \ot \d_i \vec{\tau})).\qedhere
\end{align*}
\end{proof}

In Lemma \ref{lemma:omegar} we will give a criterion to construct a family $\{\Psiom^n\}_n$ satisfying the conditions of Proposition \ref{prop:omegarm}. In Propositions \ref{prop:chain} and \ref{prop:condition} we will reduce this criterion to the construction of a map $f$ satisfying the properties of Construction \ref{cons:1}. Finally, in Proposition \ref{prop:assemblage} and Lemma \ref{lemma:asymmetry} we will explain how to build the map $f$ from an $r$-cyclic asymmetry with duality. Altogether they provide the main theorem of this article:

\begin{theorem}\label{thm2:mainthm} Every $r$-cyclic asymmetry yields a natural $r$-cyclic connected diagonal for augmented semi-simplicial objects in $\cC$
\[
	\mu \colon \rWd(r) \hotimes \rchains(X) \lra \ucadenas(X)^{\ot r}
\]
defined as the composition
	$
	\mu = \alpha\circ\beta\circ\gamma\circ \Psi.
	$
\end{theorem}

\begin{proof}
	To check naturality, suppose $f \colon X \to Y$ is a map of augmented semi-simplicial objects. If $\tau\in \chains[n](X) = \abel(X)_n$, then, we first have that
	\[
		L(\Delta)\circ L(f)(\tau) = L(\Delta\circ f)(\tau) = L(f^{\ot r}\circ \Delta)(\tau) = L(f)^{\ot r}\circ L(\Delta)(\tau).
	\]
	therefore, writing $\Delta$ for $L(\Delta)$ as in the previous proposition, we have
	\begin{align*}
	\mu(e^\dd_{-q} \ot f_*(\vec{\tau}))
		&=	\alpha\circ\beta\circ\gamma\circ \Psi(e_{-q}^\vee \ot f_*(\vec{\tau}))
		\\
		&= \alpha\circ\beta\circ\gamma\circ(\Psiom^n)^\vee(e_{-q}^\vee) \ot \Delta\circ L(f)(\tau))
		\\
		&= \alpha\circ\beta\circ\gamma\circ(\Psiom^n)^\vee(e_{-q}^\vee) \ot L(f)^{\ot r}(\Delta(\tau))
		\\
		&= L(f)^{\ot r}(\alpha\circ\beta\circ\gamma\circ(\Psiom^n)^\vee(e_{-q}^\vee) \ot \Delta(\tau))
		\\
		&= L(f)^{\ot r}(\mu(e^\dd_{-q} \ot \vec{\tau})).
	\end{align*}
	In the penultimate equality we use that $f$ commutes with the face maps.
\end{proof}

\begin{example}\label{ex:omegarn} If $r=3$ and $n=7$, using Examples \ref{example:omegar} and \ref{example:psi3}:
	\[
		\Psiom_6^7((0,0,1,3,4,6),(0,1,2,0,2,1)) = -e_6-\rho^3e_6.
	\]
	Therefore, the coefficient of $((0,0,1,3,4,6),(0,1,2,0,2,1))$ in $\psi(e_6 \ot \tau)$ is $-1$.
 If $r=3$ and $n=6$, using Examples \ref{example:omegar} and \ref{example:psi3}:
	\[
		\Psiom_6^6((0,0,1,3,4,6),(0,1,2,0,2,1)) = -e_6-\rho^3e_6.
	\]
	Therefore, the coefficient of $((0,0,1,3,4,6),(0,1,2,0,2,1))$ in $\psi(e_6 \ot \tau)$ is $-1$.
 If $r=5$ and $n=2$, using Examples \ref{example:omegar} and \ref{example:psi5}:
	\[
		\Psiom_7^2((0,1,1,1,2,2,2),(1,2,3,0,4,1,3)) = 2^2e_7.
	\]
Therefore, the coefficient of $((0,1,1,1,2,2,2),(1,2,3,0,4,1,3))$ in $\psi(e_7 \ot \tau)$ is $2^2$.
\end{example}

\begin{example}\label{ex:omegarnfinal} If $r=3$ and $n=7$, using Examples \ref{example:alpha}, \ref{example:beta}, \ref{example:gamma} and \ref{ex:omegarn} for the computation of $\alpha,\beta,\gamma$ and $\psi$:
	\begin{itemize}
		\item If $r = 3$ and $\tau = [0,1,2,3,4,5,6,7]$, the coefficient of $\tau \ot [1,2,3,4,5] \ot [2,5,6,7]$ in $\mu(e_{-6} \ot \tau)$ is $-1$.
		\item If $r = 3$ and $\tau = [0,2,3,4,5,6,9]$, the coefficient of $\tau \ot [2,3,4,5,6] \ot [3,6,9]$ in $\mu(e_{-6} \ot \tau)$ is $-1$.
		\item If $r = 5$ and $\tau = [0,1,2]$, the coefficient of $[0,2] \ot [0] \ot [0,1,2] \ot \emptyset \ot [0,1]$ in $\mu(e_{-7} \ot \tau)$ is $2^2$.
	\end{itemize}
\end{example}

\subsection{Another bar resolution of \texorpdfstring{$\Cyc_r$}{the cyclic group}}

After identifying the vertices of the standard augmented simplex $\asimplex^{r-1}$ with $\{0,1,\dots,r-1\}$, the agumented simplex carries an action of $\Cyc_r$ that is free away from the empty simplex and the top simplex. Define $\Om(r)$ as the quotient of $\chains(\asimplex^{r-1}) \ot \chains(\asimplex^{r-1}) \ot \dots$ by the relation $(A_0 \ot \dots \ot A_{j-1} \ot \emptyset \ot A_j \ot \dots) \sim (A_0 \ot \dots \ot A_{j-1} \ot A_j \ot \dots)$. Specifically, it is generated on degree $m>0$ by tuples $A = A_0\barra A_1\barra\dots \barra A_k$ with each $A_j$ a non-empty generator of $\chains(\asimplex^{r-1})$ and $\sum_j |A_j| = m$. In degree $0$ it is generated by the empty sequence, which may be denoted $\emptyset$ or $1$, the unit of the ring $R$. Each $A_j$ will be called a \emph{piece} of $A$ and if $A_j$ is of length $r$, it will be called \emph{full piece}. A tuple $A$ as above will be called a \emph{pieced word}. Its differential is $\sum_{j} (-1)^j \diff_j A$, where $\diff_jA$ is the result of removing the $j$th element of $A_0\cup \dots\cup A_k$ from $A$.

Let $\Om(r)^{\nf} \subset \Om(r)$ be the chain subcomplex generated by pieced words without full pieces. Let $\Om(r)^{\f} \subset \Om(r)$ be the chain subcomplex generated by pieced words with at most one full piece. The differential on $\Om(r)^{\f}$ decomposes as $\partial^{\nf}+\partial^{\f}$, where $\partial^{\nf}$ consists on summands without full pieces and $\partial^{\f}$ on summands with a single full piece.

Let $\DDD \colon \Om(r)^{\f} \to \Om(r)^{\nf}$ be the chain map that sends a pieced word without full pieces to zero and a pieced word $A_0\barra\dots\barra A_k$ of degree $m$ with a full piece piece $A_j$ to
\[
\DDD(A_0\barra \dots\barra A_k) = (-1)^{r\left(m+\sum_{k<j} |A_k|\right)} A_0\barra \dots\barra \hat{A}_{j}\barra \rho A_{j+1}\barra \dots\barra \rho A_k
\]
If $(U,A)$ is a generator of $\Om(r,n)$, then the word $A$ has a canonical piece decomposition: a subsequence $(a_i,\dots,a_{i+k})$ of $A$ is a piece if and only if $u_{i-1}<u_i =\dots= u_{i+k}<u_{i+k+1}$. For each $n \geq 0$, there is a chain map
\[\eta^n \colon \Om(r,n) \to \Om(r)\]
that sends a pair $(U,A)$ of degree $m$ to the word $(-1)^{m(n+1)}A$ with its canonical piece decomposition. Additionally, it satisfies the following equations:
\begin{align}\label{eq:lambda2}
\eta^n\circ \partial^{\nf} &= \partial^{\nf}\circ \eta^n
\\
\label{eq:lambda0}
	\eta^{n-1}\left(\sum_i (-1)^{ri}\d_i(U,A)\right) &= \DDD\circ \eta^n(U,A).
\end{align}
%
\begin{example}\label{example:omegar} To lighten the notation, we represent a piece $A_j = [a_0,\dots,a_{m-1}]$ as $a_0\dots a_{m-1}$: If $r=3$ and $n=7$,
	\[
		\eta^7((0,0,1,3,4,6),(0,1,2,0,2,1)) = (-1)^{6\cdot 8}(01\barra 2\barra 0\barra 2\barra 1).
	\]
	If $r=3$ and $n=6$,
	\[
		\eta^6((0,0,1,3,4,6),(0,1,2,0,2,1)) = (-1)^{6\cdot 7}(01\barra 2\barra 0\barra 2\barra 1).
	\]
	If $r=5$ and $n=2$:
	\[
		\eta^2((0,1,1,1,2,2,2),(1,2,3,0,4,1,3)) = (-1)^{7\cdot 3} 2^{}(1\barra 230\barra 413).
	\]
\end{example}

\begin{lemma}\label{lemma:omegar}
	Let $r$ be odd, and suppose one is given a $\Cyc_r$-equivariant chain map
	\begin{align*}
		\Psiom \colon \Om(r)^{\nf}& \lra \rW(r)
	\end{align*}
	such that if $A$ is a generator of $\Om(r)^{\f}$ with a full piece,
	\begin{equation}\label{eq:lambda}
		(\tilde{r}!)\Psiom_{q-1}(\partial^{\nf} (A)) = (-1)^q\theta_{1-r}\circ\Psiom_{q-r}\circ\DDD(A).
	\end{equation}
	Then the family of maps $\Psiom^n = (\tilde{r}!)^{n+1}\Psiom\circ \eta^n$ satisfies the condition of Proposition \ref{prop:omegarm}.
\end{lemma}

\begin{proof} If $(U,A)\in \Om[q](r,n)$ has a full piece,
\begin{align*}
	\theta_{1-r}\circ \Psiom_{q-r}^{n-1}&\left(\sum_i (-1)^{ri} \d_i(U,A)\right) =
	\\
	&= (\tilde{r}!)^{n}\theta_{1-r}\circ \Psiom_{q-r}\circ \eta^{n-1}\left(\sum_i (-1)^{ri} \d_i(U,A)\right)
	\\
	&\overset{\eqref{eq:lambda0}}{=} (\tilde{r}!)^{n}\theta_{1-r}\circ \Psiom_{q-r}\circ \DDD\circ \eta^{n}(U,A)
	\\
	&\overset{\eqref{eq:lambda}}{=} (-1)^{q}(\tilde{r}!)^{n} \cdot \tilde{r}!\Psiom_{q-1}\circ \partial^{\nf}\circ \eta^{n}(U,A)
	\\
		&\overset{\eqref{eq:lambda2}}{=} (-1)^{q-1}(\tilde{r}!)^{n+1}\Psiom_{q-1}\circ \eta^n\circ \partial^{\nf}(U,A)
	\\
	&= \Psiom^{n}_{q-1}\circ \partial^{\nf}(U,A).\qedhere
\end{align*}
\end{proof}

\begin{remark}
	A particular homomorphism satisfying the hypotheses of Lemma \ref{lemma:omegar} will be found in the next section. Though it is defined with coefficients in $\bZ[\frac{1}{\tilde{r}!}]$, the maps $\{\Psiom^m\}$ obtained in the lemma are well-defined with integer coefficients.
\end{remark}
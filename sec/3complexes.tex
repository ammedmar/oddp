% !TEX root = ../oddp.tex

%\section{Some cochain complexes}
\subsection{Simplicial functor tensor products}
Recall that if $A$ is an augmented cosemi-simplicial chain complex and $B$ is an augmented semi-simplicial chain complex, the functor tensor product $A\otimes_{\asimplexinj} B$ is isomorphic to the quotient of the tensor product $A\otimes B$ by the relation generated by $d_i a\otimes b \sim a\otimes d_i b$.

The chain complexes $\chains_*(\asimplex^\bullet)$ form an augmented cosimplicial chain complex while the cochain complexes $\chains^*(\asimplex^\bullet)$ form an augmented simplicial cochain complex. If $X_\bullet$ is a semi-simplicial object in $\cC$, the chain complex $\uchains_*(X)$ can alternatively be described as the functor tensor product over $\asimplexinj$ of the augmented cosimplicial chain complex $\chains_*(\asimplex^{\bullet})$ and the augmented semi-simplicial $R$-module $\rA(X_\bullet)$.

Via Alexander duality, we may replace $\chains_*(\asimplex^\bullet )$ by $\chains^*(\asimplex^\bullet)$ to obtain a new \emph{cosimplicial} chain complex $\NN^*(\asimplex^\bullet)$. Specifically, it sends the ordinal $[n]$ to $\chains^*(\asimplex^n)$ and the face maps act on a cochain $(u_0,\ldots,u_m)$ as
\begin{align*}
	d^i((u_0,\ldots,u_m)) &=
	(-1)^{r(j+1)}(u_0,\ldots,u_{j},i,u_{j+1}+1,\ldots,u_{m}+1)
	& \text{if $u_j<i \leq u_{j+1}$}
	\\
	s^i((u_0,\ldots,u_m)) &= (-1)^{rj}(u_0,\ldots,\hat{u}_j,\ldots,u_m)
	& \text{ if $u_j = u_{j+1} = i$.}
\end{align*}
Similarly, we define $\AA(X_\bullet)$ to be the semi-simplicial graded $R$-module with $\AA(X_m) = \sus{m}\rA(X_m)$, and the face maps are induced by those of $\rA_\bullet(X)$ and have degree $-1$.

\begin{lemma}\label{lemma:1} The homomorphism
	\[
	\NN^*(\asimplex^\bullet)\otimes_{\asimplexinj} \AA_\bullet(X)\lra \chains_*(X)
	\]
	that sends $U\otimes \tau$ to $(-1)^{\lambda(U)}d_U\tau$ is an isomorphism of chain complexes.
\end{lemma}

%The $r$-fold tensor product $\chains_*(X)^{\otimes r}$ has a similar description using augmented $r$-multisimplicial $R$-modules. We now produce a model of the subchain complex of $\chains_*(X)^{\otimes r}$ generated by tensor products $\tau_0\otimes \ldots\otimes \tau_{r-1}$ of common faces of some bigger simplex $\sigma$ in $X$. This model is a functor tensor product of augmented simplicial objects, instead of multisimplicial objects.

%Let $\AA(X)^{\otimes r}$ be the augmented simplicial graded module that is trivial in all degrees that are not multiples of $r$, and that equals $\sus{rk}\rA_k(X)$ in degree $rk$. The face maps $d_i$ have degree $r$ and are induced by the face maps of $\rA(X)$. If $\tau$ is an element of $\rA_k(X)$, the corresponding element of $\AA_k(X)$ will be denoted $\vec{\tau}$.

%We consider first the subset $\Delta^r(\asimplex^{\bn})\subset \asimplex^{\bn}*\overset{r}{\ldots} \asimplex^{\bn}$ of \emph{diagonal join simplices}, i.e., simplices in the join of the form $\vec{\tau} = \tau*\overset{r}{\ldots}* \tau$. Define $\rA^r(X)$ as the graded module on the diagonal join simplices. Since $\rA^r(X)$ is isomorphic to $\rA(X)$ up to a degree shift (which depends on the dimension), we do obtain an augmented simplicial $R$-module
%\[
%\AA^r(X)\colon \asimplex^\op \lra \Mod{r}.
%\]
%Here, the generator $\vec{\tau}$ has degree $rm$, where $m$ is the dimension of $\tau$. Additionally, the face maps $d_i\colon \AA^r(X)\to \AA^r(X)$ have degree $r$.

%Next, we consider the augmented cosimplicial graded $R$-module
%\[
%\asimplex^\op \overset{N^*(\asimplex^{\bullet})}{\lra} \Ch{R} \overset{(-)^{\otimes r}}{\lra} \Ch{R}
%\]
%whose $m$-simplices form the chain complex $\NN^*(\asimplex^\bm)^{\otimes r}$.

Let $\AA^r(X_\bullet)$ be the semi-simplicial graded $R$-module whose $m$-simplices are $\sus{rm}\rA(X_m\otimes \overset{r}{\ldots}\otimes X_m)$. The face map $d_i$ is the tensor product $d_i'\otimes \ldots\otimes d_i'$ of the face map $d_i'$ of $\rA(X_\bullet)$ and has degree $-r$. Given $\vec{U} = U_0\otimes \ldots\otimes U_{r-1}$ and $\vec{\tau} = \tau_0\otimes\ldots\otimes \tau_{r-1}$, let $\nu(\vec{U},\vec{\tau})$ be the sign that permutes these factors to $U_0\otimes \tau_0\otimes\ldots\otimes U_{r-1}\otimes \tau_{r-1}$. Let $\lambda(\vec{U})$ be the sum $\sum_j\lambda(U_j)$.

\begin{lemma}
	The homomorphism
	\[\alpha\colon \NN^*(\asimplex^\bullet)^{\otimes r}\otimes_{\asimplexinj} \AA^r_\bullet(X)\lra \chains_*(X)^{\otimes r}\]
	given by
	\begin{align*}
		(U_0\otimes \ldots\otimes U_{r-1})\otimes (\tau_0\otimes\ldots\otimes \tau_{r-1}) %&\mapsto (-1)^{\sum_j j\cdot|U_{j}|}(U_0\otimes \tau)\otimes \ldots\otimes (U_{r-1}\otimes \tau) \\
		&\mapsto (-1)^{ \nu(\vec{U},\vec{\tau})+ \lambda(\vec{U})}(d_{U_0}\tau_0\otimes \ldots\otimes d_{U_{r-1}}\tau_{r-1})
	\end{align*}
	is a $\Cyc_r$-equivariant monomorphism. %and surjects onto the tensor products of simplices that are faces of a common simplex.
\end{lemma}

\begin{example}\label{ex:102}
	Let $X = \asimplex^n$, let $r= 3$. The generator $[5,7,9]\otimes [3,4,5]\otimes [0,2,3]\in \chains_9(X)^{\otimes 3}$ is the image under $\alpha$ of the generator $((0,1,2,3)\otimes (0,1,5,6)\otimes (3,4,5,6))\otimes [0,2,3,4,5,7,9]$ with the signs $\lambda(U_i) = 0$ for $i=0,1,2$ and $\mu(U_0,U_1,U_2,\vec{\tau}) = 0$, hence the sign is $+$.
\end{example}
Let $\Omega_*(r,m)$ be the quotient of $\chains_*(\asimplex^{\bm}\times \EC_r)$ by the equivalence relation generated as follows: If $U = (u_0,\ldots,u_q)\in C_*(\asimplex^m)$ is an unnormalised generator with $u_i\leq u_{i+1}$ and $A = (a_0,\ldots,a_q)\in C_*(\EC_r)$ is another unnormalised generator, then
\begin{itemize}
	\item $(U,A)\sim (-1)^{|\sigma|}(U,A')$ if there is an interval $u_{i-1}<u_i =\ldots =u_{i+k}<u_{i+k+1}$ such that the sequences $(a_i,\ldots,a_{i+k})$ and $(a'_i,\ldots,a'_{i+k})$ differ by a permutation $\sigma$ and $A$ and $A'$ agree outside that interval. Notice that if all elements of $U$ are different, then $A=A'$.
\end{itemize}
Define a cosemi-simplicial chain complex $\Omega^*(r,\bullet)\colon \asimplex\to \Ch{R}$ as follows: Its value on an ordinal $[m]$ is $\Omega^*(r,m)$, the linear dual of $\Omega_*(r,m)$, and the face maps act as follows:
\begin{align*}
	d_i((u_0,\ldots,u_m), (a_0,\ldots,a_m)) &= (-1)^{r\cdot |U_{<i}|}((u'_0,\ldots,u'_{m+r}),(a'_0,\ldots,a'_{m+r}))
\end{align*}
with
\begin{align}\label{eq:Theta}
	u'_j &=
	\begin{cases}
		u_j &\text{ if $j<i$} \\
		i & \text{ if $i\leq j < i+r$} \\
		u_{j-r} + 1 & \text{ if $j\geq i+r$.}
	\end{cases}
	&
	a'_j &=
	\begin{cases}
		a_j &\text{ if $j<i$} \\
		j-i & \text{if $i\leq j<i+r$} \\
		a_{j-r}-1 & \text{ if $j\geq i+r$.}
	\end{cases}
\end{align}
Define another cosemi-simplicial chain complex $\hat{\Omega}^*(r,\bullet)\colon \asimplex\to \Ch{R}$ whose value on an ordinal $[m]$ is again $\Omega^*(r,m)$, but the effect of face maps is the following:
\[d_i((u_0,\ldots,u_m), (a_0,\ldots,a_m)) = (-1)^{r\cdot |U_{< i}|}((u'_0,\ldots,u'_{m+r}),(a'_0,\ldots,a'_{m+r}))\] with
\begin{align}\label{eq:Omegahat}
	u'_j &=
	\begin{cases}
		u_j &\text{ if $j<i$} \\
		i & \text{ if $i\leq j < i+r$} \\
		u_{j-r} + 1 & \text{ if $j\geq i+r$.}
	\end{cases}
	&
	a'_j &=
	\begin{cases}
		a_j &\text{ if $j<i$} \\
		j-i & \text{if $i\leq j<i+r$} \\
		a_{j-r} & \text{ if $j\geq i+r$.}
	\end{cases}
\end{align}
The cyclic group $\Cyc_r$ acts on $\Omega_*(r,m)$ as follows: $\rho(U,A) = (U,\rho A)$, with $\rho(a_0,\ldots,a_m) = (\rho a_0,\ldots,\rho a_m)$. On $\Omega^*(r,m)$ we have the dual action.

We have natural isomorphisms
\[\Omega^*(r,\bullet)\overset{\gamma}{\lra} \hat{\Omega}^*(r,\bullet)\overset{\beta}{\lra} \NN^*(\asimplex^\bullet)^{\otimes r}\]
defined as follows:
\begin{align*}
	\gamma(U,A) &= (U,A')
	&
	\beta(U,A) &= (-1)^{\sigma(A)}\ U_A^0\otimes \ldots\otimes U_A^{r-1}
\end{align*}
where $A' = (a_0+u_0,\ldots,a_m+u_m)$ (the sum is taken $\mod r$) and $U_A^i = \{u_j\in U\mid a_j=i\}$. The sign $\sigma(A)$ is that of the permutation that arranges $A$ in ascending order without permuting entries with the same label. For example, we can arrange $A=01210$ in ascending order in two steps: first, move the last zero to the second position $A' = 00121$, which is an odd permutation. Second, move the last one to the penultimate position obtaining the ascending word $00112$, which is again an odd permutation, therefore $\sigma(A) \equiv 0$.

\begin{example}\label{ex:103} For $r=3,m=6$, we have
	\begin{align*}                  \gamma(001123345566,012010212012) &= (001123345566,010100221212).
		\\
		\beta(001123345566,010100221212) &= (0,1,2,3)\otimes (0,1,5,6)\otimes (3,4,5,6)
	\end{align*}
	with $\sigma(A) = 0$.
\end{example}
Taking functor tensor product with the augmented simplicial $R$-module $\AA^r(X_\bullet)$, we obtain chain isomorphisms
\[\Omega^*(r,\bullet)\otimes_{\asimplexinj} \AA^r(X_\bullet)\overset{\gamma}{\lra} \hat{\Omega}^*(r,\bullet)\otimes_{\asimplexinj} \AA^r(X_\bullet)\overset{\beta}{\lra} \NN^*(\asimplexinj^\bullet)^{\otimes r}\otimes_{\asimplex} \AA^r(X_\bullet)\]

\subsection{A connected comultiplication} A \emph{full piece} in $(U,A)\in \Omega_*(r,m)$ is a nondegenerate subpair $(u_i,\ldots,u_{i+r-1}),(a_i,\ldots,a_{i+r-1})$ of $(U,A)$ such that $u_i = u_{i+1} = \ldots = u_{i+r-1}$ (and therefore $a_i,\ldots,a_{i+r-1}$ are all distinct).

Let $\Omega_*(r,m)^{\nf}\subset \Omega_*(r,m)$ be the subcomplex generated by those pairs $(U,A)$ without full pieces. If $(U,A)$ has a single full piece, then $\partial(U,A) = \partial^{\nf}(U,A) + \partial^{\f}(U,A)$, where the summands in $\partial^{\nf}(U,A)$ contain no full pieces and the summands in $\partial^{\f}(U,A)$ contain one full piece. Recall that $\Delta$ is the natural diagonal in $\cC$.

\begin{proposition} \label{prop:omegarm}
	Let $r$ be odd or $r=2$ with $R=\mathbb{F}_2$. Suppose given a family of $\Cyc_r$-equivariant homomorphisms
	\begin{align*}
		\hat{\Psi}_*^m\colon \Omega_*(r,m)^{\nf}&\lra \rW_*(r)
	\end{align*}
	related by the following equation:
	\begin{itemize}
		\item \label{it:1a}
		$\hat{\Psi}^m_{q-1}(\partial^{\nf} (U,A)) = (-1)^{q-r}\theta_{1-r}\hat{\Psi}^{m-1}_{q-r}\left(\sum_i d_i^\vee(U,A)\right)$
		if $(U,A)$ has a full piece.
	\end{itemize}
	Then there is a chain map
	\[
	\Psi\colon \rW^*(r)\hotimes \chains^r_*(X)\lra \Omega^*(r,\bullet)\otimes_{\asimplexinj} \AA^r(X_\bullet)
	\]
	defined as $\Psi(e_q\otimes \vec{\tau}) = (\Psiom^m)^\vee(e_q^\vee)\otimes \Delta(\tau)$ if $\tau$ has dimension $m$.
\end{proposition}
\def\diaglin{\Delta}

\begin{proof}
	\begin{align*}
		\delta(\Psiom^m)^{\vee}(e_q^\vee)\otimes \diaglin(\tau)
		&= (\delta(\Psiom^m)^{\vee}(e_q^\vee))^{\nf}\otimes\diaglin(\tau) + (\delta(\Psiom^m)^{\vee}(e_q^\vee))^{\f}\otimes\diaglin(\tau) \\
		&= (\Psiom^m)^{\vee}(\delta e_q^\vee)\otimes \diaglin(\tau) + (-1)^{q+1-r}\sum d_i(\Psiom^{m-1})^\vee_{q-r+1}(e_{q-r+1})\otimes\diaglin(\tau) \\
		%&= \hat{\Psi}^{\vee}(\delta e_{q})\otimes \vec{\tau} + (-1)^q \sum_i (-1)^{\nu_i(-)} D_i(\Psiom^\vee_{q-r+1}(e_{q-r+1}))\otimes \vec{\tau}\\
		&= (\Psiom^m)^{\vee}(\delta e_q^\vee)\otimes \diaglin(\tau) + (-1)^{q+1-r}\sum_i \Psiom^\vee_{q-r+1}(e_{q-r+1})\otimes d_i\diaglin(\tau) \\
		&= (\Psiom^m)^{\vee}(\delta e_q^\vee)\otimes \diaglin(\tau) + (-1)^{q+1-r}\sum_i (-1)^i\Psiom^\vee_{q-r+1}(e_{q-r+1})\otimes \diaglin(d_i\tau) \\
		&= \Psi(\delta(e_q^\vee\otimes \vec{\tau}))
	\end{align*}
	where the sign $(-1)^i$ comes from $\lambda(\{i\}) = i + \binom{m+1}{2}$.
\end{proof}

In the next section we will explain that the map required in the hypothesis of this theorem exists if $r$ is an odd prime, and we build one such map from every $r$-cyclic asymmetry. From that we deduce the main theorem of this article:
\begin{theorem}\label{thm2:mainthm} Every $r$-cyclic asymmetry yields a map $\Psi\colon W_*(r)\hotimes C_*(X)\to \Omega_*(r,\bullet)\otimes_{\asimplexinj} \AA(X_\bullet)$, and the composition
	\[ \alpha\circ\beta\circ\gamma\circ \Psi
	\]
	is a natural $r$-cyclic connected comultiplication for augmented semi-simplicial objects in $\cC$.
\end{theorem}

\begin{proof}
	To check naturality, suppose $f\colon X\to Y$ is a map of augmented semi-simplicial objects. Then
	\begin{align*}
		\alpha\circ\beta\circ\gamma\circ \Psi(e_q^\vee\otimes L(f)(\tau))
		&= \alpha\circ\beta\circ\gamma\circ(\Psiom^m)^\vee(e_q^\vee)\otimes \Delta(L(f)(\tau)) \\
		&= \alpha\circ\beta\circ\gamma\circ(\Psiom^m)^\vee(e_q^\vee)\otimes L(f)^{\otimes r}(\Delta(\tau))
	\end{align*}
	and since $f$ commutes with the face maps, the latter equals
	\[
	L(f)^{\otimes r}\circ \alpha\circ\beta\circ\gamma\circ(\Psiom^m)^\vee(e_q^\vee)\otimes \Delta(\tau).\qedhere
	\]
\end{proof}

\subsection{Another bar resolution of $\Cyc_r$}

After identifying the vertices of $\asimplex^{r-1}$ with $\{0,1,\ldots,r-1\}$, the agumented simplex carries an action of $\Cyc_r$ that is free away from the empty simplex and the top simplex. Define $\Omega_*(r)$ as the quotient of $\chains_*(\asimplex^{r-1}*\asimplex^{r-1}*\ldots)$ by the relation $(a_0\otimes \ldots \otimes \emptyset \otimes a_m\otimes \ldots) \sim (a_0\otimes \ldots \otimes a_m \otimes \emptyset \otimes \ldots)$. It is generated on degree $q>0$ by sequences $A = A_0\barra A_1\barra\ldots \barra A_k$ with $A_i$ a non-empty generator of $\chains_*(\asimplex^{r-1})$. In degree $0$ it is generated by the empty word $A=\emptyset$. Its differential is $\sum_{i} (-1)^i d_i A$, where $d_iA$ is the result of removing the $i$th term of $A$. If that removal yields an empty generator, suppress it from the sequence. Each $A_i$ will be a \emph{piece} and if $A_i$ is of length $r$, it will be called \emph{full piece}. A sequence $A$ as above will be called a \emph{pieced word}.

Let $\Omega_*(r)^{\nf}\subset \Omega_*(r)$ (resp. $\Omega_*(r^{\f})\subset \Omega_*(r)$) be the subchain complex generated by words with no full pieces (resp., at most one full piece) and let $\DDD\colon \Omega_*(r)^{\f}\to \Omega_*(r)^{\nf}$ be the chain map that sends a pieced word  without full pieces to zero and a pieced word $A_0\barra\ldots\barra A_k$ with full piece piece $A_j$ to
\[
\DDD(A_0\barra \ldots\barra A_k) = (-1)^{\nu(A_{j},A)}A_0\barra \ldots\barra \hat{A}_{j}\barra \rho A_{j+1}\barra \ldots\barra \rho A_k
\]
The quantity $\nu(A_j,A)$ equals the length of all the pieces to the right\footnote{for $\DDD^\vee$ this becomes the length of the pieces to the left} of $A_j$ if $r$ is odd, and equals $0$ if $r$ is even.

$\Omega_*(r)^{\nf}$ has a free $\Cyc_r$ action away from the empty simplex. The differential decomposes as $\partial^{\f}+\partial^{\f}$, where $\partial^{\nf}$ consists on summands without full pieces and $\partial^{\f}$ on summands with a single full piece.

If $(U,A)$ is a generator of $\Omega_*(r,m)$, then the word $A$ has a canonical piece decomposition: a sequence $(a_i,\ldots,a_{i+k})$ is a piece if and only if $u_{i-1}<u_i =\ldots= u_{i+k}<u_{i+k+1}$. This defines, for each $m\geq 0$, a chain map
\[\lambda_m\colon \Omega_*(r,m)\to \Omega_*(r).\]
that satisfies
\begin{align*}
	\lambda_m\left(\sum_i d_i^\vee(U,A)\right) &= \DDD\circ \lambda_m.
\end{align*}
%
\begin{example}\label{ex:106}
	$\lambda_6((001123345566,012010212012)) = 01|20|1|02|1|20|12$
\end{example}

%If $w'$ is a subword of $w$ and $(U,w)$ is a pair, define the pair $(U|_{w'},w')$ in the obvious way.
%where $w^{*}=w$ if $w$ has no full piece and $0$ otherwise and $w^{**}$ is the opposite.

%A pieced word $w$ and a generator $U\in C^q(\Delta^m)$ are \emph{compatible} if they have the same length and if and only if the sequence $(w_i,\ldots,w_{i+k})$ is a piece. If $U$ and $w$ are compatible, write $(U,w)$ for the generator of $\hat{\Omega}^*(r,m)$ that is obtained by forgetting the pieces of $w$. Otherwise, define $(U,w) = 0$.

\begin{lemma} \label{lemma:omegar}
	Suppose one is given a $\Cyc_r$-equivariant chain map
	\begin{align*}
		\Psiom\colon \Omega_*(r)^{\nf}&\lra \rW_*(r)
	\end{align*}
	such that
	\begin{itemize}
		\item $(\tilde{r}!)\Psiom_{q-1}((\partial^0 (A)) = (-1)^{q-r}\theta_{1-r}\check{\Psi}_{q-r}\DDD(A)$ if $A$ has a full piece.
	\end{itemize}
	Then the family of maps $\Psiom^m = (\tilde{r}!)^{m}\Psiom\lambda_m$ satisfies the conditions of Proposition \ref{prop:omegarm}.
\end{lemma}

\begin{proof}
	Immediate.
\end{proof}

\begin{remark}
	A particular homomorphism satisfying the hypotheses of Lemma \ref{lemma:omegar} will be found in the next section. Though it is defined with coefficients in $\bZ[\frac{1}{\tilde{r}!}]$, the maps $\{\Psiom^m\}$ obtained in the lemma are well-defined with integer coefficients.
\end{remark}
% !TEX root = ../oddp.tex

\section{Alexander duality on the simplex and tensor products}\label{s:3complexes}

In this section we present a duality on the normalized chains of the augmented simplex. The main result of this section is \cref{prop:milnor_to_minimal}, where we find that building a connected $r$-cyclic diagonal is equivalent to build a map between two augmented resolutions of the cyclic group that intertwines certain endomorphisms of the resolutions.
\subsection{Alexander duality on the augmented standard simplex}\label{s:alexander}

%Given a simplicial set $X$, let $\chains(X) = \sus{}\cadenas(X)$ be the right suspension of the normalized chains of $X$. \federico{I propose that }

\begin{definition}\label{d:poincare_duality_algebra}
	Let $A$ be a connected commutative algebra that is finite dimensional for each degree.
	We say $A$ is a \textit{Poincar\'e duality algebra} of \textit{formal dimension} $d$ if:
	\begin{enumerate}
		\item\label{i:pd1} $A_i = 0$ for $i > d$,
		\item\label{i:pd2} $\dim A_d = 1$,
		\item\label{i:pd3} $A_i \ot A_{d-i} \to A_d$ is non-degenerate.
	\end{enumerate}
\end{definition}

\begin{definition}\label{d:join_product}
	For any $[n] \in \ob\asimplex$ the \textit{join product} $\ast \colon \chains(\asimplex^{n})^{\ot 2} \to \chains(\asimplex^{n})$ is the linear map defined by sending a basis element $[v_0, \dots, v_{p-1}] \ot [v_{p},\dots,v_{m-1}]$ to $(-1)^{\sign{\perm}}[v_{\perm(0)}, \dots, v_{\perm(m-1)}]$	if $v_i \neq v_j$ for all $i \neq j$, where $\perm$ is the permutation ordering the vertices, and to $0$ otherwise. \federico{Here we can avoid the permutation $\pi$ writing $[v_0,\ldots,v_{p-1},v_p,\ldots,v_{m-1}]$}
\end{definition}

\begin{notation}
	If $\tau = [v_0,\dots,v_{k-1}]$ and $\tau'=[u_0,\dots,u_{m}-1]$ are two non-degenerate simplices of $\asimplex^n$, let $\lambda(\tau,\tau')$ be the parity of the permutation that orders the sequence $[v_0,\dots,v_{k-1},u_0,\dots,u_{m-1}]$. Let $\tau^c$ be the complementary face of $\tau$. The top generator $[0,1,\dots,n]$ will be denoted $\sigma$.
\end{notation}

\begin{theorem}
	The join product defines on $\chains(\asimplex^{n})$ the structure of a Poincar\'e duality algebra with unit the empty simplex $[0] \to [n+1]$ and formal dimension $n+1$. This structure is natural with respect to simplicial maps.
\end{theorem}

\begin{proof}
	The complex $\chains(\asimplex^{n})$ is connected and satisfies \cref{i:pd2,i:pd3} in \cref{d:poincare_duality_algebra} since $\chains(\asimplex^{n})_0 \cong \Z\{[0] \to [n+1]\}$, $\chains(\asimplex^{n})_{n+1} \cong \Z\{[n+1] \to [n+1]\}$, and $\chains(\asimplex^{n})_{n+k} \cong 0$ for $k>1$.

	That the join product is a natural chain map can be easily verified and a complete proof is presented in \cite[p.19]{medina2020prop1}.

	Thinking about the join product in terms of the union of sets with a permutation sign leads to a direct verification of its commutativity (in the graded sense) and unitality with respect to the empty simplex.

	To verify \cref{i:pd3} consider a basis element $x = [v_1,\dots,v_i]$.
	Let $x^c$ be the ordered complement of $\set{v_1,\dots,v_i}$ in $\{0,\dots,n\}$ and notice that $x \ast x^c = \pm [0,\dots,n]$ as required.
\end{proof}

This pairing thus induces an isomorphism
\begin{equation}\label{eq:iso1}
	\Lambda \colon \chains(\asimplex^n) \to \susp{n+1}\cochains(\asimplex^n)
\end{equation}
that sends a chain $\tau$ to $(-1)^{\lambda(\tau,\tau^c)}(\tau^c)^\vee$. We will refer to this isomorphism as \emph{Alexander duality}.

\begin{remark}\label{remark:alex}
	If $\tau = [v_0,\dots,v_k]$ and $\tau^c = [u_0,\dots,u_m]$ are the ordered representatives, then
	\begin{align*}
		\lambda(\tau,\tau^c)&\equiv \sum_{i=0}^k (v_i-i) \equiv \sum_{i=0}^k v_i - \binom{k+1}{2} \\
		\lambda(\tau,\tau^c)&\equiv \sum_{j=0}^m (n-u_j-m+j) \equiv \sum_{j=0}^m u_j +(n-m)(m+1)+\binom{m+1}{2}.
	\end{align*}
\end{remark}

\begin{notation}
	We will denote normalized chains between brackets and normalized cochains between parentheses.
\end{notation}

\subsection{Simplicial functor tensor products}

Recall that if $A$ is an augmented cosimplicial chain complex and $B$ is an augmented simplicial chain complex, the functor tensor product $A \ot_{\asimplex} B$ is the quotient of the tensor product $A \ot B$ by the relation generated by
\begin{align*}
	\d^i a \ot b &\sim (-1)^{|\d^i||a|}a \ot \d_i b
	&
	\s^ia \ot b &\sim (-1)^{|\s^i||a|}a \ot \s_ib.
\end{align*}
The chain complexes $\chains(\asimplex^\bullet)$ form an augmented cosimplicial chain complex with
\begin{align*}
	\d^i \colon \chains(\asimplex^n)& \lra \chains(\asimplex^{n+1})
	&
	\s^i \colon \chains(\asimplex^{n})& \lra \chains(\asimplex^{n-1})
\end{align*}
given on an ordered representative $[v_0,\dots,v_{k-1}]\in \chains(\asimplex^n)$ by
\begin{align*}
\d^i([v_0,\dots,v_{k-1}]) &= [v_0,\dots,v_{j-1},v_j+1,\dots,v_{k-1}+1] \quad \text{if $v_{j-1}<i\leq v_j$}
\\
\s^i([v_0,\dots,v_{k-1}]) &= [v_0,\dots,v_{j-1},v_j-1,\dots,v_{k-1}-1] \quad \text{if $v_{j-1} \leq i < v_j$}.
\end{align*}
If $X$ is an augmented simplicial object in $\cC$, the chain complex $\chains(X)$ can alternatively be described as the functor tensor product over $\asimplex$ of the augmented cosimplicial chain complex $\chains(\asimplex^{\bullet})$ and the augmented simplicial $R$-module $\abel(X)_\bullet$:
\begin{equation}\label{eq:1}
	\chains(X)
 \cong 
 \chains(\asimplex^\bullet) \ot_{\asimplex} \abel(X)_\bullet.
\end{equation}
This sends a simplex $[k] \to [n]$ and a simplex $[n] \to X$ to the composition $[k] \to X$.

\begin{proposition}\label{prop:cosimplicial}
	A cosimplicial map
	\[
		\Wdualaug{r}\ot_\theta \rchains(\asimplex^n)\lra \chains(\asimplex^n)^{\ot r}
	\]
	gives rise to a connected comultiplication.
\end{proposition}
\begin{proof}
	The tensor product of this map and the map $L(X)\to L(X)^{\ot r}$ induced by the diagonal yields an equivariant map 
	\[
		\Wdualaug{r}\ot_\theta \rchains(\asimplex^n)\ot_{\asimplex} L(X)\lra \chains(\asimplex^n)^{\ot r}\ot_{\asimplex} L(X)^{\ot r}
	\]
	and therefore a connected comultiplication.
\end{proof}

Via Alexander duality, we may endow $\cochains(\asimplex^n)$ with the following augmented \emph{cosimplicial} structure
\begin{align*}
	\d^i \colon \cochains(\asimplex^n) & \lra \susp{}\cochains(\asimplex^{n+1})
	&
	\s^i \colon \cochains(\asimplex^{n+1}) & \lra \susp{}\cochains(\asimplex^n)
\end{align*}
given on a generator $U=(u_0,\dots,u_{m-1})$ by
\begin{align*}
	\d^i(U) &=
	(-1)^{i+j+m+n+1}(u_0,\dots,u_{j-1},i,u_{j}+1,\dots,u_{m-1}+1)
	\quad \text{if $u_{j-1}<i \leq u_{j}$}
	\\
	\s^i(U) &= \begin{cases}
	(-1)^{i+j+m+n+1}(u_0,\dots,u_{j-1},u_{j+1}-1\dots,u_{m-1}-1) & \text{if $u_{j} = i$,}
	\\
	0 & \text{otherwise.}
	\end{cases}
\end{align*}
The face maps of this cosimplicial structure have degree $1$ while degeneracies have degree $-1$. %Let $\rA(X)$ be the augmented simplicial graded $R$-module with $\rA(X)_n = \susp{n+1}\abel(X)_{n}$, and face maps and degeneracies are induced by those of $\abel(X)$ and have degree $-1$ and $+1$ respectively.

\subsection{Another presentation of $\chains(\asimplex^n)^{\ot r}$}\label{s:join} The duality maps
\begin{align*}
	\Lambda^r\colon \chains^r(\asimplex^n) &\lra \Sigma^{r(n+1)}\chains^{r,\vee}(\asimplex^n)
	&
	\Lambda^{\ot r}\colon \chains(\asimplex^n)^{\ot r} & \lra \Sigma^{r(n+1)}\cochains(\asimplex^n)^{\ot r}
\end{align*}
have both degree $r(n+1)$, thus building a cosimplicial map (of degree $0$)
\[
	\Wdualaug{r}\ot_\theta \rchains(\asimplex^\bullet)  \lra \chains(\asimplex^\bullet)^{\ot r}
\]
is equivalent to build a cosimplicial map (of degree $0$)
\[
	\Wdualaug{r}\ot_\theta \chains^{r,\vee}(\asimplex^\bullet) \lra \cochains(\asimplex^\bullet)^{\ot r}
\]
and, taking linear duals, equivalent to build a simplicial map (of degree $0$)
\[
	\chains(\asimplex^\bullet)^{\ot r}  \lra \Waug{r}\ot_\theta \rchains(\asimplex^\bullet)
\]
%The explicit simplicial structure on the simplicial chain complex of the last equation is the following:

%RELLENAR

The tensor product $\chains(\asimplex^n)^{\otimes r}$ is isomorphic to the chain complex $\chains(\asimplex^n*\overset{r}{\cdots}*\asimplex^n)$, which in turn is isomorphic to the chain complex $\chains(\asimplex^{r-1}*\overset{n+1}{\cdots}*\asimplex^{r-1})$, which is isomorphic to $\chains(\asimplex^{r-1})^{\otimes n+1}$. The cyclic group $\Cyc_{r}$ acts on $\chains(\asimplex^n)^{\otimes r}$ permuting the factors and on $\asimplex^{r-1}$ permuting the vertices. These isomorphisms are equivariant with respect to these actions, and we denote their composition by $\alpha$. Let $\beta$ be the endomorphism of $\chains(\asimplex^{r-1})^{\otimes n+1}$ given by $\rho^0\otimes \rho^{-1}\otimes \ldots \otimes \rho^{-n}$. Then, via the isomorphism
\[
	\chains(\asimplex^n)^{\otimes r} \overset{\alpha}{\lra}
	\chains(\asimplex^{r-1})^{\otimes n+1} \overset{\beta}{\lra}
	\chains(\asimplex^{r-1})^{\otimes n+1}   
\]
$\chains(\asimplex^{r-1})^{\otimes \bullet}$ becomes a simplicial chain complex with the following explicit face maps of degree $r$.
\begin{align*}
	\d_i\colon \chains(\asimplex^{r-1})^{\ot (n+1)} & \lra \sus{r}\chains(\asimplex^{r-1})^{\ot n} \\
	\tau_0\ot \ldots \ot \tau_n &\longmapsto 
	%\begin{cases}
	(-1)^{i+\sum_{j>i} |\tau_j| + n+1}\tau_0\ot \ldots \ot \tau_{i-1}\ot \rho(\tau_{i+1})\ot \ldots \ot \rho(\tau_n) %& \text{ } %\\
	%0 & \text{ if $|\tau_i|<r$.}
	%\end{cases}
\end{align*}
if $\tau_i = \sigma$ and $0$ otherwise. Define the chain map $\theta_j\colon \sus{r-1}\chains(\partial\asimplex^{r-1})^{\ot m}\to \chains(\partial\asimplex^{r-1})^{\ot (m+1)}$ as 
\[
	\theta_j(\tau_0\ot \ldots \ot \tau_m) = (\tau_0\ot \ldots \tau_{j}\ot \partial\sigma \ot \rho^{-1}(\tau_{j+1})\ot \ldots \ot \rho^{-1}(\tau_{m})).
\]
If $b = [b_0,\ldots,b_{n-1}]\in \rchains(\asimplex^n)$, let $\varphi(j,b) = j-b_j$.
\federico{Let $\mathbf{2^n}$ be the ordered poset of subsets of $\{0,1,\ldots,n-1\}$. There is a contravariant automorphism $\Lambda$ of this category that sends a subset to its complement. If $F,G\colon \mathbf{2^n}\to \Mod{R}$ are two functors with the same variance, define their tensor product $F\ot G\colon \mathbf{2^n}\to \Mod{R}$ levelwise. There is a functor $\mathbf{2^n}\to \asimplex$, and taking Kan extension along it produces a functor that takes presheaves on $\mathbf{2^n}$ to presheaves on $\asimplex$. 

The poset of subsets of $\{0,1,\ldots,n-1\}$ is in bijection with the poset of faces of $\asimplex^n$. Thus, we define $\bar{\simplex}_+^n\colon \mathbf{2^n}\to \Mod{R}$ as the constant functor with value a singleton. Define now $F\colon \mathbf{2^n}\to \Mod{R}$ as $F(A) = \bigotimes_{a\in A} \chains(\asimplex^{r-1})$. If $A\subset B$ and they only differ on the $j$th element of $B$, then define $F(A\subset B) = \theta_j$.
}


There is an isomorphism of graded modules
\[
	\gamma\colon \chains(\asimplex^{r-1})^{\ot (n+1)}\lra \bigoplus_{p+q=n}\chains(\partial \asimplex^{r-1})^{\ot p}\ot \rchains(\asimplex^{n})_{rq}
\]
that sends a generator $(\tau_0\ot \ldots \ot \tau_n)$ whose top simplices are $\tau_{i_1}\ldots \tau_{i_q}$ to the pair $\pm \tau'\ot \tau''$ where $\tau'$ is the result of removing the top simplicies from $\tau$ and $\tau'' = [i_1,\dots,i_q]$ is the list of positions of these simplices and $\pm$ is the Koszul sign of this rearrangement.

Under this isomorphism, the differential on the right hand-side becomes
\[
	\partial(a\ot b) = \partial(a)\ot b + (-1)^{|a|}\sum_j (-1)^j \theta_{\varphi(j,b)}(a)\ot \d_j(b).
\]
We denote this chain complex by
\[
%	\chains(\asimplex^{r-1})^{\ot (n+1)}\cong 
\chains(\partial\asimplex^{r-1})^{\ot \bullet}\ot_\theta \rchains(\asimplex^n).
\]
We summarize these reflections in the following proposition.
\begin{proposition}\label{prop:milnor_to_minimal}
	A family of chain maps $\Psi_n\colon \chains(\partial \asimplex^{r-1})^{\ot n}\to \Waug{r}$ such that $ \theta\circ \Psi_n = c \cdot  \Psi_{n+1} \circ \theta_j$ for each $j$ and for some constant $c$ yields a 
simplicial chain map 
\[
	\mu_\bullet\colon \Wdualaug{r}\ot_\theta \rchains(\asimplex^\bullet)\to \chains(\asimplex^\bullet)^{\ot r}
\]
given by 
\[
	\mu(e_q^\dd\ot \vec{\tau}) = c^{n+1}(\alex^{-1})^{\ot r}\circ(\alpha\circ \beta\circ \gamma)^\dd (\Psi_n^\dd(e_q^\dd)\ot \alex^r(\vec{\tau})).
\]
\end{proposition}
\begin{remark}\label{remark:many_coefficients}
	In our use of the proposition, the constant will be $c= \tilde{r}!$, where $\tilde{r} = \frac{r-1}{2}$. A careful look at the definition of the map of \cref{prop:chains_to_minimal} shows that after multiplying by $\tilde{r}^{n+1}$, one may take arbitrary coefficients (instead of $R[\frac{1}{\tilde{r}!}]$ coefficients).
\end{remark}


% !TEX root = ../oddp.tex

\section{Some cochain complexes}

There is a Poincaré duality isomorphism
\[PD\colon N_*(\Delta^n)\lra N^{n-1-*}(\Delta^n)\]
that sends a simplex to its complement in the maximal simplex.

\subsection{Models of $N_*(\Delta^n)^{\otimes r}$}

Consider the functor tensor product $F\otimes_{\Delta}G$ of the following two functors:
\begin{align*}
	F\colon \Delta&\lra \Ch(R)& G\colon \Delta^{\op}&\lra \Ch(R)\\
	F([m])&=\Sigma^{(r-1)m-r}(N^*(\Delta^m)^{\otimes r})& G[m] &= N_m(\Delta^n).
\end{align*}

\begin{lemma}
	The map $\alpha\colon F\otimes_{\Delta} G\lra N_*(\Delta^n)^{\otimes r}$ given by
	\[\alpha((U_1,\ldots,U_r)\otimes \tau) = (d_{U_1}^\tau(\tau),\ldots, d_{U_r}^\tau(\tau))\]
	is a $C_r$-isomorphism of chain complexes.
\end{lemma}

Consider now the chain complex $\bar{\Omega}_*(r,n)$ that in degree $k$ is generated by equivalence classes of triples $(U,w,\tau)$ where \fcnote{This is actually a quotient of a functor tensor product. Maybe it is worthy defining it like that.}
\begin{itemize}
	\item $U$ is a generator of $C_{(r-1)m-1-k}(\Delta^m)$,
	\item $w$ is a generator of $C_{(r-1)m-1-k}(EC_r)$ (a word),
	\item $\tau$ is a generator of $N_m(\Delta^n)$
\end{itemize}
such that
\begin{itemize}
	\item $(U,w)$ is a non-degenerate simplex in $EC_r\times \Delta^m$
\end{itemize}
subject to the following relations:
\begin{itemize}
	\item (Inner reorderings) $(U,w,\tau)\sim (-1)^{|\sigma|}(U,w',\tau)$ if there is an interval $u_{i-1}<u_i =\ldots u_{i+k}<u_{i+k+1}$ such that the sequences $(w_i,\ldots,w_{i+k})$ and $(w'_i,\ldots,w'_{i+k})$ differ by a permutation $\sigma$ and $w$ and $w'$ agree outside that interval. Notice that if all elements of $U$ are different, then $w=w'$.
	\item (Full piece removal) $(U,w,d_i(\tau))\sim (-1)^r(D_i(U,w),\tau)$, where $D_i(U,w) = (U',w')$ and
	\begin{align*}
		U'_j &= \begin{cases} U_j &\text{ if $j<i$} \\ i & \text{ if $i\leq j < i+r$} \\ U_{j-r} + 1 & \text{ if $j\geq i+r$.}\end{cases} &
		w'_j &= \begin{cases} w_j &\text{ if $j<i$} \\ w_{j-r} - 1 & \text{ if $j\geq i+r$.}\end{cases}
	\end{align*}
	and $(w'_{i},w'_{i+1},\ldots,w'_{i+r-1}) = (0,1,\ldots,r-1)$ (a \emph{full piece}).
\end{itemize}
This complex has a differential defined as follows: The full piece removal defines an order relation on each equivalence class. Then $\partial([(U,w,\tau)]) = [\partial(\bar{U},\bar{w},\tau)]$ where $(\bar{U},\bar{w},\tau)$ is a maximal representative of $[(U,w,\tau)]$ (in order to compute it it is enough to consider representatives that have a single full piece).

\begin{lemma}
	The map $\beta'\colon \bar{\Omega}^*(r,n)\lra \bar{\Omega}^*(r,n)$ given by sending $(U,w,\tau)$ to $(U,w',\tau)$, where $w_i' = u_i+w_1\mod r$ is a $C_r$-equivariant isomorphism of chain complexes.
\end{lemma}

Given a generator $[(U,w,\tau)]$, define for each $0\leq i\leq r-1$, $U_w^i \subset U$ as the subset of those entries $u_j$ such that $w_j=i$. Define the word $w_U$ as $(\omega_U)_j = \omega_j$ and $\sigma(U,w)$ as the sign of the permutation that arranges $\omega_U$ in ascending order. \fcnote{This map may be further factored?}

\begin{lemma}
	The map $\beta\colon \bar{\Omega}^*(r,n)\lra F\otimes_\Delta G$ given by
	\[\beta([U,w,\tau]) = (-1)^{\sigma(U,w)}(U_w^0\otimes\ldots\otimes U_w^{r-1})\otimes \tau\]
	is a $C_r$-equivariant isomorphism of chain complexes.
\end{lemma}

\subsection{Models of $W_*(r)\otimes N_*(\Delta^n)$}

Recall that $W_*(r)$ is the augmented minimal resolution of $C_r$.

\begin{definition}
	If $k\geq -1$ is odd, define the \emph{flip map} $\varphi_k\colon W_*(r)\to W^{k-*}(r)$ as $e_q\mapsto e^{\vee}_{k-q}$. If $k$ is an even integer, define the \emph{suspension map} $\theta_k\colon W_*(r)\to W_{*-k}(r)$ as $\theta_k(e_q) = e_{q-k}$.
\end{definition}

Let $r$ be odd. Define the graded complex
\[V^{*}(r) = \bigoplus_{m}W^{(r-1)m-1-*}(r)\otimes N_m(\Delta^n)\]
with the differential $d(e^\vee_q\otimes \tau) = d(e^\vee_q)\otimes \tau + (-1)^q \theta_{r-1}(e^\vee_q)\otimes d(\tau)$%\fcnote{the sign works even better with $(-1)^n$. Maybe the other order is better?}. Define a $C_r$-equivariant homomorphism of chain complexes
\[\varphi\colon W_*(r)\otimes N_*(\Delta^n)\lra V^{*}(r)\]
as $e_{q}\otimes \tau\mapsto \varphi(e_q)\otimes \tau$.


% !TEX root = ../oddp.tex

\section{Some cochain complexes}


All complexes will be finitely generated free complexes with a prescribed basis, we will denote the dual basis with the same letter as the original basis unless we want to emphasize the duality, in which case we will denote the dual of $u$ by $u^\vee$. Often it will be convenient to shift by one the degree of normalised and unnormalised chain complexes of simplicial sets. We will write $\NN_*(X_\bullet)$ for this shifted complex. 

\subsection{Poincaré duality}
There is a Poincaré duality isomorphism
\[PD\colon N_*(\Delta^n)\lra N^{n-1-*}(\Delta^n)\]
that sends a simplex to its complement in the maximal simplex. In this isomorphism, the differential on the normalised cochain complex is not the dual of the usual differential: Writing $\delta^\dagger$ for the new differential and $\delta$ for the old differential, we have: 
\begin{align*}
    \delta^\dagger(U) &= \sum_{u\notin U} (-1)^{t_U(u)}U\cup \{u\}
    &
    \delta(U) &= \sum_{u\notin U}(-1)^{s_U(u)}U\cup \{u\},
\end{align*}
where $s_U(i)$ is the number of elements smaller than $u$ in $U$ and $t_U(u)$ is the number of elements smaller than $u$ in $\{0,\ldots,n\}\smallsetminus U$. We will always use the old differential unless specified.

There is an isomorphism of cochain complexes
\[(N^*(\Delta^n),\delta^\dagger)\longrightarrow N^*(\Delta^n),\delta)\]
given by sending a simplex $\tau$ to $(-1)^{\lambda(\tau)}\tau$, where $ \lambda(\tau)$ is the sum of the entries of $\tau$ (for example, $s(0,3,4) = 7$).

There is a cap product map
\[Q\colon \NN^*(\Delta^m)\otimes N_m(\Delta^n)\lra N_{m-*}(\Delta^n)\]
that sends $Q(U\otimes \tau) = (-1)^{\lambda(U)}d_U(\tau)$. Here (and later) $\NN^*$ is the suspension of $N^*$ that does not change the sign of the differential.

\subsection{A reordering} Consider the $r$-fold tensor product $\chains(\Delta^n)^{\otimes r}$ with the backwards cyclic action of $C_r$:
\[\rho(U_0\otimes \ldots\otimes U_{r-1}) = U_1\otimes \ldots \otimes U_{r-1}\otimes U_0\]
and the forward cyclic action:
\[\rho(U_0\otimes \ldots\otimes U_{r-1}) = U_{r-1}\otimes U_0\otimes \otimes \ldots \otimes U_{r-2}.\]
Let $\mathrm{rev}\colon N_*(\Delta^n)^{\otimes r}\to N_*(\Delta^n)^{\otimes r}$ be the $C_r$-equivariant homomorphism that sends a product $(\tau_0\otimes\ldots\otimes \tau_{r-1})$ to the product $(-1)^{\xi(\tau_{0},\ldots,\tau_{r-1})}(\tau_{r-1}\otimes\ldots\otimes \tau_0)$, where $\xi(\tau_0,\ldots,\tau_{r-1})$ is the Koszul sign associated to the reversal. 
\begin{example}\label{ex:101}
    If $r=3$ and $(\tau_0,\tau_1,\tau_2) = [0,2,3]\otimes [3,4,5] \otimes [5,7,9]$, then its reversal is $-[5,7,9]\otimes [3,4,5]\otimes [0,2,3]$.
\end{example}


\subsection{First model of $N_*(\Delta^n)^{\otimes r}$} Recall that $\Delta^n\colon \Delta^{\op}\to \Set$ is the standard simplex of dimension $n$. Consider the subset $\Delta(\Delta^n)^{*r}$ of $\Delta^n* \overset{r}{\ldots}* \Delta^n$ of \emph{diagonal join simplices}, i.e., simplices in the join of the form $\vec{\tau} = (\tau,\ldots,\tau)$, where $\tau\in \Delta^n[m]$ is an $m$-simplex. Define $N_m^r(\Delta^n)$ to be the free graded $R$-module on the $rm$-simplices of $\Delta(\Delta^n)^r$. There is an isomorphism $N_m(\Delta^n)\lra N_m^r(\Delta^n)$ of degree $(r-1)m$ given by sending a simplex $\tau$ to the product simplex $\vect{\tau}$. Additionally, every face map $d_i\colon [m-1]\to [m]$ induces a homomorphism $N_m^r(\Delta^n)\to N_{m-1}^r(\Delta^n)$ of degree $r$ that sends a diagonal product simplex $\vec{\tau}$ to $d_i\vec{\tau}:= (\tau\circ d_i,\ldots,\tau\circ d_i)$. Therefore, we have defined a functor $N_\bullet^r(\Delta^n)\colon \Delta^{\op}\to \Ch(R)$ with values in graded $R$-modules.

Define now another functor $\NN_*(\Delta^\bullet)\colon \Delta\to \Ch(R)$ whose value on a finite ordinal $m$ is the cochain complex $\NN_*(\Delta^m)$ and, for a face map $d_i\colon [m]\to [m+1]$, define $d_i(U) = (-1)^{|U_{\geq i}|}D_i(U)$, where 
\[D_i(U) = \left\{
x\left| \begin{array}{l}
x<i \text{ and } x\in U \\
x=i \\
x>i \text{ and } x-1 \in U.
\end{array}\right.
\right\}
\]
(here the point of taking suspension is that Poicaré Duality has degree $n$ instead of $n-1$, and also that the join respects degrees, and also the sign convention for the tensor product and the join agrees).


Consider the functor tensor product $\NN^*(\Delta^\bullet)^{\otimes r}\otimes_{\Delta} N_\bullet^r(\Delta^n)$. %of the functors:
%\begin{align*}
%	F\colon \Delta&\lra \Ch(R)& G\colon \Delta^{\op}&\lra \Ch(R)\\
%	F([m])&=(\mathrm{sh} N^*(\Delta^m))^{\otimes r}& G[m] &= N_{rm}(\Delta(\Delta^n)^{*r}),
%\end{align*}
 The functor tensor product is generated by tuples $ (U_0,\ldots,U_{r-1})\otimes \vec{\tau}$, up to the equivalence relation generated by: 
\begin{equation}\label{eq:59}(U_0,\ldots,U_{r-1})\otimes d_j\tau\sim (-1)^{\sum_k |(U_k)_{\geq j}|} (D_j(U_0),\ldots,D_j(U_{r-1}))\otimes \tau.
\end{equation}
Every equivalence class has a unique representative of minimal dimension, characterised by $\bigcap U_j = \emptyset$ and a maximal representative characterised by $\vec{\tau}$ being the top simplex in $N_n^r(\Delta^n)$. The differential of a class $[(U_0,\ldots,U_{r-1})\otimes \tau]$ is the class of the differential of its minimal element. In the dual complex $N_*(\Delta^\bullet)^{\otimes r}\otimes_\Delta N_\bullet(\Delta^n)$, the differential on a class $[\tau\otimes (U_1,\ldots,U_r)]$ is the class of the differential of its maximal element. The cyclic group $C_r$ acts by permuting the tensor factors of $\NN_*(\Delta^\bullet)^{\otimes r}$.
\begin{definition} Given a generator $(U_0,\ldots,U_{r-1})$ of $(\NN^*(\Delta^m))^{\otimes r}$, let $U$ be the union of the sequences $U_0,\ldots U_{r-1}$. Consider the following quatities:
\begin{itemize}
    \item $\lambda(U) = \sum_{u\in U} u$. The sign $(-1)^{\lambda(U)}$ is used to change the differential $\delta^\dagger$ to $\delta$.
    \item $\mu(U_0,\ldots,U_{r-1},\tau) &= |\tau|\cdot \sum_{i\text{ odd}} |U_i|$. The sign $(-1)^{\mu(U_0,\ldots,U_{r-1},\tau)}$ is the sign that has to be paid to permute the tuple $(U_0,\ldots,U_{r-1})\otimes (\tau,\ldots,\tau)$ to the tuple $(U_0\otimes \tau), \ldots\otimes (U_{r-1}\otimes \tau)$.
    \item $\nu_i(U) = |U_{\geq i}|$. The sign $(-1)^{\nu_i(U)}$ is the sign that has to be paid in \eqref{eq:59}.
\end{itemize}
\end{definition}
\begin{lemma}
	The map $\alpha\colon \NN_*(\Delta^\bullet)^{\otimes r}\otimes_{\Delta} N_\bullet^r(\Delta^n)\lra N_*(\Delta^n)^{\otimes r}$ given on minimal representatives by
	\[\alpha((U_0,\ldots,U_{r-1})\otimes \vec{\tau}) = (-1)^{\lambda(U)+ \mu(U_0,\ldots,U_{r-1},\tau)}(d_{U_0}^\tau(\tau)\otimes \ldots\otimes d_{U_{r-1}}^\tau(\tau))\]
	is a $C_r$-isomorphism of chain complexes.
\end{lemma}
\begin{proof} This map is the composition of two maps: First, the map that rearranges the tuple $(U_0,\ldots,U_{r-1})\otimes (\tau,\ldots,\tau)$ to the tuple $(U_0\otimes \tau),\ldots,(U_{r-1})\otimes \tau$, in which one has to pay the sign $(-1)^{\mu(U_0,\ldots,U_{r-1},\tau)}$. Second, the $r$-fold tensor product of the map $Q$, in which one has to pay the sign $(-1)^{\lambda(U)}$. Finally, this is well-defined on equivalence classes [TO BE DONE].

It has an inverse given by sending $\tau_1\otimes\ldots \otimes \tau_r$ to $(\sigma\otimes (PD(\tau_1),\ldots,PD(\tau_r)))$ with appropriate sign, where $\sigma$ is the generator of $N_n(\Delta^n)$. Equivariance is also immediate.
\end{proof}
\begin{example}\label{ex:102}
    The element $[5,7,9]\otimes [3,4,5]\otimes [0,2,3]$ is the image of the element $((0,1,2,3),(0,1,5,6),(3,4,5,6))\otimes [0,2,3,4,5,7,9]$ with the signs $\lambda(U) = 36$ and $\mu(U_0,U_1,U_2,\tau) = 0$, hence the sign is $+1$.
\end{example}

%which now we describe: Let $\tau\otimes (U_1,\ldots,U_r)$ be a minimal element in an equivalence class, and let $[\tau\otimes (U_1,\ldots,U_r)]^f$ be the set of those representatives such that $|\bigcap U_j| = 1$. Then
%\[\partial [\tau\otimes (U_1,\ldots,U_r)] = 
%\sum_{(\hat{\tau}\otimes(\hat{U}_1,\ldots,\hat{U}_r))\in [\tau\otimes (U_1,\ldots,U_r)]^f} [\hat(tau)\otimes \partial (U_1,\ldots,d_iU_k,\ldots,U_r)]- [\tau\otimes (U_1,\ldots,U_k,\ldots,U_r)] 
%\]

%\[\partial[\tau\otimes (U_1,\ldots,U_r)] = 
%\sum_{j,\tau'\mid d_j\tau' = \tau} (-1)^{j} [\tau',(\xi_j(U_1),\ldots,d_j\xi_j(U_k),\xi_j(U_r))] + 
%\]



\subsection{Second model of $N_*(\Delta^n)^{\otimes r}$} Consider now the quotient $\Theta^*(r,n)$ of the functor tensor product $\NN^*(\Delta^\bullet\times EC_r)\otimes_\Delta N_\bullet(\Delta^n)$
by the following relation:
\begin{itemize}
    	\item (Inner reorderings) $(U,w)\otimes \vec{\tau}\sim (-1)^{|\sigma|}(U,w')\otimes \vec{\tau}$ if there is an interval $u_{i-1}<u_i =\ldots u_{i+k}<u_{i+k+1}$ such that the sequences $(w_i,\ldots,w_{i+k})$ and $(w'_i,\ldots,w'_{i+k})$ differ by a permutation $\sigma$ and $w$ and $w'$ agree outside that interval. Notice that if all elements of $U$ are different, then $w=w'$.
\end{itemize}
In detail, $\Theta^*(r,n)$ is generated by equivalence classes of triples $(U,w)\otimes \tau$ where
\begin{itemize}
	\item $U$ is a generator of $\CC^q(\Delta^m)$,
	\item $w$ is a generator of $\CC^q(EC_r)$ (a word),
	\item $\vec{\tau}$ is a generator of $N_m^r(\Delta^n)$.
\end{itemize}
such that
\begin{itemize}
	\item $(U,w)$ is a non-degenerate simplex in $\Delta^m\times EC_r$
%	\item There is at most one \emph{full piece}, i.e., a sequence $u_i=u_{i+1}=\ldots=u_{i+r-1}$ of $r$ consecutive equal elements in $U$.
\end{itemize}
subject to the following relations:
\begin{itemize}
	\item (Inner reorderings) $(U,w)\otimes\vec{\tau}\sim (-1)^{|\sigma|}(U,w')\otimes \vec{\tau}$ if there is an interval $u_{i-1}<u_i =\ldots u_{i+k}<u_{i+k+1}$ such that the sequences $(w_i,\ldots,w_{i+k})$ and $(w'_i,\ldots,w'_{i+k})$ differ by a permutation $\sigma$ and $w$ and $w'$ agree outside that interval.
	\item (Full piece removal) $(U,w)\otimes d_i(\vec{\tau})\sim (-1)^{|U_{\ge i}|}D_i^r(U,w)\otimes \vec{\tau}$, where $D_i^r(U,w) = (U',w')$ and
	\begin{align*}
		U'_j &= \begin{cases} U_j &\text{ if $j<i$} \\ i & \text{ if $i\leq j < i+r$} \\ U_{j-r} + 1 & \text{ if $j\geq i+r$.}\end{cases} &
		w'_j &= \begin{cases} w_j &\text{ if $j<i$} \\ w_{j-r} - 1 & \text{ if $j\geq i+r$.}\end{cases}
	\end{align*}
	and $(w'_{i},w'_{i+1},\ldots,w'_{i+r-1}) = (0,1,\ldots,r-1)$.
\end{itemize}
Again, every class has a unique minimal (maximal) representative $(U,w)\otimes \vec{\tau}$ up to inner reorderings, characterised by not having full pieces ($\tau$ being the top simplex). The differential of a class is the class of the differential of its minimal representative. There is a dual complex $\Theta_*(r,n)$ that can be understood by replacing cochains in $\Delta^m\times EC_r$ by chains. Its differential on a class is the class of the differential of its maximal representative. The cyclic group $C_r$ acts on $\Theta^*(r,n)$ as $\rho([(U,w,\vec{\tau})]) = [(U,\rho(w),\vec{\tau})]$. with $\rho((w_0,w_1,\ldots,w_q)) = (w_0-1,w_1-1,\ldots,w_q-1)$ the backwards action, because of dualisation.  



%The full piece removal defines an order relation on each equivalence class. Then $\partial([(U,w,\tau)]) = [\partial(\bar{U},\bar{w},\tau)]$ where $(\bar{U},\bar{w},\tau)$ is a maximal representative of $[(U,w,\tau)]$ (in order to compute it it is enough to consider representatives that have a single full piece).
\begin{definition} Let $\check{\Theta}^*(r,n)$ be defined as $\Theta^*(r,n)$, but redefining $w'_j = w_j$ in the full piece removal relation.
\end{definition}
\begin{lemma}
	The map $\beta'\colon \Theta^*(r,n)\lra \check{\Theta}^*(r,n)$ given by sending $(U,w)\otimes \vec{\tau}$ to $(U,w')\otimes \vec{\tau}$, where $w_i' = u_i+w_1\mod r$ is a $C_r$-equivariant isomorphism of chain complexes.
\end{lemma}
\begin{proof} Immediate.
\end{proof}

Given $(U,w)\in N^*(\Delta^m\times EC_r)$, define
\[U_w^i = \{u_j\in U\mid w_j=i\} \qquad 0\leq i\leq r-1\] 
Let $\sigma(w)$ be the sign of the permutation that arranges $w$ in ascending order without permuting entries with the same label. For example, we can arrange $w=01210$ in ascending order in two steps: first, move the last zero to the second position $w' = 00121$, which is an odd permutation. Second, move the last one to the penultimate position obtaining the ascending word $00112$, which is again an odd permutation, therefore $\sigma(w) = 0$.
\begin{lemma}
	The map $\beta\colon \bar{\Omega}^*(r,n)\lra (F\otimes_\Delta G)$ given on any representatives by
	\[\beta([U,w,\vec{\tau}]) = (-1)^{\sigma(w)}(U_w^0,\ldots, U_w^{r-1})\otimes \vec{\tau}\]
	is a $C_r$-equivariant isomorphism of chain complexes.
\end{lemma}
\begin{proof} Immediate. The simple description of the sign uses that we are working with the shifted complex.\qedhere
%Let $(\sigma,V,\omega)$ be the maximal representative of the class $[\tau,U,w]$, where we have set every inner reordering to be ascending. Then $\sigma\otimes (V_\omega^0,\ldots,V_\omega^{r-1})$ is also the maximal representative in its class. In both cases the differential is given by the class of the differential of these maximal representatives.
%\[\partial \beta([\sigma,V,\omega]) &= 
%\partial\left((-1)^{\sigma(V,\omega)}(\sigma\otimes(V_\omega^0,\ldots,V_\omega^{r-1}))\right) \\
%&= \sum_{k=0}^{r-1}(-1)^{\sigma(V,\omega)}(-1)^{\sum_{j<k} %|V_\omega^j|} \sum_{j=0}^{|V_{\omega}^k|}(-1)^i (\sigma\otimes(V_\omega^0,\ldots,d_iV_\omega^{k},\ldots,V_\omega^{r-1}) 
%\]
%\[\beta(\partial([\sigma,V,\omega]))) 
%&= \beta\left(\sum_{i}(-1)^i[(\sigma,d_iV,d_i\omega)]\right) \\
%&= \sum_i(-1)^i(-1)^{\sigma(d_iV,d_i\omega)}\left([(\sigma,(d_iV)_\omega^0,\ldots,(d_iV)_\omega^{r-1})]\right) \\
%&= \sum_{k=0}^{r-1}\sum_{i\mid \omega_i = %k}(-1)^i\left([(\sigma,V_\omega^0,\ldots,(d_iV)_\omega^{k},\ldots,V_%\omega^{r-1})] \\
%\right)
%\]
%Let $v_j$ be the $j$th element in $V_\omega^k$, so that $d_jV_\omega^k = (-1)^{j}V_\omega^k\smallsetminus \{v_j\}$. Let $i$ be the number of elements in $V$ smaller than $v_j$ plus the number of elements equal to $v_j$ such that $\omega_j<k$. Then the summand indexed by $(k,j)$ in the first sum equals the summand indexed by $i$ in the second sum up to a sign. This correspondence defines a bijection between the summands, so it only rests to check that the signs coincide, which is a thorough computation.

%Finally, it is an isomorphism because it sends generators to generators up to sign and each generator has a unique generator in its preimage.

%The formula is also independent of the chosen representative...
\end{proof}
%Because it sends generators to generators, the dual map of the inverse of $\beta$ is defined with the same formulas as $\beta$. 
%\begin{corollary} 
%	The map $\eta\colon \bar{\Omega}^*(r,n)\lra (F\otimes_\Delta G)$ given by
%	\[\eta([U,w,\tau]) = (-1)^{\sigma(U,w)}(U_w^0\otimes\ldots\otimes U_w^{r-1})\otimes \tau\]
%	is a $C_r$-equivariant isomorphism of chain complexes.
%\end{corollary}
\begin{example}\label{ex:103}
    The element $((0,1,2,3),(0,1,5,6),(3,4,5,6))\otimes [0,2,3,4,5,7,9]$ is the image of $[001123345566,010100221212)\otimes [0,2,3,4,5,7,9]$ under $\beta$ with sign $\sigma(w) = +1$.
\end{example}
\begin{example}\label{ex:104}
    The element $[001123345566,010100221212)\otimes [0,2,3,4,5,7,9]$ is the image of $[001123345566,012010212012)\otimes [0,2,3,4,5,7,9]]$ under $\beta'$
\end{example}
\subsection{Models of $W_*(r)\otimes N_*(\Delta^n)$}
Recall that $W_*(r)$ is the augmented minimal resolution of $C_r$. It will be very convenient to shift the minimal resolution one degree up, we denote it by $\WW_*$ (note the reindexing of the generators):
\[
    \ldots\lra R\langle e_3\rangle \overset{N}{\lra} R\langle e_2\rangle \overset{T-1}{\lra} R\langle e_1\rangle \overset{N}{\lra} R
\]
Recall from the previous section that $N_m^r(\Delta^n)$ is a graded $R$-module concentrated in degree $rm$ and that $N_\bullet^r(\Delta^n)$ is a simplicial $R$-module where $m$-simplices are endowed with degree $rm$. There is an associated chain complex with differential of degree $r$ that we denote by $N_*^{r}(\Delta^n)$.

\begin{definition}
	If $k\geq -1$ is even, define the \emph{flip map} $\varphi_k$ and the \emph{suspension map} $\theta_k$ as
	\begin{align*}
	    \varphi_k\colon W_*(r)&\lra \WW^{k-*}(r) &
	    \theta_k\colon \WW^*(r)&\lra \WW^{*-k}(r) \\
	    e_q&\longmapsto e^{\vee}_{k-q} &
	    e_q^\vee &\longmapsto e_{q-k}^\vee
	\end{align*}
\end{definition}
Endow the graded $R$-module $\WW^*(r)\otimes N_*^r(\Delta^n)$ with the differential $d(e^\vee_q\otimes \tau) = d(e^\vee_q)\otimes \tau + (-1)^q \theta_{r-1}(e^\vee_q)\otimes d(\tau)$, and define a $C_r$-equivariant homomorphism of chain complexes
\[\varphi\colon  W_*(r)\otimes N_*(\Delta^n) \lra \WW^*(r)\otimes N_*^r(\Delta^n)\]
as $e_{q}\otimes \tau\mapsto \varphi_{(r-1)m}(e_q)\otimes \vec{\tau}$ if the simplex $\tau$ has dimension $m$. Notice that the cyclic action on $W_*(r)$ is a forward action, but the action on $\WW^*(r)$ is a backwards action, because of the dualisation.
\begin{example}\label{ex:105}
    $\varphi(e_0\otimes [0,2,3,4,5,7,9]) = e^\vee_{12}\otimes [0,2,3,4,5,7,9]$.
\end{example}

%\[V^{*}(r) = \bigoplus_{m}\left( \Sigma W^{(r-1)m-*}(r) \otimes N_m(\Delta^n)\right)\]
%with the differential $d(e^\vee_q\otimes \tau) = d(e^\vee_q)\otimes \tau + (-1)^q \theta_{r-1}(e^\vee_q)\otimes d(\tau)$
%$d(\tau\otimes e^\vee_q) =  d\tau\otimes \theta_{r-1}(e^\vee_q) + (-1)^{|\tau|} \tau\otimes d(e^\vee_q)$.


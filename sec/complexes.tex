% !TEX root = ../oddp.tex

\section{Some cochain complexes}

There is a Poincaré duality isomorphism
\[PD\colon N_*(\Delta^n)\lra N^{n-1-*}(\Delta^n)\]
that sends a simplex to its complement in the maximal simplex. In this isomorphism, the differential on the normalised cochain complex is not the dual of the usual differential: if $u\notin U$, define $\delta^u(U)$ by sending a generator $U$ to the union $U\cup \{u\}$, where $u\notin U$. The new differential is denoted $\delta^\dagger$ and the old differential is denoted $\partial^{\vee}$, and they are defined as: 
\begin{align*}
    \delta^\dagger(U) &= \sum_{u\notin U} (-1)^{t_U(u)}U\cup \{u\}
    &
    \delta(U) &= \sum_{u\notin U}(-1)^{s_U(u)}U\cup \{u\},
\end{align*}
where $s_U(i)$ is the number of elements smaller than $u$ in $U$ and $t_U(u)$ is the number of elements smaller than $u$ in $\{0,\ldots,n\}\smallsetminus U$.

There is an isomorphism of cochain complexes
\[(N^*(\Delta^n),\delta)\longrightarrow N^*(\Delta^n),\partial^\#)\]
given by sending a simplex $\tau$ to $(-1)^{s(\tau)}\tau$, where $s(\tau)$ is the sum of the entries of $\tau$ (for example, $s(0,3,4) = 7$).

\subsection{Models of $N_*(\Delta^n)^{\otimes r}$}

Consider the functor tensor product $F\otimes_{\Delta}G$ of the following two functors:
\begin{align*}
	F\colon \Delta&\lra \Ch(R)& G\colon \Delta^{\op}&\lra \Ch(R)\\
	F([m])&=\Sigma^{(r-1)m-r}(N^*(\Delta^m)^{\otimes r})& G[m] &= N_m(\Delta^n),
\end{align*}
where $N^*(\Delta^m)$ is endowed with the differential $\partial^\vee$.
\begin{lemma}
	The map $\alpha\colon G\otimes_{\Delta} F\lra N_*(\Delta^n)^{\otimes r}$ given by
	\[\alpha(\tau\otimes (U_1,\ldots,U_r)) = (-1)^{\sum_k s(U_k))}(d_{U_1}^\tau(\tau)\otimes \ldots\otimes d_{U_r}^\tau(\tau))\]
	is a $C_r$-isomorphism of chain complexes.
\end{lemma}
\begin{proof} If we endow $G\otimes F$ with the differential $$ On one hand, $\partial \alpha(\tau\otimes (U_1,\ldots,U_r))$ equals
\[ 
\sum_k(-1)^{|\tau|-\sum_{j=1}^{k-1}|U_{j}|}\sum_{i=0}^{|\tau|-|U_k|} d_{U_1}(\tau)\otimes \ldots\otimes d_id_{U_k}(\tau)\otimes \ldots \otimes d_{U_{r}}(\tau)
\]
while $\alpha(\delta (\tau\otimes (U_1,\ldots,U_r)))$ equals
\[
(-1)^{|\tau|}\sum_k(-1)^{\sum_{j=1}^{k-1}|U_{j}|}\sum_{i=0}^{|\tau|-|U_k|} d_{U_1}(\tau)\otimes \ldots\otimes d_{\delta^i U_k}(\tau)\otimes \ldots \otimes d_{U_{r}}(\tau)
\]
and both summations coincide, hence $\alpha$ is a chain homomorphism. It has an inverse given by sending $\tau_1\otimes\ldots \otimes \tau_r$ to $(\sigma\otimes (PD(\tau_1),\ldots,PD(\tau_r)))$, where $\sigma$ is the generator of $N_n(\Delta^n)$.
\end{proof}
The functor tensor product is generated by tuples $\tau\otimes (U_1,\ldots,U_r)$, up to the equivalence relation generated by identifying two tuples $\tau\otimes (U_1,\ldots,U_r)$ and $\tau'\otimes (U_1',\ldots,U_r')$ if there is a $j$ such that $d_j\tau\otimes (U_1,\ldots,U_r) \sim \tau\otimes U_1\cup \{u_j\},\ldots,U_r\cup \{u_j\}$.

Every equivalence class has a unique minimal element, in which $\bigcap U_j = \emptyset$, and the differential on the class is the class of the differential on that minimal element. 

We will need to consider the differential in the dual complex, which now we describe: Let $\tau\otimes (U_1,\ldots,U_r)$ be a minimal element in an equivalence class, and let $[\tau\otimes (U_1,\ldots,U_r)]^f$ be the set of those representatives such that $|\bigcap U_j| = 1$. Then
\[\partial [\tau\otimes (U_1,\ldots,U_r)] = 
\sum_{(\hat{\tau}\otimes(\hat{U}_1,\ldots,\hat{U}_r))\in [\tau\otimes (U_1,\ldots,U_r)]} \hat(tau)\otimes (U_1,\ldots,d_iU_k,\ldots,U_r)
\]



Consider now the chain complex $\bar{\Omega}_*(r,n)$ that in degree $k$ is generated by equivalence classes of triples $(U,w,\tau)$ where \fcnote{This is actually a quotient of a functor tensor product. Maybe it is worthy defining it like that.}
\begin{itemize}
	\item $U$ is a generator of $C_{(r-1)m-1-k}(\Delta^m)$,
	\item $w$ is a generator of $C_{(r-1)m-1-k}(EC_r)$ (a word),
	\item $\tau$ is a generator of $N_m(\Delta^n)$
\end{itemize}
such that
\begin{itemize}
	\item $(U,w)$ is a non-degenerate simplex in $EC_r\times \Delta^m$
	\item There is at most one \emph{full piece}, i.e., a sequence $u_i=u_{i+1}=\ldots=u_{i+r-1}$ of $r$ consecutive equal elements in $U$.
\end{itemize}
subject to the following relations:
\begin{itemize}
	\item (Inner reorderings) $(U,w,\tau)\sim (-1)^{|\sigma|}(U,w',\tau)$ if there is an interval $u_{i-1}<u_i =\ldots u_{i+k}<u_{i+k+1}$ such that the sequences $(w_i,\ldots,w_{i+k})$ and $(w'_i,\ldots,w'_{i+k})$ differ by a permutation $\sigma$ and $w$ and $w'$ agree outside that interval. Notice that if all elements of $U$ are different, then $w=w'$.
	\item (Full piece removal) $(U,w,d_i(\tau))\sim (-1)^{|U_{\ge i}|}(D_i(U,w),\tau)$, where $D_i(U,w) = (U',w')$ and
	\begin{align*}
		U'_j &= \begin{cases} U_j &\text{ if $j<i$} \\ i & \text{ if $i\leq j < i+r$} \\ U_{j-r} + 1 & \text{ if $j\geq i+r$.}\end{cases} &
		w'_j &= \begin{cases} w_j &\text{ if $j<i$} \\ w_{j-r} - 1 & \text{ if $j\geq i+r$.}\end{cases}
	\end{align*}
	and $(w'_{i},w'_{i+1},\ldots,w'_{i+r-1}) = (0,1,\ldots,r-1)$.
\end{itemize}
This complex has a differential defined as follows on a class $[(U,w,\tau)]$ with $(U,w,\tau)$ without full pieces:
\[
\partial [(U,w,\tau)] = 
\sum_{(\hat{U},\hat{w},\hat{\tau})\in [(U,w,\tau)]^{f}} \left([(\partial(\hat{U},\hat{w}),\hat{\tau})] - [(\partial(U,w),\tau)]\right) + [(\partial(U,w),\tau)]
\]
where $\partial(U,w)$ is the differential in the simplicial product and $[(U,w,\tau)]^f$ is the class of those representatives of $(U,w,\tau)$ with a full piece. The cyclic group $C_r$ acts on each class $[(U,w,\tau)]$ of $\bar{\Omega}^*(r,n)$ as $\rho([(U,w,\tau)]) = [(U,\rho(w),\tau)]$.  


%The full piece removal defines an order relation on each equivalence class. Then $\partial([(U,w,\tau)]) = [\partial(\bar{U},\bar{w},\tau)]$ where $(\bar{U},\bar{w},\tau)$ is a maximal representative of $[(U,w,\tau)]$ (in order to compute it it is enough to consider representatives that have a single full piece).

\begin{lemma}
	The map $\beta'\colon \bar{\Omega}^*(r,n)\lra \bar{\Omega}^*(r,n)$ given by sending $(U,w,\tau)$ to $(U,w',\tau)$, where $w_i' = u_i+w_1\mod r$ is a $C_r$-equivariant isomorphism of chain complexes.
\end{lemma}

Given a generator $[(U,w,\tau)]$, define for each $0\leq i\leq r-1$, $U_w^i \subset U$ as the subset of those entries $u_j$ such that $w_j=i$. Define the word $w_U$ as $(\omega_U)_j = \omega_j$ and $\sigma(U,w)$ as the sign of the permutation that arranges $\omega_U$ in ascending order. \fcnote{This map may be further factored?}

\begin{lemma}
	The map $\beta\colon \bar{\Omega}^*(r,n)\lra F\otimes_\Delta G$ given by
	\[\beta([U,w,\tau]) = (-1)^{\sigma(U,w)}(U_w^0\otimes\ldots\otimes U_w^{r-1})\otimes \tau\]
	is a $C_r$-equivariant isomorphism of chain complexes.
\end{lemma}

\subsection{Models of $W_*(r)\otimes N_*(\Delta^n)$}

Recall that $W_*(r)$ is the augmented minimal resolution of $C_r$.

\begin{definition}
	If $k\geq -1$ is odd, define the \emph{flip map} $\varphi_k\colon W_*(r)\to W^{k-*}(r)$ as $e_q\mapsto e^{\vee}_{k-q}$. If $k$ is an even integer, define the \emph{suspension map} $\theta_k\colon W_*(r)\to W_{*-k}(r)$ as $\theta_k(e_q) = e_{q-k}$.
\end{definition}

Let $r$ be odd. Define the graded complex
\[V^{*}(r) = \bigoplus_{m}W^{(r-1)m-1-*}(r)\otimes N_m(\Delta^n)\]
with the differential $d(e^\vee_q\otimes \tau) = d(e^\vee_q)\otimes \tau + (-1)^q \theta_{r-1}(e^\vee_q)\otimes d(\tau)$%\fcnote{the sign works even better with $(-1)^n$. Maybe the other order is better?}. Define a $C_r$-equivariant homomorphism of chain complexes
\[\varphi\colon W_*(r)\otimes N_*(\Delta^n)\lra V^{*}(r)\]
as $e_{q}\otimes \tau\mapsto \varphi(e_q)\otimes \tau$.


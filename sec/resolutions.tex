% !TEX root = ../oddp.tex

\section{The minimal resolution of $C_r$ and the Milnor construction}

Let $r$ be an odd number, let $C_r$ be the cyclic group of order $r$ and let $\rho$ be a generator.

\subsection{Cyclic resolutions}

Recall the augmented minimal resolution $W_*(r)$ of the cyclic group $C_r$ by $R$-modules:
\[R[C_r]\lra\ldots \lra R[C_r]\overset{T}{\lra} R[C_r]\overset{N}{\lra}R[C_r]\overset{T}{\lra} R[C_r] \overset{N}{\lra} R.\]
The generator $\rho$ in degree $k\geq -1$ will be denoted by $e_k$ (here $e_{-1}=1$), and
\begin{align*}
	N(e_k) &= \sum_{i=0}^{r-1} \rho^ie_{k-1} &
	T(e_k) &= (1-\rho)e_{k-1}
\end{align*}

Recall the Milnor construction applied to the cyclic group $C_r$, which yields a contractible augmented simplicial set with a free $C_r$-action:
\begin{align*}
	(EC_r)_k &= C_r^{k+1}\\
	d_i(a_0\ldots a_k) &= (a_0\ldots\hat{a}_i\ldots a_k) \\
	s_i(a_0\ldots a_k) &= (a_0\ldots a_i,a_i\ldots a_k)
\end{align*}
The $i$th coface $d^i\colon C^*(EC_r)\to C^{*+1}(EC_r)$ is given by inserting a number $b\in \{0,\ldots,r-1\}$ in $i$-th position in all possible ways (some of them yield degenerate cochains). Write $d^{i,b}(a_0\ldots a_n) = (a_0\ldots a_{i-1},b,a_i,\ldots, a_n)$, so that
\[\delta(a_0\ldots a_n) = \sum_{0\leq i\leq n}\sum_{0\leq b\leq r-1}(-1)^i \delta^{i,b}(a_0\ldots a_n).\]
\begin{warning}
	Observe that $\rho$ acts on the dual complex as $\rho^{-1}$.
\end{warning}
There are \emph{suspension maps}
\begin{align*}
	\theta_{r-1}\colon W_*(r) & \lra \Sigma^{r-1}W_*(r) \\
	e_k & \longmapsto e_{k-r+1} \\
	% S\colon \Lambda(r)^*&\lra \Sigma^{r-1} \Lambda(r)^* \\
	% (a_0,\ldots,a_k) & \longmapsto \sum_{(b_{0},\ldots,b_{r-1})\in \Perm_r} (-1)^{\sigma(a_k\ldots a_{k+r-1})} %(a_0,\ldots,a_{k},a_{k+1},\ldots,a_{k+r-1})
	S\colon C_*(EC_r)&\lra \Sigma^{r-1}C_*(EC_r) \\
	(a_0,\ldots,a_k) & \longmapsto \begin{cases}
		{\sgn(a_{0},\ldots,a_{r-1})} (a_{r-1},\ldots,a_{k}) & \text{ if $k\geq r-1$} \\
		\emptyset & \text{if $k = r-2$ and $a_i\neq a_j$ for all $i,j$} \\
		0 & \text{otherwise.}
	\end{cases}
\end{align*}

\begin{remark}
	For any even integer $k$, the suspension $\theta_{k}\colon W_*(r)\to \Sigma^{k}W_*(r)$ is a homomorphism of chain complexes.
\end{remark}

%The following lemma is very similar to its version for the surjection operad (whose complex in arity $r$ is a quotient of $\Lambda(r)_*$).
\begin{lemma}
	The suspension $S\colon C_*(EC_r)\to \Sigma^{r-1}C_*(EC_r)$ is a homomorphism of chain complexes.
\end{lemma}

\begin{proof}
	Consider first a simplex $(a_0,\ldots,a_k)$ of dimension $k\geq r$. If $A=(a_{0},\ldots,a_{r-1})$ has no repeated entries (i.e., it is a permutation), we choose the unique $j<r-1$ such that $a_j = a_r$. Let $A' = (a_{0},\ldots,\hat{a}_j,\ldots,a_r)$, and observe that the permutations $A$ and $A'$ differ by $r-j-1$ transpositions, hence $\sgn(A') = (-1)^{r-j+1}\sgn(A) = (-1)^j\sgn(A)$.
	\begin{align*}
		S\left(\sum_{i=0}^k (-1)^id_i (a_0,\ldots,a_k)\right)
		&= \sum_{i=0}^{r-1} (-1)^iS(d_i (a_0,\ldots,a_k)) + \sum_{i=r}^{k} (-1)^iS(d_i (a_0,\ldots,a_k))
	\end{align*}
	All the terms in the first summation on the RHS are zero but the one indexed by $j$, because $(a_0,\ldots,\hat{a}_i,\ldots,a_r)$ is not a permutation in those cases. Additionally, $Sd_i = d_{i-r+1}S$ in each term in the second summation on the RHS, hence we have:
	\begin{align*}
		S\left(\sum_{i=0}^k (-1)^id_i (a_0,\ldots,a_k)\right)
		&= (-1)^jS(d_j (a_0,\ldots,a_k)) + \sum_{i=1}^{k-r+1} (-1)^i d_i S(a_0,\ldots,a_k)\\
		&= (-1)^{j}\sgn(A')(a_r,\ldots,a_k) + \sum_{i=1}^{k-r+1} (-1)^i d_i S(a_0,\ldots,a_k)\\
		&= (-1)^{0}\sgn(A)(a_r,\ldots,a_k) + \sum_{i=1}^{k-r+1} (-1)^i d_i S(a_0,\ldots,a_k)\\
		&= \sum_{i=0}^{k-r+1} (-1)^i d_i S(a_0,\ldots,a_k)
	\end{align*}
	If $A$ has repeated entries, then $S(a_0,\ldots,a_k) = 0$. We consider two cases: if the repeated entries of $A$ are different from $a_k$ or there are more than two repeated entries, then all summands in $d(a_0,\ldots,a_k)$ have repeated entries in its first $r$ positions, therefore $S(dA)=0$. If there are exactly two repeated entries $a_j=a_m = a_r$ with $j,m<r$, let $A' = (a_0,\ldots,\hat{a}_j,\ldots,a_{r})$, $A'' = (a_0,\ldots,\hat{a}_m,\ldots,a_{r})$, and observe that these two words differ by $|m-j-1|$ transpositions, hence $\sgn(A') = (-1)^{m-j-1}\sgn(A'')$.
	\begin{align*}
		S\left(\sum_{i=0}^k (-1)^id_i (a_0,\ldots,a_k)\right)
		&= \sum_{i=0}^{r-1} (-1)^iS(d_i (a_0,\ldots,a_k)) + \sum_{i=r}^{k} (-1)^iS(d_i (a_0,\ldots,a_k)) \\
		&= (-1)^jS(d_j (a_0,\ldots,a_k)) + (-1)^mS(d_m (a_0,\ldots,a_k)) \\
		&= \left((-1)^{j}\sgn(A') + (-1)^m\sgn(A'')\right)(a_r,\ldots,a_k)\\
		&= 0
	\end{align*}
	The case $k=r-1$ is left to the reader.
\end{proof}

\subsection{A comparison homomorphism}

Here we introduce a homomorphism between the previous two cochain complexes.

\begin{definition}
	For each $0\leq k<r-1$, let $L_k\subset C_k(EC_r)$ be the set of those increasing words $(a_0\ldots a_k)$ such that $a_{2i}$ is even and different from $r-1$, and $a_{2i+1}$ is odd. Define
	\begin{equation}\label{eq:111}
		\psi_k = \sum_{(a_0,\ldots,a_k)\in L_k}\sum_{\sigma\in \Sigma_{k+1}} \sgn(\sigma)(a_{\sigma(0)},\ldots,a_{\sigma(k)})\in C^k(EC_r)
	\end{equation}
	For each $k = \ell (r-1) + s\geq r-2$ with $s<r-1$, define
	\begin{equation}\label{eq:112}
		\psi_{k} = (\dual{S})^\ell(\psi_s)\in C^k(EC_r).
	\end{equation}
\end{definition}

\begin{notation}
	If $A= (a_{\sigma(0)},\ldots,a_{\sigma(k)})$, we will sometimes denote the sign of the permutation $\sigma$ as $\sgn(A)$. If $a$ is a non-negative number, write $\tilde{a} = \left\lfloor \frac{a}{2}\right\rfloor$.
\end{notation}

Define the $C_r$-equivariant map $\Psi\colon W^*(r;\bZ[\frac{1}{\tilde{r}!}])\lra C^*(EC_r; \bZ[\frac{1}{\tilde{r}!}])$ as
\[e_k\mapsto \frac{(\tilde{r}-\tilde{s}-1)!}{(\tilde{r}!)^{\ell+1}}\psi_k\]

\begin{theorem}
	$\Psi$ is a homomorphism of cochain complexes.
\end{theorem}
%If $r>3$ and $R\neq \bZ_r$, then the $C_r$-equivariant map $\Psi\colon W^{\leq m(r-1)}(r)\lra N^*(EC_r)_{\leq m(r-1)}$ given by $e_k\mapsto (\tilde{r}!)^{m-\ell-1}(\tilde{r}-\tilde{s}-1)!\psi_k$ is a homomorphism of cochain complexes.
\begin{remark}
	If $r$ is prime, the coefficient $\tilde{r}!\mod r$ is, by Wilson's theorem, one of the two values of $\sqrt{(-1)^{\tilde{r}+1}}$, though determining which of them is not immediate\footnote{\url{https://mathoverflow.net/questions/121678/the-value-pm-1-for-the-square-root-of-wilsons-theorem-p-1-2-mod-p}}.
\end{remark}

\begin{remark}
	If $r=3$, then $\bZ[\frac{1}{\tilde{r}!}] = \bZ$ and if $r$ is prime, then $\bZ/r$ is a $\bZ[\frac{1}{\tilde{r}!}]$-algebra, hence these coefficients are also allowed.
\end{remark}

\begin{proof}
	We need to prove that
	\begin{align*}
		N\psi_{k} & = (\tilde{r}-\tilde{s})\delta(\psi_{k-1}) & \text{ if $k$ is even} \\
		T\psi_{k} &= d(\psi_{k-1}) & \text{ if $k$ is odd.}
	\end{align*}
	For the case $k\leq r-2$ we need to check that
	\begin{align*}
		\sum_{A\in L_{k-1}}\sum_{\sigma\in \Sigma_{k-1}}\sum_{0\leq b\leq r-1}(-1)^i\sgn(A)\delta^{i,b}A &= \bar{\psi}_k. & \text{ if $k$ is even,}\\
		\sum_{A\in L_{k-1}}\sum_{\sigma\in \Sigma_{k-1}}\sum_{0\leq b\leq r-1}(-1)^i\sgn(A)\delta^{i,b}A &= T\psi_k & \text{if $k$ is odd.}
	\end{align*}
	This will be proved in Lemmas \ref{lemma:pair} and \ref{lemma:cancel}. In higher degrees we have, by induction ($\lambda = \tilde{r}-\tilde{s})$,
	\begin{align*}
		N\psi_{k} &= N\dual{S}\psi_{k-r+1} = \dual{S} N\psi_{k-r+1} = \lambda\dual{S} d\psi_{k-r} = \lambda d\dual{S} \psi_{k-r} = \lambda d\psi_{k-1}\\
		T\psi_{k} &= T\dual{S}\psi_{k-r+1} = \dual{S} T\psi_{k-r+1} = \dual{S} d\psi_{k-r} = d\dual{S} \psi_{k-r} = d\psi_{k-1}.
	\end{align*}
	This also sets the case when because both definitions of $\psi_{r-2}$ agree.
\end{proof}

%If $k<r$, the \emph{shape} of such a simplex $(a_0,\ldots,a_k)$ is the cyclic set of cyclic differences $\{a_i - \bar{a}_{i-1}\}_{i=0}^k$. By this we mean that two shapes are equivalent if they differ by a cyclic permutation.
%\begin{lemma} If $k<r$ is even, the cyclic orbit of each summand of $\psi_k$ has $(\tilde{r}-\tilde{k}-1)$ elements in $\psi_k$.
%\end{lemma}
%\begin{proof}
%Since $(a_0,\ldots,a_k)$ has $\tilde{k}+1$ elements, this is the number of cyclic permutations that contain $r-1$. Since there are $\tilde{r}$ cyclic permutations in total, the result follows.
%\end{proof}

A number $b$ is \emph{friend} of a word $A$ of $\psi_k$ under any of the following conditions:
\begin{itemize}
	\item $b>A$ and $b$ has the same parity as $a_{k}$,
	\item $b<A$ and $b$ is odd,
	\item $b=r-1$.
\end{itemize}
A number is \emph{enemy} of a word $A$ of $\psi_k$ if it is not its friend. Write $b\vdash A$ if $b$ and $A$ are friends and $b\not\vdash A$ if they are enemies.

Say that a word $A = (a_{\sigma(0)},\ldots,a_{\sigma(k)})\in N_k(EC_r)$ is \emph{even} if it b (\emph{odd}) if $a_{2i}$ is even (odd) and $a_{2i+1}$ is odd (even). A word is \emph{good even} if it is even and $r-1\notin A$. A word $A$ is \emph{good odd} if $\rho^{-1}(A)$ is even good, i.e., if $r-2\notin A$ and either $A$ is odd, or $r-1\in A$ and $A\smallsetminus \{r-1\}$ is even. Observe that the summands of $\psi_k$ are precisely on the good even words, and that if $k$ is odd, then all even words are good even. Let $\bar{\psi}_k$ be the sum of all even and odd words in $\Lambda(r)_k$ with their permutation signs.
%For the next lemma, observe that if $A$ is even, then $\rho^b(A)$ is either even or odd. Observe also that, if $k$ is even, removing the minimum of an odd sequence, or removing the maximum of an even sequence yields a summand of $\psi_{k-1}$.

\begin{lemma} \label{lemma:simp}
	If $k\leq r-2$ is even, then $N\psi_k = (\tilde{r}-\tilde{k})\bar{\psi}_k$. If $k$ is odd, then $T\psi_k$ is the sum of all good odd words minus the (good) even words with their permutation signs.
\end{lemma}

\begin{proof}
	Suppose first that $k$ is even. We will first establish that every summand in $N\psi_k$ is a summand in $\bar{\psi}_k$. Then, we will prove that every summand in $\bar{\psi}_k$ appears exactly $(\tilde{r}-\tilde{k})$ times in $N\psi_k$, with the same signs.

	If $A$ is even and $0\notin A$ ($0\in A$), then $\rho(A)$ is odd (even), while if $A$ is odd, then $\rho(A)$ is even,. This has two consequences: first, since the summands in $\psi_k^0$ are good even summands, every summand in $N\psi_k^0$ is either even or odd; second, if $A$ is an even (odd) word, then $\rho^{b+1}(A)$ is a summand in $\psi_k$ if and only if $b\notin A$ (this condition assures that $r-1\notin \rho^{b+1}(A)$) and has opposite parity to $|\{a\in A\mid a<b\}|$ (this assures that $\rho^{b+1}(A)$ is even). There are $r-k-1$ many summands satisfying the first condition, and half of them satisfy the second condition, hence there are exactly $\tilde{r}-\tilde{k}$ such $b$'s, therefore the first statement follows. Regarding the sign, observe that if $A$ is even, then the permutation that reorders $\rho(A)$ is either the same as the permutation $\sigma$ that reorders $A$, or it is the result of precomposing $\sigma$ with a cyclic permutation of $\Sigma_k$, hence its sign is that of $\sigma$. If $A$ is odd, then the permutation that reorders $\rho(A)$ is $\sigma$.

	The second statement follows by inspection.
	%
	%The second statement follows because there is a summand in $\psi_k^0$ for each orbit of $C_r$ in $\psi_k$.
	%
	%On the other hand, if $A$ is an even (odd) word such that either $0\notin A$ or $r-1\notin A$, then $\rho^{\mi}A\in \psi_0$. If both $0\in A$ and $r-1\in A$, then choose two consecutive values $b,b+1\notin A$. Then either $\rho^{b+1}(A)$ or $\rho^{b+1}(A)$ is a summand in $\psi_k$, and therefore a summand in $\psi_k^0$, by the previous lemma.
\end{proof}

\begin{lemma} \label{lemma:pair}
	If $k\leq r-2$ is even,
	\begin{align*}
		\sum_{A\in L_{k-1}}\sum_{\sigma\in \Sigma_{k-1}}\sum_{b\vdash A}(-1)^i\sgn(A)\delta^{i,b}A &= \bar{\psi}_k.
	\end{align*}
	If $k$ is odd,
	\begin{align*}
		\sum_{A\in L_{k-1}}\sum_{\sigma\in \Sigma_{k-1}}\sum_{b\vdash A}(-1)^i\sgn(A)\delta^{i,b}A &= T\psi_k.
	\end{align*}
\end{lemma}

\begin{proof}
	Since each summation has no repeated summands, it will be enough to prove that every summand in the left appears in the right with the same sign, and viceversa.

	%Observe first that the summands in $N\psi_k^0$ and $T\psi_k$ are the precisely the even and the odd words. Moreover, when $k$ is odd, the summands in $\psi_k$ are the even words, while the summands in $\rho\psi_k$ are the odd words.

	Let $k$ be even. Since $b$ is friend of $A$, and $A$ is good even, the word $\delta^{i,b}A$ is either even or odd (depending on whether $b>A$ or $b<A$). Hence the summands in the left appear as summands in the right. Reciprocally, if $A$ is an even (odd) word, then removing its maximal (minimal) element $b=a_{\ma}$ ($b=a_{\mi}$) yields a good even word in $\psi_{k-1}$, hence $A = \delta^{\ma,b}A\smallsetminus\{b\}$ ($A = \delta^{\mi,b}A\smallsetminus\{b\}$).

	Let $k$ be odd. Since $b$ is friend of $A$ and $A$ is good even, the word $\delta^{i,b}A$ is again even or odd (depending on whether $b>A$ or $b<A$). If $b<A$, since $A$ is even, we have that $\max \delta^{i,b}(A) = \max A \leq r-3$, hence $\delta^{i,b}A$ is good. If $b>A$, then either $b$ is odd, in which case $\delta^{i,b}$ is good even, or $b=r-1$, in which case removing $r-1$ from $\delta{i,b}A$ yields the good even word $A$, hence $A$ is good odd. Reciprocally, if $A$ is a good even (odd) word, then removing its maximal (minimal) element $b=a_{\ma}$ ($b=a_{\mi}$) yields a good even word in $\psi_{k-1}$, hence $A = \delta^{\ma,b}A\smallsetminus\{b\}$ ($A = \delta^{\mi,b}A\smallsetminus\{b\}$).

	%If $b>A$, as $b$ is odd, it is different from $r-1$, hence $\delta^{i,b}A$ is good. If $b<A$, then
	%%
	%%if $A = (a_{\sigma(0)},\ldots,a_{\sigma(k)})$ is a summand in the right (i.e., an even or odd word), then
	%%\[
	%%\rho^{b}(A) = \begin{cases}
	%%\delta^{\sigma(s),a_{\sigma(s)}}\left(A\smallsetminus \{a_{\sigma(s)}\}\right) & \text{if $A$ is odd and $a_{\sigma(s)} = \min A$}, \\
	%%\delta^{\sigma(s),a_{\sigma(s)}}\left(A\smallsetminus \{a_{\sigma(s)}\}\right) & \text{if $A$ is even and $a_{\sigma(s)} = \max A$},
	%%\end{cases}
	%%\]
	%%with $A\smallsetminus \{a_{\sigma(s)}\}\in \psi_{k-1}$ because it is even and does not contain $r-1$.

	%Moreover, if $k$ is odd, then $\delta^{i,b}A$ is always good.

	Regarding the signs, observe that the sign of the summand in the left indexed by $A = (a_{\sigma(0)},\ldots,a_{\sigma(k-1)})$, $b$ and $i$ is $(-1)^i\sgn(A)$ while the sign of the summand $\delta^{i,b}A$ in the right is computed as follows:
	\[
		\sgn(\delta^{i,b}A) =
		\begin{cases}
			(-1)^i\sgn(\delta^{0,b}A) = (-1)^i\sgn(A) &\text{ if $b<A$,}\\
			(-1)^{k-i}\sgn(\delta^{k,b}A) = (-1)^k(-1)^i\sgn(A) &\text{ if $b>A$,}
		\end{cases}
	\]
	where the first equality follows because $\delta^{i,b}A$ and $\delta^{0,b}A$ (\resp $\delta^{k,b}A$) differ by $i$ (\resp $k-i$) transpositions, and the second equality follows because the permutation that reorders $\delta^{0,b}A$ (\resp $\delta^{k,b}A$) fixes the first (last) entry, and because $k$ is even.

	Hence, if $k$ is even, then all summands in the left appear with the sign $\sgn(\delta^{i,b}A)$, as in the right, while if $k$ is odd, the even summands appear with opposite sign and the odd summands with the same sign.
\end{proof}

\begin{lemma}\label{lemma:cancel}
	Let $k\leq r-2$.
	\begin{align*}
		\sum_{A\in L_{k-1}}\sum_{\sigma\in \Sigma_{k-1}}\sum_{b\not{\vdash} A}(-1)^i\sgn(A)\delta^{i,b}A &= 0.
	\end{align*}
\end{lemma}

\begin{proof}
	Let $A$ be a summand of $\psi_k$ and let $0\leq b\leq r-1$ be enemy of $A$. Then, if $b >\min(A)$ and $b<\max(A)$, then both $\max\{a\in A\mid a<b\}$ and $\min\{a\in A\mid a>b\}$ are well defined and have different parity, hence one of them ($c$) has the same parity as $b$. If $b<\min(A)$ or $b>\max(A)$, we let $c = \min(A)$ or $c = \max(A)$ respectively. Let $j$ be the position of $c$ in $A$ and let $A' = (\delta^{i,b} A)\smallsetminus \{c\}$. Then $c$ is enemy of $A'$ and $\delta^{j,c}A = \delta^{i,b}A'$. Let us show that they appear with opposite signs: The permutation that reorders $A'$ differs from the permutation that reorders $A$ by $|j-i-1|$ transpositions, therefore
	\[(-1)^{i}\sgn(A) = (-1)^{i}(-1)^{j-i-1}\sgn(A') = -(-1)^j\sgn(A').\]
	Finally, observe that $A'' = A$, hence the summands cancel in pairs.
\end{proof}
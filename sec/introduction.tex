% !TEX root = ../oddp.tex

\section{Introduction} \label{s:introduction}

In 1946, Steenrod was looking for invariants of maps between spheres generalizing the Hopf invariant when he found certain cochain operations on simplicial complexes
\[\smile_i\colon \ucocadenas[k](X;\bF_2)\otimes \ucocadenas[j](X;\bF_2)\lra \ucocadenas[k+j+i](X;\bF_2),\qquad i\geq 0\]
which in cohomology gave rise to the celebrated \emph{Steenrod squares}
\[\Sq^{k}\colon H^j(X;\bF_2)\lra H^{j+k}(X;\bF_2).\]
Shortly after, these cohomology operations were given a more abstract and axiomatic treatment that avoided the explicit chain maps $\smile_i$. This viewpoint exhibited new power operations
\begin{align*}
	P^{k}\colon H^j(X;\bF_r)&\lra H^{j+2k(r-1)}(X;\bF_r) & k\geq 0 \\
	\beta P^k\colon H^j(X;\bF_r)&\lra H^{j+2k(r-1)+1}(X;\bF_r) & k\geq 0
\end{align*}
for $r$ an odd prime.

From now on, we let $R$ be a commutative ring for which $\bF_r$ is an $R$-algebra. Chain complexes are taken with coefficients in $R$ unless otherwise specified.

\subsection{Unstable comultiplications} In \cite{may1970general}, May gave a general approach at the cochain level to power operations. He defined a category whose objects are chain complexes $\ucadenas$ endowed with an associative comultiplication (he actually worked with homotopy associative multiplications) together with an equivariant homomorphism $\mu\colon W_*(r)\otimes \ucadenas\to \ucadenas^{\otimes r}$ from the tensor product of the minimal free resolution $W_*(r)$ of the cyclic group $\Cyc_r$ and the chain complex $\ucadenas$ to the $r$-fold tensor product of $\ucadenas$, satisfying two additional properties:
\begin{itemize}
	\item[(i)] The restriction of $\mu$ to $W_0(r)$ is homotopic to the $r$-fold associative comultiplication.
	\item[(ii)] The map $\mu$ factors as $W_*(r)\otimes \ucadenas \to V_*(r)\otimes \ucadenas\to \ucadenas^{\otimes r}$, where $V_*(r)$ is a free resolution of the symmetric group on $r$ letters.
\end{itemize}
In this article we refer to this structure as an \emph{$r$-cyclic May structure} (compare with the notion of \emph{May--Steenrod structure} by the second author \cite{medina2021may_st}). In this article we coin the term \emph{$r$-cyclic unstable comultiplication} for an $r$-cyclic May structure without these two extra properties. If $r$ is prime, a chain complex $\ucadenas$ with an $r$-cyclic May structure has natural power operations on cohomology (note the re-indexing)
\begin{align*}
	\power^i\colon H^*(\ucocadenas;\bF_r)&\lra H^{*+i}(\ucocadenas;\bF_r), & i\in \bZ
\end{align*}
and the second property assures that they vanish unless $i = 2k(r-1), 2k(r-1)+1$ if $r$ is odd.
\anibal{Notice that up to this point both $C$ and $C_*$ have been used. I would prefer $C$ as we once discuss but I am happy with either consistent choice.}
\federico{Fixed. At this point I use a convention as standard as possible, since it is the introduction. In the rest of the paper I use the convention that a chain complex is denoted $C_*$ and its dual $C_*^\dd$ and is negatively graded, as you suggested. I prefer to keep the lower star because it is a symbol that tells the reader immediately that something is a chain complex}
These operations always satisfy:
\begin{itemize}
	\item[(iii)] If $x\in H^*(C^*)$ \anibal{is this a typo? I am not clear yet with your notational conventions.}\federico{Up to here I have not set notational conventions, and I try to use the traditional ones. With the conventions of later in the paper this should be called $H(\ucocadenas)$} has even degree $m$, then $\power^{m(r-1)}(x)$ is the $r$-fold power of the multiplication of $x$ with itself and all higher power operations of $x$ vanish.
\end{itemize}
Often they also satisfy:
\begin{enumerate}
	\item[(iv)] $\power^0(x) = x$ and $\power^i(x) = 0$ for negative $i$.
\end{enumerate}
Showing Property (iv) tends to be hard to prove.

If $r=2$, there is a special $r$-cyclic May structure on simplicial sets that has been described in multiple occasions \cite{steenrod1947products,gonzalez-diaz1999steenrod,medina2021fast_sq}. If $r$ is odd, an explicit $r$-cyclic May structure was found by the second author \cite{medina2021may_st} as the composition of a certain map $W_*(r)\to \cE_*(r)$ to the arity $r$ operations in the Barratt-Eccles operad, another map $\cE_*(r)\to \chi_*(r)$ to the arity $r$ operations in the surjection operad \cite{berger2004combinatorial} and a third map $\chi_*(r)\to EZ(r)$ to the arity $r$ operation in the Eilenberg--Zilber operad\cite{mcclure2003multivariable}.\anibal{In this paragraph you mean May structure right?}\federico{Yes, your construction give a May--Steenrod structure and the others a May structure, is that correct?}

\subsection{Stable comultiplications} If $r$ is a prime number, the singular cohomology with coefficients in $\bF_r$ of a spectrum $E$ admits power operations
\[
\power^i\colon H^*(E;\bF_r)\lra H^{*+i}(E;\bF_r)
\]
that vanish unless $i = 2(r-1)k$ or $i = 2(r-1)k+1$, but it does not admit any associative multiplication. If the spectrum $E$ is given as a sequence $\{E_m\}_{n\geq 0}$ of simplicial pointed sets with structural maps $\varepsilon_m\colon \Sigma E_m\to E_{m+1}$, then its cohomology can be computed as the cohomology of the chain complex
\[
\colim \left(\ucadenas(E_0)\lra \ldots \lra \Sigma^{-m}\ucadenas(E_m)\to \Sigma^{-m-1} \ucadenas(E_{m+1})\lra\ldots\right)
\]
This chain complex admits an action of the stabilization of the $E_{\infty}$-operad \cite{Gill2020}, and in Section \ref{s:suspension} we will show that the $r$-cyclic May structure of \cite{medina2021may_st} on the chain complex of a simplicial set is good enough to endow the chain complex of a spectrum with a stable version of an $r$-cyclic May structure (see Section \ref{s:stable} for the definition).

Dropping Property (ii) from the stable version of an $r$-cyclic May structure yields the notion of \emph{stable $r$-cyclic comultiplication}, which again defines cohomology operations $\power^i$ that need not to vanish when $i \neq 2(r-1)k, 2(r-1)k+1$.

\subsection{Connected comultiplications} The traditional way to understand the $r$-cyclic May structure on the chain complex of a simplicial set $X$ is to observe that the operad of natural endomorphisms of the chain complexes $\ucadenas(\simplex^n)$ is contractible in each arity. Thus, its homology is concentrated in degree $0$, and in that degree one finds the cup product. Then, one proceeds to build the cochain operations in higher degrees using the acyclicity of the rest of the complex. On the other hand, when one tries to produce a stable $r$-cyclic May structure, there is no special operation of degree $0$ to start with (the stable $E_{\infty}$ operad is acyclic, instead of contractible). Thus, endowing a family of chain complexes with such structure may be challenging.

In this article we introduce the notion of \emph{$r$-cyclic connected comultiplication} on a chain complex. A chain complex endowed with such structure inherits a stable comultiplication as well, but now, there is a special operation from which to start: the operation that induces $P^0$ on cohomology. Moreover, the cohomology operations induced by this stable comultiplication have Property (iv), but need not to be compatible with any associative multiplication a priori or have Property (iii). %In fact, we prove with some work that in some cases it is compatible with a certain multiplication.

%This new formalism changes the way the cochain level operations are constructed in practice: With the formalism of May, one is given first an associative product that will compute $\power^{(r-1)m}(x)$ if $x$ has degree $m$, and from that builds the rest of the cochain operations downwards towards $\power^0(x)$ ---and possibly beyond, in case $\power^0(x)\neq x$. With the new formalism, one builds first the cochain operation that induces $\power^0$, and then proceeds to build the higher operations upwards.

\subsection{Main result} We then use this new formalism to define cochain level formulas for power operations on simplicial objects as follows: Let $\cC$ be a distributive monoidal category with a natural distributive $r$-cyclic diagonal $\Delta\colon X\to X^{\otimes r}$ for each object $X$, and an $r$-cyclic coproduct-preserving monoidal functor $\abel\colon \cC\to \Mod{R}$ to the category of $R$-modules. By virtue of $L$, every simplicial object in $\cC$ has an associated simplicial $R$-module, from which one gets a chain complex. The main examples of $\cC$ are the categories $\Set$ and $\Setp$ of sets and pointed sets.

Define an \emph{$r$-cyclic assymmetry with duality} as a $\Cyc_r$-equivariant choice of an element $x_\tau\in \tau$ for every proper non-empty subset $\tau\subset \{0,1,\ldots,r-1\}$ such that the predecessor of $x_\tau$ is not in $\tau$. Such a choice exists if and only if $r$ is prime. Our main result is the following.

\begin{theorem}\label{thm:main}
	If $r$ is an odd prime, each $r$-cyclic asymmetry with duality yields a natural $r$-cyclic connected comultiplication on the chain complex of an augmented simplicial object in $\cC$.
\end{theorem}

If $r=2$ the theorem still holds, but only with coefficients in $\bF_2$. In that case there is a unique $r$-cyclic asymmetry with duality, and therefore our theorem yields a unique connected comultiplication. The induced unstable comultiplication is the same as the one of \cite{medina2021may_st}. In fact, the presentation of this $2$-cyclic May structure in \cite{medina2021fast_sq} contains the germ of the notion of a connected comultiplication. If $r=3$ there is again a unique $r$-cyclic asymmetry, and we have empirically tested that the $3$-cyclic May structure of \cite{medina2021may_st} coincides with the unstable comultiplication associated to our connected comultiplication. If $r= 5$, there are four $r$-cyclic assymetries, which yield four connected comultiplications, but none of them yield the unstable comultiplications obtained in \cite{medina2021may_st}: as we will see in Section \ref{s:suspension}, they behave differently with respect to suspension.

This new viewpoint has some advantages: first, the construction is very explicit: it is manageable to find the coefficient of any given summand. %In contrast, most of the summands in the description of the composition of the maps $\cE(r)\to \chi(r)\to EZ(r)$ vanish universally.
Second, our formulas hold for the chain complexes of augmented simplicial objects too ---these do not admit an unstable comultiplication in general. %Such upgrade in the May--Steenrod structure would involve a replacement of the complex of surjections by a complex of functions.
%Third, power operations $\power^i$ for $i$ small are simpler to compute with connected comultiplications, while if $i$ is close to $(r-1)m$, then they are simpler to compute with unstable comultiplications.

Third, the formula for a power operation $\power^i([x])$ of an unstable comultiplication depends heavily on the degree of the class $x$: they are parametrized by overlapping intervals and surjections, and both increase in size as the degree of $x$ increases. On the other hand our formulas are parametrized by pairs $(U,A)$, of which only $U$ changes when the degree of $x$ increases. As a consequence, computing $\power^i([x])$ for small $i$'s tends to be simpler for a connected comultiplication than for an unstable comultiplication, while computing them for $i$'s close to $(m-1)r$ is simpler for an unstable comultiplication than for a connected comultiplication.
\begin{question}
	When $\cC$ is the category of pointed sets, these connected comultiplications yield stable comultiplications that compute cohomology operations. Improve this stable comultiplication to a stable $r$-cyclic May structure.
\end{question}
%We assume that $r$ is prime from Section \ref{s:assembly} on, and we assume that $r$ is odd from Section \ref{s:mainresult} on.
\begin{question}
	Which constructions in this paper can be generalized to composite or even numbers? ($r$ is assumed to be odd from Section \ref{s:mainresult} on, and is assumed to be prime from Section \ref{s:assembly} on)\anibal{These seems like a randomly placed comment.}\federico{Is it better to place it here?}
\end{question}
\anibal{Are these questions for me or left in the paper?}\federico{left in the paper}
%The main example of a category $\cC$ satisfying the hypotheses are the category of sets, the category of pointed sets. If $\cC$ has a cyclic diagonal for $r= 2$, then simplicial objects in $\cC$ have natural diagonal maps that induce an associative product in cohomology. With some effort we will prove that

%\begin{theorem}
%	If $X$ is a simplicial object in $\cC$, the power operation $\power^{(r-1)m}(x)$ of a class of degree $m$ is the $m$-fold power of the multiplication of $x$ with itself.
%\end{theorem}
%This theorem applies if the category $\cC$ is the category of sets with the cartesian product or the category of pointed sets with the smash product.
%As a consequence, our connected comultiplications yield natural unstable comultiplications too when $X$ is not augmented.

%\begin{question}
%	The cochain complex of a simplicial object in $\cC$ has always a multiplication. If the tensor product in $\cC$ is symmetric monoidal, how clear is it that this multiplication extends to an $E_{\infty}$-structure? \federico{I believe that the answer is yes, since all the machinery of McClure-Smith et al will work fine. Even if it is not symmetric monoidal, the machinery works the same, the only difference being that the endomorphism operad will have an action of the cyclic group, instead of the full symmetric group. }
%\end{question}

%\begin{question}
%	Can one replace in this article the requirement that $L$ goes to $R$-modules by the requirement that $L$ goes to chain complexes?
%\end{question}

%\subsection{Simplicial sets}

\subsection{Kan spectra} Kan spectra \cite{Kan1963} are models of spectra introduced by Kan, that are not widespread used because of a lack of a convenient smash product, a situation that has been recently treated \cite{CKP2023}. Our viewpoint applies in this setting too, and yields natural connected comultiplications on the chain complexes of Kan spectra. Again, they do not extend to unstable comultiplications.

%{\bf Stabilization:} The cochain operations are not compatible with suspension: the suspension should send the $i$th cochain operation to the $i+r-1$ cochain operation in the suspension, but these two operations have different formulas. In [Gill20] these cochain operations in the cochain complex of a $\Sigma$-spectrum are studied, and they have the handicap that the formulas are indexed over infinite dimensional objects (the stable commutative operad is not of finite type)...

%\subsection{Desymmetrisations} The notion of $r$-cyclic desymmetrisation is new to the extend of our knowledge. A $2$-cyclic desymmetrisation was implicitely used by the first author in \cite{cantero-moran2020khovanov} to find Steenrod squares in augmented semi-simplicial objects in the Burnside $2$-category. In a future work we expect to find such structures for $r$ an odd prime, thus yielding expicit formulas for the odd power operations in Khovanov homology. For the reader acquainted with the surjection operad we remark that the operations that are involved in these desymmetrisations would correspond to summands indexed by non-overlapping intervals.

\subsection{Outline} In Section \ref{s:formulas} we give an algorithmic presentation of the formulas. In Section \ref{s:preliminaries} we set some sign conventions for chain complexes and the conventions for simplicial objects. The notion of connected comultiplication is presented in \cref{s:2bcomultiplications}. The construction of the connected comultiplication for augmented semi-simplicial objects of Theorem \ref{thm:main} has three main ingredients.
First, an algebra structure on the chain complex of the standard augmented simplex inducing a duality theorem on normalized chains, that is used in \cref{s:3complexes} to build a convenient chain subcomplex of the tensor product $\chains(X)^{\otimes r}$. Afterwards, a criterion to build a connected comultiplication is given.
Second, an explicit $\Cyc_r$-equivariant chain map $\Omega^*(r)^{\nf}\to W_*(r)$ between two resolutions of $\Cyc_r$ that fulfills the hypotheses of the criterion. This map is constructed in \cref{s:resolutions} up to the definition of certain map $f$. Third, the construction of this latter map $f$ is pursued using the barycentric and the pair subdivision of the simplex in Section \ref{section:atlast}. In \cref{s:suspension} we show that these operations behave well with respect to suspensions. In \cref{s:9Kanspectra} we briefly explain how our method yields natural connected comultiplications in the chain complex of a Kan spectrum.
% !TEX root = ../oddp.tex

\section{Introduction}\label{s:introduction}

\subsection{Steenrod power operations}

In 1946, Steenrod was looking for generalizations of the Hopf invariant when he found for $i \geq 0$ certain natural linear maps
\[
\Delta_i \colon \ucadenas(X) \to \ucadenas(X) \ot \ucadenas(X)
\]
on the singular chains of a space $X$, which extended the Alexander--Whitney diagonal $\Delta_0$.
These so-called \textit{cup-$i$ coproducts} satisfy $\bd\Delta_i + \Delta_i\bd = (1+\rho)\Delta_{i-1}$ for $i > 0$, where $\rho$ is the transposition of tensor factors, and give rise in mod~2 cohomology to the celebrated \emph{Steenrod squares}
\begin{equation}\label{eq:intro Sq}
	\begin{tikzcd}[column sep=small, row sep=0]
		\Sq^{k} \colon &[-15pt] H^*(X;\mathbb{F}_2) \arrow[r] & H^{*+k}(X;\mathbb{F}_2) \\
		& {[\alpha]} \arrow[r, mapsto] & {\big[(\alpha \otimes \alpha)\Delta_{k-\bars{\alpha}}\big]}.
	\end{tikzcd}
\end{equation}
The maps $\Delta_i$ effectively introduced by Steenrod in \cite{steenrod1947products} have been given alternative formulas in \cite{real1996computability,medina2021fast_sq} and are axiomatically characterized \cite{medina2022axiomatic}.
They can be reinterpreted as a single $\Sym_2$-equivariant chain map
\[
\mu \colon W_*(2) \ot \ucadenas(X) \to \ucadenas(X)^{\ot 2},
\]
where $W_*(2)$ is the minimal free resolution of the ground ring as a module over the group ring of $\Sym_2$, the symmetric group in two letter.

Using the acyclic carrier theorem, Steenrod showed in \cite{steenrod1952reduced} the existence for each $r$ of an $\Sym_r$-equivariant chain map
\[
\mu \colon V_\ast(r) \ot \ucadenas(X) \to \ucadenas(X)^{\ot r}
\]
where $V_\ast(r)$ is a free resolution of the base ring as module over the group ring of $\Sym_r$, and used it to define a cohomology operation for each class in the homology of $\Sym_r$.

In \cite{steenrod1953cyclic}, Steenrod concentrated in those operations coming from classes induced by an inclusion $\Cyc_r \leq \Sym_r$ of the cyclic group of order $r$, with $r$ assumed to be a prime.
In this case, the inclusion is surjective in homology with mod~$r$ coefficients and the resulting operations in the mod~$r$ cohomology of a space are known as \textit{Steenrod power operations}.
%\[
%\power^i \colon H^*(X;\bF_r) \to H^{*+i}(X;\bF_r)
%\]
%\begin{align*}
%	P^{k} \colon H^j(X;\bF_r)& \to H^{j+2k(r-1)}(X;\bF_r) & k \geq 0, \\
%	\beta P^k \colon H^j(X;\bF_r)& \to H^{j+2k(r-1)+1}(X;\bF_r) & k \geq 0,
%\end{align*}
%\[
%\begin{tikzcd}[column sep=tiny, row sep=0]
%	\power^i \colon &[-15pt] H^*(X;\bF_r) \arrow[r] & H^{*+i}(X;\bF_r) \\
%	& {[\alpha]} \arrow[r, mapsto] & (\alpha \ot \overset{r}{\dots} \ot \alpha)\Delta_{(r-1)\bars{x}-i}
%\end{tikzcd}
%\]

\subsection{Unstable diagonals}

In \cite{may1970general}, May gave a general approach to power operations including Steenrod's power operations \cite{steenrod1962cohomology} and Dyer--Lashof operations on the homology of infinite loop spaces \cite{dyer62lashof}.
The basic structure used in \cite{may1970general} for the construction of power operations on a chain complex $\ucadenas$ is a $\Cyc_r$-equivariant chain map
\[
\mu \colon W_*(r) \ot \ucadenas \to \ucadenas^{\ot r}
\]
where $W_*(r)$ is the minimal free resolution of the ground ring $R$ as a module over the group ring of $\Cyc_r$:
\[
\dotsb \overset{T}{\lra} \Cyc_r\langle e_2\rangle \overset{N}{\lra}
\Cyc_r\langle e_1\rangle \overset{T}{\lra}
\Cyc_r\langle e_0\rangle.
\]
We will refer to $\mu$ as an \textit{unstable $r$-cyclic diagonal} on $\ucadenas$, or simply an \textit{unstable $r$-diagonal}, saying that it is \textit{May--Steenrod} if it factors as
\[
\mu \colon W_*(r) \ot \ucadenas \xra{f \ot \id} V_*(r) \ot \ucadenas \xra{\nu} \ucadenas^{\ot r}
\]
with $f$ and $\nu$ a $\Cyc_r$- and a $\Sym_r$-equivariant map respectively.
We denote the maps $\mu(e_i \ot -) \colon \ucadenas \to \ucadenas^{\ot r}$ by $\Delta_{r,i}$ and refer to them as \textit{cup-$(r,i)$ coproducts}.

Generalizing the definition of $\Sq^i$ in \cref{eq:intro Sq} to a general prime $r$, a chain complex $\ucadenas$ with an unstable $r$-diagonal has power operations on its cohomology defined by
\begin{equation}\label{eq:intro P}
	\begin{tikzcd}[column sep=tiny, row sep=0]
		\power^i \colon &[-15pt] H^*(\ucadenas;\bF_r) \arrow[r] & H^{*+i}(\ucadenas;\bF_r) \\
		& {[\alpha]} \arrow[r, mapsto] & \big{[}(\alpha \ot \overset{r}{\dots} \ot \alpha)\Delta_{(r-1)\bars{\alpha}-i}\big{]},
	\end{tikzcd}
\end{equation}
where we have suppressed certain invertible coefficient.
We remark that the grading of this operations is different from the one used by Steenrod and May.

If the prime $r$ is odd and $\mu$ is May--Steenrod then
\[
\power^i = 0 \text{ unless } i = 2k(r-1) \text{ or }i = 2k(r-1)+1.
\]
Additionally, these operations often satisfy
\[
\power^i(x) = 0 \text{ for negative } i.
\]
For instance, this is the case with spaces.
Generally, verifying this property in practice is not straightforward.

An effective construction of cup-$(r,i)$ coproducts for simplicial sets depending solely on the combinatorics of simplices was introduced in \cite{medina2021may_st} and implemented in \cite{medina2021comch}.

\subsection{Stable diagonals}

Similarly to how Steenrod power operations on the cohomology of spaces motivated the definition of an unstable $r$-diagonal on a chain complex, we can consider spectra to motivate the definition of stable $r$-diagonals.

Let $E = \{E_m\}_{m \geq 0}$ be a spectrum. Its chains are given by
\[
\ucadenas(E) = \colim \left(\ucadenas(E_0) \to \dots \to \Sigma^{-m}\ucadenas(E_m) \to \Sigma^{-m-1} \ucadenas(E_{m+1}) \to \dotsb\right).
\]
The homotopy fibre of the norm map
\[
\W(r) \overset{N}{\lra} \Wd(r)
\]
is isomorphic to the unbounded chain complex $\Wst(r)$ defined as
\[
\dotsb \lla \Cyc_r\langle e_{-2}\rangle
\overset{T}{\lla}
\Cyc_r\langle e_{-1}\rangle
\overset{N}{\lla}
\Cyc_r\langle e_0\rangle
\overset{T}{\lla}
\Cyc_r\langle e_{1}\rangle
\overset{N}{\lla}
\Cyc_r\langle e_{2}\rangle
\overset{T}{\lla}
\dotsb,
\]
Any unstable diagonal suitably compatible with suspensions gives rise to an $\Cyc_r$-equivariant chain map
\[
\Wst(r) \ot \ucadenas(E) \lra \ucadenas(E)^{\ot r},
\]
which factors through a stable version $V_*^\mathrm{st}(r)$ of $V_*(r)$. We will see in Section \ref{s:suspension} that the effective construction of \cite{medina2021may_st} is indeed compatible with right suspension.
%If $r$ is a prime number, the singular cohomology with coefficients in $\bF_r$ of a spectrum $E$ admits power operations that vanish unless $i = 2(r-1)k$ or $i = 2(r-1)k+1$.

Abstracting this example we define a \emph{stable $r$-cyclic diagonal} on a chain complex $\ucadenas$, or simply a \textit{stable $r$-diagonal} on $\ucadenas$, as a $\Cyc_r$-equivariant chain map
\[
\mu \colon \Wst(r) \ot \ucadenas \lra \ucadenas^{\ot r}.
\]
We say $\mu$ is \textit{May--Steenrod} if it factors through $V_*^\mathrm{st}$ as in the previous section.

The same definition of power operations given in \cref{eq:intro P} applies in this context.

%If the spectrum $E$ is given as a sequence $\{E_m\}_{n \geq 0}$ of simplicial pointed sets with structural maps $\varepsilon_m \colon \Sigma E_m \to E_{m+1}$, then its cohomology can be computed as the cohomology of the chain complex
%\[
%\colim \left(\ucadenas(E_0) \to \dots \to \Sigma^{-m}\ucadenas(E_m) \to \Sigma^{-m-1} \ucadenas(E_{m+1}) \to \dots\right)
%\]
%This chain complex admits an action of the stabilization of the $E_{\infty}$-operad \cite{Gill2020}, and in \cref{s:suspension} we will show that the $r$-structure of \cite{medina2021may_st} on the chain complex of a simplicial set is good enough to endow the chain complex of a spectrum with a stable version of an $r$-cyclic May structure (see \cref{s:stable} for the definition).

%Dropping Property (ii) from the stable version of an $r$-cyclic May structure yields the notion of \emph{stable $r$-cyclic diagonal}, which again defines cohomology operations $\power^i$ that need not to vanish when $i \neq 2(r-1)k, 2(r-1)k+1$.

\subsection{Connected diagonals}

The negative power operations of a stable $r$-diagonal $\mu$ are supported on a certain graded chain subcomplex $A$ of the bicomplex $\Wst \otimes \ucadenas$. We introduce the notion of \emph{connected $r$-cyclic diagonal}, or \textit{connected $r$-diagonal} for short, as a stable $r$-diagonal that vanishes on that subcomplex. In particular, its power operations satisfy are such that $\power^i = 0$ for $i<0$.

Let $\rWd(r)$ be the augmented dual of the right suspension of $\W(r)$:
\[
\dotsb
\Cyc_r\langle e_{-3}^\dd\rangle
\overset{-N}{\lla}
\Cyc_r\langle e_{-2}^\dd\rangle
\overset{T}{\lla}
\Cyc_r\langle e_{-1}^{\dd}\rangle
\overset{-N}{\lla}
R
\]
We will define a certain chain complex $\rWd(r)\hotimes \rucadenas$ isomorphic as a $\Cyc_r$-module to $\rWd(r)\otimes \ucadenas$, and isomorphic as a bigraded $\Cyc_r$-module to the quotient of $\Wst(r)\otimes \ucadenas$ by the subcomplex $A$, so that a connected $r$-diagonal is the same as a chain map
\[
\mu\colon \rWd(r)\hotimes \rucadenas\lra \ucadenas^{\otimes r}.
\]
%The traditional way to understand the unstable $r$-diagonal structure on the chain complex of a simplicial set $X$ is to observe that the operad of natural endomorphisms of the chain complexes $\ucadenas(\simplex^n)$ is contractible in each arity.
%Thus, its homology is concentrated in degree $0$, and in that degree one finds the cup product.
%Then, one proceeds to build the cochain operations in higher degrees using the acyclicity of the rest of the complex.
%On the other hand,
%When one tries to produce a stable $r$-diagonal, there is no special operation of degree $0$ to start an inductive process.
%Thus, endowing a family of chain complexes with such structure is challenging in practice.
%We introduce the notion of \emph{connected $r$-cyclic diagonal} on a chain complex, or connected $r$-diagonal for short, and describe an effective construction naturally defining it on a general class of augmented simplicial objects.
%A chain complex endowed with a connected $r$-diagonal inherits a stable $r$-diagonal and its power operations satisfy $P^0 = \id$ and $P^i = 0$ for negative $i$.


%and define a connected $r$-diagonal to be a $\Cyc_r$-equivariant chain map
%\[
%\mu \colon \rWd(r) \hotimes \rucadenas \lra \ucadenas^{\ot r},
%\]
%where $\rWd(r) \hotimes \rucadenas$ is certain chain complex isomorphic as $\Cyc_r$-module to $\Wd(r) \ot \ucadenas$.
%\anibal{I think that one could define a single (bi?) functor from (pairs) of chain complexes to chain complexes. i.e. include the $()^r$ construction into the $\hotimes$.}

%Moreover, the cohomology operations induced by this stable diagonal have Property (iv), but need not to be compatible with any associative multiplication a priori or have Property (iii).
%With the new formalism, one builds first the cochain operation that induces $\power^0$, and then proceeds to build the higher operations upwards.

\subsection{Connected diagonals for simplicial objects}

Let $\cC$ be a monoidal category with a natural diagonal $\Delta \colon X \to X^{\ot r}$ for each object $X$, and a strong monoidal functor $\abel \colon \cC \to \Mod{R}$ to the category of $R$-modules.
By virtue of $L$, every simplicial object in $\cC$ has an associated simplicial $R$-module, from which one gets a chain complex.
The main examples of $\cC$ are the categories $\Set$ and $\Setp$ of sets and pointed sets.
Our main result is an effective construction of connected $r$-diagonals from the following combinatorial structure.

\begin{definition}
	An \emph{$r$-cyclic straightening with duality} is a $\Cyc_r$-equivariant choice of element $x_\tau\in \tau$ for every proper non-empty subset $\tau \subset \{0,1,\dots,r-1\}$ such that the predecessor of $x_\tau$ is not in $\tau$.
\end{definition}

We mention that this structure can exist only if $r$ is prime.

\begin{theorem}\label{thm:main}
	If $r$ is an odd prime, each $r$-cyclic straightening with duality yields a natural connected $r$-diagonal on the chain complex of an augmented simplicial object in $\cC$.
\end{theorem}

If $r=2$ the theorem still holds, but only with coefficients in $\bF_2$.
In that case there is a unique $r$-cyclic straightening with duality, and therefore our theorem yields a unique connected diagonal.
The induced unstable diagonal is the one axiomatized in \cite{medina2022axiomatic}.
%In fact, the presentation of this $2$-cyclic May structure in \cite{medina2021fast_sq} contains the germ of the notion of a connected diagonal.
If $r = 3$ there is again a unique $r$-cyclic straightening, and we have empirically tested that the unstable $3$-diagonal of \cite{medina2021may_st} coincides with the unstable diagonal induced by our connected diagonal.
If $r = 5$, there are four $r$-cyclic straightening, which yield four connected diagonals, but none of them yield the unstable diagonals obtained in \cite{medina2021may_st}; as we will see in \cref{s:suspension}, they behave differently with respect to suspension.

This new viewpoint has some advantages:
First, the construction is very explicit since it is manageable to find the coefficient of any given summand.
%In contrast, most of the summands in the description of the composition of the maps $\cE(r) \to \chi(r) \to EZ(r)$ vanish universally.
Second, our formulas hold for the chain complexes of augmented simplicial objects too ---these do not admit an unstable diagonal in general.\anibal{This sounds like a very important motivation}
%Such upgrade in the May--Steenrod structure would involve a replacement of the complex of surjections by a complex of functions.
%Third, power operations $\power^i$ for $i$ small are simpler to compute with connected diagonals, while if $i$ is close to $(r-1)m$, then they are simpler to compute with unstable diagonals.
%Third, the formula for a power operation $\power^i([x])$ of an unstable diagonal depends heavily on the degree of the class $x$: they are parametrized by overlapping intervals and surjections, and both increase in size as the degree of $x$ increases.
%On the other hand our formulas are parametrized by pairs $(U,A)$, of which only $U$ changes when the degree of $x$ increases.
%As a consequence,
Third, computing $\power^i([x])$ for small $i$'s tends to be simpler for a connected diagonal than for an unstable diagonal, while computing them for $i$'s close to $(\bars{x}-1)r$ is simpler for an unstable diagonal than for a connected diagonal.
%\begin{question}
%	When $\cC$ is the category of pointed sets, these connected diagonals yield stable diagonals that compute cohomology operations.
%	Improve this stable diagonal to a stable $r$-cyclic May structure.
%\end{question}
%We assume that $r$ is prime from \cref{s:assembly} on, and we assume that $r$ is odd from \cref{s:mainresult} on.
%\begin{question}
%	Which constructions in this paper can be generalized to composite or even numbers? ($r$ is assumed to be odd from \cref{s:mainresult} on, and is assumed to be prime from \cref{s:assembly} on)
%\end{question}
%The main example of a category $\cC$ satisfying the hypotheses are the category of sets, the category of pointed sets.
%If $\cC$ has a cyclic diagonal for $r= 2$, then simplicial objects in $\cC$ have natural diagonal maps that induce an associative product in cohomology.
%With some effort we will prove that

%\begin{theorem}
%	If $X$ is a simplicial object in $\cC$, the power operation $\power^{(r-1)m}(x)$ of a class of degree $m$ is the $m$-fold power of the multiplication of $x$ with itself.
%\end{theorem}
%This theorem applies if the category $\cC$ is the category of sets with the Cartesian product or the category of pointed sets with the smash product.
%As a consequence, our connected diagonals yield natural unstable diagonals too when $X$ is not augmented.

%\begin{question}
%	The cochain complex of a simplicial object in $\cC$ has always a multiplication.
%If the tensor product in $\cC$ is symmetric monoidal, how clear is it that this multiplication extends to an $E_{\infty}$-structure? \federico{I believe that the answer is yes, since all the machinery of McClure-Smith et al will work fine.
%	Even if it is not symmetric monoidal, the machinery works the same, the only difference being that the endomorphism operad will have an action of the cyclic group, instead of the full symmetric group.
%}
%\end{question}

%\begin{question}
%	Can one replace in this article the requirement that $L$ goes to $R$-modules by the requirement that $L$ goes to chain complexes?
%\end{question}

%\subsection{Kan spectra} Kan spectra \cite{Kan1963} are models of spectra introduced by Kan, that are not widespread used because of a lack of a convenient smash product, a situation that has been recently treated \cite{CKP2023}.
%Our viewpoint applies in this setting too, and yields natural connected diagonals on the chain complexes of Kan spectra.
%Again, they do not extend to unstable diagonals. \anibal{Let's talk about this. Can it be moved to the stable subsection?}

%{\bf Stabilization:} The cochain operations are not compatible with suspension: the suspension should send the $i$th cochain operation to the $i+r-1$ cochain operation in the suspension, but these two operations have different formulas.
%In [Gill20] these cochain operations in the cochain complex of a $\Sigma$-spectrum are studied, and they have the handicap that the formulas are indexed over infinite dimensional objects (the stable commutative operad is not of finite type)...

%\subsection{Desymmetrisations} The notion of $r$-cyclic desymmetrisation is new to the extend of our knowledge.

\subsection{Other connected diagonals}

We will see in Section \ref{s:suspension} that the connected diagonals of our main theorem are compatible both with right and left suspension of simplicial sets.
As a consequence, they yield also connected diagonals for suspension spectra.

As we have seen, unstable diagonals and connected diagonals on the normalized chain complex of a pointed simplicial set that are compatible with suspension yield stable diagonals in suspension spectra. This is no longer true for other models of spectra.

One of these models of spectra are the cubes in the Burnside $2$-category of Lawson, Lipshitz and Sarkar \cite{LLS20} which model Khovanov spectra \cite{LS14}. A connected $2$-cyclic diagonal for these objects was found by the first author in \cite{cantero-moran2020khovanov}, thus giving rise to explicit formulas to compute Steenrod squares in Khovanov homology. In a future work we expect to find connected $r$-cyclic diagonals for $r$ an odd prime, thus yielding explicit formulas for the odd power operations in Khovanov homology.

For the reader acquainted with the surjection operad we remark that the operations that are involved in these connected $r$-cyclic diagonals would correspond to summands indexed by non-overlapping intervals.

\subsection{Augmented simplex duality}

A key observation facilitating our construction, which we deem interesting in its own right, is the following structure on the normalized chains of the standard augmented simplices:

\begin{theorem}
	The join product defines on $\chains(\asimplex^{n})$ the structure of a Poincar\'e duality algebra with unit the empty simplex and formal dimension $n+1$.
\end{theorem}

\subsection{Outline}

In \cref{s:preliminaries} we set some sign conventions for chain complexes and the conventions for simplicial objects.
The notion of connected diagonal is presented in \cref{s:2bdiagonals}.
The construction of the connected diagonal for augmented semi-simplicial objects of Theorem \ref{thm:main} has three main ingredients.


First, an algebra structure on the chain complex of the standard augmented simplex inducing a duality theorem on normalized chains, that is used in \cref{s:3complexes} to build a convenient chain subcomplex of the tensor product $\chains(X)^{\ot r}$.
As an application of this first ingredient, we obtain a criterion to build a connected diagonal on the normalized chains of a simplicial object.


Second, in Section \ref{s:resolutions} we give a recipy to construct an explicit $\Cyc_r$-equivariant chain map $\Omega_*(r)^{\nf} \to W_*(r)$ between two resolutions of the cyclic group $\Cyc_r$. This chain map is enough to fulfil the hypotheses of the criterion.

Third, the latter recipy requires the construction of another chain map $f$, which is pursued using the barycentric and the pair subdivision of the simplex in \cref{section:atlast}. This completes the proof of Theorem \ref{thm:main}

In \cref{s:suspension} we show that these operations behave well with respect to suspensions.
In \cref{s:9Kanspectra} we briefly explain how our method yields natural connected diagonals in the chain complex of a Kan spectrum.
Finally, in \cref{s:formulas} we give an algorithmic presentation of the formulas.
These formulas are meant to be understandable after reading Sections \ref{s:preliminaries} and \ref{s:2bdiagonals}.
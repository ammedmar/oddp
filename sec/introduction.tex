% !TEX root = ../oddp.tex

\section{Introduction}\label{s:introduction}

\subsection{Steenrod power operations}

In 1946, Steenrod was looking for generalizations of the Hopf invariant when he found for $i \geq 0$ certain natural linear maps
\[
\Delta_i \colon \ucadenas(X) \to \ucadenas(X) \ot \ucadenas(X)
\]
on the singular chains of a space $X$, which extended the Alexander--Whitney diagonal $\Delta_0$.
These so-called \textit{cup-$i$ coproducts} satisfy $\bd\Delta_i + \Delta_i\bd = (1+\rho)\Delta_{i-1}$ for $i > 0$, where $\rho$ is the transposition of tensor factors, and give rise in mod~2 cohomology to the celebrated \emph{Steenrod squares}
\begin{equation}\label{eq:intro Sq}
	\begin{tikzcd}[column sep=small, row sep=0]
		\Sq^{k} \colon &[-15pt] H^*(X;\mathbb{F}_2) \arrow[r] & H^{*+k}(X;\mathbb{F}_2) \\
		& {[\alpha]} \arrow[r, mapsto] & {\big[(\alpha \otimes \alpha)\Delta_{k-\bars{\alpha}}\big]}.
	\end{tikzcd}
\end{equation}
The maps $\Delta_i$ effectively introduced by Steenrod in \cite{steenrod1947products} have been given alternative formulas in \cite{real1996computability,medina2021fast_sq} and are axiomatically characterized \cite{medina2022axiomatic}.
They can be reinterpreted as a single $\Cyc_2$-equivariant chain map
\[
\mu \colon W_*(2) \ot \ucadenas(X) \to \ucadenas(X)^{\ot 2},
\]
where $W_*(2)$ is the minimal free resolution of the ground ring as a module over the group ring of $\Cyc_2$, the cyclic group of order two.

Using the acyclic carrier theorem, Steenrod showed in \cite{steenrod1952reduced} the existence for each $r$ of an $\Sym_r$-equivariant chain map
\[
\mu \colon V_\ast(r) \ot \ucadenas(X) \to \ucadenas(X)^{\ot r}
\]
where $V_\ast(r)$ is a free resolution of the base ring as module over the group ring of the symmetric group $\Sym_r$, and used it to define a cohomology operation for each class in the homology of $\Sym_r$.

In \cite{steenrod1953cyclic}, Steenrod concentrated in those operations coming from classes induced by an inclusion $\Cyc_r \leq \Sym_r$ of the cyclic group of order $r$, with $r$ assumed to be a prime.
In this case, the inclusion is surjective in homology with mod~$r$ coefficients and the resulting operations in the mod~$r$ cohomology of a space are known as \textit{Steenrod power operations}.

\subsection{Unstable diagonals}

In \cite{may1970general}, May gave a general approach to power operations including Steenrod's power operations \cite{steenrod1962cohomology} and Dyer--Lashof operations on the homology of infinite loop spaces \cite{dyer62lashof}.
The basic structure used in \cite{may1970general} for the construction of power operations on a chain complex $\ucadenas$ is a $\Cyc_r$-equivariant chain map
\[
\mu \colon W_*(r) \ot \ucadenas \to \ucadenas^{\ot r}
\]
where $W_*(r)$ is the minimal free resolution of the ground ring $R$ as a module over the group ring of the cyclic group $\Cyc_r$:
\[
\Cyc_r\langle e_0\rangle
\overset{T}{\lla}
\Cyc_r\langle e_1\rangle 
\overset{N}{\lla}
\Cyc_r\langle e_2\rangle \overset{T}{\lla} 
\dotsb 
\]
We will refer to $\mu$ as an \textit{unstable $r$-cyclic diagonal} on $\ucadenas$, or simply an \textit{unstable $r$-diagonal}, saying that it is \textit{May--Steenrod} if it factors as
\[
\mu \colon W_*(r) \ot \ucadenas \xra{f \ot \id} V_*(r) \ot \ucadenas \xra{\nu} \ucadenas^{\ot r}
\]
with $f$ and $\nu$ a $\Cyc_r$- and a $\Sym_r$-equivariant map respectively.
We denote the maps $\mu(e_i \ot -) \colon \ucadenas \to \ucadenas^{\ot r}$ by $\Delta_{r,i}$ and refer to them as \textit{cup-$(r,i)$ coproducts}.

Generalizing the definition of $\Sq^i$ in \cref{eq:intro Sq} to a general prime $r$, a chain complex $\ucadenas$ with an unstable $r$-diagonal has power operations on its cohomology defined by
\begin{equation}\label{eq:intro P}
	\begin{tikzcd}[column sep=tiny, row sep=0]
		\power^i \colon &[-15pt] H^*(\ucadenas;\bF_r) \arrow[r] & H^{*+i}(\ucadenas;\bF_r) \\
		& {[\alpha]} \arrow[r, mapsto] & \big{[}(\alpha \ot \overset{r}{\dots} \ot \alpha)\Delta_{(r-1)\bars{\alpha}-i}\big{]},
	\end{tikzcd}
\end{equation}
where we have suppressed certain invertible coefficient.
We remark that the grading of this operations is different from the one used by Steenrod and May.

If the prime $r$ is odd and $\mu$ is May--Steenrod then
\[
\power^i = 0 \text{ unless } i = 2k(r-1) \text{ or }i = 2k(r-1)+1.
\]
Additionally, these operations often satisfy
\[
\power^i(x) = 0 \text{ for negative } i.
\]
For instance, this is the case with spaces.
Generally, verifying this property in practice is not straightforward.

An effective construction of cup-$(r,i)$ coproducts for simplicial sets depending solely on the combinatorics of simplices was introduced in \cite{medina2021may_st} and implemented in \cite{medina2021comch}.

\subsection{Stable diagonals}

Similarly to how Steenrod power operations on the cohomology of spaces motivated the definition of an unstable $r$-diagonal on a chain complex, we can consider spectra to motivate the definition of stable $r$-diagonals. Let $E = \{E_m\}_{m \geq 0}$ be a suspension spectrum. Its chains are given by
\[
\ucadenas(E) = \colim \left(\ucadenas(E_0) \to \dots \to \Sigma^{-m}\ucadenas(E_m) \to \Sigma^{-m-1} \ucadenas(E_{m+1}) \to \dotsb\right).
\]
The homotopy fibre of the norm map
\[
\W(r) \overset{N}{\lra} \Wd(r)
\]
is isomorphic to the acyclic unbounded chain complex $\Wst(r)$ defined as
\[
\dotsb \lla \Cyc_r\langle e_{-2}\rangle
\overset{T}{\lla}
\Cyc_r\langle e_{-1}\rangle
\overset{N}{\lla}
\Cyc_r\langle e_0\rangle
\overset{T}{\lla}
\Cyc_r\langle e_{1}\rangle
\overset{N}{\lla}
\Cyc_r\langle e_{2}\rangle
\overset{T}{\lla}
\dotsb,
\]
Any unstable diagonal suitably compatible with suspensions gives rise to an $\Cyc_r$-equivariant chain map
\[
\Wst(r) \ot \ucadenas(E) \lra \ucadenas(E)^{\ot r},
\]
which factors through a stable version $V_*^\mathrm{st}(r)$ of $V_*(r)$. We will see in Section \ref{s:suspension} that the effective construction of \cite{medina2021may_st} is indeed compatible with suspension.


Abstracting this example we define a \emph{stable $r$-cyclic diagonal} on a chain complex $\ucadenas$, or simply a \textit{stable $r$-diagonal} on $\ucadenas$, as a $\Cyc_r$-equivariant chain map
\[
\mu \colon \Wst(r) \ot \ucadenas \lra \ucadenas^{\ot r}.
\]
We say $\mu$ is \textit{May--Steenrod} if it factors through $V_*^\mathrm{st}$ as in the previous section.

The same definition of power operations given in \cref{eq:intro P} applies in this context.


\subsection{Connected diagonals}

The negative power operations of a stable $r$-diagonal $\mu$ are supported on a certain graded chain subcomplex $A$ of the bicomplex $\Wst \otimes \ucadenas$. We introduce the notion of \emph{connected $r$-cyclic diagonal}, or \textit{connected $r$-diagonal} for short, as a stable $r$-diagonal that vanishes on that subcomplex. In particular, its power operations are such that $\power^i = 0$ for $i<0$.

Let $\rWd(r)$ be the augmented dual of the right suspension of $\W(r)$:
\[
\dotsb
\Cyc_r\langle e_{-3}^\dd\rangle
\overset{-N}{\lla}
\Cyc_r\langle e_{-2}^\dd\rangle
\overset{T}{\lla}
\Cyc_r\langle e_{-1}^{\dd}\rangle
\overset{-N}{\lla}
R
\]
We will define a certain chain complex $\rWd(r)\hotimes \rucadenas$ isomorphic as a $\Cyc_r$-module to $\rWd(r)\otimes \ucadenas$, and isomorphic as a bigraded $\Cyc_r$-module to the quotient of $\Wst(r)\otimes \ucadenas$ by the subcomplex $A$, so that a connected $r$-diagonal is the same as a chain map
\[
\mu\colon \rWd(r)\hotimes \rucadenas\lra \ucadenas^{\otimes r}.
\]

\subsection{Connected diagonals for simplicial objects}

Let $\cC$ be a monoidal category with a natural diagonal $\Delta \colon X \to X^{\ot r}$ for each object $X$, and a strong monoidal functor $\abel \colon \cC \to \Mod{R}$ to the category of $R$-modules.
By virtue of $L$, every simplicial object in $\cC$ has an associated simplicial $R$-module, from which one gets a chain complex.
The main examples of $\cC$ are the categories $\Set$ and $\Setp$ of sets and pointed sets.
Our main result is an effective construction of connected $r$-diagonals from the following combinatorial structure.

\begin{definition}
	An \emph{$r$-cyclic straightening with duality} is a $\Cyc_r$-equivariant choice of element $x_\tau\in \tau$ for every proper non-empty subset $\tau \subset \{0,1,\dots,r-1\}$ such that the predecessor of $x_\tau$ is not in $\tau$.
\end{definition}

We mention that this structure can exist only if $r$ is prime.

\begin{theorem}\label{thm:main}
	If $r$ is an odd prime, each $r$-cyclic straightening with duality yields a natural connected $r$-diagonal on the chain complex of an augmented simplicial object in $\cC$.
\end{theorem}

If $r=2$ the theorem still holds, but only with coefficients in $\bF_2$.
In that case there is a unique $r$-cyclic straightening with duality, and therefore our theorem yields a unique connected diagonal.
When applied to a non-augmented simplicial set, it coincides with the unstable diagonal axiomatized in \cite{medina2022axiomatic}.
If $r = 3$ there is again a unique $r$-cyclic straightening, and we have empirically tested that the unstable $3$-diagonal of \cite{medina2021may_st} coincides with the unstable diagonal induced by our connected diagonal.
If $r = 5$, there are four $r$-cyclic straightenings, which yield four connected diagonals, but none of them yield the unstable diagonals obtained in \cite{medina2021may_st}; as we will see in \cref{s:suspension}, they behave differently with respect to suspension.

This new viewpoint has some advantages:
First, the construction is very explicit since it is manageable to find the coefficient of any given summand. Second, our formulas hold for the chain complexes of augmented simplicial objects too ---these do not admit an unstable diagonal in general.
Third, computing $\power^i([x])$ for small $i$'s tends to be simpler for a connected diagonal than for an unstable diagonal, while computing them for $i$'s close to $(\bars{x}-1)r$ is simpler for an unstable diagonal than for a connected diagonal.

\subsection{Other connected diagonals}

%We will see in Section \ref{s:suspension} that the connected diagonals of our main theorem are compatible with suspension.
%As a consequence, they yield also connected diagonals for the chain complex of a suspension spectra.

As we have seen, unstable diagonals and connected diagonals on the chain complex of a pointed simplicial set that are compatible with suspension yield stable diagonals in the chain complex of a suspension spectrum. This is no longer true for other models of spectra.

One of these models of spectra are the cubes in the Burnside $2$-category of Lawson, Lipshitz and Sarkar \cite{LLS20} which model Khovanov spectra \cite{LS14}. A connected $2$-cyclic diagonal for these objects was found by the first author in \cite{cantero-moran2020khovanov}, thus giving rise to explicit formulas to compute Steenrod squares in Khovanov homology. In a future work we expect to use the formulas of Theorem \ref{thm:main} to find connected $r$-cyclic diagonals for $r$ an odd prime, thus yielding explicit formulas for the odd power operations in Khovanov homology.

For the reader acquainted with the surjection operad we remark that the operations that are involved in these connected $r$-cyclic diagonals would correspond to summands indexed by non-overlapping intervals.

\subsection{Poincar\'e duality on the augmented simplex}

A key observation facilitating our construction, which we deem interesting in its own right, is the following structure on the normalized chains of the standard augmented simplices:

\begin{theorem}
	The join product defines on $\chains(\asimplex^{n})$ the structure of a Poincar\'e duality algebra with unit the empty simplex and formal dimension $n+1$.
\end{theorem}

\subsection{Outline}

In \cref{s:preliminaries} we set some sign conventions for chain complexes and the conventions for simplicial objects.
The notion of connected diagonal is presented in \cref{s:2bdiagonals}.
The construction of the connected diagonal for augmented semi-simplicial objects of Theorem \ref{thm:main} has three main ingredients.


First, an algebra structure on the chain complex of the standard augmented simplex inducing a duality theorem on normalized chains, that is used in \cref{s:3complexes} to build a convenient chain subcomplex of the tensor product $\chains(X)^{\ot r}$.
As an application of this first ingredient, we obtain a criterion to build a connected diagonal on the normalized chains of a simplicial object.


Second, in Section \ref{s:resolutions} we give a recipy to construct an explicit $\Cyc_r$-equivariant chain map $\Omega_*(r)^{\nf} \to W_*(r)$ between two resolutions of the cyclic group $\Cyc_r$. This chain map is enough to fulfil the hypotheses of the criterion.

Third, the latter recipy requires the construction of another chain map $f$, which is pursued using the barycentric and the pair subdivision of the simplex in \cref{section:atlast}. This completes the proof of Theorem \ref{thm:main}

In \cref{s:suspension} we show that these operations behave well with respect to suspensions.
In \cref{s:9Kanspectra} we briefly explain how our method yields natural connected diagonals in the chain complex of a Kan spectrum.
Finally, in \cref{s:formulas} we give an algorithmic presentation of the formulas.
These formulas are meant to be understandable after reading Sections \ref{s:preliminaries} and \ref{s:2bdiagonals}.
% !TEX root = ../oddp.tex

\section{Introduction} \label{s:introduction}

In 1946, Steenrod was looking for invariants of maps between spheres generalizing the Hopf invariant when he found certain cochain operations on simplicial complexes
\[\smile_i\colon C^k(X;\bF_2)\otimes C^j(X;\bF_2)\lra C^{k+j+i}(X;\bF_2),\qquad i\geq 0\]
which in cohomology gave rise to the celebrated \emph{Steenrod squares}
\[\Sq^{k}\colon H^j(X;\bF_2)\lra H^{j+k}(X;\bF_2).\]
Shortly after, these cohomology operations were given a more abstract and axiomatic treatment that avoided the explicit chain maps $\smile_i$. This new viewpoint exhibited new power operations
\begin{align*}
	\mathrm{P}^{k}\colon H^j(X;\bF_r)&\lra H^{j+2k(r-1)}(X;\bF_r) & k\geq 0 \\
	\beta \mathrm{P}^k\colon H^j(X;\bF_r)&\lra H^{j+2k(r-1)+1}(X;\bF_r) & k\geq 0
\end{align*}
for $r$ an odd prime. In \cite{may1970general}, May gave an axiomatic approach to the power operations via chain maps. He defined a category whose objects are chain complexes $C_*$ endowed with an associative multiplication together with an equivariant homomorphism $W_*(r)\otimes C_*\to C_*^{\otimes r}$ from the tensor product of the minimal free resolution of the cyclic group $\Cyc_r$ and the chain complex $C_*$ to the $r$-fold tensor product of $C_*$, satisfying two additional axioms regarding the multiplication and the action of the symmetric group. This structure was later baptised by the second author as a \emph{May-Steenrod structure}. In this article we coin the term \emph{$r$-cyclic unstable comultiplication} for a May-Steenrod structure without the second axiom. Every such object induces power operations (note the reindexing)
\begin{align*}
	\power^i\colon H^*(C)&\lra H^{*+i}(C), & i\geq 0
\end{align*}
and the second axiom assures that they vanish unless $i = 2k(r-1), 2k(r-1)+1$ if $r$ is odd. These operations satisfy that, if $x\in H^*(C)$ has even degree $m$, then $\power^{m(r-1)}(x)$ is the $r$-fold power of the multiplication of $x$ with itself and all higher power operations of $x$ vanish. Often they satisfy the additional condition that $\power^0(x) = x$ and $\power^i(x) = 0$ for negative $i$, but not always, and this tends to be hard to prove.

In this article we introduce an alternative formalism to encode power operations at the cochain level. We introduce a category of chain complexes with additional structure that we call \emph{$r$-cyclic connected comultiplication} with one extra axiom that plays a role similar to the first axiom of May (but not the same). As before, every object in this category has cohomology operations
\begin{align*}
	\power^i\colon H^*(C)&\lra H^{*+i}(C), &i\geq 0
\end{align*}
These operations satisfy always that $\power^0(x) = x$ and $\power^{i}(x) = 0$ for negative $i$, but need not to be compatible with any associative multiplication a priori. %In fact, we prove with some work that in some cases it is compatible with a certain multiplication.

This new formalism changes the way the cochain level operations are constructed in practice: With the formalism of May, one is given first an associative product that will compute $\power^{(r-1)m}(x)$ if $x$ has degree $m$, and from that builds the rest of the cochain operations downwards towards $\power^0(x)$ ---and possibly beyond, in case $\power^0(x)\neq x$. With the new formalism, one builds first the cochain operation that induces $\power^0$, and then proceeds to build the higher operations upwards.

We then use this new formalism to define power operations on simplicial objects as follows: We let $\cC$ be a distributive monoidal category with a natural distributive $r$-cyclic diagonal $\Delta\colon X\to X^{\otimes r}$ for each object $X$, and an $r$-cyclic coproduct-preserving monoidal functor $\abel\colon \cC\to \Mod{R}$ to the category of $R$-modules. By virtue of $L$, every simplicial object in $\cC$ has an associated simplicial $R$-module, from which one gets a chain complex.

Define an \emph{$r$-cyclic assymmetry} as an equivariant choice of an element $a\in A$ for every proper non-empty subset $A\subset \{0,1,\ldots,r-1\}$ and observe that they exist if and only if $r$ is prime. Our main result is the following.

\begin{theorem}\label{thm:main}
	Each $r$-cyclic asymmetry yields a natural $r$-cyclic connected comultiplication on the chain complex of augmented semi-simplicial objects in $\cC$.
\end{theorem}

Examples of categories $\cC$ satisfying the hypotheses are the category of sets, the category of pointed sets, the category of groups or the category of associative algebras. If $\cC$ has a cyclic diagonal for $r= 2$, then simplicial objects in $\cC$ have natural diagonal maps that induce an associative product in cohomology. With some effort we will prove that

\begin{theorem}
	If $X$ is a simplicial object in $\cC$, the power operation $\power^{(r-1)m}(x)$ of a class of degree $m$ is the $m$-fold power of the multiplication of $x$ with itself.
\end{theorem}
%This theorem applies if the category $\cC$ is the category of sets with the cartesian product or the category of pointed sets with the smash product.
As a consequence, our connected comultiplications yield natural unstable comultiplications too when $X$ is not augmented.

\begin{question}
	The cochain complex of a simplicial object in $\cC$ has always a multiplication. If the tensor product in $\cC$ is symmetric monoidal, how clear is it that this multiplication extends to an $E_{\infty}$-structure? \federico{I believe that the answer is yes, since all the machinery of McClure-Smith et al will work fine. Even if it is not symmetric monoidal, the machinery works the same, the only difference being that the endomorphism operad will have an action of the cyclic group, instead of the full symmetric group. }
\end{question}

\begin{question}
	Can one replace in this article the requirement that $L$ goes to $R$-modules by the requirement that $L$ goes to chain complexes?
\end{question}

\subsection*{Simplicial sets}

If $r=2$, there is a special May-Steenrod structure on simplicial sets that has been described in multiple occasions \cite{steenrod1947products,gonzalez-diaz1999steenrod,medina2021fast_sq}. If $r$ is odd, an explicit May-Steenrod structure was found by the second author \cite{medina2021may_st} as the composition of a certain map $W_*(r)\to \cE_*(r)$ to the arity $r$ operations in the Barratt-Eccles operad, another map $\cE_*(r)\to \chi_*(r)$ to the arity $r$ operations in the surjection operad \cite{berger2004combinatorial} and a third map $\chi_*(r)\to EZ(r)$ to the arity $r$ operation in the Eilenberg-Zilber operad\cite{mcclure2003multivariable}.

If $r=2$ there is a unique $r$-cyclic asymmetry, and therefore our theorem yields a unique connected comultiplication. The induced unstable comultiplication is the same as the one of the previous paragraph. In fact, the presentation of this May-Steenrod structure in \cite{medina2021fast_sq} contains the germ of the notion of a connected comultiplication. If $r=3$ there is again a unique $r$-cyclic asymmetry, and we have empirically tested that the May-Steenrod structure of \cite{medina2021may_st} coincides, up to a permutation of the tensor factors, with the unstable comultiplication associated to our connected comultiplication. If $r= 5$, there are four $r$-cyclic assymetries, which yield four connected comultiplications, but none of them yield the unstable comultiplications obtained in \cite{medina2021may_st}.

This new viewpoint has some advantages: first, the construction is very explicit: it is easy to find the coefficient of any given summand and it is easy to list all the non-vanishing summands. In contrast, most of the summands in the description of the maps $\cE(r)\to \chi(r)\to EZ(r)$ vanish universally. Second, our formulas hold for the chain complexes of augmented simplicial objects too ---these do not admit an unstable comultiplication in general. Such upgrade in the May-Steenrod structure would involve a replacement of the complex of surjections by a complex of functions. Third, power operations $\power^i$ for $i$ small are simpler to compute with connected comultiplications, while if $i$ is close to $(r-1)m$, then they are simpler to compute with unstable comultiplications. This latter fact makes connected comultiplications well-suited for computations of power operations in the cochain complex of a simplicial spectrum \cite{Gill2020}, that have no associative multiplication but satisfy that $\power^0([x])=[x]$. In \cref{s:suspension} we will show that the operations of this article are well-behaved both with left and right suspension, while the operations of \cite{medina2021may_st} are only well-behaved with right suspension.

\subsection*{Kan spectra} Kan spectra \cite{Kan1963} are models of spectra introduced by Kan, that fell in disgrace because of the lack of a convenient smash product, a situation that has been recently treated \cite{CKP2023}. Our viewpoint applies in this setting too, and yields natural connected comultiplications on the chain complexes of Kan spectra. Again, they do not extend to unstable comultiplications.

%{\bf Stabilization:} The cochain operations are not compatible with suspension: the suspension should send the $i$th cochain operation to the $i+r-1$ cochain operation in the suspension, but these two operations have different formulas. In [Gill20] these cochain operations in the cochain complex of a $\Sigma$-spectrum are studied, and they have the handicap that the formulas are indexed over infinite dimensional objects (the stable commutative operad is not of finite type)...

%\subsection{Desymmetrisations} The notion of $r$-cyclic desymmetrisation is new to the extend of our knowledge. A $2$-cyclic desymmetrisation was implicitely used by the first author in \cite{cantero-moran2020khovanov} to find Steenrod squares in augmented semi-simplicial objects in the Burnside $2$-category. In a future work we expect to find such structures for $r$ an odd prime, thus yielding expicit formulas for the odd power operations in Khovanov homology. For the reader acquainted with the surjection operad we remark that the operations that are involved in these desymmetrisations would correspond to summands indexed by non-overlapping intervals.

\subsection*{Outline}

The notion of connected comultiplication is presented in \cref{s:2bcomultiplications}.
The construction of the connected comultiplication for augmented semi-simplicial objects of Theorem \ref{thm:main} has two main ingredients.
First, an algebra structure on the chain complex of the standard augmented simplex inducing a duality theorem on normalized chains, that is used in \cref{s:3complexes} to build a manageable model of the tensor product $\chains_*(X)^{\otimes r}$.
Second, an explicit $\Cyc_r$-equivariant homomorphism $\Omega^*(r)^{\nf}\to W_*(r)$ between two resolutions of $\Cyc_r$, which is constructed in \cref{s:resolutions}.
In \cref{s:suspension} we show that these operations behave well with respect to suspensions, and in \cref{s:8products} we prove the second theorem.
In \cref{s:9Kanspectra} we explain how our method yields natural connected comultiplications in the chain complex of a Kan spectrum.